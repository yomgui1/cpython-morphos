\section{Standard Module \sectcode{mailbox}}
\stmodindex{mailbox}

\renewcommand{\indexsubitem}{(in module mailbox)}

This module defines a number of classes that allow easy and uniform
access to mail messages in a (unix) mailbox.

\begin{funcdesc}{UnixMailbox}{fp}
Access a classic unix-style mailbox, where all messages are contained
in a single file and separated by ``From name time'' lines. \var{Fp}
is the file object pointing to the mailbox file.
\end{funcdesc}

\begin{funcdesc}{MmdfMailbox}{fp}
Access an MMDF-style mailbox, where all messages are contained
in a single file and separated by lines consisting of 4 control-A
characters. \var{Fp}
is the file object pointing to the mailbox file.
\end{funcdesc}

\begin{funcdesc}{MHMailbox}{dirname}
Access an MH mailbox, a directory with each message in a separate
file with a numeric name. \var{Dirname}
is the name of the mailbox directory.
\end{funcdesc}

\subsection{Mailbox Objects}

All implementations of Mailbox objects have one externally visible
method:

\begin{funcdesc}{next}{}
Return the next message in the mailbox, as a \code{rfc822.Message} object.
Depending on the mailbox implementation the \var{fp} attribute of this
object may be a true file object or a class instance simulating a file object,
taking care of things like message boundaries if multiple mail messages are
contained in a single file, etc.
\end{funcdesc}

