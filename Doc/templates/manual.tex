\documentclass{manual}

\title{Big Python Manual}

\author{Your Name Here}

% Please at least include a long-lived email address;
% the rest is at your discretion.
\authoraddress{
	Organization name, if applicable \\
	Street address, if you want to use it \\
	E-mail: \email{your-email@your.domain}
}

\date{April 30, 1999}		% update before release!
				% Use an explicit date so that reformatting
				% doesn't cause a new date to be used.  Setting
				% the date to \today can be used during draft
				% stages to make it easier to handle versions.

\release{x.y}			% release version; this is used to define the
				% \version macro

\makeindex			% tell \index to actually write the .idx file
\makemodindex			% If this contains a lot of module sections.


\begin{document}

\maketitle

% This makes the contents more accessible from the front page of the HTML.
\ifhtml
\chapter*{Front Matter\label{front}}
\fi

%Copyright 1991, 1992, 1993, 1994 by Stichting Mathematisch Centrum,
Amsterdam, The Netherlands.

\begin{center}
All Rights Reserved
\end{center}

Permission to use, copy, modify, and distribute this software and its
documentation for any purpose and without fee is hereby granted,
provided that the above copyright notice appear in all copies and that
both that copyright notice and this permission notice appear in
supporting documentation, and that the names of Stichting Mathematisch
Centrum or CWI not be used in advertising or publicity pertaining to
distribution of the software without specific, written prior permission.

STICHTING MATHEMATISCH CENTRUM DISCLAIMS ALL WARRANTIES WITH REGARD TO
THIS SOFTWARE, INCLUDING ALL IMPLIED WARRANTIES OF MERCHANTABILITY AND
FITNESS, IN NO EVENT SHALL STICHTING MATHEMATISCH CENTRUM BE LIABLE
FOR ANY SPECIAL, INDIRECT OR CONSEQUENTIAL DAMAGES OR ANY DAMAGES
WHATSOEVER RESULTING FROM LOSS OF USE, DATA OR PROFITS, WHETHER IN AN
ACTION OF CONTRACT, NEGLIGENCE OR OTHER TORTIOUS ACTION, ARISING OUT
OF OR IN CONNECTION WITH THE USE OR PERFORMANCE OF THIS SOFTWARE.


\begin{abstract}

\noindent
Big Python is a special version of Python for users who require larger 
keys on their keyboards.  It accomodates their special needs by ...

\end{abstract}

\tableofcontents


\chapter{...}

My chapter.


\appendix
\chapter{...}

My appendix.

The \code{\e appendix} markup need not be repeated for additional
appendices.


%
%  The ugly "%begin{latexonly}" pseudo-environments are really just to
%  keep LaTeX2HTML quiet during the \renewcommand{} macros; they're
%  not really valuable.
%
%  If you don't want the Module Index, you can remove all of this up
%  until the second \input line.
%
%begin{latexonly}
\renewcommand{\indexname}{Module Index}
%end{latexonly}
\input{mod\jobname.ind}		% Module Index

%begin{latexonly}
\renewcommand{\indexname}{Index}
%end{latexonly}
\input{\jobname.ind}			% Index

\end{document}
