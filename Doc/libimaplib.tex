% Based on HTML documentation by Piers Lauders <piers@staff.cs.usyd.edu.au>;
% converted by Fred L. Drake, Jr. <fdrake@acm.org>.
%
% The imaplib module was written by Piers Lauders.

\section{Standard Module \module{imaplib}}
\stmodindex{imaplib}
\label{module-imaplib}

This module defines a class, \class{IMAP4}, which encapsulates a
connection to an IMAP4 server and implements the IMAP4rev1 client
protocol as defined in \rfc{2060}. It is backward compatible with
IMAP4 (\rfc{1730}) servers, but note that the \samp{STATUS} command is
not supported in IMAP4.

A single class is provided by the \code{imaplib} module:

\begin{classdesc}{IMAP4}{\optional{host\optional{, port}}}
This class implements the actual IMAP4 protocol.  The connection is
created and protocol version (IMAP4 or IMAP4rev1) is determined when
the instance is initialized.
If \var{host} is not specified, \code{''} (the local host) is used.
If \var{port} is omitted, the standard IMAP4 port (143) is used.
\end{classdesc}

Two exceptions are defined as attributes of the \class{IMAP4} class:

\begin{excdesc}{IMAP4.error}
Exception raised on any errors.  The reason for the exception is
passed to the constructor as a string.
\end{excdesc}

\begin{excdesc}{IMAP4.abort}
IMAP4 server errors cause this exception to be raised.  This is a
sub-class of \exception{IMAP4.error}.  Note that closing the instance
and instantiating a new one will usually allow recovery from this
exception.
\end{excdesc}

The following utility functions are defined:

\begin{funcdesc}{Internaldate2tuple}{datestr}
  Converts an IMAP4 INTERNALDATE string to Coordinated Universal
  Time. Returns a \module{time} module tuple.
\end{funcdesc}

\begin{funcdesc}{Int2AP}{num}
  Converts an integer into a string representation using characters
  from the set [\code{A} .. \code{P}].
\end{funcdesc}

\begin{funcdesc}{ParseFlags}{flagstr}
  Converts an IMAP4 \samp{FLAGS} response to a tuple of individual
  flags.
\end{funcdesc}

\begin{funcdesc}{Time2Internaldate}{date_time}
  Converts a \module{time} module tuple to an IMAP4
  \samp{INTERNALDATE} representation.  Returns a string in the form:
  \code{"DD-Mmm-YYYY HH:MM:SS +HHMM"} (including double-quotes).
\end{funcdesc}


\subsection{IMAP4 Objects}

All IMAP4rev1 commands are represented by methods of the same name,
either upper-case or lower-case.

Each command returns a tuple: \code{(}\var{type}, \code{[}\var{data},
...\code{])} where \var{type} is usually \code{'OK'} or \code{'NO'},
and \var{data} is either the text from the command response, or
mandated results from the command.

An \class{IMAP4} instance has the following methods:


\begin{methoddesc}{append}{mailbox, flags, date_time, message}
  Append message to named mailbox. 
\end{methoddesc}

\begin{methoddesc}{authenticate}{func}
  Authenticate command --- requires response processing. This is
  currently unimplemented, and raises an exception. 
\end{methoddesc}

\begin{methoddesc}{check}{}
  Checkpoint mailbox on server. 
\end{methoddesc}

\begin{methoddesc}{close}{}
  Close currently selected mailbox. Deleted messages are removed from
  writable mailbox. This is the recommended command before
  \samp{LOGOUT}.
\end{methoddesc}

\begin{methoddesc}{copy}{message_set, new_mailbox}
  Copy \var{message_set} messages onto end of \var{new_mailbox}. 
\end{methoddesc}

\begin{methoddesc}{create}{mailbox}
  Create new mailbox named \var{mailbox}.
\end{methoddesc}

\begin{methoddesc}{delete}{mailbox}
  Delete old mailbox named \var{mailbox}.
\end{methoddesc}

\begin{methoddesc}{expunge}{}
  Permanently remove deleted items from selected mailbox. Generates an
  \samp{EXPUNGE} response for each deleted message. Returned data
  contains a list of \samp{EXPUNGE} message numbers in order
  received.
\end{methoddesc}

\begin{methoddesc}{fetch}{message_set, message_parts}
  Fetch (parts of) messages. Returned data are tuples of message part
  envelope and data.
\end{methoddesc}

\begin{methoddesc}{list}{directory='""', pattern='*'}
  List mailbox names in directory matching pattern. Returned data contains a
  list of \samp{LIST} responses.
\end{methoddesc}

\begin{methoddesc}{login}{user, password}
  Identify the client using a plaintext password.
\end{methoddesc}

\begin{methoddesc}{logout}{}
  Shutdown connection to server. Returns server \samp{BYE} response.
\end{methoddesc}

\begin{methoddesc}{lsub}{directory='""', pattern='*'}
  List subscribed mailbox names in directory matching
  pattern. Returned data are tuples of message part envelope and data.
\end{methoddesc}

\begin{methoddesc}{recent}{}
  Prompt server for an update. Returned data is \code{None} if no new
  messages, else value of \samp{RECENT} response.
\end{methoddesc}

\begin{methoddesc}{rename}{oldmailbox, newmailbox}
  Rename mailbox named \var{oldmailbox} to \var{newmailbox}.
\end{methoddesc}

\begin{methoddesc}{response}{code}
  Return data for response \var{code} if received, or
  \code{None}. Returns the given code, instead of the usual type.
\end{methoddesc}

\begin{methoddesc}{search}{charset, criteria}
  Search mailbox for matching messages. Returned data contains a space
  separated list of matching message numbers.
\end{methoddesc}

\begin{methoddesc}{select}{\optional{mailbox\optional{, readonly}}}
  Select a mailbox. Returned data is the count of messages in
  \var{mailbox} (\samp{EXISTS} response).  The default \var{mailbox}
  is \code{'INBOX'}.  If the \var{readonly} flag is set, modifications
  to the mailbox are not allowed.
\end{methoddesc}

\begin{methoddesc}{status}{mailbox, names}
  Request named status conditions for \var{mailbox}. 
\end{methoddesc}

\begin{methoddesc}{store}{message_set, command, flag_list}
  Alters flag dispositions for messages in mailbox.
\end{methoddesc}

\begin{methoddesc}{subscribe}{mailbox}
  Subscribe to new mailbox.
\end{methoddesc}

\begin{methoddesc}{uid}{command, args}
  Execute command args with messages identified by UID, rather than
  message number. Returns response appropriate to command.
\end{methoddesc}

\begin{methoddesc}{unsubscribe}{mailbox}
  Unsubscribe from old mailbox.
\end{methoddesc}

\begin{methoddesc}{xatom}{name\optional{, arg1\optional{, arg2}}}
  Allow simple extension commands notified by server in
  \samp{CAPABILITY} response.
\end{methoddesc}


\class{IMAP4} instances have a variable \member{PROTOCOL_VERSION} that
is set to the most recent supported protocol in the \samp{CAPABILITY}
response.

Finally, \class{IMAP4} instances have a variable debug which can be
set to an integer to turn on debugging.  Values greater than 3 trace
each command.


\subsection{IMAP4 Example}

Here is a minimal example (without error checking) that opens a
mailbox and retrieves and prints all messages:

\begin{verbatim}
import getpass, imaplib, string
M = imaplib.IMAP4()
M.LOGIN(getpass.getuser(), getpass.getpass())
M.SELECT()
typ, data = M.SEARCH(None, 'ALL')
for num in string.split(data[0]):
    typ, data - M.FETCH(num, '(RFC822)')
    print 'Message %s\n%s\n' % (num, data[0][1])
M.LOGOUT()
\end{verbatim}

Note that IMAP4 message numbers change as the mailbox changes, so it
is highly advisable to use UIDs instead, with the UID command.

At the end of the module, there is a test section that contains a more
extensive example of usage.

\begin{seealso}
\seetext{Documents describing the protocol, and sources and binaries
for servers implementing it, can all be found at the University of
Washington's \emph{IMAP Information Center}
(\url{http://www.cac.washington.edu/imap/}).}
\end{seealso}
