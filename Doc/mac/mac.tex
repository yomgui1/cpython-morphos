\documentclass{manual}

\title{Macintosh Library Modules}

\author{Guido van Rossum\\
	Fred L. Drake, Jr., editor}
\authoraddress{
	PythonLabs\\
	E-mail: \email{python-docs@python.org}
}

\date{June 15, 2001}		% XXX update before release!
\release{2.0.1c1}		% software release, not documentation
\setshortversion{2.0}		% major.minor only for software


\makeindex              % tell \index to actually write the .idx file
\makemodindex           % ... and the module index as well.


\begin{document}

\maketitle

\ifhtml
\chapter*{Front Matter\label{front}}
\fi

Copyright 1991, 1992, 1993, 1994 by Stichting Mathematisch Centrum,
Amsterdam, The Netherlands.

\begin{center}
All Rights Reserved
\end{center}

Permission to use, copy, modify, and distribute this software and its
documentation for any purpose and without fee is hereby granted,
provided that the above copyright notice appear in all copies and that
both that copyright notice and this permission notice appear in
supporting documentation, and that the names of Stichting Mathematisch
Centrum or CWI not be used in advertising or publicity pertaining to
distribution of the software without specific, written prior permission.

STICHTING MATHEMATISCH CENTRUM DISCLAIMS ALL WARRANTIES WITH REGARD TO
THIS SOFTWARE, INCLUDING ALL IMPLIED WARRANTIES OF MERCHANTABILITY AND
FITNESS, IN NO EVENT SHALL STICHTING MATHEMATISCH CENTRUM BE LIABLE
FOR ANY SPECIAL, INDIRECT OR CONSEQUENTIAL DAMAGES OR ANY DAMAGES
WHATSOEVER RESULTING FROM LOSS OF USE, DATA OR PROFITS, WHETHER IN AN
ACTION OF CONTRACT, NEGLIGENCE OR OTHER TORTIOUS ACTION, ARISING OUT
OF OR IN CONNECTION WITH THE USE OR PERFORMANCE OF THIS SOFTWARE.


\begin{abstract}

\noindent
This library reference manual documents Python's extensions for the
Macintosh.  It should be used in conjunction with the
\citetitle[../lib/lib.html]{Python Library Reference}, which documents
the standard library and built-in types.

This manual assumes basic knowledge about the Python language.  For an
informal introduction to Python, see the
\citetitle[../tut/tut.html]{Python Tutorial}; the
\citetitle[../ref/ref.html]{Python Reference Manual} remains the
highest authority on syntactic and semantic questions.  Finally, the
manual entitled \citetitle[../ext/ext.html]{Extending and Embedding
the Python Interpreter} describes how to add new extensions to Python
and how to embed it in other applications.

\end{abstract}

\tableofcontents


\chapter{Using Python on a Mac OS 9 Macintosh \label{using}}
\sectionauthor{Bob Savage}{bobsavage@mac.com}

Using Python on a Mac OS 9 Macintosh can seem like something completely
different than using it on a \UNIX-like or Windows system. Most of the
Python documentation, both the ``official'' documentation and
published books, describe only how Python is used on these systems,
causing confusion for the new user of MacPython-OS9. This chapter gives a
brief introduction to the specifics of using Python on a Macintosh.

Note that this chapter is mainly relevant to Mac OS 9: MacPython-OSX
is a superset of a normal unix Python. While MacPython-OS9 runs fine
on Mac OS X it is a better choice to use MacPython-OSX there.

The section on the IDE (see Section \ref{IDE}) is relevant to MacPython-OSX
too.

\section{Getting and Installing MacPython-OS9 \label{getting}}

The most recent release version as well as possible newer experimental
versions are best found at the MacPython page maintained by Jack
Jansen: \url{http://www.cwi.nl/\textasciitilde jack/macpython.html}.


Please refer to the \file{README} included with your distribution for
the most up-to-date instructions.


\section{Entering the interactive Interpreter
         \label{interpreter}}

The interactive interpreter that you will see used in Python
documentation is started by double-clicking the
\program{PythonInterpreter} icon, which looks like a 16-ton weight
falling. You should see the version information and the
\samp{>\code{>}>~} prompt.  Use it exactly as described in the
standard documentation.


\section{How to run a Python script}

There are several ways to run an existing Python script; two common
ways to run a Python script are ``drag and drop'' and ``double
clicking''.  Other ways include running it from within the IDE (see
Section \ref{IDE}), or launching via AppleScript.


\subsection{Drag and drop}

One of the easiest ways to launch a Python script is via ``Drag and
Drop''. This is just like launching a text file in the Finder by
``dragging'' it over your word processor's icon and ``dropping'' it
there. Make sure that you use an icon referring to the
\program{PythonInterpreter}, not the \program{IDE} or \program{Idle}
icons which have different behaviour which is described below.

Some things that might have gone wrong:

\begin{itemize}
\item
A window flashes after dropping the script onto the
\program{PythonInterpreter}, but then disappears. Most likely this is a
configuration issue; your \program{PythonInterpreter} is setup to exit
immediately upon completion, but your script assumes that if it prints
something that text will stick around for a while. To fix this, see
section \ref{defaults}.

\item
When you waved the script icon over the \program{PythonInterpreter},
the \program{PythonInterpreter} icon did not hilight.  Most likely the
Creator code and document type is unset (or set incorrectly) -- this
often happens when a file originates on a non-Mac computer.  See
section \ref{creator-code} for more details.
\end{itemize}


\subsection{Set Creator and Double Click \label{creator-code}}

If the script that you want to launch has the appropriate Creator Code
and File Type you can simply double-click on the script to launch it.
To be ``double-clickable'' a file needs to be of type \samp{TEXT},
with a creator code of \samp{Pyth}.

Setting the creator code and filetype can be done with the IDE (see
sections \ref{IDEwrite} and \ref{IDEapplet}), with an editor with a
Python mode (\program{BBEdit}) -- see section
\ref{scripting-with-BBedit}, or with assorted other Mac utilities, but
a script (\file{fixfiletypes.py}) has been included in the MacPython
distribution, making it possible to set the proper Type and Creator
Codes with Python.

The \file{fixfiletypes.py} script will change the file type and
creator codes for the indicated directory.  To use
\file{fixfiletypes.py}:

\begin{enumerate}
\item
Locate it in the \file{scripts} folder of the \file{Mac} folder of the
MacPython distribution.

\item
Put all of the scripts that you want to fix in a folder with nothing
else in it.

\item
Double-click on the \file{fixfiletypes.py} icon.

\item
Navigate into the folder of files you want to fix, and press the
``Select current folder'' button.
\end{enumerate}


\section{Simulating command line arguments
         \label{argv}}

There are two ways to simulate command-line arguments with MacPython-OS9.
 
\begin{enumerate}
\item via Interpreter options
\begin{itemize} % nestable? I hope so!
  \item Hold the option-key down when launching your script. This will
        bring up a dialog box of Python Interpreter options.
  \item Click ``Set \UNIX-style command line..'' button. 
  \item Type the arguments into the ``Argument'' field.
  \item Click ``OK''
  \item Click ``Run''.
\end{itemize} % end

\item via drag and drop
If you save the script as an applet (see Section \ref{IDEapplet}), you
can also simulate some command-line arguments via
``Drag-and-Drop''. In this case, the names of the files that were
dropped onto the applet will be appended to \code{sys.argv}, so that
it will appear to the script as though they had been typed on a
command line.  As on \UNIX\ systems, the first item in \code{sys.srgv} is
the path to the applet, and the rest are the files dropped on the
applet.
\end{enumerate}


\section{Creating a Python script}

Since Python scripts are simply text files, they can be created in any
way that text files can be created, but some special tools also exist
with extra features.


\subsection{In an editor}

You can create a text file with any word processing program such as
\program{MSWord} or \program{AppleWorks} but you need to make sure
that the file is saved as ``\ASCII'' or ``plain text''.


\subsubsection{Editors with Python modes}

Several text editors have additional features that add functionality
when you are creating a Python script.  These can include coloring
Python keywords to make your code easier to read, module browsing, or
a built-in debugger. These include \program{Alpha}, \program{Pepper},
and \program{BBedit}, and the MacPython IDE (Section \ref{IDE}).

%\subsubsection{Alpha}
% **NEED INFO HERE**
 
\subsubsection{BBedit \label{scripting-with-BBedit}}

If you use \program{BBEdit} to create your scripts you will want to tell it about the Python creator code so that
you can simply double click on the saved file to launch it.
\begin{itemize}
  \item Launch \program{BBEdit}.
  \item Select ``Preferences'' from the ``Edit'' menu.
  \item Select ``File Types'' from the scrolling list.
  \item click on the ``Add...'' button and navigate to
        \program{PythonInterpreter} in the main directory of the
        MacPython distribution; click ``open''.
  \item Click on the ``Save'' button in the Preferences panel.
\end{itemize}
% Are there additional BBedit Python-specific features? I'm not aware of any.
 
%\subsubsection{IDE}
%You can use the \program{Python IDE} supplied in the MacPython Distribution to create longer Python scripts 
%-- see Section \ref{IDEwrite} for details.
 
%\subsubsection{IDLE}
%Idle is an IDE for Python that was written in Python, using TKInter. You should be able to use it on a Mac by following
%the standard documentation, but see Section \ref{TKInter} for guidance on using TKInter with MacPython.

%\subsubsection{Pepper}
% **NEED INFO HERE**


\section{The IDE\label{IDE}}

The \program{Python IDE} (Integrated Development Environment) is a
separate application that acts as a text editor for your Python code,
a class browser, a graphical debugger, and more.


\subsection{Using the ``Python Interactive'' window}

Use this window like you would the \program{PythonInterpreter}, except
that you cannot use the ``Drag and drop'' method above. Instead,
dropping a script onto the \program{Python IDE} icon will open the
file in a separate script window (which you can then execute manually
-- see section \ref{IDEexecution}).


\subsection{Writing a Python Script \label{IDEwrite}}

In addition to using the \program{Python IDE} interactively, you can
also type out a complete Python program, saving it incrementally, and
execute it or smaller selections of it.

You can create a new script, open a previously saved script, and save
your currently open script by selecting the appropriate item in the
``File'' menu. Dropping a Python script onto the
\program{Python IDE} will open it for editting.

If you try to open a script with the \program{Python IDE} but either
can't locate it from the ``Open'' dialog box, or you get an error
message like ``Can't open file of type ...'' see section
\ref{creator-code}.

When the \program{Python IDE} saves a script, it uses the creator code
settings which are available by clicking on the small black triangle
on the top right of the document window, and selecting ``save
options''. The default is to save the file with the \program{Python
IDE} as the creator, this means that you can open the file for editing
by simply double-clicking on its icon. You might want to change this
behaviour so that it will be opened by the
\program{PythonInterpreter}, and run. To do this simply choose
``Python Interpreter'' from the ``save options''. Note that these
options are associated with the \emph{file} not the application.


\subsection{Executing a script from within the IDE
            \label{IDEexecution}}

You can run the script in the frontmost window of the \program{Python
IDE} by hitting the run all button.  You should be aware, however that
if you use the Python convention \samp{if __name__ == "__main__":} the
script will \emph{not} be ``__main__'' by default. To get that
behaviour you must select the ``Run as __main__'' option from the
small black triangle on the top right of the document window.  Note
that this option is associated with the \emph{file} not the
application. It \emph{will} stay active after a save, however; to shut
this feature off simply select it again.
 

\subsection{``Save as'' versus ``Save as Applet''
            \label{IDEapplet}}

When you are done writing your Python script you have the option of
saving it as an ``applet'' (by selecting ``Save as applet'' from the
``File'' menu). This has a significant advantage in that you can drop
files or folders onto it, to pass them to the applet the way
command-line users would type them onto the command-line to pass them
as arguments to the script. However, you should make sure to save the
applet as a separate file, do not overwrite the script you are
writing, because you will not be able to edit it again.

Accessing the items passed to the applet via ``drag-and-drop'' is done
using the standard \member{sys.argv} mechanism. See the general
documentation for more
% need to link to the appropriate place in non-Mac docs

Note that saving a script as an applet will not make it runnable on a
system without a Python installation.

%\subsection{Debugger}
% **NEED INFO HERE**
 
%\subsection{Module Browser}
% **NEED INFO HERE**
 
%\subsection{Profiler}
% **NEED INFO HERE**
% end IDE

%\subsection{The ``Scripts'' menu}
% **NEED INFO HERE**
 
\section{Configuration \label{configuration}}

The MacPython distribution comes with \program{EditPythonPrefs}, an
applet which will help you to customize the MacPython environment for
your working habits.
 
\subsection{EditPythonPrefs\label{EditPythonPrefs}}

\program{EditPythonPrefs} gives you the capability to configure Python
to behave the way you want it to.  There are two ways to use
\program{EditPythonPrefs}, you can use it to set the preferences in
general, or you can drop a particular Python engine onto it to
customize only that version. The latter can be handy if, for example,
you want to have a second copy of the \program{PythonInterpreter} that
keeps the output window open on a normal exit even though you prefer
to normally not work that way.

To change the default preferences, simply double-click on
\program{EditPythonPrefs}. To change the preferences only for one copy
of the Interpreter, drop the icon for that copy onto
\program{EditPythonPrefs}.  You can also use \program{EditPythonPrefs}
in this fashion to set the preferences of the \program{Python IDE} and
any applets you create -- see section %s \ref{BuildApplet} and
\ref{IDEapplet}.

\subsection{Adding modules to the Module Search Path
            \label{search-path}}

When executing an \keyword{import} statement, Python looks for modules
in places defined by the \member{sys.path} To edit the
\member{sys.path} on a Mac, launch \program{EditPythonPrefs}, and
enter them into the largish field at the top (one per line).

Since MacPython defines a main Python directory, the easiest thing is
to add folders to search within the main Python directory. To add a
folder of scripts that you created called ``My Folder'' located in the
main Python Folder, enter \samp{\$(PYTHON):My Folder} onto a new line.

To add the Desktop under OS 9 or below, add
\samp{StartupDriveName:Desktop Folder} on a new line.

\subsection{Default startup options \label{defaults}}

% I'm assuming that there exists some other documentation on the
% rest of the options so I only go over a couple here.

The ``Default startup options...'' button in the
\program{EditPythonPrefs} dialog box gives you many options including
the ability to keep the ``Output'' window open after the script
terminates, and the ability to enter interactive mode after the
termination of the run script. The latter can be very helpful if you
want to examine the objects that were created during your script.

%\section{Nifty Tools}
%There are many other tools included with the MacPython
%distribution. In addition to those discussed here, make 
%sure to check the \file{Mac} directory.

%\subsection{BuildApplet \label{BuildApplet}}
% **NEED INFO HERE**

%\subsection{BuildApplication}
% **NEED INFO HERE**
 
%\section{TKInter on the Mac \label{TKInter}}

%TKinter is installed by default with the MacPython distribution, but
%you may need to add the \file{lib-tk} folder to the Python Path (see
%section \ref{search-path}).  Also, it is important that you do not
%try to launch Tk from within the \program{Python IDE} because the two
%event loops will collide -- always run a script which uses Tkinter
%with the \program{PythonInterpreter} instead -- see section
%\ref{interpreter}.
 
%\section{CGI on the Mac with Python \label{CGI}}
%**NEED INFO HERE**
                       % Using Python on the Macintosh


\chapter{MacPython Modules \label{macpython-modules}}

The following modules are only available on the Macintosh, and are
documented here:

\localmoduletable

\chapter{MACINTOSH ONLY}

The modules in this chapter are available on the Apple Macintosh only.

\section{Built-in module \sectcode{mac}}

\bimodindex{mac}
This module provides a subset of the operating system dependent
functionality provided by the optional built-in module \code{posix}.
It is best accessed through the more portable standard module
\code{os}.

The following functions are available in this module:
\code{chdir},
\code{getcwd},
\code{listdir},
\code{mkdir},
\code{rename},
\code{rmdir},
\code{stat},
\code{sync},
\code{unlink},
as well as the exception \code{error}.

\section{Standard module \sectcode{macpath}}

\stmodindex{macpath}
This module provides a subset of the pathname manipulation functions
available from the optional standard module \code{posixpath}.  It is
best accessed through the more portable standard module \code{os}, as
\code{os.path}.

The following functions are available in this module:
\code{normcase},
\code{isabs},
\code{join},
\code{split},
\code{isdir},
\code{isfile},
\code{exists}.

\section{Built-in Module \sectcode{ctb}}
\bimodindex{ctb}
\renewcommand{\indexsubitem}{(in module ctb)}

This module provides a partial interface to the Macintosh
Communications Toolbox. Currently, only Connection Manager tools are
supported.  It may not be available in all Mac Python versions.

\begin{datadesc}{error}
The exception raised on errors.
\end{datadesc}

\begin{datadesc}{cmData}
\dataline{cmCntl}
\dataline{cmAttn}
Flags for the \var{channel} argument of the \var{Read} and \var{Write}
methods.
\end{datadesc}

\begin{datadesc}{cmFlagsEOM}
End-of-message flag for \var{Read} and \var{Write}.
\end{datadesc}

\begin{datadesc}{choose*}
Values returned by \var{Choose}.
\end{datadesc}

\begin{datadesc}{cmStatus*}
Bits in the status as returned by \var{Status}.
\end{datadesc}

\begin{funcdesc}{available}{}
Return 1 if the communication toolbox is available, zero otherwise.
\end{funcdesc}

\begin{funcdesc}{CMNew}{name\, sizes}
Create a connection object using the connection tool named
\var{name}. \var{sizes} is a 6-tuple given buffer sizes for data in,
data out, control in, control out, attention in and attention out.
Alternatively, passing \code{None} will result in default buffer sizes.
\end{funcdesc}

\subsection{connection object}
For all connection methods that take a \var{timeout} argument, a value
of \code{-1} is indefinite, meaning that the command runs to completion.

\renewcommand{\indexsubitem}{(connection object attribute)}

\begin{datadesc}{callback}
If this member is set to a value other than \code{None} it should point
to a function accepting a single argument (the connection
object). This will make all connection object methods work
asynchronously, with the callback routine being called upon
completion.

{\em Note:} for reasons beyond my understanding the callback routine
is currently never called. You are advised against using asynchronous
calls for the time being.
\end{datadesc}


\renewcommand{\indexsubitem}{(connection object method)}

\begin{funcdesc}{Open}{timeout}
Open an outgoing connection, waiting at most \var{timeout} seconds for
the connection to be established.
\end{funcdesc}

\begin{funcdesc}{Listen}{timeout}
Wait for an incoming connection. Stop waiting after \var{timeout}
seconds. This call is only meaningful to some tools.
\end{funcdesc}

\begin{funcdesc}{accept}{yesno}
Accept (when \var{yesno} is non-zero) or reject an incoming call after
\var{Listen} returned.
\end{funcdesc}

\begin{funcdesc}{Close}{timeout\, now}
Close a connection. When \var{now} is zero, the close is orderly
(i.e.\ outstanding output is flushed, etc.)\ with a timeout of
\var{timeout} seconds. When \var{now} is non-zero the close is
immediate, discarding output.
\end{funcdesc}

\begin{funcdesc}{Read}{len\, chan\, timeout}
Read \var{len} bytes, or until \var{timeout} seconds have passed, from
the channel \var{chan} (which is one of \var{cmData}, \var{cmCntl} or
\var{cmAttn}). Return a 2-tuple:\ the data read and the end-of-message
flag.
\end{funcdesc}

\begin{funcdesc}{Write}{buf\, chan\, timeout\, eom}
Write \var{buf} to channel \var{chan}, aborting after \var{timeout}
seconds. When \var{eom} has the value \var{cmFlagsEOM} an
end-of-message indicator will be written after the data (if this
concept has a meaning for this communication tool). The method returns
the number of bytes written.
\end{funcdesc}

\begin{funcdesc}{Status}{}
Return connection status as the 2-tuple \code{(\var{sizes},
\var{flags})}. \var{sizes} is a 6-tuple giving the actual buffer sizes used
(see \var{CMNew}), \var{flags} is a set of bits describing the state
of the connection.
\end{funcdesc}

\begin{funcdesc}{GetConfig}{}
Return the configuration string of the communication tool. These
configuration strings are tool-dependent, but usually easily parsed
and modified.
\end{funcdesc}

\begin{funcdesc}{SetConfig}{str}
Set the configuration string for the tool. The strings are parsed
left-to-right, with later values taking precedence. This means
individual configuration parameters can be modified by simply appending
something like \code{'baud 4800'} to the end of the string returned by
\var{GetConfig} and passing that to this method. The method returns
the number of characters actually parsed by the tool before it
encountered an error (or completed successfully).
\end{funcdesc}

\begin{funcdesc}{Choose}{}
Present the user with a dialog to choose a communication tool and
configure it. If there is an outstanding connection some choices (like
selecting a different tool) may cause the connection to be
aborted. The return value (one of the \var{choose*} constants) will
indicate this.
\end{funcdesc}

\begin{funcdesc}{Idle}{}
Give the tool a chance to use the processor. You should call this
method regularly.
\end{funcdesc}

\begin{funcdesc}{Abort}{}
Abort an outstanding asynchronous \var{Open} or \var{Listen}.
\end{funcdesc}

\begin{funcdesc}{Reset}{}
Reset a connection. Exact meaning depends on the tool.
\end{funcdesc}

\begin{funcdesc}{Break}{length}
Send a break. Whether this means anything, what it means and
interpretation of the \var{length} parameter depend on the tool in
use.
\end{funcdesc}

\section{\module{macfs} ---
         Various file system services}

\declaremodule{builtin}{macfs}
  \platform{Mac}
\modulesynopsis{Support for FSSpec, the Alias Manager,
                \program{finder} aliases, and the Standard File package.}


This module provides access to Macintosh FSSpec handling, the Alias
Manager, \program{finder} aliases and the Standard File package.
\index{Macintosh Alias Manager}
\index{Alias Manager, Macintosh}
\index{Standard File}

Whenever a function or method expects a \var{file} argument, this
argument can be one of three things:\ (1) a full or partial Macintosh
pathname, (2) an \pytype{FSSpec} object or (3) a 3-tuple \code{(\var{wdRefNum},
\var{parID}, \var{name})} as described in \citetitle{Inside
Macintosh:\ Files}. A description of aliases and the Standard File
package can also be found there.

\begin{funcdesc}{FSSpec}{file}
Create an \pytype{FSSpec} object for the specified file.
\end{funcdesc}

\begin{funcdesc}{RawFSSpec}{data}
Create an \pytype{FSSpec} object given the raw data for the \C{}
structure for the \pytype{FSSpec} as a string.  This is mainly useful
if you have obtained an \pytype{FSSpec} structure over a network.
\end{funcdesc}

\begin{funcdesc}{RawAlias}{data}
Create an \pytype{Alias} object given the raw data for the \C{}
structure for the alias as a string.  This is mainly useful if you
have obtained an \pytype{FSSpec} structure over a network.
\end{funcdesc}

\begin{funcdesc}{FInfo}{}
Create a zero-filled \pytype{FInfo} object.
\end{funcdesc}

\begin{funcdesc}{ResolveAliasFile}{file}
Resolve an alias file. Returns a 3-tuple \code{(\var{fsspec},
\var{isfolder}, \var{aliased})} where \var{fsspec} is the resulting
\pytype{FSSpec} object, \var{isfolder} is true if \var{fsspec} points
to a folder and \var{aliased} is true if the file was an alias in the
first place (otherwise the \pytype{FSSpec} object for the file itself
is returned).
\end{funcdesc}

\begin{funcdesc}{StandardGetFile}{\optional{type, ...}}
Present the user with a standard ``open input file''
dialog. Optionally, you can pass up to four 4-character file types to limit
the files the user can choose from. The function returns an \pytype{FSSpec}
object and a flag indicating that the user completed the dialog
without cancelling.
\end{funcdesc}

\begin{funcdesc}{PromptGetFile}{prompt\optional{, type, ...}}
Similar to \function{StandardGetFile()} but allows you to specify a
prompt.
\end{funcdesc}

\begin{funcdesc}{StandardPutFile}{prompt, \optional{default}}
Present the user with a standard ``open output file''
dialog. \var{prompt} is the prompt string, and the optional
\var{default} argument initializes the output file name. The function
returns an \pytype{FSSpec} object and a flag indicating that the user
completed the dialog without cancelling.
\end{funcdesc}

\begin{funcdesc}{GetDirectory}{\optional{prompt}}
Present the user with a non-standard ``select a directory''
dialog. \var{prompt} is the prompt string, and the optional.
Return an \pytype{FSSpec} object and a success-indicator.
\end{funcdesc}

\begin{funcdesc}{SetFolder}{\optional{fsspec}}
Set the folder that is initially presented to the user when one of
the file selection dialogs is presented. \var{fsspec} should point to
a file in the folder, not the folder itself (the file need not exist,
though). If no argument is passed the folder will be set to the
current directory, i.e. what \function{os.getcwd()} returns.

Note that starting with system 7.5 the user can change Standard File
behaviour with the ``general controls'' controlpanel, thereby making
this call inoperative.
\end{funcdesc}

\begin{funcdesc}{FindFolder}{where, which, create}
Locates one of the ``special'' folders that MacOS knows about, such as
the trash or the Preferences folder. \var{where} is the disk to
search, \var{which} is the 4-character string specifying which folder to
locate. Setting \var{create} causes the folder to be created if it
does not exist. Returns a \code{(\var{vrefnum}, \var{dirid})} tuple.
\end{funcdesc}

\begin{funcdesc}{NewAliasMinimalFromFullPath}{pathname}
Return a minimal \pytype{alias} object that points to the given file, which
must be specified as a full pathname. This is the only way to create an
\pytype{Alias} pointing to a non-existing file.

The constants for \var{where} and \var{which} can be obtained from the
standard module \var{MACFS}.
\end{funcdesc}

\begin{funcdesc}{FindApplication}{creator}
Locate the application with 4-char creator code \var{creator}. The
function returns an \pytype{FSSpec} object pointing to the application.
\end{funcdesc}


\subsection{FSSpec objects \label{fsspec-objects}}

\begin{memberdesc}[FSSpec]{data}
The raw data from the FSSpec object, suitable for passing
to other applications, for instance.
\end{memberdesc}

\begin{methoddesc}[FSSpec]{as_pathname}{}
Return the full pathname of the file described by the \pytype{FSSpec}
object.
\end{methoddesc}

\begin{methoddesc}[FSSpec]{as_tuple}{}
Return the \code{(\var{wdRefNum}, \var{parID}, \var{name})} tuple of
the file described by the \pytype{FSSpec} object.
\end{methoddesc}

\begin{methoddesc}[FSSpec]{NewAlias}{\optional{file}}
Create an Alias object pointing to the file described by this
FSSpec. If the optional \var{file} parameter is present the alias
will be relative to that file, otherwise it will be absolute.
\end{methoddesc}

\begin{methoddesc}[FSSpec]{NewAliasMinimal}{}
Create a minimal alias pointing to this file.
\end{methoddesc}

\begin{methoddesc}[FSSpec]{GetCreatorType}{}
Return the 4-character creator and type of the file.
\end{methoddesc}

\begin{methoddesc}[FSSpec]{SetCreatorType}{creator, type}
Set the 4-character creator and type of the file.
\end{methoddesc}

\begin{methoddesc}[FSSpec]{GetFInfo}{}
Return a \pytype{FInfo} object describing the finder info for the file.
\end{methoddesc}

\begin{methoddesc}[FSSpec]{SetFInfo}{finfo}
Set the finder info for the file to the values given as \var{finfo}
(an \pytype{FInfo} object).
\end{methoddesc}

\begin{methoddesc}[FSSpec]{GetDates}{}
Return a tuple with three floating point values representing the
creation date, modification date and backup date of the file.
\end{methoddesc}

\begin{methoddesc}[FSSpec]{SetDates}{crdate, moddate, backupdate}
Set the creation, modification and backup date of the file. The values
are in the standard floating point format used for times throughout
Python.
\end{methoddesc}


\subsection{Alias Objects \label{alias-objects}}

\begin{memberdesc}[Alias]{data}
The raw data for the Alias record, suitable for storing in a resource
or transmitting to other programs.
\end{memberdesc}

\begin{methoddesc}[Alias]{Resolve}{\optional{file}}
Resolve the alias. If the alias was created as a relative alias you
should pass the file relative to which it is. Return the FSSpec for
the file pointed to and a flag indicating whether the \pytype{Alias} object
itself was modified during the search process. If the file does
not exist but the path leading up to it does exist a valid fsspec
is returned.
\end{methoddesc}

\begin{methoddesc}[Alias]{GetInfo}{num}
An interface to the \C{} routine \cfunction{GetAliasInfo()}.
\end{methoddesc}

\begin{methoddesc}[Alias]{Update}{file, \optional{file2}}
Update the alias to point to the \var{file} given. If \var{file2} is
present a relative alias will be created.
\end{methoddesc}

Note that it is currently not possible to directly manipulate a
resource as an \pytype{Alias} object. Hence, after calling
\method{Update()} or after \method{Resolve()} indicates that the alias
has changed the Python program is responsible for getting the
\member{data} value from the \pytype{Alias} object and modifying the
resource.


\subsection{FInfo Objects \label{finfo-objects}}

See \citetitle{Inside Macintosh: Files} for a complete description of what
the various fields mean.

\begin{memberdesc}[FInfo]{Creator}
The 4-character creator code of the file.
\end{memberdesc}

\begin{memberdesc}[FInfo]{Type}
The 4-character type code of the file.
\end{memberdesc}

\begin{memberdesc}[FInfo]{Flags}
The finder flags for the file as 16-bit integer. The bit values in
\var{Flags} are defined in standard module \module{MACFS}.
\end{memberdesc}

\begin{memberdesc}[FInfo]{Location}
A Point giving the position of the file's icon in its folder.
\end{memberdesc}

\begin{memberdesc}[FInfo]{Fldr}
The folder the file is in (as an integer).
\end{memberdesc}

\section{Standard Module \sectcode{ic}}
\label{module-ic}
\bimodindex{ic}


This module provides access to Macintosh Internet Config package,
which stores preferences for Internet programs such as mail address,
default homepage, etc. Also, Internet Config contains an elaborate set
of mappings from Macintosh creator/type codes to foreign filename
extensions plus information on how to transfer files (binary, ascii,
etc).

There is a low-level companion module
\module{icglue}\refbimodindex{icglue} which provides the basic
Internet Config access functionality.  This low-level module is not
documented, but the docstrings of the routines document the parameters
and the routine names are the same as for the Pascal or \C{} API to
Internet Config, so the standard IC programmers' documentation can be
used if this module is needed.

The \module{ic} module defines the \exception{error} exception and
symbolic names for all error codes Internet Config can produce; see
the source for details.

\begin{excdesc}{error}
Exception raised on errors in the \module{ic} module.
\end{excdesc}


The \module{ic} module defines the following functions:

\begin{funcdesc}{IC}{\optional{signature\optional{, ic}}}
Create an internet config object. The signature is a 4-char creator
code of the current application (default \code{'Pyth'}) which may
influence some of ICs settings. The optional \var{ic} argument is a
low-level \code{icglue.icinstance} created beforehand, this may be
useful if you want to get preferences from a different config file,
etc.
\end{funcdesc}

\begin{funcdesc}{launchurl}{url\optional{, hint}}
\funcline{parseurl}{data\optional{, start\optional{, end\optional{, hint}}}}
\funcline{mapfile}{file}
\funcline{maptypecreator}{type, creator\optional{, filename}}
\funcline{settypecreator}{file}
These functions are ``shortcuts'' to the methods of the same name,
described below.
\end{funcdesc}


\subsection{IC objects}

IC objects have a mapping interface, hence to obtain the mail address
you simply get \code{\var{ic}['MailAddress']}. Assignment also works,
and changes the option in the configuration file.

The module knows about various datatypes, and converts the internal IC
representation to a ``logical'' Python datastructure. Running the
\module{ic} module standalone will run a test program that lists all
keys and values in your IC database, this will have to server as
documentation.

If the module does not know how to represent the data it returns an
instance of the \code{ICOpaqueData} type, with the raw data in its
\var{data} attribute. Objects of this type are also acceptable values
for assignment.

Besides the dictionary interface IC objects have the following methods:

\setindexsubitem{(IC attribute)}

\begin{funcdesc}{launchurl}{url\optional{, hint}}
Parse the given URL, lauch the correct application and pass it the
URL. The optional \var{hint} can be a scheme name such as
\code{'mailto:'}, in which case incomplete URLs are completed with this
scheme.  If \var{hint} is not provided, incomplete URLs are invalid.
\end{funcdesc}

\begin{funcdesc}{parseurl}{data\optional{, start\optional{, end\optional{, hint}}}}
Find an URL somewhere in \var{data} and return start position, end
position and the URL. The optional \var{start} and \var{end} can be
used to limit the search, so for instance if a user clicks in a long
textfield you can pass the whole textfield and the click-position in
\var{start} and this routine will return the whole URL in which the
user clicked.  \var{Hint} is again an optional scheme used to complete
incomplete URLs.
\end{funcdesc}

\begin{funcdesc}{mapfile}{file}
Return the mapping entry for the given \var{file}, which can be passed
as either a filename or an \code{macfs.FSSpec} object, and which need
not exist.

The mapping entry is returned as a tuple \code{(}\var{version},
\var{type}, \var{creator}, \var{postcreator}, \var{flags},
\var{extension}, \var{appname}, \var{postappname}, \var{mimetype},
\var{entryname}\code{)}, where \var{version} is the entry version
number, \var{type} is the 4-char filetype, \var{creator} is the 4-char
creator type, \var{postcreator} is the 4-char creator code of an
optional application to post-process the file after downloading,
\var{flags} are various bits specifying whether to transfer in binary
or ascii and such, \var{extension} is the filename extension for this
file type, \var{appname} is the printable name of the application to
which this file belongs, \var{postappname} is the name of the
postprocessing application, \var{mimetype} is the MIME type of this
file and \var{entryname} is the name of this entry.
\end{funcdesc}

\begin{funcdesc}{maptypecreator}{type, creator\optional{, filename}}
Return the mapping entry for files with given 4-char \var{type} and
\var{creator} codes. The optional \var{filename} may be specified to
further help finding the correct entry (if the creator code is
\code{'????'}, for instance).

The mapping entry is returned in the same format as for \var{mapfile}.
\end{funcdesc}

\begin{funcdesc}{settypecreator}{file}
Given an existing \var{file}, specified either as a filename or as an
\code{macfs.FSSpec} record, set its creator and type correctly based
on its extension.  The finder is told about the change, so the finder
icon will be updated quickly.
\end{funcdesc}

\section{Built-in Module \module{MacOS}}
\label{module-MacOS}
\bimodindex{MacOS}


This module provides access to MacOS specific functionality in the
Python interpreter, such as how the interpreter eventloop functions
and the like. Use with care.

Note the capitalisation of the module name, this is a historical
artifact.

\begin{excdesc}{Error}
This exception is raised on MacOS generated errors, either from
functions in this module or from other mac-specific modules like the
toolbox interfaces. The arguments are the integer error code (the
\cdata{OSErr} value) and a textual description of the error code.
Symbolic names for all known error codes are defined in the standard
module \module{macerrors}\refstmodindex{macerrors}.
\end{excdesc}

\begin{funcdesc}{SetEventHandler}{handler}
In the inner interpreter loop Python will occasionally check for events,
unless disabled with \function{ScheduleParams()}. With this function you
can pass a Python event-handler function that will be called if an event
is available. The event is passed as parameter and the function should return
non-zero if the event has been fully processed, otherwise event processing
continues (by passing the event to the console window package, for instance).

Call \function{SetEventHandler()} without a parameter to clear the
event handler. Setting an event handler while one is already set is an
error.
\end{funcdesc}

\begin{funcdesc}{SchedParams}{\optional{doint\optional{, evtmask\optional{,
                              besocial\optional{, interval\optional{,
                              bgyield}}}}}}
Influence the interpreter inner loop event handling. \var{Interval}
specifies how often (in seconds, floating point) the interpreter
should enter the event processing code. When true, \var{doint} causes
interrupt (command-dot) checking to be done. \var{evtmask} tells the
interpreter to do event processing for events in the mask (redraws,
mouseclicks to switch to other applications, etc). The \var{besocial}
flag gives other processes a chance to run. They are granted minimal
runtime when Python is in the foreground and \var{bgyield} seconds per
\var{interval} when Python runs in the background.

All parameters are optional, and default to the current value. The return
value of this function is a tuple with the old values of these options.
Initial defaults are that all processing is enabled, checking is done every
quarter second and the CPU is given up for a quarter second when in the
background.
\end{funcdesc}

\begin{funcdesc}{HandleEvent}{ev}
Pass the event record \var{ev} back to the Python event loop, or
possibly to the handler for the \code{sys.stdout} window (based on the
compiler used to build Python). This allows Python programs that do
their own event handling to still have some command-period and
window-switching capability.

If you attempt to call this function from an event handler set through
\function{SetEventHandler()} you will get an exception.
\end{funcdesc}

\begin{funcdesc}{GetErrorString}{errno}
Return the textual description of MacOS error code \var{errno}.
\end{funcdesc}

\begin{funcdesc}{splash}{resid}
This function will put a splash window
on-screen, with the contents of the DLOG resource specified by
\var{resid}. Calling with a zero argument will remove the splash
screen. This function is useful if you want an applet to post a splash screen
early in initialization without first having to load numerous
extension modules.
\end{funcdesc}

\begin{funcdesc}{DebugStr}{message \optional{, object}}
Drop to the low-level debugger with message \var{message}. The
optional \var{object} argument is not used, but can easily be
inspected from the debugger.

Note that you should use this function with extreme care: if no
low-level debugger like MacsBug is installed this call will crash your
system. It is intended mainly for developers of Python extension
modules.
\end{funcdesc}

\begin{funcdesc}{openrf}{name \optional{, mode}}
Open the resource fork of a file. Arguments are the same as for the
built-in function \function{open()}. The object returned has file-like
semantics, but it is not a Python file object, so there may be subtle
differences.
\end{funcdesc}

\section{Standard Module \sectcode{macostools}}
\label{module-macostools}
\stmodindex{macostools}

This module contains some convenience routines for file-manipulation
on the Macintosh.

The \code{macostools} module defines the following functions:

\setindexsubitem{(in module macostools)}

\begin{funcdesc}{copy}{src, dst\optional{, createpath, copytimes}}
Copy file \var{src} to \var{dst}. The files can be specified as
pathnames or \code{FSSpec} objects. If \var{createpath} is non-zero
\var{dst} must be a pathname and the folders leading to the
destination are created if necessary.  The method copies data and
resource fork and some finder information (creator, type, flags) and
optionally the creation, modification and backup times (default is to
copy them). Custom icons, comments and icon position are not copied.

If the source is an alias the original to which the alias points is
copied, not the aliasfile.
\end{funcdesc}

\begin{funcdesc}{copytree}{src, dst}
Recursively copy a file tree from \var{src} to \var{dst}, creating
folders as needed. \var{Src} and \var{dst} should be specified as
pathnames.
\end{funcdesc}

\begin{funcdesc}{mkalias}{src, dst}
Create a finder alias \var{dst} pointing to \var{src}. Both may be
specified as pathnames or \var{FSSpec} objects.
\end{funcdesc}

\begin{funcdesc}{touched}{dst}
Tell the finder that some bits of finder-information such as creator
or type for file \var{dst} has changed. The file can be specified by
pathname or fsspec. This call should prod the finder into redrawing the
files icon.
\end{funcdesc}

\begin{datadesc}{BUFSIZ}
The buffer size for \code{copy}, default 1 megabyte.
\end{datadesc}

Note that the process of creating finder aliases is not specified in
the Apple documentation. Hence, aliases created with \code{mkalias}
could conceivably have incompatible behaviour in some cases.

\section{Standard Module \sectcode{findertools}}
\label{module-findertools}
\stmodindex{findertools}

This module contains routines that give Python programs access to some
functionality provided by the finder. They are implemented as wrappers
around the AppleEvent interface to the finder.

All file and folder parameters can be specified either as full
pathnames or as \code{FSSpec} objects.

The \code{findertools} module defines the following functions:

\setindexsubitem{(in module macostools)}

\begin{funcdesc}{launch}{file}
Tell the finder to launch \var{file}. What launching means depends on the file:
applications are started, folders are opened and documents are opened
in the correct application.
\end{funcdesc}

\begin{funcdesc}{Print}{file}
Tell the finder to print a file (again specified by full pathname or
FSSpec). The behaviour is identical to selecting the file and using
the print command in the finder.
\end{funcdesc}

\begin{funcdesc}{copy}{file, destdir}
Tell the finder to copy a file or folder \var{file} to folder
\var{destdir}. The function returns an \code{Alias} object pointing to
the new file.
\end{funcdesc}

\begin{funcdesc}{move}{file, destdir}
Tell the finder to move a file or folder \var{file} to folder
\var{destdir}. The function returns an \code{Alias} object pointing to
the new file.
\end{funcdesc}

\begin{funcdesc}{sleep}{}
Tell the finder to put the mac to sleep, if your machine supports it.
\end{funcdesc}

\begin{funcdesc}{restart}{}
Tell the finder to perform an orderly restart of the machine.
\end{funcdesc}

\begin{funcdesc}{shutdown}{}
Tell the finder to perform an orderly shutdown of the machine.
\end{funcdesc}

\section{Built-in Module \module{macspeech}}
\declaremodule{builtin}{macspeech}

\modulesynopsis{Interface to the Macintosh Speech Manager.}



This module provides an interface to the Macintosh Speech Manager,
\index{Macintosh Speech Manager}
\index{Speech Manager, Macintosh}
allowing you to let the Macintosh utter phrases. You need a version of
the Speech Manager extension (version 1 and 2 have been tested) in
your \file{Extensions} folder for this to work. The module does not
provide full access to all features of the Speech Manager yet.  It may
not be available in all Mac Python versions.

\begin{funcdesc}{Available}{}
Test availability of the Speech Manager extension (and, on the
PowerPC, the Speech Manager shared library). Return \code{0} or
\code{1}.
\end{funcdesc}

\begin{funcdesc}{Version}{}
Return the (integer) version number of the Speech Manager.
\end{funcdesc}

\begin{funcdesc}{SpeakString}{str}
Utter the string \var{str} using the default voice,
asynchronously. This aborts any speech that may still be active from
prior \function{SpeakString()} invocations.
\end{funcdesc}

\begin{funcdesc}{Busy}{}
Return the number of speech channels busy, system-wide.
\end{funcdesc}

\begin{funcdesc}{CountVoices}{}
Return the number of different voices available.
\end{funcdesc}

\begin{funcdesc}{GetIndVoice}{num}
Return a \pytype{Voice} object for voice number \var{num}.
\end{funcdesc}

\subsection{Voice Objects}
\label{voice-objects}

Voice objects contain the description of a voice. It is currently not
yet possible to access the parameters of a voice.

\setindexsubitem{(voice object method)}

\begin{methoddesc}[Voice]{GetGender}{}
Return the gender of the voice: \code{0} for male, \code{1} for female
and \code{-1} for neuter.
\end{methoddesc}

\begin{methoddesc}[Voice]{NewChannel}{}
Return a new Speech Channel object using this voice.
\end{methoddesc}

\subsection{Speech Channel Objects}
\label{speech-channel-objects}

A Speech Channel object allows you to speak strings with slightly more
control than \function{SpeakString()}, and allows you to use multiple
speakers at the same time. Please note that channel pitch and rate are
interrelated in some way, so that to make your Macintosh sing you will
have to adjust both.

\begin{methoddesc}[Speech Channel]{SpeakText}{str}
Start uttering the given string.
\end{methoddesc}

\begin{methoddesc}[Speech Channel]{Stop}{}
Stop babbling.
\end{methoddesc}

\begin{methoddesc}[Speech Channel]{GetPitch}{}
Return the current pitch of the channel, as a floating-point number.
\end{methoddesc}

\begin{methoddesc}[Speech Channel]{SetPitch}{pitch}
Set the pitch of the channel.
\end{methoddesc}

\begin{methoddesc}[Speech Channel]{GetRate}{}
Get the speech rate (utterances per minute) of the channel as a
floating point number.
\end{methoddesc}

\begin{methoddesc}[Speech Channel]{SetRate}{rate}
Set the speech rate of the channel.
\end{methoddesc}


\section{Standard Module \sectcode{EasyDialogs}}
\label{module-EasyDialogs}
\stmodindex{EasyDialogs}

The \code{EasyDialogs} module contains some simple dialogs for
the Macintosh, modelled after the \code{stdwin} dialogs with similar
names. All routines have an optional parameter \var{id} with which you
can override the DLOG resource used for the dialog, as long as the
item numbers correspond. See the source for details.

The \code{EasyDialogs} module defines the following functions:

\setindexsubitem{(in module EasyDialogs)}

\begin{funcdesc}{Message}{str}
A modal dialog with the message text \var{str}, which should be at
most 255 characters long, is displayed. Control is returned when the
user clicks ``OK''.
\end{funcdesc}

\begin{funcdesc}{AskString}{prompt\optional{, default}}
Ask the user to input a string value, in a modal dialog. \var{Prompt}
is the promt message, the optional \var{default} arg is the initial
value for the string. All strings can be at most 255 bytes
long. \var{AskString} returns the string entered or \code{None} in
case the user cancelled.
\end{funcdesc}

\begin{funcdesc}{AskYesNoCancel}{question\optional{, default}}
Present a dialog with text \var{question} and three buttons labelled
``yes'', ``no'' and ``cancel''. Return \code{1} for yes, \code{0} for
no and \code{-1} for cancel. The default return value chosen by
hitting return is \code{0}. This can be changed with the optional
\var{default} argument.
\end{funcdesc}

\begin{funcdesc}{ProgressBar}{\optional{label, maxval}}
Display a modeless progress dialog with a thermometer bar. \var{Label}
is the textstring displayed (default ``Working...''), \var{maxval} is
the value at which progress is complete (default 100). The returned
object has one method, \code{set(value)}, which sets the value of the
progress bar. The bar remains visible until the object returned is
discarded.

The progress bar has a ``cancel'' button, but it is currently
non-functional.
\end{funcdesc}

Note that \code{EasyDialogs} does not currently use the notification
manager. This means that displaying dialogs while the program is in
the background will lead to unexpected results and possibly
crashes. Also, all dialogs are modeless and hence expect to be at the
top of the stacking order. This is true when the dialogs are created,
but windows that pop-up later (like a console window) may also result
in crashes.

\section{\module{FrameWork} ---
         Interactive application framework}

\declaremodule{standard}{FrameWork}
  \platform{Mac}
\modulesynopsis{Interactive application framework.}


The \module{FrameWork} module contains classes that together provide a
framework for an interactive Macintosh application. The programmer
builds an application by creating subclasses that override various
methods of the bases classes, thereby implementing the functionality
wanted. Overriding functionality can often be done on various
different levels, i.e. to handle clicks in a single dialog window in a
non-standard way it is not necessary to override the complete event
handling.

The \module{FrameWork} is still very much work-in-progress, and the
documentation describes only the most important functionality, and not
in the most logical manner at that. Examine the source or the examples
for more details.  The following are some comments posted on the
MacPython newsgroup about the strengths and limitations of
\module{FrameWork}:

\begin{quotation}
The strong point of \module{FrameWork} is that it allows you to break
into the control-flow at many different places. \refmodule{W}, for
instance, uses a different way to enable/disable menus and that plugs
right in leaving the rest intact.  The weak points of
\module{FrameWork} are that it has no abstract command interface (but
that shouldn't be difficult), that it's dialog support is minimal and
that it's control/toolbar support is non-existent.
\end{quotation}


The \module{FrameWork} module defines the following functions:


\begin{funcdesc}{Application}{}
An object representing the complete application. See below for a
description of the methods. The default \method{__init__()} routine
creates an empty window dictionary and a menu bar with an apple menu.
\end{funcdesc}

\begin{funcdesc}{MenuBar}{}
An object representing the menubar. This object is usually not created
by the user.
\end{funcdesc}

\begin{funcdesc}{Menu}{bar, title\optional{, after}}
An object representing a menu. Upon creation you pass the
\code{MenuBar} the menu appears in, the \var{title} string and a
position (1-based) \var{after} where the menu should appear (default:
at the end).
\end{funcdesc}

\begin{funcdesc}{MenuItem}{menu, title\optional{, shortcut, callback}}
Create a menu item object. The arguments are the menu to create, the
item title string and optionally the keyboard shortcut
and a callback routine. The callback is called with the arguments
menu-id, item number within menu (1-based), current front window and
the event record.

Instead of a callable object the callback can also be a string. In
this case menu selection causes the lookup of a method in the topmost
window and the application. The method name is the callback string
with \code{'domenu_'} prepended.

Calling the \code{MenuBar} \method{fixmenudimstate()} method sets the
correct dimming for all menu items based on the current front window.
\end{funcdesc}

\begin{funcdesc}{Separator}{menu}
Add a separator to the end of a menu.
\end{funcdesc}

\begin{funcdesc}{SubMenu}{menu, label}
Create a submenu named \var{label} under menu \var{menu}. The menu
object is returned.
\end{funcdesc}

\begin{funcdesc}{Window}{parent}
Creates a (modeless) window. \var{Parent} is the application object to
which the window belongs. The window is not displayed until later.
\end{funcdesc}

\begin{funcdesc}{DialogWindow}{parent}
Creates a modeless dialog window.
\end{funcdesc}

\begin{funcdesc}{windowbounds}{width, height}
Return a \code{(\var{left}, \var{top}, \var{right}, \var{bottom})}
tuple suitable for creation of a window of given width and height. The
window will be staggered with respect to previous windows, and an
attempt is made to keep the whole window on-screen. However, the window will
however always be the exact size given, so parts may be offscreen.
\end{funcdesc}

\begin{funcdesc}{setwatchcursor}{}
Set the mouse cursor to a watch.
\end{funcdesc}

\begin{funcdesc}{setarrowcursor}{}
Set the mouse cursor to an arrow.
\end{funcdesc}


\subsection{Application Objects \label{application-objects}}

Application objects have the following methods, among others:


\begin{methoddesc}[Application]{makeusermenus}{}
Override this method if you need menus in your application. Append the
menus to the attribute \member{menubar}.
\end{methoddesc}

\begin{methoddesc}[Application]{getabouttext}{}
Override this method to return a text string describing your
application.  Alternatively, override the \method{do_about()} method
for more elaborate ``about'' messages.
\end{methoddesc}

\begin{methoddesc}[Application]{mainloop}{\optional{mask\optional{, wait}}}
This routine is the main event loop, call it to set your application
rolling. \var{Mask} is the mask of events you want to handle,
\var{wait} is the number of ticks you want to leave to other
concurrent application (default 0, which is probably not a good
idea). While raising \var{self} to exit the mainloop is still
supported it is not recommended: call \code{self._quit()} instead.

The event loop is split into many small parts, each of which can be
overridden. The default methods take care of dispatching events to
windows and dialogs, handling drags and resizes, Apple Events, events
for non-FrameWork windows, etc.

In general, all event handlers should return \code{1} if the event is fully
handled and \code{0} otherwise (because the front window was not a FrameWork
window, for instance). This is needed so that update events and such
can be passed on to other windows like the Sioux console window.
Calling \function{MacOS.HandleEvent()} is not allowed within
\var{our_dispatch} or its callees, since this may result in an
infinite loop if the code is called through the Python inner-loop
event handler.
\end{methoddesc}

\begin{methoddesc}[Application]{asyncevents}{onoff}
Call this method with a nonzero parameter to enable
asynchronous event handling. This will tell the inner interpreter loop
to call the application event handler \var{async_dispatch} whenever events
are available. This will cause FrameWork window updates and the user
interface to remain working during long computations, but will slow the
interpreter down and may cause surprising results in non-reentrant code
(such as FrameWork itself). By default \var{async_dispatch} will immedeately
call \var{our_dispatch} but you may override this to handle only certain
events asynchronously. Events you do not handle will be passed to Sioux
and such.

The old on/off value is returned.
\end{methoddesc}

\begin{methoddesc}[Application]{_quit}{}
Terminate the running \method{mainloop()} call at the next convenient
moment.
\end{methoddesc}

\begin{methoddesc}[Application]{do_char}{c, event}
The user typed character \var{c}. The complete details of the event
can be found in the \var{event} structure. This method can also be
provided in a \code{Window} object, which overrides the
application-wide handler if the window is frontmost.
\end{methoddesc}

\begin{methoddesc}[Application]{do_dialogevent}{event}
Called early in the event loop to handle modeless dialog events. The
default method simply dispatches the event to the relevant dialog (not
through the \code{DialogWindow} object involved). Override if you
need special handling of dialog events (keyboard shortcuts, etc).
\end{methoddesc}

\begin{methoddesc}[Application]{idle}{event}
Called by the main event loop when no events are available. The
null-event is passed (so you can look at mouse position, etc).
\end{methoddesc}


\subsection{Window Objects \label{window-objects}}

Window objects have the following methods, among others:

\setindexsubitem{(Window method)}

\begin{methoddesc}[Window]{open}{}
Override this method to open a window. Store the MacOS window-id in
\member{self.wid} and call the \method{do_postopen()} method to
register the window with the parent application.
\end{methoddesc}

\begin{methoddesc}[Window]{close}{}
Override this method to do any special processing on window
close. Call the \method{do_postclose()} method to cleanup the parent
state.
\end{methoddesc}

\begin{methoddesc}[Window]{do_postresize}{width, height, macoswindowid}
Called after the window is resized. Override if more needs to be done
than calling \code{InvalRect}.
\end{methoddesc}

\begin{methoddesc}[Window]{do_contentclick}{local, modifiers, event}
The user clicked in the content part of a window. The arguments are
the coordinates (window-relative), the key modifiers and the raw
event.
\end{methoddesc}

\begin{methoddesc}[Window]{do_update}{macoswindowid, event}
An update event for the window was received. Redraw the window.
\end{methoddesc}

\begin{methoddesc}{do_activate}{activate, event}
The window was activated (\code{\var{activate} == 1}) or deactivated
(\code{\var{activate} == 0}). Handle things like focus highlighting,
etc.
\end{methoddesc}


\subsection{ControlsWindow Object \label{controlswindow-object}}

ControlsWindow objects have the following methods besides those of
\code{Window} objects:


\begin{methoddesc}[ControlsWindow]{do_controlhit}{window, control,
                                                  pcode, event}
Part \var{pcode} of control \var{control} was hit by the
user. Tracking and such has already been taken care of.
\end{methoddesc}


\subsection{ScrolledWindow Object \label{scrolledwindow-object}}

ScrolledWindow objects are ControlsWindow objects with the following
extra methods:


\begin{methoddesc}[ScrolledWindow]{scrollbars}{\optional{wantx\optional{,
                                               wanty}}}
Create (or destroy) horizontal and vertical scrollbars. The arguments
specify which you want (default: both). The scrollbars always have
minimum \code{0} and maximum \code{32767}.
\end{methoddesc}

\begin{methoddesc}[ScrolledWindow]{getscrollbarvalues}{}
You must supply this method. It should return a tuple \code{(\var{x},
\var{y})} giving the current position of the scrollbars (between
\code{0} and \code{32767}). You can return \code{None} for either to
indicate the whole document is visible in that direction.
\end{methoddesc}

\begin{methoddesc}[ScrolledWindow]{updatescrollbars}{}
Call this method when the document has changed. It will call
\method{getscrollbarvalues()} and update the scrollbars.
\end{methoddesc}

\begin{methoddesc}[ScrolledWindow]{scrollbar_callback}{which, what, value}
Supplied by you and called after user interaction. \var{which} will
be \code{'x'} or \code{'y'}, \var{what} will be \code{'-'},
\code{'--'}, \code{'set'}, \code{'++'} or \code{'+'}. For
\code{'set'}, \var{value} will contain the new scrollbar position.
\end{methoddesc}

\begin{methoddesc}[ScrolledWindow]{scalebarvalues}{absmin, absmax,
                                                   curmin, curmax}
Auxiliary method to help you calculate values to return from
\method{getscrollbarvalues()}. You pass document minimum and maximum value
and topmost (leftmost) and bottommost (rightmost) visible values and
it returns the correct number or \code{None}.
\end{methoddesc}

\begin{methoddesc}[ScrolledWindow]{do_activate}{onoff, event}
Takes care of dimming/highlighting scrollbars when a window becomes
frontmost. If you override this method, call this one at the end of
your method.
\end{methoddesc}

\begin{methoddesc}[ScrolledWindow]{do_postresize}{width, height, window}
Moves scrollbars to the correct position. Call this method initially
if you override it.
\end{methoddesc}

\begin{methoddesc}[ScrolledWindow]{do_controlhit}{window, control,
                                                  pcode, event}
Handles scrollbar interaction. If you override it call this method
first, a nonzero return value indicates the hit was in the scrollbars
and has been handled.
\end{methoddesc}


\subsection{DialogWindow Objects \label{dialogwindow-objects}}

DialogWindow objects have the following methods besides those of
\code{Window} objects:


\begin{methoddesc}[DialogWindow]{open}{resid}
Create the dialog window, from the DLOG resource with id
\var{resid}. The dialog object is stored in \member{self.wid}.
\end{methoddesc}

\begin{methoddesc}[DialogWindow]{do_itemhit}{item, event}
Item number \var{item} was hit. You are responsible for redrawing
toggle buttons, etc.
\end{methoddesc}

\section{Standard Module \sectcode{MiniAEFrame}}
\stmodindex{MiniAEFrame}
\label{module-MiniAEFrame}

The module \var{MiniAEFrame} provides a framework for an application
that can function as an OSA server, i.e. receive and process
AppleEvents. It can be used in conjunction with \var{FrameWork} or
standalone.

This module is temporary, it will eventually be replaced by a module
that handles argument names better and possibly automates making your
application scriptable.

The \var{MiniAEFrame} module defines the following classes:

\setindexsubitem{(in module MiniAEFrame)}

\begin{funcdesc}{AEServer}{}
A class that handles AppleEvent dispatch. Your application should
subclass this class together with either
\code{MiniAEFrame.MiniApplication} or
\code{FrameWork.Application}. Your \code{__init__} method should call
the \code{__init__} method for both classes.
\end{funcdesc}

\begin{funcdesc}{MiniApplication}{}
A class that is more or less compatible with
\code{FrameWork.Application} but with less functionality. Its
eventloop supports the apple menu, command-dot and AppleEvents, other
events are passed on to the Python interpreter and/or Sioux.
Useful if your application wants to use \code{AEServer} but does not
provide its own windows, etc.
\end{funcdesc}

\subsection{AEServer Objects}

\setindexsubitem{(AEServer method)}

\begin{funcdesc}{installaehandler}{classe\, type\, callback}
Installs an AppleEvent handler. \code{Classe} and \code{type} are the
four-char OSA Class and Type designators, \code{'****'} wildcards are
allowed. When a matching AppleEvent is received the parameters are
decoded and your callback is invoked.
\end{funcdesc}

\begin{funcdesc}{callback}{_object\, **kwargs}
Your callback is called with the OSA Direct Object as first positional
parameter. The other parameters are passed as keyword arguments, with
the 4-char designator as name. Three extra keyword parameters are
passed: \code{_class} and \code{_type} are the Class and Type
designators and \code{_attributes} is a dictionary with the AppleEvent
attributes.

The return value of your method is packed with
\code{aetools.packevent} and sent as reply.
\end{funcdesc}

Note that there are some serious problems with the current
design. AppleEvents which have non-identifier 4-char designators for
arguments are not implementable, and it is not possible to return an
error to the originator. This will be addressed in a future release.

\section{\module{aepack} ---
         Conversion between Python variables and AppleEvent data containers}

\declaremodule{standard}{aepack}
  \platform{Mac}
%\moduleauthor{Jack Jansen?}{email}
\modulesynopsis{Conversion between Python variables and AppleEvent
                data containers.}
\sectionauthor{Vincent Marchetti}{vincem@en.com}


The \module{aepack} module defines functions for converting (packing)
Python variables to AppleEvent descriptors and back (unpacking).
Within Python the AppleEvent descriptor is handled by Python objects
of built-in type \class{AEDesc}, defined in module \refmodule{AE}.

The \module{aepack} module defines the following functions:


\begin{funcdesc}{pack}{x\optional{, forcetype}}
Returns an \class{AEDesc} object  containing a conversion of Python
value x. If \var{forcetype} is provided it specifies the descriptor
type of the result. Otherwise, a default mapping of Python types to
Apple Event descriptor types is used, as follows:

\begin{tableii}{l|l}{textrm}{Python type}{descriptor type}
  \lineii{\class{FSSpec}}{typeFSS}
  \lineii{\class{FSRef}}{typeFSRef}
  \lineii{\class{Alias}}{typeAlias}
  \lineii{integer}{typeLong (32 bit integer)}
  \lineii{float}{typeFloat (64 bit floating point)}
  \lineii{string}{typeText}
  \lineii{unicode}{typeUnicodeText}
  \lineii{list}{typeAEList}
  \lineii{dictionary}{typeAERecord}
  \lineii{instance}{\emph{see below}}
\end{tableii}  
 
If \var{x} is a Python instance then this function attempts to call an
\method{__aepack__()} method.  This method should return an
\class{AE.AEDesc} object.

If the conversion \var{x} is not defined above, this function returns
the Python string representation of a value (the repr() function)
encoded as a text descriptor.
\end{funcdesc}

\begin{funcdesc}{unpack}{x\optional{, formodulename}}
  \var{x} must be an object of type \class{AEDesc}. This function
  returns a Python object representation of the data in the Apple
  Event descriptor \var{x}. Simple AppleEvent data types (integer,
  text, float) are returned as their obvious Python counterparts.
  Apple Event lists are returned as Python lists, and the list
  elements are recursively unpacked.  Object references
  (ex. \code{line 3 of document 1}) are returned as instances of
  \class{aetypes.ObjectSpecifier}, unless \code{formodulename}
  is specified.  AppleEvent descriptors with
  descriptor type typeFSS are returned as \class{FSSpec}
  objects.  AppleEvent record descriptors are returned as Python
  dictionaries, with 4-character string keys and elements recursively
  unpacked.
  
  The optional \code{formodulename} argument is used by the stub packages
  generated by \module{gensuitemodule}, and ensures that the OSA classes
  for object specifiers are looked up in the correct module. This ensures
  that if, say, the Finder returns an object specifier for a window
  you get an instance of \code{Finder.Window} and not a generic
  \code{aetypes.Window}. The former knows about all the properties
  and elements a window has in the Finder, while the latter knows
  no such things.
\end{funcdesc}


\begin{seealso}
  \seemodule{Carbon.AE}{Built-in access to Apple Event Manager routines.}
  \seemodule{aetypes}{Python definitions of codes for Apple Event
                      descriptor types.}
  \seetitle[http://developer.apple.com/techpubs/mac/IAC/IAC-2.html]{
            Inside Macintosh: Interapplication
            Communication}{Information about inter-process
            communications on the Macintosh.}
\end{seealso}

\section{\module{aetypes} ---
         AppleEvent objects}

\declaremodule{standard}{aetypes}
  \platform{Mac}
%\moduleauthor{Jack Jansen?}{email}
\modulesynopsis{Python representation of the Apple Event Object Model.}
\sectionauthor{Vincent Marchetti}{vincem@en.com}


The \module{aetypes} defines classes used to represent Apple Event data
descriptors and Apple Event object specifiers.

Apple Event data is contained in descriptors, and these descriptors
are typed. For many descriptors the Python representation is simply the
corresponding Python type: \code{typeText} in OSA is a Python string,
\code{typeFloat} is a float, etc. For OSA types that have no direct
Python counterpart this module declares classes. Packing and unpacking
instances of these classes is handled automatically by \module{aepack}.

An object specifier is essentially an address of an object implemented
in a Apple Event server. An Apple Event specifier is used as the direct
object for an Apple Event or as the argument of an optional parameter.
The \module{aetypes} module contains the base classes for OSA classes
and properties, which are used by the packages generated by
\module{gensuitemodule} to populate the classes and properties in a
given suite.

For reasons of backward compatibility, and for cases where you need to
script an application for which you have not generated the stub package
this module also contains object specifiers for a number of common OSA
classes such as \code{Document}, \code{Window}, \code{Character}, etc.



The \module{AEObjects} module defines the following classes to represent
Apple Event descriptor data:

\begin{classdesc}{Unknown}{type, data}
The representation of OSA descriptor data for which the \module{aepack}
and \module{aetypes} modules have no support, i.e. anything that is not
represented by the other classes here and that is not equivalent to a
simple Python value.
\end{classdesc}

\begin{classdesc}{Enum}{enum}
An enumeration value with the given 4-character string value.
\end{classdesc}

\begin{classdesc}{InsertionLoc}{of, pos}
Position \code{pos} in object \code{of}.
\end{classdesc}

\begin{classdesc}{Boolean}{bool}
A boolean.
\end{classdesc}

\begin{classdesc}{StyledText}{style, text}
Text with style information (font, face, etc) included.
\end{classdesc}

\begin{classdesc}{AEText}{script, style, text}
Text with script system and style information included.
\end{classdesc}

\begin{classdesc}{IntlText}{script, language, text}
Text with script system and language information included.
\end{classdesc}

\begin{classdesc}{IntlWritingCode}{script, language}
Script system and language information.
\end{classdesc}

\begin{classdesc}{QDPoint}{v, h}
A quickdraw point.
\end{classdesc}

\begin{classdesc}{QDRectangle}{v0, h0, v1, h1}
A quickdraw rectangle.
\end{classdesc}

\begin{classdesc}{RGBColor}{r, g, b}
A color.
\end{classdesc}

\begin{classdesc}{Type}{type}
An OSA type value with the given 4-character name.
\end{classdesc}

\begin{classdesc}{Keyword}{name}
An OSA keyword with the given 4-character name.
\end{classdesc}

\begin{classdesc}{Range}{start, stop}
A range.
\end{classdesc}

\begin{classdesc}{Ordinal}{abso}
Non-numeric absolute positions, such as \code{"firs"}, first, or \code{"midd"},
middle.
\end{classdesc}

\begin{classdesc}{Logical}{logc, term}
The logical expression of applying operator \code{logc} to
\code{term}.
\end{classdesc}

\begin{classdesc}{Comparison}{obj1, relo, obj2}
The comparison \code{relo} of \code{obj1} to \code{obj2}.
\end{classdesc}

The following classes are used as base classes by the generated stub
packages to represent AppleScript classes and properties in Python:

\begin{classdesc}{ComponentItem}{which\optional{, fr}}
Abstract baseclass for an OSA class. The subclass should set the class
attribute \code{want} to the 4-character OSA class code. Instances of
subclasses of this class are equivalent to AppleScript Object
Specifiers. Upon instantiation you should pass a selector in
\code{which}, and optionally a parent object in \code{fr}.
\end{classdesc}

\begin{classdesc}{NProperty}{fr}
Abstract baseclass for an OSA property. The subclass should set the class
attributes \code{want} and \code{which} to designate which property we
are talking about. Instances of subclasses of this class are Object
Specifiers.
\end{classdesc}

\begin{classdesc}{ObjectSpecifier}{want, form, seld\optional{, fr}}
Base class of \code{ComponentItem} and \code{NProperty}, a general
OSA Object Specifier. See the Apple Open Scripting Architecture
documentation for the parameters. Note that this class is not abstract.
\end{classdesc}



\chapter{MacOS Toolbox Modules \label{toolbox}}

There are a set of modules that provide interfaces to various MacOS
toolboxes.  If applicable the module will define a number of Python
objects for the various structures declared by the toolbox, and
operations will be implemented as methods of the object.  Other
operations will be implemented as functions in the module.  Not all
operations possible in C will also be possible in Python (callbacks
are often a problem), and parameters will occasionally be different in
Python (input and output buffers, especially).  All methods and
functions have a \member{__doc__} string describing their arguments
and return values, and for additional description you are referred to
\citetitle[http://developer.apple.com/documentation/macos8/mac8.html]{Inside
Macintosh} or similar works.

These modules all live in a package called \module{Carbon}. Despite that name
they are not all part of the Carbon framework: CF is really in the CoreFoundation
framework and Qt is in the QuickTime framework.
The normal use pattern is

\begin{verbatim}
from Carbon import AE
\end{verbatim}

\strong{Warning!}  These modules are not yet documented.  If you
wish to contribute documentation of any of these modules, please get
in touch with \email{docs@python.org}.

\localmoduletable


%\section{Argument Handling for Toolbox Modules}


\section{\module{Carbon.AE} --- Apple Events}
\declaremodule{standard}{Carbon.AE}
  \platform{Mac}
\modulesynopsis{Interface to the Apple Events toolbox.}

\section{\module{Carbon.AH} --- Apple Help}
\declaremodule{standard}{Carbon.AH}
  \platform{Mac}
\modulesynopsis{Interface to the Apple Help manager.}


\section{\module{Carbon.App} --- Appearance Manager}
\declaremodule{standard}{Carbon.App}
  \platform{Mac}
\modulesynopsis{Interface to the Appearance Manager.}


\section{\module{Carbon.CF} --- Core Foundation}
\declaremodule{standard}{Carbon.CF}
  \platform{Mac}
\modulesynopsis{Interface to the Core Foundation.}

The
\code{CFBase}, \code{CFArray}, \code{CFData}, \code{CFDictionary},
\code{CFString} and \code{CFURL} objects are supported, some
only partially.

\section{\module{Carbon.CG} --- Core Graphics}
\declaremodule{standard}{Carbon.CG}
  \platform{Mac}
\modulesynopsis{Interface to the Component Manager.}

\section{\module{Carbon.CarbonEvt} --- Carbon Event Manager}
\declaremodule{standard}{Carbon.CaronEvt}
  \platform{Mac}
\modulesynopsis{Interface to the Carbon Event Manager.}

\section{\module{Carbon.Cm} --- Component Manager}
\declaremodule{standard}{Carbon.Cm}
  \platform{Mac}
\modulesynopsis{Interface to the Component Manager.}


\section{\module{Carbon.Ctl} --- Control Manager}
\declaremodule{standard}{Carbon.Ctl}
  \platform{Mac}
\modulesynopsis{Interface to the Control Manager.}


\section{\module{Carbon.Dlg} --- Dialog Manager}
\declaremodule{standard}{Carbon.Dlg}
  \platform{Mac}
\modulesynopsis{Interface to the Dialog Manager.}


\section{\module{Carbon.Evt} --- Event Manager}
\declaremodule{standard}{Carbon.Evt}
  \platform{Mac}
\modulesynopsis{Interface to the classic Event Manager.}


\section{\module{Carbon.Fm} --- Font Manager}
\declaremodule{standard}{Carbon.Fm}
  \platform{Mac}
\modulesynopsis{Interface to the Font Manager.}

\section{\module{Carbon.Folder} --- Folder Manager}
\declaremodule{standard}{Carbon.Folder}
  \platform{Mac}
\modulesynopsis{Interface to the Folder Manager.}


\section{\module{Carbon.Help} --- Help Manager}
\declaremodule{standard}{Carbon.Help}
  \platform{Mac}
\modulesynopsis{Interface to the Carbon Help Manager.}

\section{\module{Carbon.List} --- List Manager}
\declaremodule{standard}{Carbon.List}
  \platform{Mac}
\modulesynopsis{Interface to the List Manager.}


\section{\module{Carbon.Menu} --- Menu Manager}
\declaremodule{standard}{Carbon.Menu}
  \platform{Mac}
\modulesynopsis{Interface to the Menu Manager.}


\section{\module{Carbon.Mlte} --- MultiLingual Text Editor}
\declaremodule{standard}{Carbon.Mlte}
  \platform{Mac}
\modulesynopsis{Interface to the MultiLingual Text Editor.}


\section{\module{Carbon.Qd} --- QuickDraw}
\declaremodule{builtin}{Carbon.Qd}
  \platform{Mac}
\modulesynopsis{Interface to the QuickDraw toolbox.}


\section{\module{Carbon.Qdoffs} --- QuickDraw Offscreen}
\declaremodule{builtin}{Carbon.Qdoffs}
  \platform{Mac}
\modulesynopsis{Interface to the QuickDraw Offscreen APIs.}


\section{\module{Carbon.Qt} --- QuickTime}
\declaremodule{standard}{Carbon.Qt}
  \platform{Mac}
\modulesynopsis{Interface to the QuickTime toolbox.}


\section{\module{Carbon.Res} --- Resource Manager and Handles}
\declaremodule{standard}{Carbon.Res}
  \platform{Mac}
\modulesynopsis{Interface to the Resource Manager and Handles.}

\section{\module{Carbon.Scrap} --- Scrap Manager}
\declaremodule{standard}{Carbon.Scrap}
  \platform{Mac}
\modulesynopsis{Interface to the Carbon Scrap Manager.}

\section{\module{Carbon.Snd} --- Sound Manager}
\declaremodule{standard}{Carbon.Snd}
  \platform{Mac}
\modulesynopsis{Interface to the Sound Manager.}


\section{\module{Carbon.TE} --- TextEdit}
\declaremodule{standard}{Carbon.TE}
  \platform{Mac}
\modulesynopsis{Interface to TextEdit.}


\section{\module{Carbon.Win} --- Window Manager}
\declaremodule{standard}{Carbon.Win}
  \platform{Mac}
\modulesynopsis{Interface to the Window Manager.}
                         % MacOS Toolbox Modules
\section{\module{ColorPicker} ---
         Color selection dialog}

\declaremodule{extension}{ColorPicker}
  \platform{Mac}
\modulesynopsis{}
\moduleauthor{Just van Rossum}{just@letterror.com}
\sectionauthor{Fred L. Drake, Jr.}{fdrake@acm.org}


The \module{ColorPicker} module provides access to the standard color
picker dialog.


\begin{funcdesc}{GetColor}{prompt, rgb}
  Show a standard color selection dialog and allow the user to select
  a color.  The user is given instruction by the \var{prompt} string,
  and the default color is set to \var{rgb}.  \var{rgb} must be a
  tuple giving the red, green, and blue components of the color.
  \function{GetColor()} returns a tuple giving the user's selected
  color and a flag indicating whether they accepted the selection of
  cancelled.
\end{funcdesc}


\chapter{Undocumented Modules \label{undocumented-modules}}


The modules in this chapter are poorly documented (if at all).  If you
wish to contribute documentation of any of these modules, please get in
touch with
\ulink{\email{docs@python.org}}{mailto:docs@python.org}.

\localmoduletable


\section{\module{applesingle} --- AppleSingle decoder}
\declaremodule{standard}{applesingle}
  \platform{Mac}
\modulesynopsis{Rudimentary decoder for AppleSingle format files.}


\section{\module{buildtools} --- Helper module for BuildApplet and Friends}
\declaremodule{standard}{buildtools}
  \platform{Mac}
\modulesynopsis{Helper module for BuildApplet, BuildApplication and
                macfreeze.}

\deprecated{2.4}{}

\section{\module{cfmfile} --- Code Fragment Resource module}
\declaremodule{standard}{cfmfile}
  \platform{Mac}
\modulesynopsis{Code Fragment Resource module.}

\module{cfmfile} is a module that understands Code Fragments and the
accompanying ``cfrg'' resources. It can parse them and merge them, and is
used by BuildApplication to combine all plugin modules to a single
executable.

\deprecated{2.4}{}

\section{\module{icopen} --- Internet Config replacement for \method{open()}}
\declaremodule{standard}{icopen}
  \platform{Mac}
\modulesynopsis{Internet Config replacement for \method{open()}.}

Importing \module{icopen} will replace the builtin \method{open()}
with a version that uses Internet Config to set file type and creator
for new files.


\section{\module{macerrors} --- Mac OS Errors}
\declaremodule{standard}{macerrors}
  \platform{Mac}
\modulesynopsis{Constant definitions for many Mac OS error codes.}

\module{macerrors} contains constant definitions for many Mac OS error
codes.


\section{\module{macresource} --- Locate script resources}
\declaremodule{standard}{macresource}
  \platform{Mac}
\modulesynopsis{Locate script resources.}

\module{macresource} helps scripts finding their resources, such as
dialogs and menus, without requiring special case code for when the
script is run under MacPython, as a MacPython applet or under OSX Python.

\section{\module{Nav} --- NavServices calls}
\declaremodule{standard}{Nav}
  \platform{Mac}
\modulesynopsis{Interface to Navigation Services.}

A low-level interface to Navigation Services. 

\section{\module{PixMapWrapper} --- Wrapper for PixMap objects}
\declaremodule{standard}{PixMapWrapper}
  \platform{Mac}
\modulesynopsis{Wrapper for PixMap objects.}

\module{PixMapWrapper} wraps a PixMap object with a Python object that
allows access to the fields by name. It also has methods to convert
to and from \module{PIL} images.

\section{\module{videoreader} --- Read QuickTime movies}
\declaremodule{standard}{videoreader}
  \platform{Mac}
\modulesynopsis{Read QuickTime movies frame by frame for further processing.}

\module{videoreader} reads and decodes QuickTime movies and passes
a stream of images to your program. It also provides some support for
audio tracks.

\section{\module{W} --- Widgets built on \module{FrameWork}}
\declaremodule{standard}{W}
  \platform{Mac}
\modulesynopsis{Widgets for the Mac, built on top of \refmodule{FrameWork}.}

The \module{W} widgets are used extensively in the \program{IDE}.

                           % Undocumented Modules

\appendix
\chapter{History and License}
\section{History of the software}

Python was created in the early 1990s by Guido van Rossum at Stichting
Mathematisch Centrum (CWI, see \url{http://www.cwi.nl/}) in the Netherlands
as a successor of a language called ABC.  Guido remains Python's
principal author, although it includes many contributions from others.

In 1995, Guido continued his work on Python at the Corporation for
National Research Initiatives (CNRI, see \url{http://www.cnri.reston.va.us/})
in Reston, Virginia where he released several versions of the
software.

In May 2000, Guido and the Python core development team moved to
BeOpen.com to form the BeOpen PythonLabs team.  In October of the same
year, the PythonLabs team moved to Digital Creations (now Zope
Corporation; see \url{http://www.zope.com/}).  In 2001, the Python
Software Foundation (PSF, see \url{http://www.python.org/psf/}) was
formed, a non-profit organization created specifically to own
Python-related Intellectual Property.  Zope Corporation is a
sponsoring member of the PSF.

All Python releases are Open Source (see
\url{http://www.opensource.org/} for the Open Source Definition).
Historically, most, but not all, Python releases have also been
GPL-compatible; the table below summarizes the various releases.

\begin{tablev}{c|c|c|c|c}{textrm}%
  {Release}{Derived from}{Year}{Owner}{GPL compatible?}
  \linev{0.9.0 thru 1.2}{n/a}{1991-1995}{CWI}{yes}
  \linev{1.3 thru 1.5.2}{1.2}{1995-1999}{CNRI}{yes}
  \linev{1.6}{1.5.2}{2000}{CNRI}{no}
  \linev{2.0}{1.6}{2000}{BeOpen.com}{no}
  \linev{1.6.1}{1.6}{2001}{CNRI}{no}
  \linev{2.1}{2.0+1.6.1}{2001}{PSF}{no}
  \linev{2.0.1}{2.0+1.6.1}{2001}{PSF}{yes}
  \linev{2.1.1}{2.1+2.0.1}{2001}{PSF}{yes}
  \linev{2.2}{2.1.1}{2001}{PSF}{yes}
  \linev{2.1.2}{2.1.1}{2002}{PSF}{yes}
  \linev{2.1.3}{2.1.2}{2002}{PSF}{yes}
  \linev{2.2.1}{2.2}{2002}{PSF}{yes}
  \linev{2.2.2}{2.2.1}{2002}{PSF}{yes}
  \linev{2.2.3}{2.2.2}{2002-2003}{PSF}{yes}
  \linev{2.3}{2.2.2}{2002-2003}{PSF}{yes}
  \linev{2.3.1}{2.3}{2002-2003}{PSF}{yes}
  \linev{2.3.2}{2.3.1}{2003}{PSF}{yes}
  \linev{2.3.3}{2.3.2}{2003}{PSF}{yes}
  \linev{2.3.4}{2.3.3}{2004}{PSF}{yes}
  \linev{2.3.5}{2.3.4}{2005}{PSF}{yes}
  \linev{2.4}{2.3}{2004}{PSF}{yes}
  \linev{2.4.1}{2.4}{2005}{PSF}{yes}
  \linev{2.4.2}{2.4.1}{2005}{PSF}{yes}
  \linev{2.4.3}{2.4.2}{2006}{PSF}{yes}
  \linev{2.5}{2.4}{2006}{PSF}{yes}
\end{tablev}

\note{GPL-compatible doesn't mean that we're distributing
Python under the GPL.  All Python licenses, unlike the GPL, let you
distribute a modified version without making your changes open source.
The GPL-compatible licenses make it possible to combine Python with
other software that is released under the GPL; the others don't.}

Thanks to the many outside volunteers who have worked under Guido's
direction to make these releases possible.


\section{Terms and conditions for accessing or otherwise using Python}

\centerline{\strong{PSF LICENSE AGREEMENT FOR PYTHON \version}}

\begin{enumerate}
\item
This LICENSE AGREEMENT is between the Python Software Foundation
(``PSF''), and the Individual or Organization (``Licensee'') accessing
and otherwise using Python \version{} software in source or binary
form and its associated documentation.

\item
Subject to the terms and conditions of this License Agreement, PSF
hereby grants Licensee a nonexclusive, royalty-free, world-wide
license to reproduce, analyze, test, perform and/or display publicly,
prepare derivative works, distribute, and otherwise use Python
\version{} alone or in any derivative version, provided, however, that
PSF's License Agreement and PSF's notice of copyright, i.e.,
``Copyright \copyright{} 2001-2006 Python Software Foundation; All
Rights Reserved'' are retained in Python \version{} alone or in any
derivative version prepared by Licensee.

\item
In the event Licensee prepares a derivative work that is based on
or incorporates Python \version{} or any part thereof, and wants to
make the derivative work available to others as provided herein, then
Licensee hereby agrees to include in any such work a brief summary of
the changes made to Python \version.

\item
PSF is making Python \version{} available to Licensee on an ``AS IS''
basis.  PSF MAKES NO REPRESENTATIONS OR WARRANTIES, EXPRESS OR
IMPLIED.  BY WAY OF EXAMPLE, BUT NOT LIMITATION, PSF MAKES NO AND
DISCLAIMS ANY REPRESENTATION OR WARRANTY OF MERCHANTABILITY OR FITNESS
FOR ANY PARTICULAR PURPOSE OR THAT THE USE OF PYTHON \version{} WILL
NOT INFRINGE ANY THIRD PARTY RIGHTS.

\item
PSF SHALL NOT BE LIABLE TO LICENSEE OR ANY OTHER USERS OF PYTHON
\version{} FOR ANY INCIDENTAL, SPECIAL, OR CONSEQUENTIAL DAMAGES OR
LOSS AS A RESULT OF MODIFYING, DISTRIBUTING, OR OTHERWISE USING PYTHON
\version, OR ANY DERIVATIVE THEREOF, EVEN IF ADVISED OF THE
POSSIBILITY THEREOF.

\item
This License Agreement will automatically terminate upon a material
breach of its terms and conditions.

\item
Nothing in this License Agreement shall be deemed to create any
relationship of agency, partnership, or joint venture between PSF and
Licensee.  This License Agreement does not grant permission to use PSF
trademarks or trade name in a trademark sense to endorse or promote
products or services of Licensee, or any third party.

\item
By copying, installing or otherwise using Python \version, Licensee
agrees to be bound by the terms and conditions of this License
Agreement.
\end{enumerate}


\centerline{\strong{BEOPEN.COM LICENSE AGREEMENT FOR PYTHON 2.0}}

\centerline{\strong{BEOPEN PYTHON OPEN SOURCE LICENSE AGREEMENT VERSION 1}}

\begin{enumerate}
\item
This LICENSE AGREEMENT is between BeOpen.com (``BeOpen''), having an
office at 160 Saratoga Avenue, Santa Clara, CA 95051, and the
Individual or Organization (``Licensee'') accessing and otherwise
using this software in source or binary form and its associated
documentation (``the Software'').

\item
Subject to the terms and conditions of this BeOpen Python License
Agreement, BeOpen hereby grants Licensee a non-exclusive,
royalty-free, world-wide license to reproduce, analyze, test, perform
and/or display publicly, prepare derivative works, distribute, and
otherwise use the Software alone or in any derivative version,
provided, however, that the BeOpen Python License is retained in the
Software, alone or in any derivative version prepared by Licensee.

\item
BeOpen is making the Software available to Licensee on an ``AS IS''
basis.  BEOPEN MAKES NO REPRESENTATIONS OR WARRANTIES, EXPRESS OR
IMPLIED.  BY WAY OF EXAMPLE, BUT NOT LIMITATION, BEOPEN MAKES NO AND
DISCLAIMS ANY REPRESENTATION OR WARRANTY OF MERCHANTABILITY OR FITNESS
FOR ANY PARTICULAR PURPOSE OR THAT THE USE OF THE SOFTWARE WILL NOT
INFRINGE ANY THIRD PARTY RIGHTS.

\item
BEOPEN SHALL NOT BE LIABLE TO LICENSEE OR ANY OTHER USERS OF THE
SOFTWARE FOR ANY INCIDENTAL, SPECIAL, OR CONSEQUENTIAL DAMAGES OR LOSS
AS A RESULT OF USING, MODIFYING OR DISTRIBUTING THE SOFTWARE, OR ANY
DERIVATIVE THEREOF, EVEN IF ADVISED OF THE POSSIBILITY THEREOF.

\item
This License Agreement will automatically terminate upon a material
breach of its terms and conditions.

\item
This License Agreement shall be governed by and interpreted in all
respects by the law of the State of California, excluding conflict of
law provisions.  Nothing in this License Agreement shall be deemed to
create any relationship of agency, partnership, or joint venture
between BeOpen and Licensee.  This License Agreement does not grant
permission to use BeOpen trademarks or trade names in a trademark
sense to endorse or promote products or services of Licensee, or any
third party.  As an exception, the ``BeOpen Python'' logos available
at http://www.pythonlabs.com/logos.html may be used according to the
permissions granted on that web page.

\item
By copying, installing or otherwise using the software, Licensee
agrees to be bound by the terms and conditions of this License
Agreement.
\end{enumerate}


\centerline{\strong{CNRI LICENSE AGREEMENT FOR PYTHON 1.6.1}}

\begin{enumerate}
\item
This LICENSE AGREEMENT is between the Corporation for National
Research Initiatives, having an office at 1895 Preston White Drive,
Reston, VA 20191 (``CNRI''), and the Individual or Organization
(``Licensee'') accessing and otherwise using Python 1.6.1 software in
source or binary form and its associated documentation.

\item
Subject to the terms and conditions of this License Agreement, CNRI
hereby grants Licensee a nonexclusive, royalty-free, world-wide
license to reproduce, analyze, test, perform and/or display publicly,
prepare derivative works, distribute, and otherwise use Python 1.6.1
alone or in any derivative version, provided, however, that CNRI's
License Agreement and CNRI's notice of copyright, i.e., ``Copyright
\copyright{} 1995-2001 Corporation for National Research Initiatives;
All Rights Reserved'' are retained in Python 1.6.1 alone or in any
derivative version prepared by Licensee.  Alternately, in lieu of
CNRI's License Agreement, Licensee may substitute the following text
(omitting the quotes): ``Python 1.6.1 is made available subject to the
terms and conditions in CNRI's License Agreement.  This Agreement
together with Python 1.6.1 may be located on the Internet using the
following unique, persistent identifier (known as a handle):
1895.22/1013.  This Agreement may also be obtained from a proxy server
on the Internet using the following URL:
\url{http://hdl.handle.net/1895.22/1013}.''

\item
In the event Licensee prepares a derivative work that is based on
or incorporates Python 1.6.1 or any part thereof, and wants to make
the derivative work available to others as provided herein, then
Licensee hereby agrees to include in any such work a brief summary of
the changes made to Python 1.6.1.

\item
CNRI is making Python 1.6.1 available to Licensee on an ``AS IS''
basis.  CNRI MAKES NO REPRESENTATIONS OR WARRANTIES, EXPRESS OR
IMPLIED.  BY WAY OF EXAMPLE, BUT NOT LIMITATION, CNRI MAKES NO AND
DISCLAIMS ANY REPRESENTATION OR WARRANTY OF MERCHANTABILITY OR FITNESS
FOR ANY PARTICULAR PURPOSE OR THAT THE USE OF PYTHON 1.6.1 WILL NOT
INFRINGE ANY THIRD PARTY RIGHTS.

\item
CNRI SHALL NOT BE LIABLE TO LICENSEE OR ANY OTHER USERS OF PYTHON
1.6.1 FOR ANY INCIDENTAL, SPECIAL, OR CONSEQUENTIAL DAMAGES OR LOSS AS
A RESULT OF MODIFYING, DISTRIBUTING, OR OTHERWISE USING PYTHON 1.6.1,
OR ANY DERIVATIVE THEREOF, EVEN IF ADVISED OF THE POSSIBILITY THEREOF.

\item
This License Agreement will automatically terminate upon a material
breach of its terms and conditions.

\item
This License Agreement shall be governed by the federal
intellectual property law of the United States, including without
limitation the federal copyright law, and, to the extent such
U.S. federal law does not apply, by the law of the Commonwealth of
Virginia, excluding Virginia's conflict of law provisions.
Notwithstanding the foregoing, with regard to derivative works based
on Python 1.6.1 that incorporate non-separable material that was
previously distributed under the GNU General Public License (GPL), the
law of the Commonwealth of Virginia shall govern this License
Agreement only as to issues arising under or with respect to
Paragraphs 4, 5, and 7 of this License Agreement.  Nothing in this
License Agreement shall be deemed to create any relationship of
agency, partnership, or joint venture between CNRI and Licensee.  This
License Agreement does not grant permission to use CNRI trademarks or
trade name in a trademark sense to endorse or promote products or
services of Licensee, or any third party.

\item
By clicking on the ``ACCEPT'' button where indicated, or by copying,
installing or otherwise using Python 1.6.1, Licensee agrees to be
bound by the terms and conditions of this License Agreement.
\end{enumerate}

\centerline{ACCEPT}



\centerline{\strong{CWI LICENSE AGREEMENT FOR PYTHON 0.9.0 THROUGH 1.2}}

Copyright \copyright{} 1991 - 1995, Stichting Mathematisch Centrum
Amsterdam, The Netherlands.  All rights reserved.

Permission to use, copy, modify, and distribute this software and its
documentation for any purpose and without fee is hereby granted,
provided that the above copyright notice appear in all copies and that
both that copyright notice and this permission notice appear in
supporting documentation, and that the name of Stichting Mathematisch
Centrum or CWI not be used in advertising or publicity pertaining to
distribution of the software without specific, written prior
permission.

STICHTING MATHEMATISCH CENTRUM DISCLAIMS ALL WARRANTIES WITH REGARD TO
THIS SOFTWARE, INCLUDING ALL IMPLIED WARRANTIES OF MERCHANTABILITY AND
FITNESS, IN NO EVENT SHALL STICHTING MATHEMATISCH CENTRUM BE LIABLE
FOR ANY SPECIAL, INDIRECT OR CONSEQUENTIAL DAMAGES OR ANY DAMAGES
WHATSOEVER RESULTING FROM LOSS OF USE, DATA OR PROFITS, WHETHER IN AN
ACTION OF CONTRACT, NEGLIGENCE OR OTHER TORTIOUS ACTION, ARISING OUT
OF OR IN CONNECTION WITH THE USE OR PERFORMANCE OF THIS SOFTWARE.


\section{Licenses and Acknowledgements for Incorporated Software}

This section is an incomplete, but growing list of licenses and
acknowledgements for third-party software incorporated in the
Python distribution.


\subsection{Mersenne Twister}

The \module{_random} module includes code based on a download from
\url{http://www.math.keio.ac.jp/~matumoto/MT2002/emt19937ar.html}.
The following are the verbatim comments from the original code:

\begin{verbatim}
A C-program for MT19937, with initialization improved 2002/1/26.
Coded by Takuji Nishimura and Makoto Matsumoto.

Before using, initialize the state by using init_genrand(seed)
or init_by_array(init_key, key_length).

Copyright (C) 1997 - 2002, Makoto Matsumoto and Takuji Nishimura,
All rights reserved.

Redistribution and use in source and binary forms, with or without
modification, are permitted provided that the following conditions
are met:

 1. Redistributions of source code must retain the above copyright
    notice, this list of conditions and the following disclaimer.

 2. Redistributions in binary form must reproduce the above copyright
    notice, this list of conditions and the following disclaimer in the
    documentation and/or other materials provided with the distribution.

 3. The names of its contributors may not be used to endorse or promote
    products derived from this software without specific prior written
    permission.

THIS SOFTWARE IS PROVIDED BY THE COPYRIGHT HOLDERS AND CONTRIBUTORS
"AS IS" AND ANY EXPRESS OR IMPLIED WARRANTIES, INCLUDING, BUT NOT
LIMITED TO, THE IMPLIED WARRANTIES OF MERCHANTABILITY AND FITNESS FOR
A PARTICULAR PURPOSE ARE DISCLAIMED.  IN NO EVENT SHALL THE COPYRIGHT OWNER OR
CONTRIBUTORS BE LIABLE FOR ANY DIRECT, INDIRECT, INCIDENTAL, SPECIAL,
EXEMPLARY, OR CONSEQUENTIAL DAMAGES (INCLUDING, BUT NOT LIMITED TO,
PROCUREMENT OF SUBSTITUTE GOODS OR SERVICES; LOSS OF USE, DATA, OR
PROFITS; OR BUSINESS INTERRUPTION) HOWEVER CAUSED AND ON ANY THEORY OF
LIABILITY, WHETHER IN CONTRACT, STRICT LIABILITY, OR TORT (INCLUDING
NEGLIGENCE OR OTHERWISE) ARISING IN ANY WAY OUT OF THE USE OF THIS
SOFTWARE, EVEN IF ADVISED OF THE POSSIBILITY OF SUCH DAMAGE.


Any feedback is very welcome.
http://www.math.keio.ac.jp/matumoto/emt.html
email: matumoto@math.keio.ac.jp
\end{verbatim}



\subsection{Sockets}

The \module{socket} module uses the functions, \function{getaddrinfo},
and \function{getnameinfo}, which are coded in separate source files
from the WIDE Project, \url{http://www.wide.ad.jp/about/index.html}.

\begin{verbatim}      
Copyright (C) 1995, 1996, 1997, and 1998 WIDE Project.
All rights reserved.
 
Redistribution and use in source and binary forms, with or without
modification, are permitted provided that the following conditions
are met:
1. Redistributions of source code must retain the above copyright
   notice, this list of conditions and the following disclaimer.
2. Redistributions in binary form must reproduce the above copyright
   notice, this list of conditions and the following disclaimer in the
   documentation and/or other materials provided with the distribution.
3. Neither the name of the project nor the names of its contributors
   may be used to endorse or promote products derived from this software
   without specific prior written permission.

THIS SOFTWARE IS PROVIDED BY THE PROJECT AND CONTRIBUTORS ``AS IS'' AND
GAI_ANY EXPRESS OR IMPLIED WARRANTIES, INCLUDING, BUT NOT LIMITED TO, THE
IMPLIED WARRANTIES OF MERCHANTABILITY AND FITNESS FOR A PARTICULAR PURPOSE
ARE DISCLAIMED.  IN NO EVENT SHALL THE PROJECT OR CONTRIBUTORS BE LIABLE
FOR GAI_ANY DIRECT, INDIRECT, INCIDENTAL, SPECIAL, EXEMPLARY, OR CONSEQUENTIAL
DAMAGES (INCLUDING, BUT NOT LIMITED TO, PROCUREMENT OF SUBSTITUTE GOODS
OR SERVICES; LOSS OF USE, DATA, OR PROFITS; OR BUSINESS INTERRUPTION)
HOWEVER CAUSED AND ON GAI_ANY THEORY OF LIABILITY, WHETHER IN CONTRACT, STRICT
LIABILITY, OR TORT (INCLUDING NEGLIGENCE OR OTHERWISE) ARISING IN GAI_ANY WAY
OUT OF THE USE OF THIS SOFTWARE, EVEN IF ADVISED OF THE POSSIBILITY OF
SUCH DAMAGE.
\end{verbatim}



\subsection{Floating point exception control}

The source for the \module{fpectl} module includes the following notice:

\begin{verbatim}
     ---------------------------------------------------------------------  
    /                       Copyright (c) 1996.                           \ 
   |          The Regents of the University of California.                 |
   |                        All rights reserved.                           |
   |                                                                       |
   |   Permission to use, copy, modify, and distribute this software for   |
   |   any purpose without fee is hereby granted, provided that this en-   |
   |   tire notice is included in all copies of any software which is or   |
   |   includes  a  copy  or  modification  of  this software and in all   |
   |   copies of the supporting documentation for such software.           |
   |                                                                       |
   |   This  work was produced at the University of California, Lawrence   |
   |   Livermore National Laboratory under  contract  no.  W-7405-ENG-48   |
   |   between  the  U.S.  Department  of  Energy and The Regents of the   |
   |   University of California for the operation of UC LLNL.              |
   |                                                                       |
   |                              DISCLAIMER                               |
   |                                                                       |
   |   This  software was prepared as an account of work sponsored by an   |
   |   agency of the United States Government. Neither the United States   |
   |   Government  nor the University of California nor any of their em-   |
   |   ployees, makes any warranty, express or implied, or  assumes  any   |
   |   liability  or  responsibility  for the accuracy, completeness, or   |
   |   usefulness of any information,  apparatus,  product,  or  process   |
   |   disclosed,   or  represents  that  its  use  would  not  infringe   |
   |   privately-owned rights. Reference herein to any specific  commer-   |
   |   cial  products,  process,  or  service  by trade name, trademark,   |
   |   manufacturer, or otherwise, does not  necessarily  constitute  or   |
   |   imply  its endorsement, recommendation, or favoring by the United   |
   |   States Government or the University of California. The views  and   |
   |   opinions  of authors expressed herein do not necessarily state or   |
   |   reflect those of the United States Government or  the  University   |
   |   of  California,  and shall not be used for advertising or product   |
    \  endorsement purposes.                                              / 
     ---------------------------------------------------------------------
\end{verbatim}



\subsection{MD5 message digest algorithm}

The source code for the \module{md5} module contains the following notice:

\begin{verbatim}
  Copyright (C) 1999, 2002 Aladdin Enterprises.  All rights reserved.

  This software is provided 'as-is', without any express or implied
  warranty.  In no event will the authors be held liable for any damages
  arising from the use of this software.

  Permission is granted to anyone to use this software for any purpose,
  including commercial applications, and to alter it and redistribute it
  freely, subject to the following restrictions:

  1. The origin of this software must not be misrepresented; you must not
     claim that you wrote the original software. If you use this software
     in a product, an acknowledgment in the product documentation would be
     appreciated but is not required.
  2. Altered source versions must be plainly marked as such, and must not be
     misrepresented as being the original software.
  3. This notice may not be removed or altered from any source distribution.

  L. Peter Deutsch
  ghost@aladdin.com

  Independent implementation of MD5 (RFC 1321).

  This code implements the MD5 Algorithm defined in RFC 1321, whose
  text is available at
	http://www.ietf.org/rfc/rfc1321.txt
  The code is derived from the text of the RFC, including the test suite
  (section A.5) but excluding the rest of Appendix A.  It does not include
  any code or documentation that is identified in the RFC as being
  copyrighted.

  The original and principal author of md5.h is L. Peter Deutsch
  <ghost@aladdin.com>.  Other authors are noted in the change history
  that follows (in reverse chronological order):

  2002-04-13 lpd Removed support for non-ANSI compilers; removed
	references to Ghostscript; clarified derivation from RFC 1321;
	now handles byte order either statically or dynamically.
  1999-11-04 lpd Edited comments slightly for automatic TOC extraction.
  1999-10-18 lpd Fixed typo in header comment (ansi2knr rather than md5);
	added conditionalization for C++ compilation from Martin
	Purschke <purschke@bnl.gov>.
  1999-05-03 lpd Original version.
\end{verbatim}



\subsection{Asynchronous socket services}

The \module{asynchat} and \module{asyncore} modules contain the
following notice:

\begin{verbatim}      
 Copyright 1996 by Sam Rushing

                         All Rights Reserved

 Permission to use, copy, modify, and distribute this software and
 its documentation for any purpose and without fee is hereby
 granted, provided that the above copyright notice appear in all
 copies and that both that copyright notice and this permission
 notice appear in supporting documentation, and that the name of Sam
 Rushing not be used in advertising or publicity pertaining to
 distribution of the software without specific, written prior
 permission.

 SAM RUSHING DISCLAIMS ALL WARRANTIES WITH REGARD TO THIS SOFTWARE,
 INCLUDING ALL IMPLIED WARRANTIES OF MERCHANTABILITY AND FITNESS, IN
 NO EVENT SHALL SAM RUSHING BE LIABLE FOR ANY SPECIAL, INDIRECT OR
 CONSEQUENTIAL DAMAGES OR ANY DAMAGES WHATSOEVER RESULTING FROM LOSS
 OF USE, DATA OR PROFITS, WHETHER IN AN ACTION OF CONTRACT,
 NEGLIGENCE OR OTHER TORTIOUS ACTION, ARISING OUT OF OR IN
 CONNECTION WITH THE USE OR PERFORMANCE OF THIS SOFTWARE.
\end{verbatim}


\subsection{Cookie management}

The \module{Cookie} module contains the following notice:

\begin{verbatim}
 Copyright 2000 by Timothy O'Malley <timo@alum.mit.edu>

                All Rights Reserved

 Permission to use, copy, modify, and distribute this software
 and its documentation for any purpose and without fee is hereby
 granted, provided that the above copyright notice appear in all
 copies and that both that copyright notice and this permission
 notice appear in supporting documentation, and that the name of
 Timothy O'Malley  not be used in advertising or publicity
 pertaining to distribution of the software without specific, written
 prior permission.

 Timothy O'Malley DISCLAIMS ALL WARRANTIES WITH REGARD TO THIS
 SOFTWARE, INCLUDING ALL IMPLIED WARRANTIES OF MERCHANTABILITY
 AND FITNESS, IN NO EVENT SHALL Timothy O'Malley BE LIABLE FOR
 ANY SPECIAL, INDIRECT OR CONSEQUENTIAL DAMAGES OR ANY DAMAGES
 WHATSOEVER RESULTING FROM LOSS OF USE, DATA OR PROFITS,
 WHETHER IN AN ACTION OF CONTRACT, NEGLIGENCE OR OTHER TORTIOUS
 ACTION, ARISING OUT OF OR IN CONNECTION WITH THE USE OR
 PERFORMANCE OF THIS SOFTWARE.
\end{verbatim}      



\subsection{Profiling}

The \module{profile} and \module{pstats} modules contain
the following notice:

\begin{verbatim}
 Copyright 1994, by InfoSeek Corporation, all rights reserved.
 Written by James Roskind

 Permission to use, copy, modify, and distribute this Python software
 and its associated documentation for any purpose (subject to the
 restriction in the following sentence) without fee is hereby granted,
 provided that the above copyright notice appears in all copies, and
 that both that copyright notice and this permission notice appear in
 supporting documentation, and that the name of InfoSeek not be used in
 advertising or publicity pertaining to distribution of the software
 without specific, written prior permission.  This permission is
 explicitly restricted to the copying and modification of the software
 to remain in Python, compiled Python, or other languages (such as C)
 wherein the modified or derived code is exclusively imported into a
 Python module.

 INFOSEEK CORPORATION DISCLAIMS ALL WARRANTIES WITH REGARD TO THIS
 SOFTWARE, INCLUDING ALL IMPLIED WARRANTIES OF MERCHANTABILITY AND
 FITNESS. IN NO EVENT SHALL INFOSEEK CORPORATION BE LIABLE FOR ANY
 SPECIAL, INDIRECT OR CONSEQUENTIAL DAMAGES OR ANY DAMAGES WHATSOEVER
 RESULTING FROM LOSS OF USE, DATA OR PROFITS, WHETHER IN AN ACTION OF
 CONTRACT, NEGLIGENCE OR OTHER TORTIOUS ACTION, ARISING OUT OF OR IN
 CONNECTION WITH THE USE OR PERFORMANCE OF THIS SOFTWARE.
\end{verbatim}



\subsection{Execution tracing}

The \module{trace} module contains the following notice:

\begin{verbatim}
 portions copyright 2001, Autonomous Zones Industries, Inc., all rights...
 err...  reserved and offered to the public under the terms of the
 Python 2.2 license.
 Author: Zooko O'Whielacronx
 http://zooko.com/
 mailto:zooko@zooko.com

 Copyright 2000, Mojam Media, Inc., all rights reserved.
 Author: Skip Montanaro

 Copyright 1999, Bioreason, Inc., all rights reserved.
 Author: Andrew Dalke

 Copyright 1995-1997, Automatrix, Inc., all rights reserved.
 Author: Skip Montanaro

 Copyright 1991-1995, Stichting Mathematisch Centrum, all rights reserved.


 Permission to use, copy, modify, and distribute this Python software and
 its associated documentation for any purpose without fee is hereby
 granted, provided that the above copyright notice appears in all copies,
 and that both that copyright notice and this permission notice appear in
 supporting documentation, and that the name of neither Automatrix,
 Bioreason or Mojam Media be used in advertising or publicity pertaining to
 distribution of the software without specific, written prior permission.
\end{verbatim} 



\subsection{UUencode and UUdecode functions}

The \module{uu} module contains the following notice:

\begin{verbatim}
 Copyright 1994 by Lance Ellinghouse
 Cathedral City, California Republic, United States of America.
                        All Rights Reserved
 Permission to use, copy, modify, and distribute this software and its
 documentation for any purpose and without fee is hereby granted,
 provided that the above copyright notice appear in all copies and that
 both that copyright notice and this permission notice appear in
 supporting documentation, and that the name of Lance Ellinghouse
 not be used in advertising or publicity pertaining to distribution
 of the software without specific, written prior permission.
 LANCE ELLINGHOUSE DISCLAIMS ALL WARRANTIES WITH REGARD TO
 THIS SOFTWARE, INCLUDING ALL IMPLIED WARRANTIES OF MERCHANTABILITY AND
 FITNESS, IN NO EVENT SHALL LANCE ELLINGHOUSE CENTRUM BE LIABLE
 FOR ANY SPECIAL, INDIRECT OR CONSEQUENTIAL DAMAGES OR ANY DAMAGES
 WHATSOEVER RESULTING FROM LOSS OF USE, DATA OR PROFITS, WHETHER IN AN
 ACTION OF CONTRACT, NEGLIGENCE OR OTHER TORTIOUS ACTION, ARISING OUT
 OF OR IN CONNECTION WITH THE USE OR PERFORMANCE OF THIS SOFTWARE.

 Modified by Jack Jansen, CWI, July 1995:
 - Use binascii module to do the actual line-by-line conversion
   between ascii and binary. This results in a 1000-fold speedup. The C
   version is still 5 times faster, though.
 - Arguments more compliant with python standard
\end{verbatim}



\subsection{XML Remote Procedure Calls}

The \module{xmlrpclib} module contains the following notice:

\begin{verbatim}
     The XML-RPC client interface is

 Copyright (c) 1999-2002 by Secret Labs AB
 Copyright (c) 1999-2002 by Fredrik Lundh

 By obtaining, using, and/or copying this software and/or its
 associated documentation, you agree that you have read, understood,
 and will comply with the following terms and conditions:

 Permission to use, copy, modify, and distribute this software and
 its associated documentation for any purpose and without fee is
 hereby granted, provided that the above copyright notice appears in
 all copies, and that both that copyright notice and this permission
 notice appear in supporting documentation, and that the name of
 Secret Labs AB or the author not be used in advertising or publicity
 pertaining to distribution of the software without specific, written
 prior permission.

 SECRET LABS AB AND THE AUTHOR DISCLAIMS ALL WARRANTIES WITH REGARD
 TO THIS SOFTWARE, INCLUDING ALL IMPLIED WARRANTIES OF MERCHANT-
 ABILITY AND FITNESS.  IN NO EVENT SHALL SECRET LABS AB OR THE AUTHOR
 BE LIABLE FOR ANY SPECIAL, INDIRECT OR CONSEQUENTIAL DAMAGES OR ANY
 DAMAGES WHATSOEVER RESULTING FROM LOSS OF USE, DATA OR PROFITS,
 WHETHER IN AN ACTION OF CONTRACT, NEGLIGENCE OR OTHER TORTIOUS
 ACTION, ARISING OUT OF OR IN CONNECTION WITH THE USE OR PERFORMANCE
 OF THIS SOFTWARE.
\end{verbatim}


%
%  The ugly "%begin{latexonly}" pseudo-environments are really just to
%  keep LaTeX2HTML quiet during the \renewcommand{} macros; they're
%  not really valuable.
%

%begin{latexonly}
\renewcommand{\indexname}{Module Index}
%end{latexonly}
\input{modmac.ind}      % Module Index

%begin{latexonly}
\renewcommand{\indexname}{Index}
%end{latexonly}
\documentclass{manual}

\title{Macintosh Library Modules}

\author{Guido van Rossum\\
	Fred L. Drake, Jr., editor}
\authoraddress{
	PythonLabs\\
	E-mail: \email{python-docs@python.org}
}

\date{June 15, 2001}		% XXX update before release!
\release{2.0.1c1}		% software release, not documentation
\setshortversion{2.0}		% major.minor only for software


\makeindex              % tell \index to actually write the .idx file
\makemodindex           % ... and the module index as well.


\begin{document}

\maketitle

\ifhtml
\chapter*{Front Matter\label{front}}
\fi

Copyright 1991, 1992, 1993, 1994 by Stichting Mathematisch Centrum,
Amsterdam, The Netherlands.

\begin{center}
All Rights Reserved
\end{center}

Permission to use, copy, modify, and distribute this software and its
documentation for any purpose and without fee is hereby granted,
provided that the above copyright notice appear in all copies and that
both that copyright notice and this permission notice appear in
supporting documentation, and that the names of Stichting Mathematisch
Centrum or CWI not be used in advertising or publicity pertaining to
distribution of the software without specific, written prior permission.

STICHTING MATHEMATISCH CENTRUM DISCLAIMS ALL WARRANTIES WITH REGARD TO
THIS SOFTWARE, INCLUDING ALL IMPLIED WARRANTIES OF MERCHANTABILITY AND
FITNESS, IN NO EVENT SHALL STICHTING MATHEMATISCH CENTRUM BE LIABLE
FOR ANY SPECIAL, INDIRECT OR CONSEQUENTIAL DAMAGES OR ANY DAMAGES
WHATSOEVER RESULTING FROM LOSS OF USE, DATA OR PROFITS, WHETHER IN AN
ACTION OF CONTRACT, NEGLIGENCE OR OTHER TORTIOUS ACTION, ARISING OUT
OF OR IN CONNECTION WITH THE USE OR PERFORMANCE OF THIS SOFTWARE.


\begin{abstract}

\noindent
This library reference manual documents Python's extensions for the
Macintosh.  It should be used in conjunction with the
\citetitle[../lib/lib.html]{Python Library Reference}, which documents
the standard library and built-in types.

This manual assumes basic knowledge about the Python language.  For an
informal introduction to Python, see the
\citetitle[../tut/tut.html]{Python Tutorial}; the
\citetitle[../ref/ref.html]{Python Reference Manual} remains the
highest authority on syntactic and semantic questions.  Finally, the
manual entitled \citetitle[../ext/ext.html]{Extending and Embedding
the Python Interpreter} describes how to add new extensions to Python
and how to embed it in other applications.

\end{abstract}

\tableofcontents


\chapter{Using Python on a Mac OS 9 Macintosh \label{using}}
\sectionauthor{Bob Savage}{bobsavage@mac.com}

Using Python on a Mac OS 9 Macintosh can seem like something completely
different than using it on a \UNIX-like or Windows system. Most of the
Python documentation, both the ``official'' documentation and
published books, describe only how Python is used on these systems,
causing confusion for the new user of MacPython-OS9. This chapter gives a
brief introduction to the specifics of using Python on a Macintosh.

Note that this chapter is mainly relevant to Mac OS 9: MacPython-OSX
is a superset of a normal unix Python. While MacPython-OS9 runs fine
on Mac OS X it is a better choice to use MacPython-OSX there.

The section on the IDE (see Section \ref{IDE}) is relevant to MacPython-OSX
too.

\section{Getting and Installing MacPython-OS9 \label{getting}}

The most recent release version as well as possible newer experimental
versions are best found at the MacPython page maintained by Jack
Jansen: \url{http://www.cwi.nl/\textasciitilde jack/macpython.html}.


Please refer to the \file{README} included with your distribution for
the most up-to-date instructions.


\section{Entering the interactive Interpreter
         \label{interpreter}}

The interactive interpreter that you will see used in Python
documentation is started by double-clicking the
\program{PythonInterpreter} icon, which looks like a 16-ton weight
falling. You should see the version information and the
\samp{>\code{>}>~} prompt.  Use it exactly as described in the
standard documentation.


\section{How to run a Python script}

There are several ways to run an existing Python script; two common
ways to run a Python script are ``drag and drop'' and ``double
clicking''.  Other ways include running it from within the IDE (see
Section \ref{IDE}), or launching via AppleScript.


\subsection{Drag and drop}

One of the easiest ways to launch a Python script is via ``Drag and
Drop''. This is just like launching a text file in the Finder by
``dragging'' it over your word processor's icon and ``dropping'' it
there. Make sure that you use an icon referring to the
\program{PythonInterpreter}, not the \program{IDE} or \program{Idle}
icons which have different behaviour which is described below.

Some things that might have gone wrong:

\begin{itemize}
\item
A window flashes after dropping the script onto the
\program{PythonInterpreter}, but then disappears. Most likely this is a
configuration issue; your \program{PythonInterpreter} is setup to exit
immediately upon completion, but your script assumes that if it prints
something that text will stick around for a while. To fix this, see
section \ref{defaults}.

\item
When you waved the script icon over the \program{PythonInterpreter},
the \program{PythonInterpreter} icon did not hilight.  Most likely the
Creator code and document type is unset (or set incorrectly) -- this
often happens when a file originates on a non-Mac computer.  See
section \ref{creator-code} for more details.
\end{itemize}


\subsection{Set Creator and Double Click \label{creator-code}}

If the script that you want to launch has the appropriate Creator Code
and File Type you can simply double-click on the script to launch it.
To be ``double-clickable'' a file needs to be of type \samp{TEXT},
with a creator code of \samp{Pyth}.

Setting the creator code and filetype can be done with the IDE (see
sections \ref{IDEwrite} and \ref{IDEapplet}), with an editor with a
Python mode (\program{BBEdit}) -- see section
\ref{scripting-with-BBedit}, or with assorted other Mac utilities, but
a script (\file{fixfiletypes.py}) has been included in the MacPython
distribution, making it possible to set the proper Type and Creator
Codes with Python.

The \file{fixfiletypes.py} script will change the file type and
creator codes for the indicated directory.  To use
\file{fixfiletypes.py}:

\begin{enumerate}
\item
Locate it in the \file{scripts} folder of the \file{Mac} folder of the
MacPython distribution.

\item
Put all of the scripts that you want to fix in a folder with nothing
else in it.

\item
Double-click on the \file{fixfiletypes.py} icon.

\item
Navigate into the folder of files you want to fix, and press the
``Select current folder'' button.
\end{enumerate}


\section{Simulating command line arguments
         \label{argv}}

There are two ways to simulate command-line arguments with MacPython-OS9.
 
\begin{enumerate}
\item via Interpreter options
\begin{itemize} % nestable? I hope so!
  \item Hold the option-key down when launching your script. This will
        bring up a dialog box of Python Interpreter options.
  \item Click ``Set \UNIX-style command line..'' button. 
  \item Type the arguments into the ``Argument'' field.
  \item Click ``OK''
  \item Click ``Run''.
\end{itemize} % end

\item via drag and drop
If you save the script as an applet (see Section \ref{IDEapplet}), you
can also simulate some command-line arguments via
``Drag-and-Drop''. In this case, the names of the files that were
dropped onto the applet will be appended to \code{sys.argv}, so that
it will appear to the script as though they had been typed on a
command line.  As on \UNIX\ systems, the first item in \code{sys.srgv} is
the path to the applet, and the rest are the files dropped on the
applet.
\end{enumerate}


\section{Creating a Python script}

Since Python scripts are simply text files, they can be created in any
way that text files can be created, but some special tools also exist
with extra features.


\subsection{In an editor}

You can create a text file with any word processing program such as
\program{MSWord} or \program{AppleWorks} but you need to make sure
that the file is saved as ``\ASCII'' or ``plain text''.


\subsubsection{Editors with Python modes}

Several text editors have additional features that add functionality
when you are creating a Python script.  These can include coloring
Python keywords to make your code easier to read, module browsing, or
a built-in debugger. These include \program{Alpha}, \program{Pepper},
and \program{BBedit}, and the MacPython IDE (Section \ref{IDE}).

%\subsubsection{Alpha}
% **NEED INFO HERE**
 
\subsubsection{BBedit \label{scripting-with-BBedit}}

If you use \program{BBEdit} to create your scripts you will want to tell it about the Python creator code so that
you can simply double click on the saved file to launch it.
\begin{itemize}
  \item Launch \program{BBEdit}.
  \item Select ``Preferences'' from the ``Edit'' menu.
  \item Select ``File Types'' from the scrolling list.
  \item click on the ``Add...'' button and navigate to
        \program{PythonInterpreter} in the main directory of the
        MacPython distribution; click ``open''.
  \item Click on the ``Save'' button in the Preferences panel.
\end{itemize}
% Are there additional BBedit Python-specific features? I'm not aware of any.
 
%\subsubsection{IDE}
%You can use the \program{Python IDE} supplied in the MacPython Distribution to create longer Python scripts 
%-- see Section \ref{IDEwrite} for details.
 
%\subsubsection{IDLE}
%Idle is an IDE for Python that was written in Python, using TKInter. You should be able to use it on a Mac by following
%the standard documentation, but see Section \ref{TKInter} for guidance on using TKInter with MacPython.

%\subsubsection{Pepper}
% **NEED INFO HERE**


\section{The IDE\label{IDE}}

The \program{Python IDE} (Integrated Development Environment) is a
separate application that acts as a text editor for your Python code,
a class browser, a graphical debugger, and more.


\subsection{Using the ``Python Interactive'' window}

Use this window like you would the \program{PythonInterpreter}, except
that you cannot use the ``Drag and drop'' method above. Instead,
dropping a script onto the \program{Python IDE} icon will open the
file in a separate script window (which you can then execute manually
-- see section \ref{IDEexecution}).


\subsection{Writing a Python Script \label{IDEwrite}}

In addition to using the \program{Python IDE} interactively, you can
also type out a complete Python program, saving it incrementally, and
execute it or smaller selections of it.

You can create a new script, open a previously saved script, and save
your currently open script by selecting the appropriate item in the
``File'' menu. Dropping a Python script onto the
\program{Python IDE} will open it for editting.

If you try to open a script with the \program{Python IDE} but either
can't locate it from the ``Open'' dialog box, or you get an error
message like ``Can't open file of type ...'' see section
\ref{creator-code}.

When the \program{Python IDE} saves a script, it uses the creator code
settings which are available by clicking on the small black triangle
on the top right of the document window, and selecting ``save
options''. The default is to save the file with the \program{Python
IDE} as the creator, this means that you can open the file for editing
by simply double-clicking on its icon. You might want to change this
behaviour so that it will be opened by the
\program{PythonInterpreter}, and run. To do this simply choose
``Python Interpreter'' from the ``save options''. Note that these
options are associated with the \emph{file} not the application.


\subsection{Executing a script from within the IDE
            \label{IDEexecution}}

You can run the script in the frontmost window of the \program{Python
IDE} by hitting the run all button.  You should be aware, however that
if you use the Python convention \samp{if __name__ == "__main__":} the
script will \emph{not} be ``__main__'' by default. To get that
behaviour you must select the ``Run as __main__'' option from the
small black triangle on the top right of the document window.  Note
that this option is associated with the \emph{file} not the
application. It \emph{will} stay active after a save, however; to shut
this feature off simply select it again.
 

\subsection{``Save as'' versus ``Save as Applet''
            \label{IDEapplet}}

When you are done writing your Python script you have the option of
saving it as an ``applet'' (by selecting ``Save as applet'' from the
``File'' menu). This has a significant advantage in that you can drop
files or folders onto it, to pass them to the applet the way
command-line users would type them onto the command-line to pass them
as arguments to the script. However, you should make sure to save the
applet as a separate file, do not overwrite the script you are
writing, because you will not be able to edit it again.

Accessing the items passed to the applet via ``drag-and-drop'' is done
using the standard \member{sys.argv} mechanism. See the general
documentation for more
% need to link to the appropriate place in non-Mac docs

Note that saving a script as an applet will not make it runnable on a
system without a Python installation.

%\subsection{Debugger}
% **NEED INFO HERE**
 
%\subsection{Module Browser}
% **NEED INFO HERE**
 
%\subsection{Profiler}
% **NEED INFO HERE**
% end IDE

%\subsection{The ``Scripts'' menu}
% **NEED INFO HERE**
 
\section{Configuration \label{configuration}}

The MacPython distribution comes with \program{EditPythonPrefs}, an
applet which will help you to customize the MacPython environment for
your working habits.
 
\subsection{EditPythonPrefs\label{EditPythonPrefs}}

\program{EditPythonPrefs} gives you the capability to configure Python
to behave the way you want it to.  There are two ways to use
\program{EditPythonPrefs}, you can use it to set the preferences in
general, or you can drop a particular Python engine onto it to
customize only that version. The latter can be handy if, for example,
you want to have a second copy of the \program{PythonInterpreter} that
keeps the output window open on a normal exit even though you prefer
to normally not work that way.

To change the default preferences, simply double-click on
\program{EditPythonPrefs}. To change the preferences only for one copy
of the Interpreter, drop the icon for that copy onto
\program{EditPythonPrefs}.  You can also use \program{EditPythonPrefs}
in this fashion to set the preferences of the \program{Python IDE} and
any applets you create -- see section %s \ref{BuildApplet} and
\ref{IDEapplet}.

\subsection{Adding modules to the Module Search Path
            \label{search-path}}

When executing an \keyword{import} statement, Python looks for modules
in places defined by the \member{sys.path} To edit the
\member{sys.path} on a Mac, launch \program{EditPythonPrefs}, and
enter them into the largish field at the top (one per line).

Since MacPython defines a main Python directory, the easiest thing is
to add folders to search within the main Python directory. To add a
folder of scripts that you created called ``My Folder'' located in the
main Python Folder, enter \samp{\$(PYTHON):My Folder} onto a new line.

To add the Desktop under OS 9 or below, add
\samp{StartupDriveName:Desktop Folder} on a new line.

\subsection{Default startup options \label{defaults}}

% I'm assuming that there exists some other documentation on the
% rest of the options so I only go over a couple here.

The ``Default startup options...'' button in the
\program{EditPythonPrefs} dialog box gives you many options including
the ability to keep the ``Output'' window open after the script
terminates, and the ability to enter interactive mode after the
termination of the run script. The latter can be very helpful if you
want to examine the objects that were created during your script.

%\section{Nifty Tools}
%There are many other tools included with the MacPython
%distribution. In addition to those discussed here, make 
%sure to check the \file{Mac} directory.

%\subsection{BuildApplet \label{BuildApplet}}
% **NEED INFO HERE**

%\subsection{BuildApplication}
% **NEED INFO HERE**
 
%\section{TKInter on the Mac \label{TKInter}}

%TKinter is installed by default with the MacPython distribution, but
%you may need to add the \file{lib-tk} folder to the Python Path (see
%section \ref{search-path}).  Also, it is important that you do not
%try to launch Tk from within the \program{Python IDE} because the two
%event loops will collide -- always run a script which uses Tkinter
%with the \program{PythonInterpreter} instead -- see section
%\ref{interpreter}.
 
%\section{CGI on the Mac with Python \label{CGI}}
%**NEED INFO HERE**
                       % Using Python on the Macintosh


\chapter{MacPython Modules \label{macpython-modules}}

The following modules are only available on the Macintosh, and are
documented here:

\localmoduletable

\chapter{MACINTOSH ONLY}

The modules in this chapter are available on the Apple Macintosh only.

\section{Built-in module \sectcode{mac}}

\bimodindex{mac}
This module provides a subset of the operating system dependent
functionality provided by the optional built-in module \code{posix}.
It is best accessed through the more portable standard module
\code{os}.

The following functions are available in this module:
\code{chdir},
\code{getcwd},
\code{listdir},
\code{mkdir},
\code{rename},
\code{rmdir},
\code{stat},
\code{sync},
\code{unlink},
as well as the exception \code{error}.

\section{Standard module \sectcode{macpath}}

\stmodindex{macpath}
This module provides a subset of the pathname manipulation functions
available from the optional standard module \code{posixpath}.  It is
best accessed through the more portable standard module \code{os}, as
\code{os.path}.

The following functions are available in this module:
\code{normcase},
\code{isabs},
\code{join},
\code{split},
\code{isdir},
\code{isfile},
\code{exists}.

\section{Built-in Module \sectcode{ctb}}
\bimodindex{ctb}
\renewcommand{\indexsubitem}{(in module ctb)}

This module provides a partial interface to the Macintosh
Communications Toolbox. Currently, only Connection Manager tools are
supported.  It may not be available in all Mac Python versions.

\begin{datadesc}{error}
The exception raised on errors.
\end{datadesc}

\begin{datadesc}{cmData}
\dataline{cmCntl}
\dataline{cmAttn}
Flags for the \var{channel} argument of the \var{Read} and \var{Write}
methods.
\end{datadesc}

\begin{datadesc}{cmFlagsEOM}
End-of-message flag for \var{Read} and \var{Write}.
\end{datadesc}

\begin{datadesc}{choose*}
Values returned by \var{Choose}.
\end{datadesc}

\begin{datadesc}{cmStatus*}
Bits in the status as returned by \var{Status}.
\end{datadesc}

\begin{funcdesc}{available}{}
Return 1 if the communication toolbox is available, zero otherwise.
\end{funcdesc}

\begin{funcdesc}{CMNew}{name\, sizes}
Create a connection object using the connection tool named
\var{name}. \var{sizes} is a 6-tuple given buffer sizes for data in,
data out, control in, control out, attention in and attention out.
Alternatively, passing \code{None} will result in default buffer sizes.
\end{funcdesc}

\subsection{connection object}
For all connection methods that take a \var{timeout} argument, a value
of \code{-1} is indefinite, meaning that the command runs to completion.

\renewcommand{\indexsubitem}{(connection object attribute)}

\begin{datadesc}{callback}
If this member is set to a value other than \code{None} it should point
to a function accepting a single argument (the connection
object). This will make all connection object methods work
asynchronously, with the callback routine being called upon
completion.

{\em Note:} for reasons beyond my understanding the callback routine
is currently never called. You are advised against using asynchronous
calls for the time being.
\end{datadesc}


\renewcommand{\indexsubitem}{(connection object method)}

\begin{funcdesc}{Open}{timeout}
Open an outgoing connection, waiting at most \var{timeout} seconds for
the connection to be established.
\end{funcdesc}

\begin{funcdesc}{Listen}{timeout}
Wait for an incoming connection. Stop waiting after \var{timeout}
seconds. This call is only meaningful to some tools.
\end{funcdesc}

\begin{funcdesc}{accept}{yesno}
Accept (when \var{yesno} is non-zero) or reject an incoming call after
\var{Listen} returned.
\end{funcdesc}

\begin{funcdesc}{Close}{timeout\, now}
Close a connection. When \var{now} is zero, the close is orderly
(i.e.\ outstanding output is flushed, etc.)\ with a timeout of
\var{timeout} seconds. When \var{now} is non-zero the close is
immediate, discarding output.
\end{funcdesc}

\begin{funcdesc}{Read}{len\, chan\, timeout}
Read \var{len} bytes, or until \var{timeout} seconds have passed, from
the channel \var{chan} (which is one of \var{cmData}, \var{cmCntl} or
\var{cmAttn}). Return a 2-tuple:\ the data read and the end-of-message
flag.
\end{funcdesc}

\begin{funcdesc}{Write}{buf\, chan\, timeout\, eom}
Write \var{buf} to channel \var{chan}, aborting after \var{timeout}
seconds. When \var{eom} has the value \var{cmFlagsEOM} an
end-of-message indicator will be written after the data (if this
concept has a meaning for this communication tool). The method returns
the number of bytes written.
\end{funcdesc}

\begin{funcdesc}{Status}{}
Return connection status as the 2-tuple \code{(\var{sizes},
\var{flags})}. \var{sizes} is a 6-tuple giving the actual buffer sizes used
(see \var{CMNew}), \var{flags} is a set of bits describing the state
of the connection.
\end{funcdesc}

\begin{funcdesc}{GetConfig}{}
Return the configuration string of the communication tool. These
configuration strings are tool-dependent, but usually easily parsed
and modified.
\end{funcdesc}

\begin{funcdesc}{SetConfig}{str}
Set the configuration string for the tool. The strings are parsed
left-to-right, with later values taking precedence. This means
individual configuration parameters can be modified by simply appending
something like \code{'baud 4800'} to the end of the string returned by
\var{GetConfig} and passing that to this method. The method returns
the number of characters actually parsed by the tool before it
encountered an error (or completed successfully).
\end{funcdesc}

\begin{funcdesc}{Choose}{}
Present the user with a dialog to choose a communication tool and
configure it. If there is an outstanding connection some choices (like
selecting a different tool) may cause the connection to be
aborted. The return value (one of the \var{choose*} constants) will
indicate this.
\end{funcdesc}

\begin{funcdesc}{Idle}{}
Give the tool a chance to use the processor. You should call this
method regularly.
\end{funcdesc}

\begin{funcdesc}{Abort}{}
Abort an outstanding asynchronous \var{Open} or \var{Listen}.
\end{funcdesc}

\begin{funcdesc}{Reset}{}
Reset a connection. Exact meaning depends on the tool.
\end{funcdesc}

\begin{funcdesc}{Break}{length}
Send a break. Whether this means anything, what it means and
interpretation of the \var{length} parameter depend on the tool in
use.
\end{funcdesc}

%\section{Built-in module \sectcode{macconsole}}
\bimodindex{macconsole}

\renewcommand{\indexsubitem}{(in module macconsole)}

This module is available on the Macintosh, provided Python has been
built using the Think C compiler. It provides an interface to the
Think console package, with which basic text windows can be created.

\begin{datadesc}{options}
An object allowing you to set various options when creating windows,
see below.
\end{datadesc}

\begin{datadesc}{C_ECHO}
\dataline{C_NOECHO}
\dataline{C_CBREAK}
\dataline{C_RAW}
Options for the \code{setmode} method. \var{C_ECHO} and \var{C_CBREAK}
enable character echo, the other two disable it, \var{C_ECHO} and
\var{C_NOECHO} enable line-oriented input (erase/kill processing,
etc).
\end{datadesc}

\begin{funcdesc}{copen}{}
Open a new console window. Returns a console window object.
\end{funcdesc}

\begin{funcdesc}{fopen}{fp}
Return the console window object corresponding with the given file
object. \var{Fp} should be one of \var{sys.stdin}, \var{sys.stdout} or
\var{sys.stderr}.
\end{funcdesc}

\subsection{macconsole options object}
These options are examined when a window is created:

\renewcommand{\indexsubitem}{(macconsole option)}
\begin{datadesc}{top}
\dataline{left}
The origin of the window.
\end{datadesc}

\begin{datadesc}{nrows}
\dataline{ncols}
The size of the window.
\end{datadesc}

\begin{datadesc}{txFont}
\dataline{txSize}
\dataline{txStyle}
The font, fontsize and fontstyle to be used in the window.
\end{datadesc}

\begin{datadesc}{title}
The title of the window.
\end{datadesc}

\begin{datadesc}{pause_atexit}
If set non-zero, the window will wait for user action before closing
the window.
\end{datadesc}

\subsection{console window object}

\renewcommand{\indexsubitem}{(console window method)}

\begin{datadesc}{file}
The file object corresponding to this console window. If the file is
buffered, you should call \code{file.flush()} between \code{write()}
and \code{read()} calls.
\end{datadesc}

\begin{funcdesc}{setmode}{mode}
Set the input mode of the console to \var{C_ECHO}, etc.
\end{funcdesc}

\begin{funcdesc}{settabs}{n}
Set the tabsize to \var{n} spaces.
\end{funcdesc}

\begin{funcdesc}{cleos}{}
Clear to end-of-screen.
\end{funcdesc}

\begin{funcdesc}{cleol}{}
Clear to end-of-line.
\end{funcdesc}

\begin{funcdesc}{inverse}{onoff}
Enable inverse-video mode: characters with the high bit set are
displayed in inverse video (this disables the upper half of a
non-ascii character set).
\end{funcdesc}

\begin{funcdesc}{gotoxy}{x\, y}
Set the cursor to position \code{(x, y)}.
\end{funcdesc}

\begin{funcdesc}{hide}{}
Hide the window, remembering the contents.
\end{funcdesc}

\begin{funcdesc}{show}{}
Show the window again.
\end{funcdesc}

\begin{funcdesc}{echo2printer}{}
Copy everything written to the window to the printer as well.
\end{funcdesc}


\section{Built-in module \sectcode{macdnr}}
\bimodindex{macdnr}

This module provides an interface to the Macintosh Domain Name
Resolver. It is usually used in conjunction with the \var{mactcp} module, to
map hostnames to IP-addresses.

The \code{macdnr} module defines the following functions:

\renewcommand{\indexsubitem}{(in module macdnr)}

\begin{funcdesc}{Open}{\optional{filename}}
Open the domain name resolver extension. If \var{filename} is given it
should be the pathname of the extension, otherwise a default is
used. Normally, this call is not needed since the other calls will
open the extension automatically.
\end{funcdesc}

\begin{funcdesc}{Close}{}
Close the resolver extension. Again, not needed for normal use.
\end{funcdesc}

\begin{funcdesc}{StrToAddr}{hostname}
Look up the IP address for \var{hostname}. This call returns a dnr
result object of the ``address'' variation.
\end{funcdesc}

\begin{funcdesc}{AddrToName}{addr}
Do a reverse lookup on the 32-bit integer IP-address
\var{addr}. Returns a dnr result object of the ``address'' variation.
\end{funcdesc}

\begin{funcdesc}{AddrToStr}{addr}
Convert the 32-bit integer IP-address \var{addr} to a dotted-decimal
string. Returns the string.
\end{funcdesc}

\begin{funcdesc}{HInfo}{hostname}
Query the nameservers for a \code{HInfo} record for host
\var{hostname}. These records contain hardware and software
information about the machine in question (if they are available in
the first place). Returns a dnr result object of the ``hinfo''
variety.
\end{funcdesc}

\begin{funcdesc}{MXInfo}{domain}
Query the nameservers for a mail exchanger for \var{domain}. This is
the hostname of a host willing to accept SMTP mail for the given
domain. Returns a dnr result object of the ``mx'' variety.
\end{funcdesc}

\subsection{dnr result object}

Since the DNR calls all execute asynchronously you do not get the
results back immedeately. In stead, you get a dnr result object. You
can check this object to see whether the query is complete, and access
its attributes to obtain the information when it is.

Alternatively, you can also reference the result attributes directly,
this will result in an implicit wait for the query to complete.

The \var{rtnCode} and \var{cname} attributes are always available, the
others depend on the type of query (address, hinfo or mx).

\renewcommand{\indexsubitem}{(dnr result object method)}

% Add args, as in {arg1\, arg2 \optional{\, arg3}}
\begin{funcdesc}{wait}{}
Wait for the query to complete.
\end{funcdesc}

% Add args, as in {arg1\, arg2 \optional{\, arg3}}
\begin{funcdesc}{isdone}{}
Return 1 if the query is complete.
\end{funcdesc}

\begin{datadesc}{rtnCode}
The error code returned by the query.
\end{datadesc}

\begin{datadesc}{cname}
The canonical name of the host that was queried.
\end{datadesc}

\begin{datadesc}{ip0}
\dataline{ip1}
\dataline{ip2}
\dataline{ip3}
At most four integer IP addresses for this host. Unused entries are
zero. Valid only for address queries.
\end{datadesc}

\begin{datadesc}{cpuType}
\dataline{osType}
Textual strings giving the machine type an OS name. Valid for hinfo
queries.
\end{datadesc}

\begin{datadesc}{exchange}
The name of a mail-exchanger host. Valid for mx queries.
\end{datadesc}

\begin{datadesc}{preference}
The preference of this mx record. Not too useful, since the Macintosh
will only return a single mx record. Mx queries only.
\end{datadesc}

The simplest way to use the module to convert names to dotted-decimal
strings, without worrying about idle time, etc:
\begin{verbatim}
>>> def gethostname(name):
...     import macdnr
...     dnrr = macdnr.StrToAddr(name)
...     return macdnr.AddrToStr(dnrr.ip0)
\end{verbatim}

\section{\module{macfs} ---
         Various file system services}

\declaremodule{builtin}{macfs}
  \platform{Mac}
\modulesynopsis{Support for FSSpec, the Alias Manager,
                \program{finder} aliases, and the Standard File package.}


This module provides access to Macintosh FSSpec handling, the Alias
Manager, \program{finder} aliases and the Standard File package.
\index{Macintosh Alias Manager}
\index{Alias Manager, Macintosh}
\index{Standard File}

Whenever a function or method expects a \var{file} argument, this
argument can be one of three things:\ (1) a full or partial Macintosh
pathname, (2) an \pytype{FSSpec} object or (3) a 3-tuple \code{(\var{wdRefNum},
\var{parID}, \var{name})} as described in \citetitle{Inside
Macintosh:\ Files}. A description of aliases and the Standard File
package can also be found there.

\begin{funcdesc}{FSSpec}{file}
Create an \pytype{FSSpec} object for the specified file.
\end{funcdesc}

\begin{funcdesc}{RawFSSpec}{data}
Create an \pytype{FSSpec} object given the raw data for the \C{}
structure for the \pytype{FSSpec} as a string.  This is mainly useful
if you have obtained an \pytype{FSSpec} structure over a network.
\end{funcdesc}

\begin{funcdesc}{RawAlias}{data}
Create an \pytype{Alias} object given the raw data for the \C{}
structure for the alias as a string.  This is mainly useful if you
have obtained an \pytype{FSSpec} structure over a network.
\end{funcdesc}

\begin{funcdesc}{FInfo}{}
Create a zero-filled \pytype{FInfo} object.
\end{funcdesc}

\begin{funcdesc}{ResolveAliasFile}{file}
Resolve an alias file. Returns a 3-tuple \code{(\var{fsspec},
\var{isfolder}, \var{aliased})} where \var{fsspec} is the resulting
\pytype{FSSpec} object, \var{isfolder} is true if \var{fsspec} points
to a folder and \var{aliased} is true if the file was an alias in the
first place (otherwise the \pytype{FSSpec} object for the file itself
is returned).
\end{funcdesc}

\begin{funcdesc}{StandardGetFile}{\optional{type, ...}}
Present the user with a standard ``open input file''
dialog. Optionally, you can pass up to four 4-character file types to limit
the files the user can choose from. The function returns an \pytype{FSSpec}
object and a flag indicating that the user completed the dialog
without cancelling.
\end{funcdesc}

\begin{funcdesc}{PromptGetFile}{prompt\optional{, type, ...}}
Similar to \function{StandardGetFile()} but allows you to specify a
prompt.
\end{funcdesc}

\begin{funcdesc}{StandardPutFile}{prompt, \optional{default}}
Present the user with a standard ``open output file''
dialog. \var{prompt} is the prompt string, and the optional
\var{default} argument initializes the output file name. The function
returns an \pytype{FSSpec} object and a flag indicating that the user
completed the dialog without cancelling.
\end{funcdesc}

\begin{funcdesc}{GetDirectory}{\optional{prompt}}
Present the user with a non-standard ``select a directory''
dialog. \var{prompt} is the prompt string, and the optional.
Return an \pytype{FSSpec} object and a success-indicator.
\end{funcdesc}

\begin{funcdesc}{SetFolder}{\optional{fsspec}}
Set the folder that is initially presented to the user when one of
the file selection dialogs is presented. \var{fsspec} should point to
a file in the folder, not the folder itself (the file need not exist,
though). If no argument is passed the folder will be set to the
current directory, i.e. what \function{os.getcwd()} returns.

Note that starting with system 7.5 the user can change Standard File
behaviour with the ``general controls'' controlpanel, thereby making
this call inoperative.
\end{funcdesc}

\begin{funcdesc}{FindFolder}{where, which, create}
Locates one of the ``special'' folders that MacOS knows about, such as
the trash or the Preferences folder. \var{where} is the disk to
search, \var{which} is the 4-character string specifying which folder to
locate. Setting \var{create} causes the folder to be created if it
does not exist. Returns a \code{(\var{vrefnum}, \var{dirid})} tuple.
\end{funcdesc}

\begin{funcdesc}{NewAliasMinimalFromFullPath}{pathname}
Return a minimal \pytype{alias} object that points to the given file, which
must be specified as a full pathname. This is the only way to create an
\pytype{Alias} pointing to a non-existing file.

The constants for \var{where} and \var{which} can be obtained from the
standard module \var{MACFS}.
\end{funcdesc}

\begin{funcdesc}{FindApplication}{creator}
Locate the application with 4-char creator code \var{creator}. The
function returns an \pytype{FSSpec} object pointing to the application.
\end{funcdesc}


\subsection{FSSpec objects \label{fsspec-objects}}

\begin{memberdesc}[FSSpec]{data}
The raw data from the FSSpec object, suitable for passing
to other applications, for instance.
\end{memberdesc}

\begin{methoddesc}[FSSpec]{as_pathname}{}
Return the full pathname of the file described by the \pytype{FSSpec}
object.
\end{methoddesc}

\begin{methoddesc}[FSSpec]{as_tuple}{}
Return the \code{(\var{wdRefNum}, \var{parID}, \var{name})} tuple of
the file described by the \pytype{FSSpec} object.
\end{methoddesc}

\begin{methoddesc}[FSSpec]{NewAlias}{\optional{file}}
Create an Alias object pointing to the file described by this
FSSpec. If the optional \var{file} parameter is present the alias
will be relative to that file, otherwise it will be absolute.
\end{methoddesc}

\begin{methoddesc}[FSSpec]{NewAliasMinimal}{}
Create a minimal alias pointing to this file.
\end{methoddesc}

\begin{methoddesc}[FSSpec]{GetCreatorType}{}
Return the 4-character creator and type of the file.
\end{methoddesc}

\begin{methoddesc}[FSSpec]{SetCreatorType}{creator, type}
Set the 4-character creator and type of the file.
\end{methoddesc}

\begin{methoddesc}[FSSpec]{GetFInfo}{}
Return a \pytype{FInfo} object describing the finder info for the file.
\end{methoddesc}

\begin{methoddesc}[FSSpec]{SetFInfo}{finfo}
Set the finder info for the file to the values given as \var{finfo}
(an \pytype{FInfo} object).
\end{methoddesc}

\begin{methoddesc}[FSSpec]{GetDates}{}
Return a tuple with three floating point values representing the
creation date, modification date and backup date of the file.
\end{methoddesc}

\begin{methoddesc}[FSSpec]{SetDates}{crdate, moddate, backupdate}
Set the creation, modification and backup date of the file. The values
are in the standard floating point format used for times throughout
Python.
\end{methoddesc}


\subsection{Alias Objects \label{alias-objects}}

\begin{memberdesc}[Alias]{data}
The raw data for the Alias record, suitable for storing in a resource
or transmitting to other programs.
\end{memberdesc}

\begin{methoddesc}[Alias]{Resolve}{\optional{file}}
Resolve the alias. If the alias was created as a relative alias you
should pass the file relative to which it is. Return the FSSpec for
the file pointed to and a flag indicating whether the \pytype{Alias} object
itself was modified during the search process. If the file does
not exist but the path leading up to it does exist a valid fsspec
is returned.
\end{methoddesc}

\begin{methoddesc}[Alias]{GetInfo}{num}
An interface to the \C{} routine \cfunction{GetAliasInfo()}.
\end{methoddesc}

\begin{methoddesc}[Alias]{Update}{file, \optional{file2}}
Update the alias to point to the \var{file} given. If \var{file2} is
present a relative alias will be created.
\end{methoddesc}

Note that it is currently not possible to directly manipulate a
resource as an \pytype{Alias} object. Hence, after calling
\method{Update()} or after \method{Resolve()} indicates that the alias
has changed the Python program is responsible for getting the
\member{data} value from the \pytype{Alias} object and modifying the
resource.


\subsection{FInfo Objects \label{finfo-objects}}

See \citetitle{Inside Macintosh: Files} for a complete description of what
the various fields mean.

\begin{memberdesc}[FInfo]{Creator}
The 4-character creator code of the file.
\end{memberdesc}

\begin{memberdesc}[FInfo]{Type}
The 4-character type code of the file.
\end{memberdesc}

\begin{memberdesc}[FInfo]{Flags}
The finder flags for the file as 16-bit integer. The bit values in
\var{Flags} are defined in standard module \module{MACFS}.
\end{memberdesc}

\begin{memberdesc}[FInfo]{Location}
A Point giving the position of the file's icon in its folder.
\end{memberdesc}

\begin{memberdesc}[FInfo]{Fldr}
The folder the file is in (as an integer).
\end{memberdesc}

\section{Standard Module \sectcode{ic}}
\label{module-ic}
\bimodindex{ic}


This module provides access to Macintosh Internet Config package,
which stores preferences for Internet programs such as mail address,
default homepage, etc. Also, Internet Config contains an elaborate set
of mappings from Macintosh creator/type codes to foreign filename
extensions plus information on how to transfer files (binary, ascii,
etc).

There is a low-level companion module
\module{icglue}\refbimodindex{icglue} which provides the basic
Internet Config access functionality.  This low-level module is not
documented, but the docstrings of the routines document the parameters
and the routine names are the same as for the Pascal or \C{} API to
Internet Config, so the standard IC programmers' documentation can be
used if this module is needed.

The \module{ic} module defines the \exception{error} exception and
symbolic names for all error codes Internet Config can produce; see
the source for details.

\begin{excdesc}{error}
Exception raised on errors in the \module{ic} module.
\end{excdesc}


The \module{ic} module defines the following functions:

\begin{funcdesc}{IC}{\optional{signature\optional{, ic}}}
Create an internet config object. The signature is a 4-char creator
code of the current application (default \code{'Pyth'}) which may
influence some of ICs settings. The optional \var{ic} argument is a
low-level \code{icglue.icinstance} created beforehand, this may be
useful if you want to get preferences from a different config file,
etc.
\end{funcdesc}

\begin{funcdesc}{launchurl}{url\optional{, hint}}
\funcline{parseurl}{data\optional{, start\optional{, end\optional{, hint}}}}
\funcline{mapfile}{file}
\funcline{maptypecreator}{type, creator\optional{, filename}}
\funcline{settypecreator}{file}
These functions are ``shortcuts'' to the methods of the same name,
described below.
\end{funcdesc}


\subsection{IC objects}

IC objects have a mapping interface, hence to obtain the mail address
you simply get \code{\var{ic}['MailAddress']}. Assignment also works,
and changes the option in the configuration file.

The module knows about various datatypes, and converts the internal IC
representation to a ``logical'' Python datastructure. Running the
\module{ic} module standalone will run a test program that lists all
keys and values in your IC database, this will have to server as
documentation.

If the module does not know how to represent the data it returns an
instance of the \code{ICOpaqueData} type, with the raw data in its
\var{data} attribute. Objects of this type are also acceptable values
for assignment.

Besides the dictionary interface IC objects have the following methods:

\setindexsubitem{(IC attribute)}

\begin{funcdesc}{launchurl}{url\optional{, hint}}
Parse the given URL, lauch the correct application and pass it the
URL. The optional \var{hint} can be a scheme name such as
\code{'mailto:'}, in which case incomplete URLs are completed with this
scheme.  If \var{hint} is not provided, incomplete URLs are invalid.
\end{funcdesc}

\begin{funcdesc}{parseurl}{data\optional{, start\optional{, end\optional{, hint}}}}
Find an URL somewhere in \var{data} and return start position, end
position and the URL. The optional \var{start} and \var{end} can be
used to limit the search, so for instance if a user clicks in a long
textfield you can pass the whole textfield and the click-position in
\var{start} and this routine will return the whole URL in which the
user clicked.  \var{Hint} is again an optional scheme used to complete
incomplete URLs.
\end{funcdesc}

\begin{funcdesc}{mapfile}{file}
Return the mapping entry for the given \var{file}, which can be passed
as either a filename or an \code{macfs.FSSpec} object, and which need
not exist.

The mapping entry is returned as a tuple \code{(}\var{version},
\var{type}, \var{creator}, \var{postcreator}, \var{flags},
\var{extension}, \var{appname}, \var{postappname}, \var{mimetype},
\var{entryname}\code{)}, where \var{version} is the entry version
number, \var{type} is the 4-char filetype, \var{creator} is the 4-char
creator type, \var{postcreator} is the 4-char creator code of an
optional application to post-process the file after downloading,
\var{flags} are various bits specifying whether to transfer in binary
or ascii and such, \var{extension} is the filename extension for this
file type, \var{appname} is the printable name of the application to
which this file belongs, \var{postappname} is the name of the
postprocessing application, \var{mimetype} is the MIME type of this
file and \var{entryname} is the name of this entry.
\end{funcdesc}

\begin{funcdesc}{maptypecreator}{type, creator\optional{, filename}}
Return the mapping entry for files with given 4-char \var{type} and
\var{creator} codes. The optional \var{filename} may be specified to
further help finding the correct entry (if the creator code is
\code{'????'}, for instance).

The mapping entry is returned in the same format as for \var{mapfile}.
\end{funcdesc}

\begin{funcdesc}{settypecreator}{file}
Given an existing \var{file}, specified either as a filename or as an
\code{macfs.FSSpec} record, set its creator and type correctly based
on its extension.  The finder is told about the change, so the finder
icon will be updated quickly.
\end{funcdesc}

\section{Built-in Module \module{MacOS}}
\label{module-MacOS}
\bimodindex{MacOS}


This module provides access to MacOS specific functionality in the
Python interpreter, such as how the interpreter eventloop functions
and the like. Use with care.

Note the capitalisation of the module name, this is a historical
artifact.

\begin{excdesc}{Error}
This exception is raised on MacOS generated errors, either from
functions in this module or from other mac-specific modules like the
toolbox interfaces. The arguments are the integer error code (the
\cdata{OSErr} value) and a textual description of the error code.
Symbolic names for all known error codes are defined in the standard
module \module{macerrors}\refstmodindex{macerrors}.
\end{excdesc}

\begin{funcdesc}{SetEventHandler}{handler}
In the inner interpreter loop Python will occasionally check for events,
unless disabled with \function{ScheduleParams()}. With this function you
can pass a Python event-handler function that will be called if an event
is available. The event is passed as parameter and the function should return
non-zero if the event has been fully processed, otherwise event processing
continues (by passing the event to the console window package, for instance).

Call \function{SetEventHandler()} without a parameter to clear the
event handler. Setting an event handler while one is already set is an
error.
\end{funcdesc}

\begin{funcdesc}{SchedParams}{\optional{doint\optional{, evtmask\optional{,
                              besocial\optional{, interval\optional{,
                              bgyield}}}}}}
Influence the interpreter inner loop event handling. \var{Interval}
specifies how often (in seconds, floating point) the interpreter
should enter the event processing code. When true, \var{doint} causes
interrupt (command-dot) checking to be done. \var{evtmask} tells the
interpreter to do event processing for events in the mask (redraws,
mouseclicks to switch to other applications, etc). The \var{besocial}
flag gives other processes a chance to run. They are granted minimal
runtime when Python is in the foreground and \var{bgyield} seconds per
\var{interval} when Python runs in the background.

All parameters are optional, and default to the current value. The return
value of this function is a tuple with the old values of these options.
Initial defaults are that all processing is enabled, checking is done every
quarter second and the CPU is given up for a quarter second when in the
background.
\end{funcdesc}

\begin{funcdesc}{HandleEvent}{ev}
Pass the event record \var{ev} back to the Python event loop, or
possibly to the handler for the \code{sys.stdout} window (based on the
compiler used to build Python). This allows Python programs that do
their own event handling to still have some command-period and
window-switching capability.

If you attempt to call this function from an event handler set through
\function{SetEventHandler()} you will get an exception.
\end{funcdesc}

\begin{funcdesc}{GetErrorString}{errno}
Return the textual description of MacOS error code \var{errno}.
\end{funcdesc}

\begin{funcdesc}{splash}{resid}
This function will put a splash window
on-screen, with the contents of the DLOG resource specified by
\var{resid}. Calling with a zero argument will remove the splash
screen. This function is useful if you want an applet to post a splash screen
early in initialization without first having to load numerous
extension modules.
\end{funcdesc}

\begin{funcdesc}{DebugStr}{message \optional{, object}}
Drop to the low-level debugger with message \var{message}. The
optional \var{object} argument is not used, but can easily be
inspected from the debugger.

Note that you should use this function with extreme care: if no
low-level debugger like MacsBug is installed this call will crash your
system. It is intended mainly for developers of Python extension
modules.
\end{funcdesc}

\begin{funcdesc}{openrf}{name \optional{, mode}}
Open the resource fork of a file. Arguments are the same as for the
built-in function \function{open()}. The object returned has file-like
semantics, but it is not a Python file object, so there may be subtle
differences.
\end{funcdesc}

\section{Standard Module \sectcode{macostools}}
\label{module-macostools}
\stmodindex{macostools}

This module contains some convenience routines for file-manipulation
on the Macintosh.

The \code{macostools} module defines the following functions:

\setindexsubitem{(in module macostools)}

\begin{funcdesc}{copy}{src, dst\optional{, createpath, copytimes}}
Copy file \var{src} to \var{dst}. The files can be specified as
pathnames or \code{FSSpec} objects. If \var{createpath} is non-zero
\var{dst} must be a pathname and the folders leading to the
destination are created if necessary.  The method copies data and
resource fork and some finder information (creator, type, flags) and
optionally the creation, modification and backup times (default is to
copy them). Custom icons, comments and icon position are not copied.

If the source is an alias the original to which the alias points is
copied, not the aliasfile.
\end{funcdesc}

\begin{funcdesc}{copytree}{src, dst}
Recursively copy a file tree from \var{src} to \var{dst}, creating
folders as needed. \var{Src} and \var{dst} should be specified as
pathnames.
\end{funcdesc}

\begin{funcdesc}{mkalias}{src, dst}
Create a finder alias \var{dst} pointing to \var{src}. Both may be
specified as pathnames or \var{FSSpec} objects.
\end{funcdesc}

\begin{funcdesc}{touched}{dst}
Tell the finder that some bits of finder-information such as creator
or type for file \var{dst} has changed. The file can be specified by
pathname or fsspec. This call should prod the finder into redrawing the
files icon.
\end{funcdesc}

\begin{datadesc}{BUFSIZ}
The buffer size for \code{copy}, default 1 megabyte.
\end{datadesc}

Note that the process of creating finder aliases is not specified in
the Apple documentation. Hence, aliases created with \code{mkalias}
could conceivably have incompatible behaviour in some cases.

\section{Standard Module \sectcode{findertools}}
\label{module-findertools}
\stmodindex{findertools}

This module contains routines that give Python programs access to some
functionality provided by the finder. They are implemented as wrappers
around the AppleEvent interface to the finder.

All file and folder parameters can be specified either as full
pathnames or as \code{FSSpec} objects.

The \code{findertools} module defines the following functions:

\setindexsubitem{(in module macostools)}

\begin{funcdesc}{launch}{file}
Tell the finder to launch \var{file}. What launching means depends on the file:
applications are started, folders are opened and documents are opened
in the correct application.
\end{funcdesc}

\begin{funcdesc}{Print}{file}
Tell the finder to print a file (again specified by full pathname or
FSSpec). The behaviour is identical to selecting the file and using
the print command in the finder.
\end{funcdesc}

\begin{funcdesc}{copy}{file, destdir}
Tell the finder to copy a file or folder \var{file} to folder
\var{destdir}. The function returns an \code{Alias} object pointing to
the new file.
\end{funcdesc}

\begin{funcdesc}{move}{file, destdir}
Tell the finder to move a file or folder \var{file} to folder
\var{destdir}. The function returns an \code{Alias} object pointing to
the new file.
\end{funcdesc}

\begin{funcdesc}{sleep}{}
Tell the finder to put the mac to sleep, if your machine supports it.
\end{funcdesc}

\begin{funcdesc}{restart}{}
Tell the finder to perform an orderly restart of the machine.
\end{funcdesc}

\begin{funcdesc}{shutdown}{}
Tell the finder to perform an orderly shutdown of the machine.
\end{funcdesc}

\section{Built-in module \sectcode{mactcp}}
\bimodindex{mactcp}
\renewcommand{\indexsubitem}{(in module mactcp)}

This module provides an interface to the Macintosh TCP/IP driver
MacTCP. There is an accompanying module \var{macdnr} which provides an
interface to the name-server (allowing you to translate hostnames to
ip-addresses), a module \var{MACTCP} which has symbolic names for
constants constants used by MacTCP and a wrapper module \var{socket}
which mimics the unix socket interface (as far as possible).

A complete description of the MacTCP interface can be found in the
Apple MacTCP API documentation.

\begin{funcdesc}{MTU}{}
Return the Maximum Transmit Unit (the packet size) of the network
interface.
\end{funcdesc}

\begin{funcdesc}{IPAddr}{}
Return the 32-bit integer IP address of the network interface.
\end{funcdesc}

\begin{funcdesc}{NetMask}{}
Return the 32-bit integer network mask of the interface.
\end{funcdesc}

\begin{funcdesc}{TCPCreate}{size}
Create a TCP Stream object. \var{Size} is the size of the receive
buffer, \code{4096} is suggested by various sources.
\end{funcdesc}

\begin{funcdesc}{UDPCreate}{size, port}
Create a UDP stream object. \var{Size} is the size of the receive
buffer (and, hence, the size of the biggest datagram you can receive
on this port). \var{Port} is the UDP port number you want to receive
datagrams on, a value of zero will make MacTCP select a free port.
\end{funcdesc}

\subsection{TCP stream objects}
\renewcommand{\indexsubitem}{(TCP stream method)}

\begin{datadesc}{asr}
When set to a value different than \var{None} this should point to a
function with two integer parameters: an event code and a detail. This
function will be called upon network-generated events such as urgent
data arrival. In addition, it is called with eventcode
\var{MACTCP.PassiveOpenDone} when a \var{PassiveOpen} completes. This
is a python addition to the MacTCP semantics.
It is safe to do further calls from the asr.
\end{datadesc}

\begin{funcdesc}{PassiveOpen}{port}
Wait for an incoming connection on TCP port \var{port} (zero makes the
system pick a free port). The call returns immedeately, and you should
use \var{wait} to wait for completion. You should not issue any method
calls other than
\var{wait}, \var{isdone} or \var{GetSockName} before the call
completes.
\end{funcdesc}

\begin{funcdesc}{wait}{}
Wait for \var{PassiveOpen} to complete.
\end{funcdesc}

\begin{funcdesc}{isdone}{}
Return 1 if a \var{PassiveOpen} is completed.
\end{funcdesc}

\begin{funcdesc}{GetSockName}{}
Return the TCP address of this side of a connection as a 2-tuple
\code{(host, port)}, both integers.
\end{funcdesc}

\begin{funcdesc}{ActiveOpen}{lport\, host\, rport}
Open an outgoing connection to TCP address \code{(host, rport)}. Use
local port \var{lport} (zero makes the system pick a free port). This
call blocks until the connection is established.
\end{funcdesc}

\begin{funcdesc}{Send}{buf\, push\, urgent}
Send data \var{buf} over the connection. \var{Push} and \var{urgent}
are flags as specified by the TCP standard.
\end{funcdesc}

\begin{funcdesc}{Rcv}{timeout}
Receive data. The call returns when \var{timeout} seconds have passed
or when (according to the MacTCP documentation) ``a reasonable amount
of data has been received''. The return value is a 3-tuple
\code{(data, urgent, mark)}. If urgent data is outstanding \var{Rcv}
will always return that before looking at any normal data. The first
call returning urgent data will have the \var{urgent} flag set, the
last will have the \var{mark} flag set.
\end{funcdesc}

\begin{funcdesc}{Close}{}
Tell MacTCP that no more data will be transmitted on this
connection. The call returnes when all data has been acknowledged by
the receiving side.
\end{funcdesc}

\begin{funcdesc}{Abort}{}
Forcibly close both sides of a connection, ignoring outstanding data.
\end{funcdesc}

\begin{funcdesc}{Status}{}
Return a TCP status object for this stream.
\end{funcdesc}

\subsection{TCP status objects}
This object has no methods, only some members holding information on
the connection. A complete description of all fields in this objects
can be found in the Apple documentation. The most interesting ones are:

\renewcommand{\indexsubitem}{(TCP status method)}
\begin{datadesc}{localHost}
\dataline{localPort}
\dataline{remoteHost}
\dataline{remotePort}
The integer IP-addresses and port numbers of both endpoints of the
connection. 
\end{datadesc}

\begin{datadesc}{sendWindow}
The current window size.
\end{datadesc}

\begin{datadesc}{amtUnackedData}
The number of bytes sent but not yet acknowledged. \code{sendWindow -
amtUnackedData} is what you can pass to \code{Send} without blocking.
\end{datadesc}

\begin{datadesc}{amtUnreadData}
The number of bytes received but not yet read (what you can \var{Recv}
without blocking).
\end{datadesc}



\subsection{UDP stream objects}
Note that, unlike the name suggests, there is nothing stream-like
about UDP.

\renewcommand{\indexsubitem}{(UDP stream method)}

\begin{datadesc}{asr}
The asynchronous service routine to be called on events such as
datagram arrival without outstanding \var{Read} call. The asr has a
single argument, the event code.
\end{datadesc}

\begin{datadesc}{port}
A read-only member giving the port number of this UDP stream.
\end{datadesc}

\begin{funcdesc}{Read}{timeout}
Read a datagram, waiting at most \var{timeout} seconds (-1 is
indefinite). Returns the data.
\end{funcdesc}

\begin{funcdesc}{Write}{host\, port\, buf}
Send \var{buf} as a datagram to IP-address \var{host}, port
\var{port}.
\end{funcdesc}

\section{Built-in Module \module{macspeech}}
\declaremodule{builtin}{macspeech}

\modulesynopsis{Interface to the Macintosh Speech Manager.}



This module provides an interface to the Macintosh Speech Manager,
\index{Macintosh Speech Manager}
\index{Speech Manager, Macintosh}
allowing you to let the Macintosh utter phrases. You need a version of
the Speech Manager extension (version 1 and 2 have been tested) in
your \file{Extensions} folder for this to work. The module does not
provide full access to all features of the Speech Manager yet.  It may
not be available in all Mac Python versions.

\begin{funcdesc}{Available}{}
Test availability of the Speech Manager extension (and, on the
PowerPC, the Speech Manager shared library). Return \code{0} or
\code{1}.
\end{funcdesc}

\begin{funcdesc}{Version}{}
Return the (integer) version number of the Speech Manager.
\end{funcdesc}

\begin{funcdesc}{SpeakString}{str}
Utter the string \var{str} using the default voice,
asynchronously. This aborts any speech that may still be active from
prior \function{SpeakString()} invocations.
\end{funcdesc}

\begin{funcdesc}{Busy}{}
Return the number of speech channels busy, system-wide.
\end{funcdesc}

\begin{funcdesc}{CountVoices}{}
Return the number of different voices available.
\end{funcdesc}

\begin{funcdesc}{GetIndVoice}{num}
Return a \pytype{Voice} object for voice number \var{num}.
\end{funcdesc}

\subsection{Voice Objects}
\label{voice-objects}

Voice objects contain the description of a voice. It is currently not
yet possible to access the parameters of a voice.

\setindexsubitem{(voice object method)}

\begin{methoddesc}[Voice]{GetGender}{}
Return the gender of the voice: \code{0} for male, \code{1} for female
and \code{-1} for neuter.
\end{methoddesc}

\begin{methoddesc}[Voice]{NewChannel}{}
Return a new Speech Channel object using this voice.
\end{methoddesc}

\subsection{Speech Channel Objects}
\label{speech-channel-objects}

A Speech Channel object allows you to speak strings with slightly more
control than \function{SpeakString()}, and allows you to use multiple
speakers at the same time. Please note that channel pitch and rate are
interrelated in some way, so that to make your Macintosh sing you will
have to adjust both.

\begin{methoddesc}[Speech Channel]{SpeakText}{str}
Start uttering the given string.
\end{methoddesc}

\begin{methoddesc}[Speech Channel]{Stop}{}
Stop babbling.
\end{methoddesc}

\begin{methoddesc}[Speech Channel]{GetPitch}{}
Return the current pitch of the channel, as a floating-point number.
\end{methoddesc}

\begin{methoddesc}[Speech Channel]{SetPitch}{pitch}
Set the pitch of the channel.
\end{methoddesc}

\begin{methoddesc}[Speech Channel]{GetRate}{}
Get the speech rate (utterances per minute) of the channel as a
floating point number.
\end{methoddesc}

\begin{methoddesc}[Speech Channel]{SetRate}{rate}
Set the speech rate of the channel.
\end{methoddesc}


\section{Standard Module \sectcode{EasyDialogs}}
\label{module-EasyDialogs}
\stmodindex{EasyDialogs}

The \code{EasyDialogs} module contains some simple dialogs for
the Macintosh, modelled after the \code{stdwin} dialogs with similar
names. All routines have an optional parameter \var{id} with which you
can override the DLOG resource used for the dialog, as long as the
item numbers correspond. See the source for details.

The \code{EasyDialogs} module defines the following functions:

\setindexsubitem{(in module EasyDialogs)}

\begin{funcdesc}{Message}{str}
A modal dialog with the message text \var{str}, which should be at
most 255 characters long, is displayed. Control is returned when the
user clicks ``OK''.
\end{funcdesc}

\begin{funcdesc}{AskString}{prompt\optional{, default}}
Ask the user to input a string value, in a modal dialog. \var{Prompt}
is the promt message, the optional \var{default} arg is the initial
value for the string. All strings can be at most 255 bytes
long. \var{AskString} returns the string entered or \code{None} in
case the user cancelled.
\end{funcdesc}

\begin{funcdesc}{AskYesNoCancel}{question\optional{, default}}
Present a dialog with text \var{question} and three buttons labelled
``yes'', ``no'' and ``cancel''. Return \code{1} for yes, \code{0} for
no and \code{-1} for cancel. The default return value chosen by
hitting return is \code{0}. This can be changed with the optional
\var{default} argument.
\end{funcdesc}

\begin{funcdesc}{ProgressBar}{\optional{label, maxval}}
Display a modeless progress dialog with a thermometer bar. \var{Label}
is the textstring displayed (default ``Working...''), \var{maxval} is
the value at which progress is complete (default 100). The returned
object has one method, \code{set(value)}, which sets the value of the
progress bar. The bar remains visible until the object returned is
discarded.

The progress bar has a ``cancel'' button, but it is currently
non-functional.
\end{funcdesc}

Note that \code{EasyDialogs} does not currently use the notification
manager. This means that displaying dialogs while the program is in
the background will lead to unexpected results and possibly
crashes. Also, all dialogs are modeless and hence expect to be at the
top of the stacking order. This is true when the dialogs are created,
but windows that pop-up later (like a console window) may also result
in crashes.

\section{\module{FrameWork} ---
         Interactive application framework}

\declaremodule{standard}{FrameWork}
  \platform{Mac}
\modulesynopsis{Interactive application framework.}


The \module{FrameWork} module contains classes that together provide a
framework for an interactive Macintosh application. The programmer
builds an application by creating subclasses that override various
methods of the bases classes, thereby implementing the functionality
wanted. Overriding functionality can often be done on various
different levels, i.e. to handle clicks in a single dialog window in a
non-standard way it is not necessary to override the complete event
handling.

The \module{FrameWork} is still very much work-in-progress, and the
documentation describes only the most important functionality, and not
in the most logical manner at that. Examine the source or the examples
for more details.  The following are some comments posted on the
MacPython newsgroup about the strengths and limitations of
\module{FrameWork}:

\begin{quotation}
The strong point of \module{FrameWork} is that it allows you to break
into the control-flow at many different places. \refmodule{W}, for
instance, uses a different way to enable/disable menus and that plugs
right in leaving the rest intact.  The weak points of
\module{FrameWork} are that it has no abstract command interface (but
that shouldn't be difficult), that it's dialog support is minimal and
that it's control/toolbar support is non-existent.
\end{quotation}


The \module{FrameWork} module defines the following functions:


\begin{funcdesc}{Application}{}
An object representing the complete application. See below for a
description of the methods. The default \method{__init__()} routine
creates an empty window dictionary and a menu bar with an apple menu.
\end{funcdesc}

\begin{funcdesc}{MenuBar}{}
An object representing the menubar. This object is usually not created
by the user.
\end{funcdesc}

\begin{funcdesc}{Menu}{bar, title\optional{, after}}
An object representing a menu. Upon creation you pass the
\code{MenuBar} the menu appears in, the \var{title} string and a
position (1-based) \var{after} where the menu should appear (default:
at the end).
\end{funcdesc}

\begin{funcdesc}{MenuItem}{menu, title\optional{, shortcut, callback}}
Create a menu item object. The arguments are the menu to create, the
item title string and optionally the keyboard shortcut
and a callback routine. The callback is called with the arguments
menu-id, item number within menu (1-based), current front window and
the event record.

Instead of a callable object the callback can also be a string. In
this case menu selection causes the lookup of a method in the topmost
window and the application. The method name is the callback string
with \code{'domenu_'} prepended.

Calling the \code{MenuBar} \method{fixmenudimstate()} method sets the
correct dimming for all menu items based on the current front window.
\end{funcdesc}

\begin{funcdesc}{Separator}{menu}
Add a separator to the end of a menu.
\end{funcdesc}

\begin{funcdesc}{SubMenu}{menu, label}
Create a submenu named \var{label} under menu \var{menu}. The menu
object is returned.
\end{funcdesc}

\begin{funcdesc}{Window}{parent}
Creates a (modeless) window. \var{Parent} is the application object to
which the window belongs. The window is not displayed until later.
\end{funcdesc}

\begin{funcdesc}{DialogWindow}{parent}
Creates a modeless dialog window.
\end{funcdesc}

\begin{funcdesc}{windowbounds}{width, height}
Return a \code{(\var{left}, \var{top}, \var{right}, \var{bottom})}
tuple suitable for creation of a window of given width and height. The
window will be staggered with respect to previous windows, and an
attempt is made to keep the whole window on-screen. However, the window will
however always be the exact size given, so parts may be offscreen.
\end{funcdesc}

\begin{funcdesc}{setwatchcursor}{}
Set the mouse cursor to a watch.
\end{funcdesc}

\begin{funcdesc}{setarrowcursor}{}
Set the mouse cursor to an arrow.
\end{funcdesc}


\subsection{Application Objects \label{application-objects}}

Application objects have the following methods, among others:


\begin{methoddesc}[Application]{makeusermenus}{}
Override this method if you need menus in your application. Append the
menus to the attribute \member{menubar}.
\end{methoddesc}

\begin{methoddesc}[Application]{getabouttext}{}
Override this method to return a text string describing your
application.  Alternatively, override the \method{do_about()} method
for more elaborate ``about'' messages.
\end{methoddesc}

\begin{methoddesc}[Application]{mainloop}{\optional{mask\optional{, wait}}}
This routine is the main event loop, call it to set your application
rolling. \var{Mask} is the mask of events you want to handle,
\var{wait} is the number of ticks you want to leave to other
concurrent application (default 0, which is probably not a good
idea). While raising \var{self} to exit the mainloop is still
supported it is not recommended: call \code{self._quit()} instead.

The event loop is split into many small parts, each of which can be
overridden. The default methods take care of dispatching events to
windows and dialogs, handling drags and resizes, Apple Events, events
for non-FrameWork windows, etc.

In general, all event handlers should return \code{1} if the event is fully
handled and \code{0} otherwise (because the front window was not a FrameWork
window, for instance). This is needed so that update events and such
can be passed on to other windows like the Sioux console window.
Calling \function{MacOS.HandleEvent()} is not allowed within
\var{our_dispatch} or its callees, since this may result in an
infinite loop if the code is called through the Python inner-loop
event handler.
\end{methoddesc}

\begin{methoddesc}[Application]{asyncevents}{onoff}
Call this method with a nonzero parameter to enable
asynchronous event handling. This will tell the inner interpreter loop
to call the application event handler \var{async_dispatch} whenever events
are available. This will cause FrameWork window updates and the user
interface to remain working during long computations, but will slow the
interpreter down and may cause surprising results in non-reentrant code
(such as FrameWork itself). By default \var{async_dispatch} will immedeately
call \var{our_dispatch} but you may override this to handle only certain
events asynchronously. Events you do not handle will be passed to Sioux
and such.

The old on/off value is returned.
\end{methoddesc}

\begin{methoddesc}[Application]{_quit}{}
Terminate the running \method{mainloop()} call at the next convenient
moment.
\end{methoddesc}

\begin{methoddesc}[Application]{do_char}{c, event}
The user typed character \var{c}. The complete details of the event
can be found in the \var{event} structure. This method can also be
provided in a \code{Window} object, which overrides the
application-wide handler if the window is frontmost.
\end{methoddesc}

\begin{methoddesc}[Application]{do_dialogevent}{event}
Called early in the event loop to handle modeless dialog events. The
default method simply dispatches the event to the relevant dialog (not
through the \code{DialogWindow} object involved). Override if you
need special handling of dialog events (keyboard shortcuts, etc).
\end{methoddesc}

\begin{methoddesc}[Application]{idle}{event}
Called by the main event loop when no events are available. The
null-event is passed (so you can look at mouse position, etc).
\end{methoddesc}


\subsection{Window Objects \label{window-objects}}

Window objects have the following methods, among others:

\setindexsubitem{(Window method)}

\begin{methoddesc}[Window]{open}{}
Override this method to open a window. Store the MacOS window-id in
\member{self.wid} and call the \method{do_postopen()} method to
register the window with the parent application.
\end{methoddesc}

\begin{methoddesc}[Window]{close}{}
Override this method to do any special processing on window
close. Call the \method{do_postclose()} method to cleanup the parent
state.
\end{methoddesc}

\begin{methoddesc}[Window]{do_postresize}{width, height, macoswindowid}
Called after the window is resized. Override if more needs to be done
than calling \code{InvalRect}.
\end{methoddesc}

\begin{methoddesc}[Window]{do_contentclick}{local, modifiers, event}
The user clicked in the content part of a window. The arguments are
the coordinates (window-relative), the key modifiers and the raw
event.
\end{methoddesc}

\begin{methoddesc}[Window]{do_update}{macoswindowid, event}
An update event for the window was received. Redraw the window.
\end{methoddesc}

\begin{methoddesc}{do_activate}{activate, event}
The window was activated (\code{\var{activate} == 1}) or deactivated
(\code{\var{activate} == 0}). Handle things like focus highlighting,
etc.
\end{methoddesc}


\subsection{ControlsWindow Object \label{controlswindow-object}}

ControlsWindow objects have the following methods besides those of
\code{Window} objects:


\begin{methoddesc}[ControlsWindow]{do_controlhit}{window, control,
                                                  pcode, event}
Part \var{pcode} of control \var{control} was hit by the
user. Tracking and such has already been taken care of.
\end{methoddesc}


\subsection{ScrolledWindow Object \label{scrolledwindow-object}}

ScrolledWindow objects are ControlsWindow objects with the following
extra methods:


\begin{methoddesc}[ScrolledWindow]{scrollbars}{\optional{wantx\optional{,
                                               wanty}}}
Create (or destroy) horizontal and vertical scrollbars. The arguments
specify which you want (default: both). The scrollbars always have
minimum \code{0} and maximum \code{32767}.
\end{methoddesc}

\begin{methoddesc}[ScrolledWindow]{getscrollbarvalues}{}
You must supply this method. It should return a tuple \code{(\var{x},
\var{y})} giving the current position of the scrollbars (between
\code{0} and \code{32767}). You can return \code{None} for either to
indicate the whole document is visible in that direction.
\end{methoddesc}

\begin{methoddesc}[ScrolledWindow]{updatescrollbars}{}
Call this method when the document has changed. It will call
\method{getscrollbarvalues()} and update the scrollbars.
\end{methoddesc}

\begin{methoddesc}[ScrolledWindow]{scrollbar_callback}{which, what, value}
Supplied by you and called after user interaction. \var{which} will
be \code{'x'} or \code{'y'}, \var{what} will be \code{'-'},
\code{'--'}, \code{'set'}, \code{'++'} or \code{'+'}. For
\code{'set'}, \var{value} will contain the new scrollbar position.
\end{methoddesc}

\begin{methoddesc}[ScrolledWindow]{scalebarvalues}{absmin, absmax,
                                                   curmin, curmax}
Auxiliary method to help you calculate values to return from
\method{getscrollbarvalues()}. You pass document minimum and maximum value
and topmost (leftmost) and bottommost (rightmost) visible values and
it returns the correct number or \code{None}.
\end{methoddesc}

\begin{methoddesc}[ScrolledWindow]{do_activate}{onoff, event}
Takes care of dimming/highlighting scrollbars when a window becomes
frontmost. If you override this method, call this one at the end of
your method.
\end{methoddesc}

\begin{methoddesc}[ScrolledWindow]{do_postresize}{width, height, window}
Moves scrollbars to the correct position. Call this method initially
if you override it.
\end{methoddesc}

\begin{methoddesc}[ScrolledWindow]{do_controlhit}{window, control,
                                                  pcode, event}
Handles scrollbar interaction. If you override it call this method
first, a nonzero return value indicates the hit was in the scrollbars
and has been handled.
\end{methoddesc}


\subsection{DialogWindow Objects \label{dialogwindow-objects}}

DialogWindow objects have the following methods besides those of
\code{Window} objects:


\begin{methoddesc}[DialogWindow]{open}{resid}
Create the dialog window, from the DLOG resource with id
\var{resid}. The dialog object is stored in \member{self.wid}.
\end{methoddesc}

\begin{methoddesc}[DialogWindow]{do_itemhit}{item, event}
Item number \var{item} was hit. You are responsible for redrawing
toggle buttons, etc.
\end{methoddesc}

\section{Standard Module \sectcode{MiniAEFrame}}
\stmodindex{MiniAEFrame}
\label{module-MiniAEFrame}

The module \var{MiniAEFrame} provides a framework for an application
that can function as an OSA server, i.e. receive and process
AppleEvents. It can be used in conjunction with \var{FrameWork} or
standalone.

This module is temporary, it will eventually be replaced by a module
that handles argument names better and possibly automates making your
application scriptable.

The \var{MiniAEFrame} module defines the following classes:

\setindexsubitem{(in module MiniAEFrame)}

\begin{funcdesc}{AEServer}{}
A class that handles AppleEvent dispatch. Your application should
subclass this class together with either
\code{MiniAEFrame.MiniApplication} or
\code{FrameWork.Application}. Your \code{__init__} method should call
the \code{__init__} method for both classes.
\end{funcdesc}

\begin{funcdesc}{MiniApplication}{}
A class that is more or less compatible with
\code{FrameWork.Application} but with less functionality. Its
eventloop supports the apple menu, command-dot and AppleEvents, other
events are passed on to the Python interpreter and/or Sioux.
Useful if your application wants to use \code{AEServer} but does not
provide its own windows, etc.
\end{funcdesc}

\subsection{AEServer Objects}

\setindexsubitem{(AEServer method)}

\begin{funcdesc}{installaehandler}{classe\, type\, callback}
Installs an AppleEvent handler. \code{Classe} and \code{type} are the
four-char OSA Class and Type designators, \code{'****'} wildcards are
allowed. When a matching AppleEvent is received the parameters are
decoded and your callback is invoked.
\end{funcdesc}

\begin{funcdesc}{callback}{_object\, **kwargs}
Your callback is called with the OSA Direct Object as first positional
parameter. The other parameters are passed as keyword arguments, with
the 4-char designator as name. Three extra keyword parameters are
passed: \code{_class} and \code{_type} are the Class and Type
designators and \code{_attributes} is a dictionary with the AppleEvent
attributes.

The return value of your method is packed with
\code{aetools.packevent} and sent as reply.
\end{funcdesc}

Note that there are some serious problems with the current
design. AppleEvents which have non-identifier 4-char designators for
arguments are not implementable, and it is not possible to return an
error to the originator. This will be addressed in a future release.

\section{\module{aepack} ---
         Conversion between Python variables and AppleEvent data containers}

\declaremodule{standard}{aepack}
  \platform{Mac}
%\moduleauthor{Jack Jansen?}{email}
\modulesynopsis{Conversion between Python variables and AppleEvent
                data containers.}
\sectionauthor{Vincent Marchetti}{vincem@en.com}


The \module{aepack} module defines functions for converting (packing)
Python variables to AppleEvent descriptors and back (unpacking).
Within Python the AppleEvent descriptor is handled by Python objects
of built-in type \class{AEDesc}, defined in module \refmodule{AE}.

The \module{aepack} module defines the following functions:


\begin{funcdesc}{pack}{x\optional{, forcetype}}
Returns an \class{AEDesc} object  containing a conversion of Python
value x. If \var{forcetype} is provided it specifies the descriptor
type of the result. Otherwise, a default mapping of Python types to
Apple Event descriptor types is used, as follows:

\begin{tableii}{l|l}{textrm}{Python type}{descriptor type}
  \lineii{\class{FSSpec}}{typeFSS}
  \lineii{\class{FSRef}}{typeFSRef}
  \lineii{\class{Alias}}{typeAlias}
  \lineii{integer}{typeLong (32 bit integer)}
  \lineii{float}{typeFloat (64 bit floating point)}
  \lineii{string}{typeText}
  \lineii{unicode}{typeUnicodeText}
  \lineii{list}{typeAEList}
  \lineii{dictionary}{typeAERecord}
  \lineii{instance}{\emph{see below}}
\end{tableii}  
 
If \var{x} is a Python instance then this function attempts to call an
\method{__aepack__()} method.  This method should return an
\class{AE.AEDesc} object.

If the conversion \var{x} is not defined above, this function returns
the Python string representation of a value (the repr() function)
encoded as a text descriptor.
\end{funcdesc}

\begin{funcdesc}{unpack}{x\optional{, formodulename}}
  \var{x} must be an object of type \class{AEDesc}. This function
  returns a Python object representation of the data in the Apple
  Event descriptor \var{x}. Simple AppleEvent data types (integer,
  text, float) are returned as their obvious Python counterparts.
  Apple Event lists are returned as Python lists, and the list
  elements are recursively unpacked.  Object references
  (ex. \code{line 3 of document 1}) are returned as instances of
  \class{aetypes.ObjectSpecifier}, unless \code{formodulename}
  is specified.  AppleEvent descriptors with
  descriptor type typeFSS are returned as \class{FSSpec}
  objects.  AppleEvent record descriptors are returned as Python
  dictionaries, with 4-character string keys and elements recursively
  unpacked.
  
  The optional \code{formodulename} argument is used by the stub packages
  generated by \module{gensuitemodule}, and ensures that the OSA classes
  for object specifiers are looked up in the correct module. This ensures
  that if, say, the Finder returns an object specifier for a window
  you get an instance of \code{Finder.Window} and not a generic
  \code{aetypes.Window}. The former knows about all the properties
  and elements a window has in the Finder, while the latter knows
  no such things.
\end{funcdesc}


\begin{seealso}
  \seemodule{Carbon.AE}{Built-in access to Apple Event Manager routines.}
  \seemodule{aetypes}{Python definitions of codes for Apple Event
                      descriptor types.}
  \seetitle[http://developer.apple.com/techpubs/mac/IAC/IAC-2.html]{
            Inside Macintosh: Interapplication
            Communication}{Information about inter-process
            communications on the Macintosh.}
\end{seealso}

\section{\module{aetypes} ---
         AppleEvent objects}

\declaremodule{standard}{aetypes}
  \platform{Mac}
%\moduleauthor{Jack Jansen?}{email}
\modulesynopsis{Python representation of the Apple Event Object Model.}
\sectionauthor{Vincent Marchetti}{vincem@en.com}


The \module{aetypes} defines classes used to represent Apple Event data
descriptors and Apple Event object specifiers.

Apple Event data is contained in descriptors, and these descriptors
are typed. For many descriptors the Python representation is simply the
corresponding Python type: \code{typeText} in OSA is a Python string,
\code{typeFloat} is a float, etc. For OSA types that have no direct
Python counterpart this module declares classes. Packing and unpacking
instances of these classes is handled automatically by \module{aepack}.

An object specifier is essentially an address of an object implemented
in a Apple Event server. An Apple Event specifier is used as the direct
object for an Apple Event or as the argument of an optional parameter.
The \module{aetypes} module contains the base classes for OSA classes
and properties, which are used by the packages generated by
\module{gensuitemodule} to populate the classes and properties in a
given suite.

For reasons of backward compatibility, and for cases where you need to
script an application for which you have not generated the stub package
this module also contains object specifiers for a number of common OSA
classes such as \code{Document}, \code{Window}, \code{Character}, etc.



The \module{AEObjects} module defines the following classes to represent
Apple Event descriptor data:

\begin{classdesc}{Unknown}{type, data}
The representation of OSA descriptor data for which the \module{aepack}
and \module{aetypes} modules have no support, i.e. anything that is not
represented by the other classes here and that is not equivalent to a
simple Python value.
\end{classdesc}

\begin{classdesc}{Enum}{enum}
An enumeration value with the given 4-character string value.
\end{classdesc}

\begin{classdesc}{InsertionLoc}{of, pos}
Position \code{pos} in object \code{of}.
\end{classdesc}

\begin{classdesc}{Boolean}{bool}
A boolean.
\end{classdesc}

\begin{classdesc}{StyledText}{style, text}
Text with style information (font, face, etc) included.
\end{classdesc}

\begin{classdesc}{AEText}{script, style, text}
Text with script system and style information included.
\end{classdesc}

\begin{classdesc}{IntlText}{script, language, text}
Text with script system and language information included.
\end{classdesc}

\begin{classdesc}{IntlWritingCode}{script, language}
Script system and language information.
\end{classdesc}

\begin{classdesc}{QDPoint}{v, h}
A quickdraw point.
\end{classdesc}

\begin{classdesc}{QDRectangle}{v0, h0, v1, h1}
A quickdraw rectangle.
\end{classdesc}

\begin{classdesc}{RGBColor}{r, g, b}
A color.
\end{classdesc}

\begin{classdesc}{Type}{type}
An OSA type value with the given 4-character name.
\end{classdesc}

\begin{classdesc}{Keyword}{name}
An OSA keyword with the given 4-character name.
\end{classdesc}

\begin{classdesc}{Range}{start, stop}
A range.
\end{classdesc}

\begin{classdesc}{Ordinal}{abso}
Non-numeric absolute positions, such as \code{"firs"}, first, or \code{"midd"},
middle.
\end{classdesc}

\begin{classdesc}{Logical}{logc, term}
The logical expression of applying operator \code{logc} to
\code{term}.
\end{classdesc}

\begin{classdesc}{Comparison}{obj1, relo, obj2}
The comparison \code{relo} of \code{obj1} to \code{obj2}.
\end{classdesc}

The following classes are used as base classes by the generated stub
packages to represent AppleScript classes and properties in Python:

\begin{classdesc}{ComponentItem}{which\optional{, fr}}
Abstract baseclass for an OSA class. The subclass should set the class
attribute \code{want} to the 4-character OSA class code. Instances of
subclasses of this class are equivalent to AppleScript Object
Specifiers. Upon instantiation you should pass a selector in
\code{which}, and optionally a parent object in \code{fr}.
\end{classdesc}

\begin{classdesc}{NProperty}{fr}
Abstract baseclass for an OSA property. The subclass should set the class
attributes \code{want} and \code{which} to designate which property we
are talking about. Instances of subclasses of this class are Object
Specifiers.
\end{classdesc}

\begin{classdesc}{ObjectSpecifier}{want, form, seld\optional{, fr}}
Base class of \code{ComponentItem} and \code{NProperty}, a general
OSA Object Specifier. See the Apple Open Scripting Architecture
documentation for the parameters. Note that this class is not abstract.
\end{classdesc}



\chapter{MacOS Toolbox Modules \label{toolbox}}

There are a set of modules that provide interfaces to various MacOS
toolboxes.  If applicable the module will define a number of Python
objects for the various structures declared by the toolbox, and
operations will be implemented as methods of the object.  Other
operations will be implemented as functions in the module.  Not all
operations possible in C will also be possible in Python (callbacks
are often a problem), and parameters will occasionally be different in
Python (input and output buffers, especially).  All methods and
functions have a \member{__doc__} string describing their arguments
and return values, and for additional description you are referred to
\citetitle[http://developer.apple.com/documentation/macos8/mac8.html]{Inside
Macintosh} or similar works.

These modules all live in a package called \module{Carbon}. Despite that name
they are not all part of the Carbon framework: CF is really in the CoreFoundation
framework and Qt is in the QuickTime framework.
The normal use pattern is

\begin{verbatim}
from Carbon import AE
\end{verbatim}

\strong{Warning!}  These modules are not yet documented.  If you
wish to contribute documentation of any of these modules, please get
in touch with \email{docs@python.org}.

\localmoduletable


%\section{Argument Handling for Toolbox Modules}


\section{\module{Carbon.AE} --- Apple Events}
\declaremodule{standard}{Carbon.AE}
  \platform{Mac}
\modulesynopsis{Interface to the Apple Events toolbox.}

\section{\module{Carbon.AH} --- Apple Help}
\declaremodule{standard}{Carbon.AH}
  \platform{Mac}
\modulesynopsis{Interface to the Apple Help manager.}


\section{\module{Carbon.App} --- Appearance Manager}
\declaremodule{standard}{Carbon.App}
  \platform{Mac}
\modulesynopsis{Interface to the Appearance Manager.}


\section{\module{Carbon.CF} --- Core Foundation}
\declaremodule{standard}{Carbon.CF}
  \platform{Mac}
\modulesynopsis{Interface to the Core Foundation.}

The
\code{CFBase}, \code{CFArray}, \code{CFData}, \code{CFDictionary},
\code{CFString} and \code{CFURL} objects are supported, some
only partially.

\section{\module{Carbon.CG} --- Core Graphics}
\declaremodule{standard}{Carbon.CG}
  \platform{Mac}
\modulesynopsis{Interface to the Component Manager.}

\section{\module{Carbon.CarbonEvt} --- Carbon Event Manager}
\declaremodule{standard}{Carbon.CaronEvt}
  \platform{Mac}
\modulesynopsis{Interface to the Carbon Event Manager.}

\section{\module{Carbon.Cm} --- Component Manager}
\declaremodule{standard}{Carbon.Cm}
  \platform{Mac}
\modulesynopsis{Interface to the Component Manager.}


\section{\module{Carbon.Ctl} --- Control Manager}
\declaremodule{standard}{Carbon.Ctl}
  \platform{Mac}
\modulesynopsis{Interface to the Control Manager.}


\section{\module{Carbon.Dlg} --- Dialog Manager}
\declaremodule{standard}{Carbon.Dlg}
  \platform{Mac}
\modulesynopsis{Interface to the Dialog Manager.}


\section{\module{Carbon.Evt} --- Event Manager}
\declaremodule{standard}{Carbon.Evt}
  \platform{Mac}
\modulesynopsis{Interface to the classic Event Manager.}


\section{\module{Carbon.Fm} --- Font Manager}
\declaremodule{standard}{Carbon.Fm}
  \platform{Mac}
\modulesynopsis{Interface to the Font Manager.}

\section{\module{Carbon.Folder} --- Folder Manager}
\declaremodule{standard}{Carbon.Folder}
  \platform{Mac}
\modulesynopsis{Interface to the Folder Manager.}


\section{\module{Carbon.Help} --- Help Manager}
\declaremodule{standard}{Carbon.Help}
  \platform{Mac}
\modulesynopsis{Interface to the Carbon Help Manager.}

\section{\module{Carbon.List} --- List Manager}
\declaremodule{standard}{Carbon.List}
  \platform{Mac}
\modulesynopsis{Interface to the List Manager.}


\section{\module{Carbon.Menu} --- Menu Manager}
\declaremodule{standard}{Carbon.Menu}
  \platform{Mac}
\modulesynopsis{Interface to the Menu Manager.}


\section{\module{Carbon.Mlte} --- MultiLingual Text Editor}
\declaremodule{standard}{Carbon.Mlte}
  \platform{Mac}
\modulesynopsis{Interface to the MultiLingual Text Editor.}


\section{\module{Carbon.Qd} --- QuickDraw}
\declaremodule{builtin}{Carbon.Qd}
  \platform{Mac}
\modulesynopsis{Interface to the QuickDraw toolbox.}


\section{\module{Carbon.Qdoffs} --- QuickDraw Offscreen}
\declaremodule{builtin}{Carbon.Qdoffs}
  \platform{Mac}
\modulesynopsis{Interface to the QuickDraw Offscreen APIs.}


\section{\module{Carbon.Qt} --- QuickTime}
\declaremodule{standard}{Carbon.Qt}
  \platform{Mac}
\modulesynopsis{Interface to the QuickTime toolbox.}


\section{\module{Carbon.Res} --- Resource Manager and Handles}
\declaremodule{standard}{Carbon.Res}
  \platform{Mac}
\modulesynopsis{Interface to the Resource Manager and Handles.}

\section{\module{Carbon.Scrap} --- Scrap Manager}
\declaremodule{standard}{Carbon.Scrap}
  \platform{Mac}
\modulesynopsis{Interface to the Carbon Scrap Manager.}

\section{\module{Carbon.Snd} --- Sound Manager}
\declaremodule{standard}{Carbon.Snd}
  \platform{Mac}
\modulesynopsis{Interface to the Sound Manager.}


\section{\module{Carbon.TE} --- TextEdit}
\declaremodule{standard}{Carbon.TE}
  \platform{Mac}
\modulesynopsis{Interface to TextEdit.}


\section{\module{Carbon.Win} --- Window Manager}
\declaremodule{standard}{Carbon.Win}
  \platform{Mac}
\modulesynopsis{Interface to the Window Manager.}
                         % MacOS Toolbox Modules
\section{\module{ColorPicker} ---
         Color selection dialog}

\declaremodule{extension}{ColorPicker}
  \platform{Mac}
\modulesynopsis{}
\moduleauthor{Just van Rossum}{just@letterror.com}
\sectionauthor{Fred L. Drake, Jr.}{fdrake@acm.org}


The \module{ColorPicker} module provides access to the standard color
picker dialog.


\begin{funcdesc}{GetColor}{prompt, rgb}
  Show a standard color selection dialog and allow the user to select
  a color.  The user is given instruction by the \var{prompt} string,
  and the default color is set to \var{rgb}.  \var{rgb} must be a
  tuple giving the red, green, and blue components of the color.
  \function{GetColor()} returns a tuple giving the user's selected
  color and a flag indicating whether they accepted the selection of
  cancelled.
\end{funcdesc}


\chapter{Undocumented Modules \label{undocumented-modules}}


The modules in this chapter are poorly documented (if at all).  If you
wish to contribute documentation of any of these modules, please get in
touch with
\ulink{\email{docs@python.org}}{mailto:docs@python.org}.

\localmoduletable


\section{\module{applesingle} --- AppleSingle decoder}
\declaremodule{standard}{applesingle}
  \platform{Mac}
\modulesynopsis{Rudimentary decoder for AppleSingle format files.}


\section{\module{buildtools} --- Helper module for BuildApplet and Friends}
\declaremodule{standard}{buildtools}
  \platform{Mac}
\modulesynopsis{Helper module for BuildApplet, BuildApplication and
                macfreeze.}

\deprecated{2.4}{}

\section{\module{cfmfile} --- Code Fragment Resource module}
\declaremodule{standard}{cfmfile}
  \platform{Mac}
\modulesynopsis{Code Fragment Resource module.}

\module{cfmfile} is a module that understands Code Fragments and the
accompanying ``cfrg'' resources. It can parse them and merge them, and is
used by BuildApplication to combine all plugin modules to a single
executable.

\deprecated{2.4}{}

\section{\module{icopen} --- Internet Config replacement for \method{open()}}
\declaremodule{standard}{icopen}
  \platform{Mac}
\modulesynopsis{Internet Config replacement for \method{open()}.}

Importing \module{icopen} will replace the builtin \method{open()}
with a version that uses Internet Config to set file type and creator
for new files.


\section{\module{macerrors} --- Mac OS Errors}
\declaremodule{standard}{macerrors}
  \platform{Mac}
\modulesynopsis{Constant definitions for many Mac OS error codes.}

\module{macerrors} contains constant definitions for many Mac OS error
codes.


\section{\module{macresource} --- Locate script resources}
\declaremodule{standard}{macresource}
  \platform{Mac}
\modulesynopsis{Locate script resources.}

\module{macresource} helps scripts finding their resources, such as
dialogs and menus, without requiring special case code for when the
script is run under MacPython, as a MacPython applet or under OSX Python.

\section{\module{Nav} --- NavServices calls}
\declaremodule{standard}{Nav}
  \platform{Mac}
\modulesynopsis{Interface to Navigation Services.}

A low-level interface to Navigation Services. 

\section{\module{PixMapWrapper} --- Wrapper for PixMap objects}
\declaremodule{standard}{PixMapWrapper}
  \platform{Mac}
\modulesynopsis{Wrapper for PixMap objects.}

\module{PixMapWrapper} wraps a PixMap object with a Python object that
allows access to the fields by name. It also has methods to convert
to and from \module{PIL} images.

\section{\module{videoreader} --- Read QuickTime movies}
\declaremodule{standard}{videoreader}
  \platform{Mac}
\modulesynopsis{Read QuickTime movies frame by frame for further processing.}

\module{videoreader} reads and decodes QuickTime movies and passes
a stream of images to your program. It also provides some support for
audio tracks.

\section{\module{W} --- Widgets built on \module{FrameWork}}
\declaremodule{standard}{W}
  \platform{Mac}
\modulesynopsis{Widgets for the Mac, built on top of \refmodule{FrameWork}.}

The \module{W} widgets are used extensively in the \program{IDE}.

                           % Undocumented Modules

\appendix
\chapter{History and License}
\section{History of the software}

Python was created in the early 1990s by Guido van Rossum at Stichting
Mathematisch Centrum (CWI, see \url{http://www.cwi.nl/}) in the Netherlands
as a successor of a language called ABC.  Guido remains Python's
principal author, although it includes many contributions from others.

In 1995, Guido continued his work on Python at the Corporation for
National Research Initiatives (CNRI, see \url{http://www.cnri.reston.va.us/})
in Reston, Virginia where he released several versions of the
software.

In May 2000, Guido and the Python core development team moved to
BeOpen.com to form the BeOpen PythonLabs team.  In October of the same
year, the PythonLabs team moved to Digital Creations (now Zope
Corporation; see \url{http://www.zope.com/}).  In 2001, the Python
Software Foundation (PSF, see \url{http://www.python.org/psf/}) was
formed, a non-profit organization created specifically to own
Python-related Intellectual Property.  Zope Corporation is a
sponsoring member of the PSF.

All Python releases are Open Source (see
\url{http://www.opensource.org/} for the Open Source Definition).
Historically, most, but not all, Python releases have also been
GPL-compatible; the table below summarizes the various releases.

\begin{tablev}{c|c|c|c|c}{textrm}%
  {Release}{Derived from}{Year}{Owner}{GPL compatible?}
  \linev{0.9.0 thru 1.2}{n/a}{1991-1995}{CWI}{yes}
  \linev{1.3 thru 1.5.2}{1.2}{1995-1999}{CNRI}{yes}
  \linev{1.6}{1.5.2}{2000}{CNRI}{no}
  \linev{2.0}{1.6}{2000}{BeOpen.com}{no}
  \linev{1.6.1}{1.6}{2001}{CNRI}{no}
  \linev{2.1}{2.0+1.6.1}{2001}{PSF}{no}
  \linev{2.0.1}{2.0+1.6.1}{2001}{PSF}{yes}
  \linev{2.1.1}{2.1+2.0.1}{2001}{PSF}{yes}
  \linev{2.2}{2.1.1}{2001}{PSF}{yes}
  \linev{2.1.2}{2.1.1}{2002}{PSF}{yes}
  \linev{2.1.3}{2.1.2}{2002}{PSF}{yes}
  \linev{2.2.1}{2.2}{2002}{PSF}{yes}
  \linev{2.2.2}{2.2.1}{2002}{PSF}{yes}
  \linev{2.2.3}{2.2.2}{2002-2003}{PSF}{yes}
  \linev{2.3}{2.2.2}{2002-2003}{PSF}{yes}
  \linev{2.3.1}{2.3}{2002-2003}{PSF}{yes}
  \linev{2.3.2}{2.3.1}{2003}{PSF}{yes}
  \linev{2.3.3}{2.3.2}{2003}{PSF}{yes}
  \linev{2.3.4}{2.3.3}{2004}{PSF}{yes}
  \linev{2.3.5}{2.3.4}{2005}{PSF}{yes}
  \linev{2.4}{2.3}{2004}{PSF}{yes}
  \linev{2.4.1}{2.4}{2005}{PSF}{yes}
  \linev{2.4.2}{2.4.1}{2005}{PSF}{yes}
  \linev{2.4.3}{2.4.2}{2006}{PSF}{yes}
  \linev{2.5}{2.4}{2006}{PSF}{yes}
\end{tablev}

\note{GPL-compatible doesn't mean that we're distributing
Python under the GPL.  All Python licenses, unlike the GPL, let you
distribute a modified version without making your changes open source.
The GPL-compatible licenses make it possible to combine Python with
other software that is released under the GPL; the others don't.}

Thanks to the many outside volunteers who have worked under Guido's
direction to make these releases possible.


\section{Terms and conditions for accessing or otherwise using Python}

\centerline{\strong{PSF LICENSE AGREEMENT FOR PYTHON \version}}

\begin{enumerate}
\item
This LICENSE AGREEMENT is between the Python Software Foundation
(``PSF''), and the Individual or Organization (``Licensee'') accessing
and otherwise using Python \version{} software in source or binary
form and its associated documentation.

\item
Subject to the terms and conditions of this License Agreement, PSF
hereby grants Licensee a nonexclusive, royalty-free, world-wide
license to reproduce, analyze, test, perform and/or display publicly,
prepare derivative works, distribute, and otherwise use Python
\version{} alone or in any derivative version, provided, however, that
PSF's License Agreement and PSF's notice of copyright, i.e.,
``Copyright \copyright{} 2001-2006 Python Software Foundation; All
Rights Reserved'' are retained in Python \version{} alone or in any
derivative version prepared by Licensee.

\item
In the event Licensee prepares a derivative work that is based on
or incorporates Python \version{} or any part thereof, and wants to
make the derivative work available to others as provided herein, then
Licensee hereby agrees to include in any such work a brief summary of
the changes made to Python \version.

\item
PSF is making Python \version{} available to Licensee on an ``AS IS''
basis.  PSF MAKES NO REPRESENTATIONS OR WARRANTIES, EXPRESS OR
IMPLIED.  BY WAY OF EXAMPLE, BUT NOT LIMITATION, PSF MAKES NO AND
DISCLAIMS ANY REPRESENTATION OR WARRANTY OF MERCHANTABILITY OR FITNESS
FOR ANY PARTICULAR PURPOSE OR THAT THE USE OF PYTHON \version{} WILL
NOT INFRINGE ANY THIRD PARTY RIGHTS.

\item
PSF SHALL NOT BE LIABLE TO LICENSEE OR ANY OTHER USERS OF PYTHON
\version{} FOR ANY INCIDENTAL, SPECIAL, OR CONSEQUENTIAL DAMAGES OR
LOSS AS A RESULT OF MODIFYING, DISTRIBUTING, OR OTHERWISE USING PYTHON
\version, OR ANY DERIVATIVE THEREOF, EVEN IF ADVISED OF THE
POSSIBILITY THEREOF.

\item
This License Agreement will automatically terminate upon a material
breach of its terms and conditions.

\item
Nothing in this License Agreement shall be deemed to create any
relationship of agency, partnership, or joint venture between PSF and
Licensee.  This License Agreement does not grant permission to use PSF
trademarks or trade name in a trademark sense to endorse or promote
products or services of Licensee, or any third party.

\item
By copying, installing or otherwise using Python \version, Licensee
agrees to be bound by the terms and conditions of this License
Agreement.
\end{enumerate}


\centerline{\strong{BEOPEN.COM LICENSE AGREEMENT FOR PYTHON 2.0}}

\centerline{\strong{BEOPEN PYTHON OPEN SOURCE LICENSE AGREEMENT VERSION 1}}

\begin{enumerate}
\item
This LICENSE AGREEMENT is between BeOpen.com (``BeOpen''), having an
office at 160 Saratoga Avenue, Santa Clara, CA 95051, and the
Individual or Organization (``Licensee'') accessing and otherwise
using this software in source or binary form and its associated
documentation (``the Software'').

\item
Subject to the terms and conditions of this BeOpen Python License
Agreement, BeOpen hereby grants Licensee a non-exclusive,
royalty-free, world-wide license to reproduce, analyze, test, perform
and/or display publicly, prepare derivative works, distribute, and
otherwise use the Software alone or in any derivative version,
provided, however, that the BeOpen Python License is retained in the
Software, alone or in any derivative version prepared by Licensee.

\item
BeOpen is making the Software available to Licensee on an ``AS IS''
basis.  BEOPEN MAKES NO REPRESENTATIONS OR WARRANTIES, EXPRESS OR
IMPLIED.  BY WAY OF EXAMPLE, BUT NOT LIMITATION, BEOPEN MAKES NO AND
DISCLAIMS ANY REPRESENTATION OR WARRANTY OF MERCHANTABILITY OR FITNESS
FOR ANY PARTICULAR PURPOSE OR THAT THE USE OF THE SOFTWARE WILL NOT
INFRINGE ANY THIRD PARTY RIGHTS.

\item
BEOPEN SHALL NOT BE LIABLE TO LICENSEE OR ANY OTHER USERS OF THE
SOFTWARE FOR ANY INCIDENTAL, SPECIAL, OR CONSEQUENTIAL DAMAGES OR LOSS
AS A RESULT OF USING, MODIFYING OR DISTRIBUTING THE SOFTWARE, OR ANY
DERIVATIVE THEREOF, EVEN IF ADVISED OF THE POSSIBILITY THEREOF.

\item
This License Agreement will automatically terminate upon a material
breach of its terms and conditions.

\item
This License Agreement shall be governed by and interpreted in all
respects by the law of the State of California, excluding conflict of
law provisions.  Nothing in this License Agreement shall be deemed to
create any relationship of agency, partnership, or joint venture
between BeOpen and Licensee.  This License Agreement does not grant
permission to use BeOpen trademarks or trade names in a trademark
sense to endorse or promote products or services of Licensee, or any
third party.  As an exception, the ``BeOpen Python'' logos available
at http://www.pythonlabs.com/logos.html may be used according to the
permissions granted on that web page.

\item
By copying, installing or otherwise using the software, Licensee
agrees to be bound by the terms and conditions of this License
Agreement.
\end{enumerate}


\centerline{\strong{CNRI LICENSE AGREEMENT FOR PYTHON 1.6.1}}

\begin{enumerate}
\item
This LICENSE AGREEMENT is between the Corporation for National
Research Initiatives, having an office at 1895 Preston White Drive,
Reston, VA 20191 (``CNRI''), and the Individual or Organization
(``Licensee'') accessing and otherwise using Python 1.6.1 software in
source or binary form and its associated documentation.

\item
Subject to the terms and conditions of this License Agreement, CNRI
hereby grants Licensee a nonexclusive, royalty-free, world-wide
license to reproduce, analyze, test, perform and/or display publicly,
prepare derivative works, distribute, and otherwise use Python 1.6.1
alone or in any derivative version, provided, however, that CNRI's
License Agreement and CNRI's notice of copyright, i.e., ``Copyright
\copyright{} 1995-2001 Corporation for National Research Initiatives;
All Rights Reserved'' are retained in Python 1.6.1 alone or in any
derivative version prepared by Licensee.  Alternately, in lieu of
CNRI's License Agreement, Licensee may substitute the following text
(omitting the quotes): ``Python 1.6.1 is made available subject to the
terms and conditions in CNRI's License Agreement.  This Agreement
together with Python 1.6.1 may be located on the Internet using the
following unique, persistent identifier (known as a handle):
1895.22/1013.  This Agreement may also be obtained from a proxy server
on the Internet using the following URL:
\url{http://hdl.handle.net/1895.22/1013}.''

\item
In the event Licensee prepares a derivative work that is based on
or incorporates Python 1.6.1 or any part thereof, and wants to make
the derivative work available to others as provided herein, then
Licensee hereby agrees to include in any such work a brief summary of
the changes made to Python 1.6.1.

\item
CNRI is making Python 1.6.1 available to Licensee on an ``AS IS''
basis.  CNRI MAKES NO REPRESENTATIONS OR WARRANTIES, EXPRESS OR
IMPLIED.  BY WAY OF EXAMPLE, BUT NOT LIMITATION, CNRI MAKES NO AND
DISCLAIMS ANY REPRESENTATION OR WARRANTY OF MERCHANTABILITY OR FITNESS
FOR ANY PARTICULAR PURPOSE OR THAT THE USE OF PYTHON 1.6.1 WILL NOT
INFRINGE ANY THIRD PARTY RIGHTS.

\item
CNRI SHALL NOT BE LIABLE TO LICENSEE OR ANY OTHER USERS OF PYTHON
1.6.1 FOR ANY INCIDENTAL, SPECIAL, OR CONSEQUENTIAL DAMAGES OR LOSS AS
A RESULT OF MODIFYING, DISTRIBUTING, OR OTHERWISE USING PYTHON 1.6.1,
OR ANY DERIVATIVE THEREOF, EVEN IF ADVISED OF THE POSSIBILITY THEREOF.

\item
This License Agreement will automatically terminate upon a material
breach of its terms and conditions.

\item
This License Agreement shall be governed by the federal
intellectual property law of the United States, including without
limitation the federal copyright law, and, to the extent such
U.S. federal law does not apply, by the law of the Commonwealth of
Virginia, excluding Virginia's conflict of law provisions.
Notwithstanding the foregoing, with regard to derivative works based
on Python 1.6.1 that incorporate non-separable material that was
previously distributed under the GNU General Public License (GPL), the
law of the Commonwealth of Virginia shall govern this License
Agreement only as to issues arising under or with respect to
Paragraphs 4, 5, and 7 of this License Agreement.  Nothing in this
License Agreement shall be deemed to create any relationship of
agency, partnership, or joint venture between CNRI and Licensee.  This
License Agreement does not grant permission to use CNRI trademarks or
trade name in a trademark sense to endorse or promote products or
services of Licensee, or any third party.

\item
By clicking on the ``ACCEPT'' button where indicated, or by copying,
installing or otherwise using Python 1.6.1, Licensee agrees to be
bound by the terms and conditions of this License Agreement.
\end{enumerate}

\centerline{ACCEPT}



\centerline{\strong{CWI LICENSE AGREEMENT FOR PYTHON 0.9.0 THROUGH 1.2}}

Copyright \copyright{} 1991 - 1995, Stichting Mathematisch Centrum
Amsterdam, The Netherlands.  All rights reserved.

Permission to use, copy, modify, and distribute this software and its
documentation for any purpose and without fee is hereby granted,
provided that the above copyright notice appear in all copies and that
both that copyright notice and this permission notice appear in
supporting documentation, and that the name of Stichting Mathematisch
Centrum or CWI not be used in advertising or publicity pertaining to
distribution of the software without specific, written prior
permission.

STICHTING MATHEMATISCH CENTRUM DISCLAIMS ALL WARRANTIES WITH REGARD TO
THIS SOFTWARE, INCLUDING ALL IMPLIED WARRANTIES OF MERCHANTABILITY AND
FITNESS, IN NO EVENT SHALL STICHTING MATHEMATISCH CENTRUM BE LIABLE
FOR ANY SPECIAL, INDIRECT OR CONSEQUENTIAL DAMAGES OR ANY DAMAGES
WHATSOEVER RESULTING FROM LOSS OF USE, DATA OR PROFITS, WHETHER IN AN
ACTION OF CONTRACT, NEGLIGENCE OR OTHER TORTIOUS ACTION, ARISING OUT
OF OR IN CONNECTION WITH THE USE OR PERFORMANCE OF THIS SOFTWARE.


\section{Licenses and Acknowledgements for Incorporated Software}

This section is an incomplete, but growing list of licenses and
acknowledgements for third-party software incorporated in the
Python distribution.


\subsection{Mersenne Twister}

The \module{_random} module includes code based on a download from
\url{http://www.math.keio.ac.jp/~matumoto/MT2002/emt19937ar.html}.
The following are the verbatim comments from the original code:

\begin{verbatim}
A C-program for MT19937, with initialization improved 2002/1/26.
Coded by Takuji Nishimura and Makoto Matsumoto.

Before using, initialize the state by using init_genrand(seed)
or init_by_array(init_key, key_length).

Copyright (C) 1997 - 2002, Makoto Matsumoto and Takuji Nishimura,
All rights reserved.

Redistribution and use in source and binary forms, with or without
modification, are permitted provided that the following conditions
are met:

 1. Redistributions of source code must retain the above copyright
    notice, this list of conditions and the following disclaimer.

 2. Redistributions in binary form must reproduce the above copyright
    notice, this list of conditions and the following disclaimer in the
    documentation and/or other materials provided with the distribution.

 3. The names of its contributors may not be used to endorse or promote
    products derived from this software without specific prior written
    permission.

THIS SOFTWARE IS PROVIDED BY THE COPYRIGHT HOLDERS AND CONTRIBUTORS
"AS IS" AND ANY EXPRESS OR IMPLIED WARRANTIES, INCLUDING, BUT NOT
LIMITED TO, THE IMPLIED WARRANTIES OF MERCHANTABILITY AND FITNESS FOR
A PARTICULAR PURPOSE ARE DISCLAIMED.  IN NO EVENT SHALL THE COPYRIGHT OWNER OR
CONTRIBUTORS BE LIABLE FOR ANY DIRECT, INDIRECT, INCIDENTAL, SPECIAL,
EXEMPLARY, OR CONSEQUENTIAL DAMAGES (INCLUDING, BUT NOT LIMITED TO,
PROCUREMENT OF SUBSTITUTE GOODS OR SERVICES; LOSS OF USE, DATA, OR
PROFITS; OR BUSINESS INTERRUPTION) HOWEVER CAUSED AND ON ANY THEORY OF
LIABILITY, WHETHER IN CONTRACT, STRICT LIABILITY, OR TORT (INCLUDING
NEGLIGENCE OR OTHERWISE) ARISING IN ANY WAY OUT OF THE USE OF THIS
SOFTWARE, EVEN IF ADVISED OF THE POSSIBILITY OF SUCH DAMAGE.


Any feedback is very welcome.
http://www.math.keio.ac.jp/matumoto/emt.html
email: matumoto@math.keio.ac.jp
\end{verbatim}



\subsection{Sockets}

The \module{socket} module uses the functions, \function{getaddrinfo},
and \function{getnameinfo}, which are coded in separate source files
from the WIDE Project, \url{http://www.wide.ad.jp/about/index.html}.

\begin{verbatim}      
Copyright (C) 1995, 1996, 1997, and 1998 WIDE Project.
All rights reserved.
 
Redistribution and use in source and binary forms, with or without
modification, are permitted provided that the following conditions
are met:
1. Redistributions of source code must retain the above copyright
   notice, this list of conditions and the following disclaimer.
2. Redistributions in binary form must reproduce the above copyright
   notice, this list of conditions and the following disclaimer in the
   documentation and/or other materials provided with the distribution.
3. Neither the name of the project nor the names of its contributors
   may be used to endorse or promote products derived from this software
   without specific prior written permission.

THIS SOFTWARE IS PROVIDED BY THE PROJECT AND CONTRIBUTORS ``AS IS'' AND
GAI_ANY EXPRESS OR IMPLIED WARRANTIES, INCLUDING, BUT NOT LIMITED TO, THE
IMPLIED WARRANTIES OF MERCHANTABILITY AND FITNESS FOR A PARTICULAR PURPOSE
ARE DISCLAIMED.  IN NO EVENT SHALL THE PROJECT OR CONTRIBUTORS BE LIABLE
FOR GAI_ANY DIRECT, INDIRECT, INCIDENTAL, SPECIAL, EXEMPLARY, OR CONSEQUENTIAL
DAMAGES (INCLUDING, BUT NOT LIMITED TO, PROCUREMENT OF SUBSTITUTE GOODS
OR SERVICES; LOSS OF USE, DATA, OR PROFITS; OR BUSINESS INTERRUPTION)
HOWEVER CAUSED AND ON GAI_ANY THEORY OF LIABILITY, WHETHER IN CONTRACT, STRICT
LIABILITY, OR TORT (INCLUDING NEGLIGENCE OR OTHERWISE) ARISING IN GAI_ANY WAY
OUT OF THE USE OF THIS SOFTWARE, EVEN IF ADVISED OF THE POSSIBILITY OF
SUCH DAMAGE.
\end{verbatim}



\subsection{Floating point exception control}

The source for the \module{fpectl} module includes the following notice:

\begin{verbatim}
     ---------------------------------------------------------------------  
    /                       Copyright (c) 1996.                           \ 
   |          The Regents of the University of California.                 |
   |                        All rights reserved.                           |
   |                                                                       |
   |   Permission to use, copy, modify, and distribute this software for   |
   |   any purpose without fee is hereby granted, provided that this en-   |
   |   tire notice is included in all copies of any software which is or   |
   |   includes  a  copy  or  modification  of  this software and in all   |
   |   copies of the supporting documentation for such software.           |
   |                                                                       |
   |   This  work was produced at the University of California, Lawrence   |
   |   Livermore National Laboratory under  contract  no.  W-7405-ENG-48   |
   |   between  the  U.S.  Department  of  Energy and The Regents of the   |
   |   University of California for the operation of UC LLNL.              |
   |                                                                       |
   |                              DISCLAIMER                               |
   |                                                                       |
   |   This  software was prepared as an account of work sponsored by an   |
   |   agency of the United States Government. Neither the United States   |
   |   Government  nor the University of California nor any of their em-   |
   |   ployees, makes any warranty, express or implied, or  assumes  any   |
   |   liability  or  responsibility  for the accuracy, completeness, or   |
   |   usefulness of any information,  apparatus,  product,  or  process   |
   |   disclosed,   or  represents  that  its  use  would  not  infringe   |
   |   privately-owned rights. Reference herein to any specific  commer-   |
   |   cial  products,  process,  or  service  by trade name, trademark,   |
   |   manufacturer, or otherwise, does not  necessarily  constitute  or   |
   |   imply  its endorsement, recommendation, or favoring by the United   |
   |   States Government or the University of California. The views  and   |
   |   opinions  of authors expressed herein do not necessarily state or   |
   |   reflect those of the United States Government or  the  University   |
   |   of  California,  and shall not be used for advertising or product   |
    \  endorsement purposes.                                              / 
     ---------------------------------------------------------------------
\end{verbatim}



\subsection{MD5 message digest algorithm}

The source code for the \module{md5} module contains the following notice:

\begin{verbatim}
  Copyright (C) 1999, 2002 Aladdin Enterprises.  All rights reserved.

  This software is provided 'as-is', without any express or implied
  warranty.  In no event will the authors be held liable for any damages
  arising from the use of this software.

  Permission is granted to anyone to use this software for any purpose,
  including commercial applications, and to alter it and redistribute it
  freely, subject to the following restrictions:

  1. The origin of this software must not be misrepresented; you must not
     claim that you wrote the original software. If you use this software
     in a product, an acknowledgment in the product documentation would be
     appreciated but is not required.
  2. Altered source versions must be plainly marked as such, and must not be
     misrepresented as being the original software.
  3. This notice may not be removed or altered from any source distribution.

  L. Peter Deutsch
  ghost@aladdin.com

  Independent implementation of MD5 (RFC 1321).

  This code implements the MD5 Algorithm defined in RFC 1321, whose
  text is available at
	http://www.ietf.org/rfc/rfc1321.txt
  The code is derived from the text of the RFC, including the test suite
  (section A.5) but excluding the rest of Appendix A.  It does not include
  any code or documentation that is identified in the RFC as being
  copyrighted.

  The original and principal author of md5.h is L. Peter Deutsch
  <ghost@aladdin.com>.  Other authors are noted in the change history
  that follows (in reverse chronological order):

  2002-04-13 lpd Removed support for non-ANSI compilers; removed
	references to Ghostscript; clarified derivation from RFC 1321;
	now handles byte order either statically or dynamically.
  1999-11-04 lpd Edited comments slightly for automatic TOC extraction.
  1999-10-18 lpd Fixed typo in header comment (ansi2knr rather than md5);
	added conditionalization for C++ compilation from Martin
	Purschke <purschke@bnl.gov>.
  1999-05-03 lpd Original version.
\end{verbatim}



\subsection{Asynchronous socket services}

The \module{asynchat} and \module{asyncore} modules contain the
following notice:

\begin{verbatim}      
 Copyright 1996 by Sam Rushing

                         All Rights Reserved

 Permission to use, copy, modify, and distribute this software and
 its documentation for any purpose and without fee is hereby
 granted, provided that the above copyright notice appear in all
 copies and that both that copyright notice and this permission
 notice appear in supporting documentation, and that the name of Sam
 Rushing not be used in advertising or publicity pertaining to
 distribution of the software without specific, written prior
 permission.

 SAM RUSHING DISCLAIMS ALL WARRANTIES WITH REGARD TO THIS SOFTWARE,
 INCLUDING ALL IMPLIED WARRANTIES OF MERCHANTABILITY AND FITNESS, IN
 NO EVENT SHALL SAM RUSHING BE LIABLE FOR ANY SPECIAL, INDIRECT OR
 CONSEQUENTIAL DAMAGES OR ANY DAMAGES WHATSOEVER RESULTING FROM LOSS
 OF USE, DATA OR PROFITS, WHETHER IN AN ACTION OF CONTRACT,
 NEGLIGENCE OR OTHER TORTIOUS ACTION, ARISING OUT OF OR IN
 CONNECTION WITH THE USE OR PERFORMANCE OF THIS SOFTWARE.
\end{verbatim}


\subsection{Cookie management}

The \module{Cookie} module contains the following notice:

\begin{verbatim}
 Copyright 2000 by Timothy O'Malley <timo@alum.mit.edu>

                All Rights Reserved

 Permission to use, copy, modify, and distribute this software
 and its documentation for any purpose and without fee is hereby
 granted, provided that the above copyright notice appear in all
 copies and that both that copyright notice and this permission
 notice appear in supporting documentation, and that the name of
 Timothy O'Malley  not be used in advertising or publicity
 pertaining to distribution of the software without specific, written
 prior permission.

 Timothy O'Malley DISCLAIMS ALL WARRANTIES WITH REGARD TO THIS
 SOFTWARE, INCLUDING ALL IMPLIED WARRANTIES OF MERCHANTABILITY
 AND FITNESS, IN NO EVENT SHALL Timothy O'Malley BE LIABLE FOR
 ANY SPECIAL, INDIRECT OR CONSEQUENTIAL DAMAGES OR ANY DAMAGES
 WHATSOEVER RESULTING FROM LOSS OF USE, DATA OR PROFITS,
 WHETHER IN AN ACTION OF CONTRACT, NEGLIGENCE OR OTHER TORTIOUS
 ACTION, ARISING OUT OF OR IN CONNECTION WITH THE USE OR
 PERFORMANCE OF THIS SOFTWARE.
\end{verbatim}      



\subsection{Profiling}

The \module{profile} and \module{pstats} modules contain
the following notice:

\begin{verbatim}
 Copyright 1994, by InfoSeek Corporation, all rights reserved.
 Written by James Roskind

 Permission to use, copy, modify, and distribute this Python software
 and its associated documentation for any purpose (subject to the
 restriction in the following sentence) without fee is hereby granted,
 provided that the above copyright notice appears in all copies, and
 that both that copyright notice and this permission notice appear in
 supporting documentation, and that the name of InfoSeek not be used in
 advertising or publicity pertaining to distribution of the software
 without specific, written prior permission.  This permission is
 explicitly restricted to the copying and modification of the software
 to remain in Python, compiled Python, or other languages (such as C)
 wherein the modified or derived code is exclusively imported into a
 Python module.

 INFOSEEK CORPORATION DISCLAIMS ALL WARRANTIES WITH REGARD TO THIS
 SOFTWARE, INCLUDING ALL IMPLIED WARRANTIES OF MERCHANTABILITY AND
 FITNESS. IN NO EVENT SHALL INFOSEEK CORPORATION BE LIABLE FOR ANY
 SPECIAL, INDIRECT OR CONSEQUENTIAL DAMAGES OR ANY DAMAGES WHATSOEVER
 RESULTING FROM LOSS OF USE, DATA OR PROFITS, WHETHER IN AN ACTION OF
 CONTRACT, NEGLIGENCE OR OTHER TORTIOUS ACTION, ARISING OUT OF OR IN
 CONNECTION WITH THE USE OR PERFORMANCE OF THIS SOFTWARE.
\end{verbatim}



\subsection{Execution tracing}

The \module{trace} module contains the following notice:

\begin{verbatim}
 portions copyright 2001, Autonomous Zones Industries, Inc., all rights...
 err...  reserved and offered to the public under the terms of the
 Python 2.2 license.
 Author: Zooko O'Whielacronx
 http://zooko.com/
 mailto:zooko@zooko.com

 Copyright 2000, Mojam Media, Inc., all rights reserved.
 Author: Skip Montanaro

 Copyright 1999, Bioreason, Inc., all rights reserved.
 Author: Andrew Dalke

 Copyright 1995-1997, Automatrix, Inc., all rights reserved.
 Author: Skip Montanaro

 Copyright 1991-1995, Stichting Mathematisch Centrum, all rights reserved.


 Permission to use, copy, modify, and distribute this Python software and
 its associated documentation for any purpose without fee is hereby
 granted, provided that the above copyright notice appears in all copies,
 and that both that copyright notice and this permission notice appear in
 supporting documentation, and that the name of neither Automatrix,
 Bioreason or Mojam Media be used in advertising or publicity pertaining to
 distribution of the software without specific, written prior permission.
\end{verbatim} 



\subsection{UUencode and UUdecode functions}

The \module{uu} module contains the following notice:

\begin{verbatim}
 Copyright 1994 by Lance Ellinghouse
 Cathedral City, California Republic, United States of America.
                        All Rights Reserved
 Permission to use, copy, modify, and distribute this software and its
 documentation for any purpose and without fee is hereby granted,
 provided that the above copyright notice appear in all copies and that
 both that copyright notice and this permission notice appear in
 supporting documentation, and that the name of Lance Ellinghouse
 not be used in advertising or publicity pertaining to distribution
 of the software without specific, written prior permission.
 LANCE ELLINGHOUSE DISCLAIMS ALL WARRANTIES WITH REGARD TO
 THIS SOFTWARE, INCLUDING ALL IMPLIED WARRANTIES OF MERCHANTABILITY AND
 FITNESS, IN NO EVENT SHALL LANCE ELLINGHOUSE CENTRUM BE LIABLE
 FOR ANY SPECIAL, INDIRECT OR CONSEQUENTIAL DAMAGES OR ANY DAMAGES
 WHATSOEVER RESULTING FROM LOSS OF USE, DATA OR PROFITS, WHETHER IN AN
 ACTION OF CONTRACT, NEGLIGENCE OR OTHER TORTIOUS ACTION, ARISING OUT
 OF OR IN CONNECTION WITH THE USE OR PERFORMANCE OF THIS SOFTWARE.

 Modified by Jack Jansen, CWI, July 1995:
 - Use binascii module to do the actual line-by-line conversion
   between ascii and binary. This results in a 1000-fold speedup. The C
   version is still 5 times faster, though.
 - Arguments more compliant with python standard
\end{verbatim}



\subsection{XML Remote Procedure Calls}

The \module{xmlrpclib} module contains the following notice:

\begin{verbatim}
     The XML-RPC client interface is

 Copyright (c) 1999-2002 by Secret Labs AB
 Copyright (c) 1999-2002 by Fredrik Lundh

 By obtaining, using, and/or copying this software and/or its
 associated documentation, you agree that you have read, understood,
 and will comply with the following terms and conditions:

 Permission to use, copy, modify, and distribute this software and
 its associated documentation for any purpose and without fee is
 hereby granted, provided that the above copyright notice appears in
 all copies, and that both that copyright notice and this permission
 notice appear in supporting documentation, and that the name of
 Secret Labs AB or the author not be used in advertising or publicity
 pertaining to distribution of the software without specific, written
 prior permission.

 SECRET LABS AB AND THE AUTHOR DISCLAIMS ALL WARRANTIES WITH REGARD
 TO THIS SOFTWARE, INCLUDING ALL IMPLIED WARRANTIES OF MERCHANT-
 ABILITY AND FITNESS.  IN NO EVENT SHALL SECRET LABS AB OR THE AUTHOR
 BE LIABLE FOR ANY SPECIAL, INDIRECT OR CONSEQUENTIAL DAMAGES OR ANY
 DAMAGES WHATSOEVER RESULTING FROM LOSS OF USE, DATA OR PROFITS,
 WHETHER IN AN ACTION OF CONTRACT, NEGLIGENCE OR OTHER TORTIOUS
 ACTION, ARISING OUT OF OR IN CONNECTION WITH THE USE OR PERFORMANCE
 OF THIS SOFTWARE.
\end{verbatim}


%
%  The ugly "%begin{latexonly}" pseudo-environments are really just to
%  keep LaTeX2HTML quiet during the \renewcommand{} macros; they're
%  not really valuable.
%

%begin{latexonly}
\renewcommand{\indexname}{Module Index}
%end{latexonly}
\input{modmac.ind}      % Module Index

%begin{latexonly}
\renewcommand{\indexname}{Index}
%end{latexonly}
\documentclass{manual}

\title{Macintosh Library Modules}

\author{Guido van Rossum\\
	Fred L. Drake, Jr., editor}
\authoraddress{
	PythonLabs\\
	E-mail: \email{python-docs@python.org}
}

\date{June 15, 2001}		% XXX update before release!
\release{2.0.1c1}		% software release, not documentation
\setshortversion{2.0}		% major.minor only for software


\makeindex              % tell \index to actually write the .idx file
\makemodindex           % ... and the module index as well.


\begin{document}

\maketitle

\ifhtml
\chapter*{Front Matter\label{front}}
\fi

Copyright 1991, 1992, 1993, 1994 by Stichting Mathematisch Centrum,
Amsterdam, The Netherlands.

\begin{center}
All Rights Reserved
\end{center}

Permission to use, copy, modify, and distribute this software and its
documentation for any purpose and without fee is hereby granted,
provided that the above copyright notice appear in all copies and that
both that copyright notice and this permission notice appear in
supporting documentation, and that the names of Stichting Mathematisch
Centrum or CWI not be used in advertising or publicity pertaining to
distribution of the software without specific, written prior permission.

STICHTING MATHEMATISCH CENTRUM DISCLAIMS ALL WARRANTIES WITH REGARD TO
THIS SOFTWARE, INCLUDING ALL IMPLIED WARRANTIES OF MERCHANTABILITY AND
FITNESS, IN NO EVENT SHALL STICHTING MATHEMATISCH CENTRUM BE LIABLE
FOR ANY SPECIAL, INDIRECT OR CONSEQUENTIAL DAMAGES OR ANY DAMAGES
WHATSOEVER RESULTING FROM LOSS OF USE, DATA OR PROFITS, WHETHER IN AN
ACTION OF CONTRACT, NEGLIGENCE OR OTHER TORTIOUS ACTION, ARISING OUT
OF OR IN CONNECTION WITH THE USE OR PERFORMANCE OF THIS SOFTWARE.


\begin{abstract}

\noindent
This library reference manual documents Python's extensions for the
Macintosh.  It should be used in conjunction with the
\citetitle[../lib/lib.html]{Python Library Reference}, which documents
the standard library and built-in types.

This manual assumes basic knowledge about the Python language.  For an
informal introduction to Python, see the
\citetitle[../tut/tut.html]{Python Tutorial}; the
\citetitle[../ref/ref.html]{Python Reference Manual} remains the
highest authority on syntactic and semantic questions.  Finally, the
manual entitled \citetitle[../ext/ext.html]{Extending and Embedding
the Python Interpreter} describes how to add new extensions to Python
and how to embed it in other applications.

\end{abstract}

\tableofcontents


\chapter{Using Python on a Mac OS 9 Macintosh \label{using}}
\sectionauthor{Bob Savage}{bobsavage@mac.com}

Using Python on a Mac OS 9 Macintosh can seem like something completely
different than using it on a \UNIX-like or Windows system. Most of the
Python documentation, both the ``official'' documentation and
published books, describe only how Python is used on these systems,
causing confusion for the new user of MacPython-OS9. This chapter gives a
brief introduction to the specifics of using Python on a Macintosh.

Note that this chapter is mainly relevant to Mac OS 9: MacPython-OSX
is a superset of a normal unix Python. While MacPython-OS9 runs fine
on Mac OS X it is a better choice to use MacPython-OSX there.

The section on the IDE (see Section \ref{IDE}) is relevant to MacPython-OSX
too.

\section{Getting and Installing MacPython-OS9 \label{getting}}

The most recent release version as well as possible newer experimental
versions are best found at the MacPython page maintained by Jack
Jansen: \url{http://www.cwi.nl/\textasciitilde jack/macpython.html}.


Please refer to the \file{README} included with your distribution for
the most up-to-date instructions.


\section{Entering the interactive Interpreter
         \label{interpreter}}

The interactive interpreter that you will see used in Python
documentation is started by double-clicking the
\program{PythonInterpreter} icon, which looks like a 16-ton weight
falling. You should see the version information and the
\samp{>\code{>}>~} prompt.  Use it exactly as described in the
standard documentation.


\section{How to run a Python script}

There are several ways to run an existing Python script; two common
ways to run a Python script are ``drag and drop'' and ``double
clicking''.  Other ways include running it from within the IDE (see
Section \ref{IDE}), or launching via AppleScript.


\subsection{Drag and drop}

One of the easiest ways to launch a Python script is via ``Drag and
Drop''. This is just like launching a text file in the Finder by
``dragging'' it over your word processor's icon and ``dropping'' it
there. Make sure that you use an icon referring to the
\program{PythonInterpreter}, not the \program{IDE} or \program{Idle}
icons which have different behaviour which is described below.

Some things that might have gone wrong:

\begin{itemize}
\item
A window flashes after dropping the script onto the
\program{PythonInterpreter}, but then disappears. Most likely this is a
configuration issue; your \program{PythonInterpreter} is setup to exit
immediately upon completion, but your script assumes that if it prints
something that text will stick around for a while. To fix this, see
section \ref{defaults}.

\item
When you waved the script icon over the \program{PythonInterpreter},
the \program{PythonInterpreter} icon did not hilight.  Most likely the
Creator code and document type is unset (or set incorrectly) -- this
often happens when a file originates on a non-Mac computer.  See
section \ref{creator-code} for more details.
\end{itemize}


\subsection{Set Creator and Double Click \label{creator-code}}

If the script that you want to launch has the appropriate Creator Code
and File Type you can simply double-click on the script to launch it.
To be ``double-clickable'' a file needs to be of type \samp{TEXT},
with a creator code of \samp{Pyth}.

Setting the creator code and filetype can be done with the IDE (see
sections \ref{IDEwrite} and \ref{IDEapplet}), with an editor with a
Python mode (\program{BBEdit}) -- see section
\ref{scripting-with-BBedit}, or with assorted other Mac utilities, but
a script (\file{fixfiletypes.py}) has been included in the MacPython
distribution, making it possible to set the proper Type and Creator
Codes with Python.

The \file{fixfiletypes.py} script will change the file type and
creator codes for the indicated directory.  To use
\file{fixfiletypes.py}:

\begin{enumerate}
\item
Locate it in the \file{scripts} folder of the \file{Mac} folder of the
MacPython distribution.

\item
Put all of the scripts that you want to fix in a folder with nothing
else in it.

\item
Double-click on the \file{fixfiletypes.py} icon.

\item
Navigate into the folder of files you want to fix, and press the
``Select current folder'' button.
\end{enumerate}


\section{Simulating command line arguments
         \label{argv}}

There are two ways to simulate command-line arguments with MacPython-OS9.
 
\begin{enumerate}
\item via Interpreter options
\begin{itemize} % nestable? I hope so!
  \item Hold the option-key down when launching your script. This will
        bring up a dialog box of Python Interpreter options.
  \item Click ``Set \UNIX-style command line..'' button. 
  \item Type the arguments into the ``Argument'' field.
  \item Click ``OK''
  \item Click ``Run''.
\end{itemize} % end

\item via drag and drop
If you save the script as an applet (see Section \ref{IDEapplet}), you
can also simulate some command-line arguments via
``Drag-and-Drop''. In this case, the names of the files that were
dropped onto the applet will be appended to \code{sys.argv}, so that
it will appear to the script as though they had been typed on a
command line.  As on \UNIX\ systems, the first item in \code{sys.srgv} is
the path to the applet, and the rest are the files dropped on the
applet.
\end{enumerate}


\section{Creating a Python script}

Since Python scripts are simply text files, they can be created in any
way that text files can be created, but some special tools also exist
with extra features.


\subsection{In an editor}

You can create a text file with any word processing program such as
\program{MSWord} or \program{AppleWorks} but you need to make sure
that the file is saved as ``\ASCII'' or ``plain text''.


\subsubsection{Editors with Python modes}

Several text editors have additional features that add functionality
when you are creating a Python script.  These can include coloring
Python keywords to make your code easier to read, module browsing, or
a built-in debugger. These include \program{Alpha}, \program{Pepper},
and \program{BBedit}, and the MacPython IDE (Section \ref{IDE}).

%\subsubsection{Alpha}
% **NEED INFO HERE**
 
\subsubsection{BBedit \label{scripting-with-BBedit}}

If you use \program{BBEdit} to create your scripts you will want to tell it about the Python creator code so that
you can simply double click on the saved file to launch it.
\begin{itemize}
  \item Launch \program{BBEdit}.
  \item Select ``Preferences'' from the ``Edit'' menu.
  \item Select ``File Types'' from the scrolling list.
  \item click on the ``Add...'' button and navigate to
        \program{PythonInterpreter} in the main directory of the
        MacPython distribution; click ``open''.
  \item Click on the ``Save'' button in the Preferences panel.
\end{itemize}
% Are there additional BBedit Python-specific features? I'm not aware of any.
 
%\subsubsection{IDE}
%You can use the \program{Python IDE} supplied in the MacPython Distribution to create longer Python scripts 
%-- see Section \ref{IDEwrite} for details.
 
%\subsubsection{IDLE}
%Idle is an IDE for Python that was written in Python, using TKInter. You should be able to use it on a Mac by following
%the standard documentation, but see Section \ref{TKInter} for guidance on using TKInter with MacPython.

%\subsubsection{Pepper}
% **NEED INFO HERE**


\section{The IDE\label{IDE}}

The \program{Python IDE} (Integrated Development Environment) is a
separate application that acts as a text editor for your Python code,
a class browser, a graphical debugger, and more.


\subsection{Using the ``Python Interactive'' window}

Use this window like you would the \program{PythonInterpreter}, except
that you cannot use the ``Drag and drop'' method above. Instead,
dropping a script onto the \program{Python IDE} icon will open the
file in a separate script window (which you can then execute manually
-- see section \ref{IDEexecution}).


\subsection{Writing a Python Script \label{IDEwrite}}

In addition to using the \program{Python IDE} interactively, you can
also type out a complete Python program, saving it incrementally, and
execute it or smaller selections of it.

You can create a new script, open a previously saved script, and save
your currently open script by selecting the appropriate item in the
``File'' menu. Dropping a Python script onto the
\program{Python IDE} will open it for editting.

If you try to open a script with the \program{Python IDE} but either
can't locate it from the ``Open'' dialog box, or you get an error
message like ``Can't open file of type ...'' see section
\ref{creator-code}.

When the \program{Python IDE} saves a script, it uses the creator code
settings which are available by clicking on the small black triangle
on the top right of the document window, and selecting ``save
options''. The default is to save the file with the \program{Python
IDE} as the creator, this means that you can open the file for editing
by simply double-clicking on its icon. You might want to change this
behaviour so that it will be opened by the
\program{PythonInterpreter}, and run. To do this simply choose
``Python Interpreter'' from the ``save options''. Note that these
options are associated with the \emph{file} not the application.


\subsection{Executing a script from within the IDE
            \label{IDEexecution}}

You can run the script in the frontmost window of the \program{Python
IDE} by hitting the run all button.  You should be aware, however that
if you use the Python convention \samp{if __name__ == "__main__":} the
script will \emph{not} be ``__main__'' by default. To get that
behaviour you must select the ``Run as __main__'' option from the
small black triangle on the top right of the document window.  Note
that this option is associated with the \emph{file} not the
application. It \emph{will} stay active after a save, however; to shut
this feature off simply select it again.
 

\subsection{``Save as'' versus ``Save as Applet''
            \label{IDEapplet}}

When you are done writing your Python script you have the option of
saving it as an ``applet'' (by selecting ``Save as applet'' from the
``File'' menu). This has a significant advantage in that you can drop
files or folders onto it, to pass them to the applet the way
command-line users would type them onto the command-line to pass them
as arguments to the script. However, you should make sure to save the
applet as a separate file, do not overwrite the script you are
writing, because you will not be able to edit it again.

Accessing the items passed to the applet via ``drag-and-drop'' is done
using the standard \member{sys.argv} mechanism. See the general
documentation for more
% need to link to the appropriate place in non-Mac docs

Note that saving a script as an applet will not make it runnable on a
system without a Python installation.

%\subsection{Debugger}
% **NEED INFO HERE**
 
%\subsection{Module Browser}
% **NEED INFO HERE**
 
%\subsection{Profiler}
% **NEED INFO HERE**
% end IDE

%\subsection{The ``Scripts'' menu}
% **NEED INFO HERE**
 
\section{Configuration \label{configuration}}

The MacPython distribution comes with \program{EditPythonPrefs}, an
applet which will help you to customize the MacPython environment for
your working habits.
 
\subsection{EditPythonPrefs\label{EditPythonPrefs}}

\program{EditPythonPrefs} gives you the capability to configure Python
to behave the way you want it to.  There are two ways to use
\program{EditPythonPrefs}, you can use it to set the preferences in
general, or you can drop a particular Python engine onto it to
customize only that version. The latter can be handy if, for example,
you want to have a second copy of the \program{PythonInterpreter} that
keeps the output window open on a normal exit even though you prefer
to normally not work that way.

To change the default preferences, simply double-click on
\program{EditPythonPrefs}. To change the preferences only for one copy
of the Interpreter, drop the icon for that copy onto
\program{EditPythonPrefs}.  You can also use \program{EditPythonPrefs}
in this fashion to set the preferences of the \program{Python IDE} and
any applets you create -- see section %s \ref{BuildApplet} and
\ref{IDEapplet}.

\subsection{Adding modules to the Module Search Path
            \label{search-path}}

When executing an \keyword{import} statement, Python looks for modules
in places defined by the \member{sys.path} To edit the
\member{sys.path} on a Mac, launch \program{EditPythonPrefs}, and
enter them into the largish field at the top (one per line).

Since MacPython defines a main Python directory, the easiest thing is
to add folders to search within the main Python directory. To add a
folder of scripts that you created called ``My Folder'' located in the
main Python Folder, enter \samp{\$(PYTHON):My Folder} onto a new line.

To add the Desktop under OS 9 or below, add
\samp{StartupDriveName:Desktop Folder} on a new line.

\subsection{Default startup options \label{defaults}}

% I'm assuming that there exists some other documentation on the
% rest of the options so I only go over a couple here.

The ``Default startup options...'' button in the
\program{EditPythonPrefs} dialog box gives you many options including
the ability to keep the ``Output'' window open after the script
terminates, and the ability to enter interactive mode after the
termination of the run script. The latter can be very helpful if you
want to examine the objects that were created during your script.

%\section{Nifty Tools}
%There are many other tools included with the MacPython
%distribution. In addition to those discussed here, make 
%sure to check the \file{Mac} directory.

%\subsection{BuildApplet \label{BuildApplet}}
% **NEED INFO HERE**

%\subsection{BuildApplication}
% **NEED INFO HERE**
 
%\section{TKInter on the Mac \label{TKInter}}

%TKinter is installed by default with the MacPython distribution, but
%you may need to add the \file{lib-tk} folder to the Python Path (see
%section \ref{search-path}).  Also, it is important that you do not
%try to launch Tk from within the \program{Python IDE} because the two
%event loops will collide -- always run a script which uses Tkinter
%with the \program{PythonInterpreter} instead -- see section
%\ref{interpreter}.
 
%\section{CGI on the Mac with Python \label{CGI}}
%**NEED INFO HERE**
                       % Using Python on the Macintosh


\chapter{MacPython Modules \label{macpython-modules}}

The following modules are only available on the Macintosh, and are
documented here:

\localmoduletable

\chapter{MACINTOSH ONLY}

The modules in this chapter are available on the Apple Macintosh only.

\section{Built-in module \sectcode{mac}}

\bimodindex{mac}
This module provides a subset of the operating system dependent
functionality provided by the optional built-in module \code{posix}.
It is best accessed through the more portable standard module
\code{os}.

The following functions are available in this module:
\code{chdir},
\code{getcwd},
\code{listdir},
\code{mkdir},
\code{rename},
\code{rmdir},
\code{stat},
\code{sync},
\code{unlink},
as well as the exception \code{error}.

\section{Standard module \sectcode{macpath}}

\stmodindex{macpath}
This module provides a subset of the pathname manipulation functions
available from the optional standard module \code{posixpath}.  It is
best accessed through the more portable standard module \code{os}, as
\code{os.path}.

The following functions are available in this module:
\code{normcase},
\code{isabs},
\code{join},
\code{split},
\code{isdir},
\code{isfile},
\code{exists}.

\section{Built-in Module \sectcode{ctb}}
\bimodindex{ctb}
\renewcommand{\indexsubitem}{(in module ctb)}

This module provides a partial interface to the Macintosh
Communications Toolbox. Currently, only Connection Manager tools are
supported.  It may not be available in all Mac Python versions.

\begin{datadesc}{error}
The exception raised on errors.
\end{datadesc}

\begin{datadesc}{cmData}
\dataline{cmCntl}
\dataline{cmAttn}
Flags for the \var{channel} argument of the \var{Read} and \var{Write}
methods.
\end{datadesc}

\begin{datadesc}{cmFlagsEOM}
End-of-message flag for \var{Read} and \var{Write}.
\end{datadesc}

\begin{datadesc}{choose*}
Values returned by \var{Choose}.
\end{datadesc}

\begin{datadesc}{cmStatus*}
Bits in the status as returned by \var{Status}.
\end{datadesc}

\begin{funcdesc}{available}{}
Return 1 if the communication toolbox is available, zero otherwise.
\end{funcdesc}

\begin{funcdesc}{CMNew}{name\, sizes}
Create a connection object using the connection tool named
\var{name}. \var{sizes} is a 6-tuple given buffer sizes for data in,
data out, control in, control out, attention in and attention out.
Alternatively, passing \code{None} will result in default buffer sizes.
\end{funcdesc}

\subsection{connection object}
For all connection methods that take a \var{timeout} argument, a value
of \code{-1} is indefinite, meaning that the command runs to completion.

\renewcommand{\indexsubitem}{(connection object attribute)}

\begin{datadesc}{callback}
If this member is set to a value other than \code{None} it should point
to a function accepting a single argument (the connection
object). This will make all connection object methods work
asynchronously, with the callback routine being called upon
completion.

{\em Note:} for reasons beyond my understanding the callback routine
is currently never called. You are advised against using asynchronous
calls for the time being.
\end{datadesc}


\renewcommand{\indexsubitem}{(connection object method)}

\begin{funcdesc}{Open}{timeout}
Open an outgoing connection, waiting at most \var{timeout} seconds for
the connection to be established.
\end{funcdesc}

\begin{funcdesc}{Listen}{timeout}
Wait for an incoming connection. Stop waiting after \var{timeout}
seconds. This call is only meaningful to some tools.
\end{funcdesc}

\begin{funcdesc}{accept}{yesno}
Accept (when \var{yesno} is non-zero) or reject an incoming call after
\var{Listen} returned.
\end{funcdesc}

\begin{funcdesc}{Close}{timeout\, now}
Close a connection. When \var{now} is zero, the close is orderly
(i.e.\ outstanding output is flushed, etc.)\ with a timeout of
\var{timeout} seconds. When \var{now} is non-zero the close is
immediate, discarding output.
\end{funcdesc}

\begin{funcdesc}{Read}{len\, chan\, timeout}
Read \var{len} bytes, or until \var{timeout} seconds have passed, from
the channel \var{chan} (which is one of \var{cmData}, \var{cmCntl} or
\var{cmAttn}). Return a 2-tuple:\ the data read and the end-of-message
flag.
\end{funcdesc}

\begin{funcdesc}{Write}{buf\, chan\, timeout\, eom}
Write \var{buf} to channel \var{chan}, aborting after \var{timeout}
seconds. When \var{eom} has the value \var{cmFlagsEOM} an
end-of-message indicator will be written after the data (if this
concept has a meaning for this communication tool). The method returns
the number of bytes written.
\end{funcdesc}

\begin{funcdesc}{Status}{}
Return connection status as the 2-tuple \code{(\var{sizes},
\var{flags})}. \var{sizes} is a 6-tuple giving the actual buffer sizes used
(see \var{CMNew}), \var{flags} is a set of bits describing the state
of the connection.
\end{funcdesc}

\begin{funcdesc}{GetConfig}{}
Return the configuration string of the communication tool. These
configuration strings are tool-dependent, but usually easily parsed
and modified.
\end{funcdesc}

\begin{funcdesc}{SetConfig}{str}
Set the configuration string for the tool. The strings are parsed
left-to-right, with later values taking precedence. This means
individual configuration parameters can be modified by simply appending
something like \code{'baud 4800'} to the end of the string returned by
\var{GetConfig} and passing that to this method. The method returns
the number of characters actually parsed by the tool before it
encountered an error (or completed successfully).
\end{funcdesc}

\begin{funcdesc}{Choose}{}
Present the user with a dialog to choose a communication tool and
configure it. If there is an outstanding connection some choices (like
selecting a different tool) may cause the connection to be
aborted. The return value (one of the \var{choose*} constants) will
indicate this.
\end{funcdesc}

\begin{funcdesc}{Idle}{}
Give the tool a chance to use the processor. You should call this
method regularly.
\end{funcdesc}

\begin{funcdesc}{Abort}{}
Abort an outstanding asynchronous \var{Open} or \var{Listen}.
\end{funcdesc}

\begin{funcdesc}{Reset}{}
Reset a connection. Exact meaning depends on the tool.
\end{funcdesc}

\begin{funcdesc}{Break}{length}
Send a break. Whether this means anything, what it means and
interpretation of the \var{length} parameter depend on the tool in
use.
\end{funcdesc}

%\section{Built-in module \sectcode{macconsole}}
\bimodindex{macconsole}

\renewcommand{\indexsubitem}{(in module macconsole)}

This module is available on the Macintosh, provided Python has been
built using the Think C compiler. It provides an interface to the
Think console package, with which basic text windows can be created.

\begin{datadesc}{options}
An object allowing you to set various options when creating windows,
see below.
\end{datadesc}

\begin{datadesc}{C_ECHO}
\dataline{C_NOECHO}
\dataline{C_CBREAK}
\dataline{C_RAW}
Options for the \code{setmode} method. \var{C_ECHO} and \var{C_CBREAK}
enable character echo, the other two disable it, \var{C_ECHO} and
\var{C_NOECHO} enable line-oriented input (erase/kill processing,
etc).
\end{datadesc}

\begin{funcdesc}{copen}{}
Open a new console window. Returns a console window object.
\end{funcdesc}

\begin{funcdesc}{fopen}{fp}
Return the console window object corresponding with the given file
object. \var{Fp} should be one of \var{sys.stdin}, \var{sys.stdout} or
\var{sys.stderr}.
\end{funcdesc}

\subsection{macconsole options object}
These options are examined when a window is created:

\renewcommand{\indexsubitem}{(macconsole option)}
\begin{datadesc}{top}
\dataline{left}
The origin of the window.
\end{datadesc}

\begin{datadesc}{nrows}
\dataline{ncols}
The size of the window.
\end{datadesc}

\begin{datadesc}{txFont}
\dataline{txSize}
\dataline{txStyle}
The font, fontsize and fontstyle to be used in the window.
\end{datadesc}

\begin{datadesc}{title}
The title of the window.
\end{datadesc}

\begin{datadesc}{pause_atexit}
If set non-zero, the window will wait for user action before closing
the window.
\end{datadesc}

\subsection{console window object}

\renewcommand{\indexsubitem}{(console window method)}

\begin{datadesc}{file}
The file object corresponding to this console window. If the file is
buffered, you should call \code{file.flush()} between \code{write()}
and \code{read()} calls.
\end{datadesc}

\begin{funcdesc}{setmode}{mode}
Set the input mode of the console to \var{C_ECHO}, etc.
\end{funcdesc}

\begin{funcdesc}{settabs}{n}
Set the tabsize to \var{n} spaces.
\end{funcdesc}

\begin{funcdesc}{cleos}{}
Clear to end-of-screen.
\end{funcdesc}

\begin{funcdesc}{cleol}{}
Clear to end-of-line.
\end{funcdesc}

\begin{funcdesc}{inverse}{onoff}
Enable inverse-video mode: characters with the high bit set are
displayed in inverse video (this disables the upper half of a
non-ascii character set).
\end{funcdesc}

\begin{funcdesc}{gotoxy}{x\, y}
Set the cursor to position \code{(x, y)}.
\end{funcdesc}

\begin{funcdesc}{hide}{}
Hide the window, remembering the contents.
\end{funcdesc}

\begin{funcdesc}{show}{}
Show the window again.
\end{funcdesc}

\begin{funcdesc}{echo2printer}{}
Copy everything written to the window to the printer as well.
\end{funcdesc}


\section{Built-in module \sectcode{macdnr}}
\bimodindex{macdnr}

This module provides an interface to the Macintosh Domain Name
Resolver. It is usually used in conjunction with the \var{mactcp} module, to
map hostnames to IP-addresses.

The \code{macdnr} module defines the following functions:

\renewcommand{\indexsubitem}{(in module macdnr)}

\begin{funcdesc}{Open}{\optional{filename}}
Open the domain name resolver extension. If \var{filename} is given it
should be the pathname of the extension, otherwise a default is
used. Normally, this call is not needed since the other calls will
open the extension automatically.
\end{funcdesc}

\begin{funcdesc}{Close}{}
Close the resolver extension. Again, not needed for normal use.
\end{funcdesc}

\begin{funcdesc}{StrToAddr}{hostname}
Look up the IP address for \var{hostname}. This call returns a dnr
result object of the ``address'' variation.
\end{funcdesc}

\begin{funcdesc}{AddrToName}{addr}
Do a reverse lookup on the 32-bit integer IP-address
\var{addr}. Returns a dnr result object of the ``address'' variation.
\end{funcdesc}

\begin{funcdesc}{AddrToStr}{addr}
Convert the 32-bit integer IP-address \var{addr} to a dotted-decimal
string. Returns the string.
\end{funcdesc}

\begin{funcdesc}{HInfo}{hostname}
Query the nameservers for a \code{HInfo} record for host
\var{hostname}. These records contain hardware and software
information about the machine in question (if they are available in
the first place). Returns a dnr result object of the ``hinfo''
variety.
\end{funcdesc}

\begin{funcdesc}{MXInfo}{domain}
Query the nameservers for a mail exchanger for \var{domain}. This is
the hostname of a host willing to accept SMTP mail for the given
domain. Returns a dnr result object of the ``mx'' variety.
\end{funcdesc}

\subsection{dnr result object}

Since the DNR calls all execute asynchronously you do not get the
results back immedeately. In stead, you get a dnr result object. You
can check this object to see whether the query is complete, and access
its attributes to obtain the information when it is.

Alternatively, you can also reference the result attributes directly,
this will result in an implicit wait for the query to complete.

The \var{rtnCode} and \var{cname} attributes are always available, the
others depend on the type of query (address, hinfo or mx).

\renewcommand{\indexsubitem}{(dnr result object method)}

% Add args, as in {arg1\, arg2 \optional{\, arg3}}
\begin{funcdesc}{wait}{}
Wait for the query to complete.
\end{funcdesc}

% Add args, as in {arg1\, arg2 \optional{\, arg3}}
\begin{funcdesc}{isdone}{}
Return 1 if the query is complete.
\end{funcdesc}

\begin{datadesc}{rtnCode}
The error code returned by the query.
\end{datadesc}

\begin{datadesc}{cname}
The canonical name of the host that was queried.
\end{datadesc}

\begin{datadesc}{ip0}
\dataline{ip1}
\dataline{ip2}
\dataline{ip3}
At most four integer IP addresses for this host. Unused entries are
zero. Valid only for address queries.
\end{datadesc}

\begin{datadesc}{cpuType}
\dataline{osType}
Textual strings giving the machine type an OS name. Valid for hinfo
queries.
\end{datadesc}

\begin{datadesc}{exchange}
The name of a mail-exchanger host. Valid for mx queries.
\end{datadesc}

\begin{datadesc}{preference}
The preference of this mx record. Not too useful, since the Macintosh
will only return a single mx record. Mx queries only.
\end{datadesc}

The simplest way to use the module to convert names to dotted-decimal
strings, without worrying about idle time, etc:
\begin{verbatim}
>>> def gethostname(name):
...     import macdnr
...     dnrr = macdnr.StrToAddr(name)
...     return macdnr.AddrToStr(dnrr.ip0)
\end{verbatim}

\section{\module{macfs} ---
         Various file system services}

\declaremodule{builtin}{macfs}
  \platform{Mac}
\modulesynopsis{Support for FSSpec, the Alias Manager,
                \program{finder} aliases, and the Standard File package.}


This module provides access to Macintosh FSSpec handling, the Alias
Manager, \program{finder} aliases and the Standard File package.
\index{Macintosh Alias Manager}
\index{Alias Manager, Macintosh}
\index{Standard File}

Whenever a function or method expects a \var{file} argument, this
argument can be one of three things:\ (1) a full or partial Macintosh
pathname, (2) an \pytype{FSSpec} object or (3) a 3-tuple \code{(\var{wdRefNum},
\var{parID}, \var{name})} as described in \citetitle{Inside
Macintosh:\ Files}. A description of aliases and the Standard File
package can also be found there.

\begin{funcdesc}{FSSpec}{file}
Create an \pytype{FSSpec} object for the specified file.
\end{funcdesc}

\begin{funcdesc}{RawFSSpec}{data}
Create an \pytype{FSSpec} object given the raw data for the \C{}
structure for the \pytype{FSSpec} as a string.  This is mainly useful
if you have obtained an \pytype{FSSpec} structure over a network.
\end{funcdesc}

\begin{funcdesc}{RawAlias}{data}
Create an \pytype{Alias} object given the raw data for the \C{}
structure for the alias as a string.  This is mainly useful if you
have obtained an \pytype{FSSpec} structure over a network.
\end{funcdesc}

\begin{funcdesc}{FInfo}{}
Create a zero-filled \pytype{FInfo} object.
\end{funcdesc}

\begin{funcdesc}{ResolveAliasFile}{file}
Resolve an alias file. Returns a 3-tuple \code{(\var{fsspec},
\var{isfolder}, \var{aliased})} where \var{fsspec} is the resulting
\pytype{FSSpec} object, \var{isfolder} is true if \var{fsspec} points
to a folder and \var{aliased} is true if the file was an alias in the
first place (otherwise the \pytype{FSSpec} object for the file itself
is returned).
\end{funcdesc}

\begin{funcdesc}{StandardGetFile}{\optional{type, ...}}
Present the user with a standard ``open input file''
dialog. Optionally, you can pass up to four 4-character file types to limit
the files the user can choose from. The function returns an \pytype{FSSpec}
object and a flag indicating that the user completed the dialog
without cancelling.
\end{funcdesc}

\begin{funcdesc}{PromptGetFile}{prompt\optional{, type, ...}}
Similar to \function{StandardGetFile()} but allows you to specify a
prompt.
\end{funcdesc}

\begin{funcdesc}{StandardPutFile}{prompt, \optional{default}}
Present the user with a standard ``open output file''
dialog. \var{prompt} is the prompt string, and the optional
\var{default} argument initializes the output file name. The function
returns an \pytype{FSSpec} object and a flag indicating that the user
completed the dialog without cancelling.
\end{funcdesc}

\begin{funcdesc}{GetDirectory}{\optional{prompt}}
Present the user with a non-standard ``select a directory''
dialog. \var{prompt} is the prompt string, and the optional.
Return an \pytype{FSSpec} object and a success-indicator.
\end{funcdesc}

\begin{funcdesc}{SetFolder}{\optional{fsspec}}
Set the folder that is initially presented to the user when one of
the file selection dialogs is presented. \var{fsspec} should point to
a file in the folder, not the folder itself (the file need not exist,
though). If no argument is passed the folder will be set to the
current directory, i.e. what \function{os.getcwd()} returns.

Note that starting with system 7.5 the user can change Standard File
behaviour with the ``general controls'' controlpanel, thereby making
this call inoperative.
\end{funcdesc}

\begin{funcdesc}{FindFolder}{where, which, create}
Locates one of the ``special'' folders that MacOS knows about, such as
the trash or the Preferences folder. \var{where} is the disk to
search, \var{which} is the 4-character string specifying which folder to
locate. Setting \var{create} causes the folder to be created if it
does not exist. Returns a \code{(\var{vrefnum}, \var{dirid})} tuple.
\end{funcdesc}

\begin{funcdesc}{NewAliasMinimalFromFullPath}{pathname}
Return a minimal \pytype{alias} object that points to the given file, which
must be specified as a full pathname. This is the only way to create an
\pytype{Alias} pointing to a non-existing file.

The constants for \var{where} and \var{which} can be obtained from the
standard module \var{MACFS}.
\end{funcdesc}

\begin{funcdesc}{FindApplication}{creator}
Locate the application with 4-char creator code \var{creator}. The
function returns an \pytype{FSSpec} object pointing to the application.
\end{funcdesc}


\subsection{FSSpec objects \label{fsspec-objects}}

\begin{memberdesc}[FSSpec]{data}
The raw data from the FSSpec object, suitable for passing
to other applications, for instance.
\end{memberdesc}

\begin{methoddesc}[FSSpec]{as_pathname}{}
Return the full pathname of the file described by the \pytype{FSSpec}
object.
\end{methoddesc}

\begin{methoddesc}[FSSpec]{as_tuple}{}
Return the \code{(\var{wdRefNum}, \var{parID}, \var{name})} tuple of
the file described by the \pytype{FSSpec} object.
\end{methoddesc}

\begin{methoddesc}[FSSpec]{NewAlias}{\optional{file}}
Create an Alias object pointing to the file described by this
FSSpec. If the optional \var{file} parameter is present the alias
will be relative to that file, otherwise it will be absolute.
\end{methoddesc}

\begin{methoddesc}[FSSpec]{NewAliasMinimal}{}
Create a minimal alias pointing to this file.
\end{methoddesc}

\begin{methoddesc}[FSSpec]{GetCreatorType}{}
Return the 4-character creator and type of the file.
\end{methoddesc}

\begin{methoddesc}[FSSpec]{SetCreatorType}{creator, type}
Set the 4-character creator and type of the file.
\end{methoddesc}

\begin{methoddesc}[FSSpec]{GetFInfo}{}
Return a \pytype{FInfo} object describing the finder info for the file.
\end{methoddesc}

\begin{methoddesc}[FSSpec]{SetFInfo}{finfo}
Set the finder info for the file to the values given as \var{finfo}
(an \pytype{FInfo} object).
\end{methoddesc}

\begin{methoddesc}[FSSpec]{GetDates}{}
Return a tuple with three floating point values representing the
creation date, modification date and backup date of the file.
\end{methoddesc}

\begin{methoddesc}[FSSpec]{SetDates}{crdate, moddate, backupdate}
Set the creation, modification and backup date of the file. The values
are in the standard floating point format used for times throughout
Python.
\end{methoddesc}


\subsection{Alias Objects \label{alias-objects}}

\begin{memberdesc}[Alias]{data}
The raw data for the Alias record, suitable for storing in a resource
or transmitting to other programs.
\end{memberdesc}

\begin{methoddesc}[Alias]{Resolve}{\optional{file}}
Resolve the alias. If the alias was created as a relative alias you
should pass the file relative to which it is. Return the FSSpec for
the file pointed to and a flag indicating whether the \pytype{Alias} object
itself was modified during the search process. If the file does
not exist but the path leading up to it does exist a valid fsspec
is returned.
\end{methoddesc}

\begin{methoddesc}[Alias]{GetInfo}{num}
An interface to the \C{} routine \cfunction{GetAliasInfo()}.
\end{methoddesc}

\begin{methoddesc}[Alias]{Update}{file, \optional{file2}}
Update the alias to point to the \var{file} given. If \var{file2} is
present a relative alias will be created.
\end{methoddesc}

Note that it is currently not possible to directly manipulate a
resource as an \pytype{Alias} object. Hence, after calling
\method{Update()} or after \method{Resolve()} indicates that the alias
has changed the Python program is responsible for getting the
\member{data} value from the \pytype{Alias} object and modifying the
resource.


\subsection{FInfo Objects \label{finfo-objects}}

See \citetitle{Inside Macintosh: Files} for a complete description of what
the various fields mean.

\begin{memberdesc}[FInfo]{Creator}
The 4-character creator code of the file.
\end{memberdesc}

\begin{memberdesc}[FInfo]{Type}
The 4-character type code of the file.
\end{memberdesc}

\begin{memberdesc}[FInfo]{Flags}
The finder flags for the file as 16-bit integer. The bit values in
\var{Flags} are defined in standard module \module{MACFS}.
\end{memberdesc}

\begin{memberdesc}[FInfo]{Location}
A Point giving the position of the file's icon in its folder.
\end{memberdesc}

\begin{memberdesc}[FInfo]{Fldr}
The folder the file is in (as an integer).
\end{memberdesc}

\section{Standard Module \sectcode{ic}}
\label{module-ic}
\bimodindex{ic}


This module provides access to Macintosh Internet Config package,
which stores preferences for Internet programs such as mail address,
default homepage, etc. Also, Internet Config contains an elaborate set
of mappings from Macintosh creator/type codes to foreign filename
extensions plus information on how to transfer files (binary, ascii,
etc).

There is a low-level companion module
\module{icglue}\refbimodindex{icglue} which provides the basic
Internet Config access functionality.  This low-level module is not
documented, but the docstrings of the routines document the parameters
and the routine names are the same as for the Pascal or \C{} API to
Internet Config, so the standard IC programmers' documentation can be
used if this module is needed.

The \module{ic} module defines the \exception{error} exception and
symbolic names for all error codes Internet Config can produce; see
the source for details.

\begin{excdesc}{error}
Exception raised on errors in the \module{ic} module.
\end{excdesc}


The \module{ic} module defines the following functions:

\begin{funcdesc}{IC}{\optional{signature\optional{, ic}}}
Create an internet config object. The signature is a 4-char creator
code of the current application (default \code{'Pyth'}) which may
influence some of ICs settings. The optional \var{ic} argument is a
low-level \code{icglue.icinstance} created beforehand, this may be
useful if you want to get preferences from a different config file,
etc.
\end{funcdesc}

\begin{funcdesc}{launchurl}{url\optional{, hint}}
\funcline{parseurl}{data\optional{, start\optional{, end\optional{, hint}}}}
\funcline{mapfile}{file}
\funcline{maptypecreator}{type, creator\optional{, filename}}
\funcline{settypecreator}{file}
These functions are ``shortcuts'' to the methods of the same name,
described below.
\end{funcdesc}


\subsection{IC objects}

IC objects have a mapping interface, hence to obtain the mail address
you simply get \code{\var{ic}['MailAddress']}. Assignment also works,
and changes the option in the configuration file.

The module knows about various datatypes, and converts the internal IC
representation to a ``logical'' Python datastructure. Running the
\module{ic} module standalone will run a test program that lists all
keys and values in your IC database, this will have to server as
documentation.

If the module does not know how to represent the data it returns an
instance of the \code{ICOpaqueData} type, with the raw data in its
\var{data} attribute. Objects of this type are also acceptable values
for assignment.

Besides the dictionary interface IC objects have the following methods:

\setindexsubitem{(IC attribute)}

\begin{funcdesc}{launchurl}{url\optional{, hint}}
Parse the given URL, lauch the correct application and pass it the
URL. The optional \var{hint} can be a scheme name such as
\code{'mailto:'}, in which case incomplete URLs are completed with this
scheme.  If \var{hint} is not provided, incomplete URLs are invalid.
\end{funcdesc}

\begin{funcdesc}{parseurl}{data\optional{, start\optional{, end\optional{, hint}}}}
Find an URL somewhere in \var{data} and return start position, end
position and the URL. The optional \var{start} and \var{end} can be
used to limit the search, so for instance if a user clicks in a long
textfield you can pass the whole textfield and the click-position in
\var{start} and this routine will return the whole URL in which the
user clicked.  \var{Hint} is again an optional scheme used to complete
incomplete URLs.
\end{funcdesc}

\begin{funcdesc}{mapfile}{file}
Return the mapping entry for the given \var{file}, which can be passed
as either a filename or an \code{macfs.FSSpec} object, and which need
not exist.

The mapping entry is returned as a tuple \code{(}\var{version},
\var{type}, \var{creator}, \var{postcreator}, \var{flags},
\var{extension}, \var{appname}, \var{postappname}, \var{mimetype},
\var{entryname}\code{)}, where \var{version} is the entry version
number, \var{type} is the 4-char filetype, \var{creator} is the 4-char
creator type, \var{postcreator} is the 4-char creator code of an
optional application to post-process the file after downloading,
\var{flags} are various bits specifying whether to transfer in binary
or ascii and such, \var{extension} is the filename extension for this
file type, \var{appname} is the printable name of the application to
which this file belongs, \var{postappname} is the name of the
postprocessing application, \var{mimetype} is the MIME type of this
file and \var{entryname} is the name of this entry.
\end{funcdesc}

\begin{funcdesc}{maptypecreator}{type, creator\optional{, filename}}
Return the mapping entry for files with given 4-char \var{type} and
\var{creator} codes. The optional \var{filename} may be specified to
further help finding the correct entry (if the creator code is
\code{'????'}, for instance).

The mapping entry is returned in the same format as for \var{mapfile}.
\end{funcdesc}

\begin{funcdesc}{settypecreator}{file}
Given an existing \var{file}, specified either as a filename or as an
\code{macfs.FSSpec} record, set its creator and type correctly based
on its extension.  The finder is told about the change, so the finder
icon will be updated quickly.
\end{funcdesc}

\section{Built-in Module \module{MacOS}}
\label{module-MacOS}
\bimodindex{MacOS}


This module provides access to MacOS specific functionality in the
Python interpreter, such as how the interpreter eventloop functions
and the like. Use with care.

Note the capitalisation of the module name, this is a historical
artifact.

\begin{excdesc}{Error}
This exception is raised on MacOS generated errors, either from
functions in this module or from other mac-specific modules like the
toolbox interfaces. The arguments are the integer error code (the
\cdata{OSErr} value) and a textual description of the error code.
Symbolic names for all known error codes are defined in the standard
module \module{macerrors}\refstmodindex{macerrors}.
\end{excdesc}

\begin{funcdesc}{SetEventHandler}{handler}
In the inner interpreter loop Python will occasionally check for events,
unless disabled with \function{ScheduleParams()}. With this function you
can pass a Python event-handler function that will be called if an event
is available. The event is passed as parameter and the function should return
non-zero if the event has been fully processed, otherwise event processing
continues (by passing the event to the console window package, for instance).

Call \function{SetEventHandler()} without a parameter to clear the
event handler. Setting an event handler while one is already set is an
error.
\end{funcdesc}

\begin{funcdesc}{SchedParams}{\optional{doint\optional{, evtmask\optional{,
                              besocial\optional{, interval\optional{,
                              bgyield}}}}}}
Influence the interpreter inner loop event handling. \var{Interval}
specifies how often (in seconds, floating point) the interpreter
should enter the event processing code. When true, \var{doint} causes
interrupt (command-dot) checking to be done. \var{evtmask} tells the
interpreter to do event processing for events in the mask (redraws,
mouseclicks to switch to other applications, etc). The \var{besocial}
flag gives other processes a chance to run. They are granted minimal
runtime when Python is in the foreground and \var{bgyield} seconds per
\var{interval} when Python runs in the background.

All parameters are optional, and default to the current value. The return
value of this function is a tuple with the old values of these options.
Initial defaults are that all processing is enabled, checking is done every
quarter second and the CPU is given up for a quarter second when in the
background.
\end{funcdesc}

\begin{funcdesc}{HandleEvent}{ev}
Pass the event record \var{ev} back to the Python event loop, or
possibly to the handler for the \code{sys.stdout} window (based on the
compiler used to build Python). This allows Python programs that do
their own event handling to still have some command-period and
window-switching capability.

If you attempt to call this function from an event handler set through
\function{SetEventHandler()} you will get an exception.
\end{funcdesc}

\begin{funcdesc}{GetErrorString}{errno}
Return the textual description of MacOS error code \var{errno}.
\end{funcdesc}

\begin{funcdesc}{splash}{resid}
This function will put a splash window
on-screen, with the contents of the DLOG resource specified by
\var{resid}. Calling with a zero argument will remove the splash
screen. This function is useful if you want an applet to post a splash screen
early in initialization without first having to load numerous
extension modules.
\end{funcdesc}

\begin{funcdesc}{DebugStr}{message \optional{, object}}
Drop to the low-level debugger with message \var{message}. The
optional \var{object} argument is not used, but can easily be
inspected from the debugger.

Note that you should use this function with extreme care: if no
low-level debugger like MacsBug is installed this call will crash your
system. It is intended mainly for developers of Python extension
modules.
\end{funcdesc}

\begin{funcdesc}{openrf}{name \optional{, mode}}
Open the resource fork of a file. Arguments are the same as for the
built-in function \function{open()}. The object returned has file-like
semantics, but it is not a Python file object, so there may be subtle
differences.
\end{funcdesc}

\section{Standard Module \sectcode{macostools}}
\label{module-macostools}
\stmodindex{macostools}

This module contains some convenience routines for file-manipulation
on the Macintosh.

The \code{macostools} module defines the following functions:

\setindexsubitem{(in module macostools)}

\begin{funcdesc}{copy}{src, dst\optional{, createpath, copytimes}}
Copy file \var{src} to \var{dst}. The files can be specified as
pathnames or \code{FSSpec} objects. If \var{createpath} is non-zero
\var{dst} must be a pathname and the folders leading to the
destination are created if necessary.  The method copies data and
resource fork and some finder information (creator, type, flags) and
optionally the creation, modification and backup times (default is to
copy them). Custom icons, comments and icon position are not copied.

If the source is an alias the original to which the alias points is
copied, not the aliasfile.
\end{funcdesc}

\begin{funcdesc}{copytree}{src, dst}
Recursively copy a file tree from \var{src} to \var{dst}, creating
folders as needed. \var{Src} and \var{dst} should be specified as
pathnames.
\end{funcdesc}

\begin{funcdesc}{mkalias}{src, dst}
Create a finder alias \var{dst} pointing to \var{src}. Both may be
specified as pathnames or \var{FSSpec} objects.
\end{funcdesc}

\begin{funcdesc}{touched}{dst}
Tell the finder that some bits of finder-information such as creator
or type for file \var{dst} has changed. The file can be specified by
pathname or fsspec. This call should prod the finder into redrawing the
files icon.
\end{funcdesc}

\begin{datadesc}{BUFSIZ}
The buffer size for \code{copy}, default 1 megabyte.
\end{datadesc}

Note that the process of creating finder aliases is not specified in
the Apple documentation. Hence, aliases created with \code{mkalias}
could conceivably have incompatible behaviour in some cases.

\section{Standard Module \sectcode{findertools}}
\label{module-findertools}
\stmodindex{findertools}

This module contains routines that give Python programs access to some
functionality provided by the finder. They are implemented as wrappers
around the AppleEvent interface to the finder.

All file and folder parameters can be specified either as full
pathnames or as \code{FSSpec} objects.

The \code{findertools} module defines the following functions:

\setindexsubitem{(in module macostools)}

\begin{funcdesc}{launch}{file}
Tell the finder to launch \var{file}. What launching means depends on the file:
applications are started, folders are opened and documents are opened
in the correct application.
\end{funcdesc}

\begin{funcdesc}{Print}{file}
Tell the finder to print a file (again specified by full pathname or
FSSpec). The behaviour is identical to selecting the file and using
the print command in the finder.
\end{funcdesc}

\begin{funcdesc}{copy}{file, destdir}
Tell the finder to copy a file or folder \var{file} to folder
\var{destdir}. The function returns an \code{Alias} object pointing to
the new file.
\end{funcdesc}

\begin{funcdesc}{move}{file, destdir}
Tell the finder to move a file or folder \var{file} to folder
\var{destdir}. The function returns an \code{Alias} object pointing to
the new file.
\end{funcdesc}

\begin{funcdesc}{sleep}{}
Tell the finder to put the mac to sleep, if your machine supports it.
\end{funcdesc}

\begin{funcdesc}{restart}{}
Tell the finder to perform an orderly restart of the machine.
\end{funcdesc}

\begin{funcdesc}{shutdown}{}
Tell the finder to perform an orderly shutdown of the machine.
\end{funcdesc}

\section{Built-in module \sectcode{mactcp}}
\bimodindex{mactcp}
\renewcommand{\indexsubitem}{(in module mactcp)}

This module provides an interface to the Macintosh TCP/IP driver
MacTCP. There is an accompanying module \var{macdnr} which provides an
interface to the name-server (allowing you to translate hostnames to
ip-addresses), a module \var{MACTCP} which has symbolic names for
constants constants used by MacTCP and a wrapper module \var{socket}
which mimics the unix socket interface (as far as possible).

A complete description of the MacTCP interface can be found in the
Apple MacTCP API documentation.

\begin{funcdesc}{MTU}{}
Return the Maximum Transmit Unit (the packet size) of the network
interface.
\end{funcdesc}

\begin{funcdesc}{IPAddr}{}
Return the 32-bit integer IP address of the network interface.
\end{funcdesc}

\begin{funcdesc}{NetMask}{}
Return the 32-bit integer network mask of the interface.
\end{funcdesc}

\begin{funcdesc}{TCPCreate}{size}
Create a TCP Stream object. \var{Size} is the size of the receive
buffer, \code{4096} is suggested by various sources.
\end{funcdesc}

\begin{funcdesc}{UDPCreate}{size, port}
Create a UDP stream object. \var{Size} is the size of the receive
buffer (and, hence, the size of the biggest datagram you can receive
on this port). \var{Port} is the UDP port number you want to receive
datagrams on, a value of zero will make MacTCP select a free port.
\end{funcdesc}

\subsection{TCP stream objects}
\renewcommand{\indexsubitem}{(TCP stream method)}

\begin{datadesc}{asr}
When set to a value different than \var{None} this should point to a
function with two integer parameters: an event code and a detail. This
function will be called upon network-generated events such as urgent
data arrival. In addition, it is called with eventcode
\var{MACTCP.PassiveOpenDone} when a \var{PassiveOpen} completes. This
is a python addition to the MacTCP semantics.
It is safe to do further calls from the asr.
\end{datadesc}

\begin{funcdesc}{PassiveOpen}{port}
Wait for an incoming connection on TCP port \var{port} (zero makes the
system pick a free port). The call returns immedeately, and you should
use \var{wait} to wait for completion. You should not issue any method
calls other than
\var{wait}, \var{isdone} or \var{GetSockName} before the call
completes.
\end{funcdesc}

\begin{funcdesc}{wait}{}
Wait for \var{PassiveOpen} to complete.
\end{funcdesc}

\begin{funcdesc}{isdone}{}
Return 1 if a \var{PassiveOpen} is completed.
\end{funcdesc}

\begin{funcdesc}{GetSockName}{}
Return the TCP address of this side of a connection as a 2-tuple
\code{(host, port)}, both integers.
\end{funcdesc}

\begin{funcdesc}{ActiveOpen}{lport\, host\, rport}
Open an outgoing connection to TCP address \code{(host, rport)}. Use
local port \var{lport} (zero makes the system pick a free port). This
call blocks until the connection is established.
\end{funcdesc}

\begin{funcdesc}{Send}{buf\, push\, urgent}
Send data \var{buf} over the connection. \var{Push} and \var{urgent}
are flags as specified by the TCP standard.
\end{funcdesc}

\begin{funcdesc}{Rcv}{timeout}
Receive data. The call returns when \var{timeout} seconds have passed
or when (according to the MacTCP documentation) ``a reasonable amount
of data has been received''. The return value is a 3-tuple
\code{(data, urgent, mark)}. If urgent data is outstanding \var{Rcv}
will always return that before looking at any normal data. The first
call returning urgent data will have the \var{urgent} flag set, the
last will have the \var{mark} flag set.
\end{funcdesc}

\begin{funcdesc}{Close}{}
Tell MacTCP that no more data will be transmitted on this
connection. The call returnes when all data has been acknowledged by
the receiving side.
\end{funcdesc}

\begin{funcdesc}{Abort}{}
Forcibly close both sides of a connection, ignoring outstanding data.
\end{funcdesc}

\begin{funcdesc}{Status}{}
Return a TCP status object for this stream.
\end{funcdesc}

\subsection{TCP status objects}
This object has no methods, only some members holding information on
the connection. A complete description of all fields in this objects
can be found in the Apple documentation. The most interesting ones are:

\renewcommand{\indexsubitem}{(TCP status method)}
\begin{datadesc}{localHost}
\dataline{localPort}
\dataline{remoteHost}
\dataline{remotePort}
The integer IP-addresses and port numbers of both endpoints of the
connection. 
\end{datadesc}

\begin{datadesc}{sendWindow}
The current window size.
\end{datadesc}

\begin{datadesc}{amtUnackedData}
The number of bytes sent but not yet acknowledged. \code{sendWindow -
amtUnackedData} is what you can pass to \code{Send} without blocking.
\end{datadesc}

\begin{datadesc}{amtUnreadData}
The number of bytes received but not yet read (what you can \var{Recv}
without blocking).
\end{datadesc}



\subsection{UDP stream objects}
Note that, unlike the name suggests, there is nothing stream-like
about UDP.

\renewcommand{\indexsubitem}{(UDP stream method)}

\begin{datadesc}{asr}
The asynchronous service routine to be called on events such as
datagram arrival without outstanding \var{Read} call. The asr has a
single argument, the event code.
\end{datadesc}

\begin{datadesc}{port}
A read-only member giving the port number of this UDP stream.
\end{datadesc}

\begin{funcdesc}{Read}{timeout}
Read a datagram, waiting at most \var{timeout} seconds (-1 is
indefinite). Returns the data.
\end{funcdesc}

\begin{funcdesc}{Write}{host\, port\, buf}
Send \var{buf} as a datagram to IP-address \var{host}, port
\var{port}.
\end{funcdesc}

\section{Built-in Module \module{macspeech}}
\declaremodule{builtin}{macspeech}

\modulesynopsis{Interface to the Macintosh Speech Manager.}



This module provides an interface to the Macintosh Speech Manager,
\index{Macintosh Speech Manager}
\index{Speech Manager, Macintosh}
allowing you to let the Macintosh utter phrases. You need a version of
the Speech Manager extension (version 1 and 2 have been tested) in
your \file{Extensions} folder for this to work. The module does not
provide full access to all features of the Speech Manager yet.  It may
not be available in all Mac Python versions.

\begin{funcdesc}{Available}{}
Test availability of the Speech Manager extension (and, on the
PowerPC, the Speech Manager shared library). Return \code{0} or
\code{1}.
\end{funcdesc}

\begin{funcdesc}{Version}{}
Return the (integer) version number of the Speech Manager.
\end{funcdesc}

\begin{funcdesc}{SpeakString}{str}
Utter the string \var{str} using the default voice,
asynchronously. This aborts any speech that may still be active from
prior \function{SpeakString()} invocations.
\end{funcdesc}

\begin{funcdesc}{Busy}{}
Return the number of speech channels busy, system-wide.
\end{funcdesc}

\begin{funcdesc}{CountVoices}{}
Return the number of different voices available.
\end{funcdesc}

\begin{funcdesc}{GetIndVoice}{num}
Return a \pytype{Voice} object for voice number \var{num}.
\end{funcdesc}

\subsection{Voice Objects}
\label{voice-objects}

Voice objects contain the description of a voice. It is currently not
yet possible to access the parameters of a voice.

\setindexsubitem{(voice object method)}

\begin{methoddesc}[Voice]{GetGender}{}
Return the gender of the voice: \code{0} for male, \code{1} for female
and \code{-1} for neuter.
\end{methoddesc}

\begin{methoddesc}[Voice]{NewChannel}{}
Return a new Speech Channel object using this voice.
\end{methoddesc}

\subsection{Speech Channel Objects}
\label{speech-channel-objects}

A Speech Channel object allows you to speak strings with slightly more
control than \function{SpeakString()}, and allows you to use multiple
speakers at the same time. Please note that channel pitch and rate are
interrelated in some way, so that to make your Macintosh sing you will
have to adjust both.

\begin{methoddesc}[Speech Channel]{SpeakText}{str}
Start uttering the given string.
\end{methoddesc}

\begin{methoddesc}[Speech Channel]{Stop}{}
Stop babbling.
\end{methoddesc}

\begin{methoddesc}[Speech Channel]{GetPitch}{}
Return the current pitch of the channel, as a floating-point number.
\end{methoddesc}

\begin{methoddesc}[Speech Channel]{SetPitch}{pitch}
Set the pitch of the channel.
\end{methoddesc}

\begin{methoddesc}[Speech Channel]{GetRate}{}
Get the speech rate (utterances per minute) of the channel as a
floating point number.
\end{methoddesc}

\begin{methoddesc}[Speech Channel]{SetRate}{rate}
Set the speech rate of the channel.
\end{methoddesc}


\section{Standard Module \sectcode{EasyDialogs}}
\label{module-EasyDialogs}
\stmodindex{EasyDialogs}

The \code{EasyDialogs} module contains some simple dialogs for
the Macintosh, modelled after the \code{stdwin} dialogs with similar
names. All routines have an optional parameter \var{id} with which you
can override the DLOG resource used for the dialog, as long as the
item numbers correspond. See the source for details.

The \code{EasyDialogs} module defines the following functions:

\setindexsubitem{(in module EasyDialogs)}

\begin{funcdesc}{Message}{str}
A modal dialog with the message text \var{str}, which should be at
most 255 characters long, is displayed. Control is returned when the
user clicks ``OK''.
\end{funcdesc}

\begin{funcdesc}{AskString}{prompt\optional{, default}}
Ask the user to input a string value, in a modal dialog. \var{Prompt}
is the promt message, the optional \var{default} arg is the initial
value for the string. All strings can be at most 255 bytes
long. \var{AskString} returns the string entered or \code{None} in
case the user cancelled.
\end{funcdesc}

\begin{funcdesc}{AskYesNoCancel}{question\optional{, default}}
Present a dialog with text \var{question} and three buttons labelled
``yes'', ``no'' and ``cancel''. Return \code{1} for yes, \code{0} for
no and \code{-1} for cancel. The default return value chosen by
hitting return is \code{0}. This can be changed with the optional
\var{default} argument.
\end{funcdesc}

\begin{funcdesc}{ProgressBar}{\optional{label, maxval}}
Display a modeless progress dialog with a thermometer bar. \var{Label}
is the textstring displayed (default ``Working...''), \var{maxval} is
the value at which progress is complete (default 100). The returned
object has one method, \code{set(value)}, which sets the value of the
progress bar. The bar remains visible until the object returned is
discarded.

The progress bar has a ``cancel'' button, but it is currently
non-functional.
\end{funcdesc}

Note that \code{EasyDialogs} does not currently use the notification
manager. This means that displaying dialogs while the program is in
the background will lead to unexpected results and possibly
crashes. Also, all dialogs are modeless and hence expect to be at the
top of the stacking order. This is true when the dialogs are created,
but windows that pop-up later (like a console window) may also result
in crashes.

\section{\module{FrameWork} ---
         Interactive application framework}

\declaremodule{standard}{FrameWork}
  \platform{Mac}
\modulesynopsis{Interactive application framework.}


The \module{FrameWork} module contains classes that together provide a
framework for an interactive Macintosh application. The programmer
builds an application by creating subclasses that override various
methods of the bases classes, thereby implementing the functionality
wanted. Overriding functionality can often be done on various
different levels, i.e. to handle clicks in a single dialog window in a
non-standard way it is not necessary to override the complete event
handling.

The \module{FrameWork} is still very much work-in-progress, and the
documentation describes only the most important functionality, and not
in the most logical manner at that. Examine the source or the examples
for more details.  The following are some comments posted on the
MacPython newsgroup about the strengths and limitations of
\module{FrameWork}:

\begin{quotation}
The strong point of \module{FrameWork} is that it allows you to break
into the control-flow at many different places. \refmodule{W}, for
instance, uses a different way to enable/disable menus and that plugs
right in leaving the rest intact.  The weak points of
\module{FrameWork} are that it has no abstract command interface (but
that shouldn't be difficult), that it's dialog support is minimal and
that it's control/toolbar support is non-existent.
\end{quotation}


The \module{FrameWork} module defines the following functions:


\begin{funcdesc}{Application}{}
An object representing the complete application. See below for a
description of the methods. The default \method{__init__()} routine
creates an empty window dictionary and a menu bar with an apple menu.
\end{funcdesc}

\begin{funcdesc}{MenuBar}{}
An object representing the menubar. This object is usually not created
by the user.
\end{funcdesc}

\begin{funcdesc}{Menu}{bar, title\optional{, after}}
An object representing a menu. Upon creation you pass the
\code{MenuBar} the menu appears in, the \var{title} string and a
position (1-based) \var{after} where the menu should appear (default:
at the end).
\end{funcdesc}

\begin{funcdesc}{MenuItem}{menu, title\optional{, shortcut, callback}}
Create a menu item object. The arguments are the menu to create, the
item title string and optionally the keyboard shortcut
and a callback routine. The callback is called with the arguments
menu-id, item number within menu (1-based), current front window and
the event record.

Instead of a callable object the callback can also be a string. In
this case menu selection causes the lookup of a method in the topmost
window and the application. The method name is the callback string
with \code{'domenu_'} prepended.

Calling the \code{MenuBar} \method{fixmenudimstate()} method sets the
correct dimming for all menu items based on the current front window.
\end{funcdesc}

\begin{funcdesc}{Separator}{menu}
Add a separator to the end of a menu.
\end{funcdesc}

\begin{funcdesc}{SubMenu}{menu, label}
Create a submenu named \var{label} under menu \var{menu}. The menu
object is returned.
\end{funcdesc}

\begin{funcdesc}{Window}{parent}
Creates a (modeless) window. \var{Parent} is the application object to
which the window belongs. The window is not displayed until later.
\end{funcdesc}

\begin{funcdesc}{DialogWindow}{parent}
Creates a modeless dialog window.
\end{funcdesc}

\begin{funcdesc}{windowbounds}{width, height}
Return a \code{(\var{left}, \var{top}, \var{right}, \var{bottom})}
tuple suitable for creation of a window of given width and height. The
window will be staggered with respect to previous windows, and an
attempt is made to keep the whole window on-screen. However, the window will
however always be the exact size given, so parts may be offscreen.
\end{funcdesc}

\begin{funcdesc}{setwatchcursor}{}
Set the mouse cursor to a watch.
\end{funcdesc}

\begin{funcdesc}{setarrowcursor}{}
Set the mouse cursor to an arrow.
\end{funcdesc}


\subsection{Application Objects \label{application-objects}}

Application objects have the following methods, among others:


\begin{methoddesc}[Application]{makeusermenus}{}
Override this method if you need menus in your application. Append the
menus to the attribute \member{menubar}.
\end{methoddesc}

\begin{methoddesc}[Application]{getabouttext}{}
Override this method to return a text string describing your
application.  Alternatively, override the \method{do_about()} method
for more elaborate ``about'' messages.
\end{methoddesc}

\begin{methoddesc}[Application]{mainloop}{\optional{mask\optional{, wait}}}
This routine is the main event loop, call it to set your application
rolling. \var{Mask} is the mask of events you want to handle,
\var{wait} is the number of ticks you want to leave to other
concurrent application (default 0, which is probably not a good
idea). While raising \var{self} to exit the mainloop is still
supported it is not recommended: call \code{self._quit()} instead.

The event loop is split into many small parts, each of which can be
overridden. The default methods take care of dispatching events to
windows and dialogs, handling drags and resizes, Apple Events, events
for non-FrameWork windows, etc.

In general, all event handlers should return \code{1} if the event is fully
handled and \code{0} otherwise (because the front window was not a FrameWork
window, for instance). This is needed so that update events and such
can be passed on to other windows like the Sioux console window.
Calling \function{MacOS.HandleEvent()} is not allowed within
\var{our_dispatch} or its callees, since this may result in an
infinite loop if the code is called through the Python inner-loop
event handler.
\end{methoddesc}

\begin{methoddesc}[Application]{asyncevents}{onoff}
Call this method with a nonzero parameter to enable
asynchronous event handling. This will tell the inner interpreter loop
to call the application event handler \var{async_dispatch} whenever events
are available. This will cause FrameWork window updates and the user
interface to remain working during long computations, but will slow the
interpreter down and may cause surprising results in non-reentrant code
(such as FrameWork itself). By default \var{async_dispatch} will immedeately
call \var{our_dispatch} but you may override this to handle only certain
events asynchronously. Events you do not handle will be passed to Sioux
and such.

The old on/off value is returned.
\end{methoddesc}

\begin{methoddesc}[Application]{_quit}{}
Terminate the running \method{mainloop()} call at the next convenient
moment.
\end{methoddesc}

\begin{methoddesc}[Application]{do_char}{c, event}
The user typed character \var{c}. The complete details of the event
can be found in the \var{event} structure. This method can also be
provided in a \code{Window} object, which overrides the
application-wide handler if the window is frontmost.
\end{methoddesc}

\begin{methoddesc}[Application]{do_dialogevent}{event}
Called early in the event loop to handle modeless dialog events. The
default method simply dispatches the event to the relevant dialog (not
through the \code{DialogWindow} object involved). Override if you
need special handling of dialog events (keyboard shortcuts, etc).
\end{methoddesc}

\begin{methoddesc}[Application]{idle}{event}
Called by the main event loop when no events are available. The
null-event is passed (so you can look at mouse position, etc).
\end{methoddesc}


\subsection{Window Objects \label{window-objects}}

Window objects have the following methods, among others:

\setindexsubitem{(Window method)}

\begin{methoddesc}[Window]{open}{}
Override this method to open a window. Store the MacOS window-id in
\member{self.wid} and call the \method{do_postopen()} method to
register the window with the parent application.
\end{methoddesc}

\begin{methoddesc}[Window]{close}{}
Override this method to do any special processing on window
close. Call the \method{do_postclose()} method to cleanup the parent
state.
\end{methoddesc}

\begin{methoddesc}[Window]{do_postresize}{width, height, macoswindowid}
Called after the window is resized. Override if more needs to be done
than calling \code{InvalRect}.
\end{methoddesc}

\begin{methoddesc}[Window]{do_contentclick}{local, modifiers, event}
The user clicked in the content part of a window. The arguments are
the coordinates (window-relative), the key modifiers and the raw
event.
\end{methoddesc}

\begin{methoddesc}[Window]{do_update}{macoswindowid, event}
An update event for the window was received. Redraw the window.
\end{methoddesc}

\begin{methoddesc}{do_activate}{activate, event}
The window was activated (\code{\var{activate} == 1}) or deactivated
(\code{\var{activate} == 0}). Handle things like focus highlighting,
etc.
\end{methoddesc}


\subsection{ControlsWindow Object \label{controlswindow-object}}

ControlsWindow objects have the following methods besides those of
\code{Window} objects:


\begin{methoddesc}[ControlsWindow]{do_controlhit}{window, control,
                                                  pcode, event}
Part \var{pcode} of control \var{control} was hit by the
user. Tracking and such has already been taken care of.
\end{methoddesc}


\subsection{ScrolledWindow Object \label{scrolledwindow-object}}

ScrolledWindow objects are ControlsWindow objects with the following
extra methods:


\begin{methoddesc}[ScrolledWindow]{scrollbars}{\optional{wantx\optional{,
                                               wanty}}}
Create (or destroy) horizontal and vertical scrollbars. The arguments
specify which you want (default: both). The scrollbars always have
minimum \code{0} and maximum \code{32767}.
\end{methoddesc}

\begin{methoddesc}[ScrolledWindow]{getscrollbarvalues}{}
You must supply this method. It should return a tuple \code{(\var{x},
\var{y})} giving the current position of the scrollbars (between
\code{0} and \code{32767}). You can return \code{None} for either to
indicate the whole document is visible in that direction.
\end{methoddesc}

\begin{methoddesc}[ScrolledWindow]{updatescrollbars}{}
Call this method when the document has changed. It will call
\method{getscrollbarvalues()} and update the scrollbars.
\end{methoddesc}

\begin{methoddesc}[ScrolledWindow]{scrollbar_callback}{which, what, value}
Supplied by you and called after user interaction. \var{which} will
be \code{'x'} or \code{'y'}, \var{what} will be \code{'-'},
\code{'--'}, \code{'set'}, \code{'++'} or \code{'+'}. For
\code{'set'}, \var{value} will contain the new scrollbar position.
\end{methoddesc}

\begin{methoddesc}[ScrolledWindow]{scalebarvalues}{absmin, absmax,
                                                   curmin, curmax}
Auxiliary method to help you calculate values to return from
\method{getscrollbarvalues()}. You pass document minimum and maximum value
and topmost (leftmost) and bottommost (rightmost) visible values and
it returns the correct number or \code{None}.
\end{methoddesc}

\begin{methoddesc}[ScrolledWindow]{do_activate}{onoff, event}
Takes care of dimming/highlighting scrollbars when a window becomes
frontmost. If you override this method, call this one at the end of
your method.
\end{methoddesc}

\begin{methoddesc}[ScrolledWindow]{do_postresize}{width, height, window}
Moves scrollbars to the correct position. Call this method initially
if you override it.
\end{methoddesc}

\begin{methoddesc}[ScrolledWindow]{do_controlhit}{window, control,
                                                  pcode, event}
Handles scrollbar interaction. If you override it call this method
first, a nonzero return value indicates the hit was in the scrollbars
and has been handled.
\end{methoddesc}


\subsection{DialogWindow Objects \label{dialogwindow-objects}}

DialogWindow objects have the following methods besides those of
\code{Window} objects:


\begin{methoddesc}[DialogWindow]{open}{resid}
Create the dialog window, from the DLOG resource with id
\var{resid}. The dialog object is stored in \member{self.wid}.
\end{methoddesc}

\begin{methoddesc}[DialogWindow]{do_itemhit}{item, event}
Item number \var{item} was hit. You are responsible for redrawing
toggle buttons, etc.
\end{methoddesc}

\section{Standard Module \sectcode{MiniAEFrame}}
\stmodindex{MiniAEFrame}
\label{module-MiniAEFrame}

The module \var{MiniAEFrame} provides a framework for an application
that can function as an OSA server, i.e. receive and process
AppleEvents. It can be used in conjunction with \var{FrameWork} or
standalone.

This module is temporary, it will eventually be replaced by a module
that handles argument names better and possibly automates making your
application scriptable.

The \var{MiniAEFrame} module defines the following classes:

\setindexsubitem{(in module MiniAEFrame)}

\begin{funcdesc}{AEServer}{}
A class that handles AppleEvent dispatch. Your application should
subclass this class together with either
\code{MiniAEFrame.MiniApplication} or
\code{FrameWork.Application}. Your \code{__init__} method should call
the \code{__init__} method for both classes.
\end{funcdesc}

\begin{funcdesc}{MiniApplication}{}
A class that is more or less compatible with
\code{FrameWork.Application} but with less functionality. Its
eventloop supports the apple menu, command-dot and AppleEvents, other
events are passed on to the Python interpreter and/or Sioux.
Useful if your application wants to use \code{AEServer} but does not
provide its own windows, etc.
\end{funcdesc}

\subsection{AEServer Objects}

\setindexsubitem{(AEServer method)}

\begin{funcdesc}{installaehandler}{classe\, type\, callback}
Installs an AppleEvent handler. \code{Classe} and \code{type} are the
four-char OSA Class and Type designators, \code{'****'} wildcards are
allowed. When a matching AppleEvent is received the parameters are
decoded and your callback is invoked.
\end{funcdesc}

\begin{funcdesc}{callback}{_object\, **kwargs}
Your callback is called with the OSA Direct Object as first positional
parameter. The other parameters are passed as keyword arguments, with
the 4-char designator as name. Three extra keyword parameters are
passed: \code{_class} and \code{_type} are the Class and Type
designators and \code{_attributes} is a dictionary with the AppleEvent
attributes.

The return value of your method is packed with
\code{aetools.packevent} and sent as reply.
\end{funcdesc}

Note that there are some serious problems with the current
design. AppleEvents which have non-identifier 4-char designators for
arguments are not implementable, and it is not possible to return an
error to the originator. This will be addressed in a future release.

\section{\module{aepack} ---
         Conversion between Python variables and AppleEvent data containers}

\declaremodule{standard}{aepack}
  \platform{Mac}
%\moduleauthor{Jack Jansen?}{email}
\modulesynopsis{Conversion between Python variables and AppleEvent
                data containers.}
\sectionauthor{Vincent Marchetti}{vincem@en.com}


The \module{aepack} module defines functions for converting (packing)
Python variables to AppleEvent descriptors and back (unpacking).
Within Python the AppleEvent descriptor is handled by Python objects
of built-in type \class{AEDesc}, defined in module \refmodule{AE}.

The \module{aepack} module defines the following functions:


\begin{funcdesc}{pack}{x\optional{, forcetype}}
Returns an \class{AEDesc} object  containing a conversion of Python
value x. If \var{forcetype} is provided it specifies the descriptor
type of the result. Otherwise, a default mapping of Python types to
Apple Event descriptor types is used, as follows:

\begin{tableii}{l|l}{textrm}{Python type}{descriptor type}
  \lineii{\class{FSSpec}}{typeFSS}
  \lineii{\class{FSRef}}{typeFSRef}
  \lineii{\class{Alias}}{typeAlias}
  \lineii{integer}{typeLong (32 bit integer)}
  \lineii{float}{typeFloat (64 bit floating point)}
  \lineii{string}{typeText}
  \lineii{unicode}{typeUnicodeText}
  \lineii{list}{typeAEList}
  \lineii{dictionary}{typeAERecord}
  \lineii{instance}{\emph{see below}}
\end{tableii}  
 
If \var{x} is a Python instance then this function attempts to call an
\method{__aepack__()} method.  This method should return an
\class{AE.AEDesc} object.

If the conversion \var{x} is not defined above, this function returns
the Python string representation of a value (the repr() function)
encoded as a text descriptor.
\end{funcdesc}

\begin{funcdesc}{unpack}{x\optional{, formodulename}}
  \var{x} must be an object of type \class{AEDesc}. This function
  returns a Python object representation of the data in the Apple
  Event descriptor \var{x}. Simple AppleEvent data types (integer,
  text, float) are returned as their obvious Python counterparts.
  Apple Event lists are returned as Python lists, and the list
  elements are recursively unpacked.  Object references
  (ex. \code{line 3 of document 1}) are returned as instances of
  \class{aetypes.ObjectSpecifier}, unless \code{formodulename}
  is specified.  AppleEvent descriptors with
  descriptor type typeFSS are returned as \class{FSSpec}
  objects.  AppleEvent record descriptors are returned as Python
  dictionaries, with 4-character string keys and elements recursively
  unpacked.
  
  The optional \code{formodulename} argument is used by the stub packages
  generated by \module{gensuitemodule}, and ensures that the OSA classes
  for object specifiers are looked up in the correct module. This ensures
  that if, say, the Finder returns an object specifier for a window
  you get an instance of \code{Finder.Window} and not a generic
  \code{aetypes.Window}. The former knows about all the properties
  and elements a window has in the Finder, while the latter knows
  no such things.
\end{funcdesc}


\begin{seealso}
  \seemodule{Carbon.AE}{Built-in access to Apple Event Manager routines.}
  \seemodule{aetypes}{Python definitions of codes for Apple Event
                      descriptor types.}
  \seetitle[http://developer.apple.com/techpubs/mac/IAC/IAC-2.html]{
            Inside Macintosh: Interapplication
            Communication}{Information about inter-process
            communications on the Macintosh.}
\end{seealso}

\section{\module{aetypes} ---
         AppleEvent objects}

\declaremodule{standard}{aetypes}
  \platform{Mac}
%\moduleauthor{Jack Jansen?}{email}
\modulesynopsis{Python representation of the Apple Event Object Model.}
\sectionauthor{Vincent Marchetti}{vincem@en.com}


The \module{aetypes} defines classes used to represent Apple Event data
descriptors and Apple Event object specifiers.

Apple Event data is contained in descriptors, and these descriptors
are typed. For many descriptors the Python representation is simply the
corresponding Python type: \code{typeText} in OSA is a Python string,
\code{typeFloat} is a float, etc. For OSA types that have no direct
Python counterpart this module declares classes. Packing and unpacking
instances of these classes is handled automatically by \module{aepack}.

An object specifier is essentially an address of an object implemented
in a Apple Event server. An Apple Event specifier is used as the direct
object for an Apple Event or as the argument of an optional parameter.
The \module{aetypes} module contains the base classes for OSA classes
and properties, which are used by the packages generated by
\module{gensuitemodule} to populate the classes and properties in a
given suite.

For reasons of backward compatibility, and for cases where you need to
script an application for which you have not generated the stub package
this module also contains object specifiers for a number of common OSA
classes such as \code{Document}, \code{Window}, \code{Character}, etc.



The \module{AEObjects} module defines the following classes to represent
Apple Event descriptor data:

\begin{classdesc}{Unknown}{type, data}
The representation of OSA descriptor data for which the \module{aepack}
and \module{aetypes} modules have no support, i.e. anything that is not
represented by the other classes here and that is not equivalent to a
simple Python value.
\end{classdesc}

\begin{classdesc}{Enum}{enum}
An enumeration value with the given 4-character string value.
\end{classdesc}

\begin{classdesc}{InsertionLoc}{of, pos}
Position \code{pos} in object \code{of}.
\end{classdesc}

\begin{classdesc}{Boolean}{bool}
A boolean.
\end{classdesc}

\begin{classdesc}{StyledText}{style, text}
Text with style information (font, face, etc) included.
\end{classdesc}

\begin{classdesc}{AEText}{script, style, text}
Text with script system and style information included.
\end{classdesc}

\begin{classdesc}{IntlText}{script, language, text}
Text with script system and language information included.
\end{classdesc}

\begin{classdesc}{IntlWritingCode}{script, language}
Script system and language information.
\end{classdesc}

\begin{classdesc}{QDPoint}{v, h}
A quickdraw point.
\end{classdesc}

\begin{classdesc}{QDRectangle}{v0, h0, v1, h1}
A quickdraw rectangle.
\end{classdesc}

\begin{classdesc}{RGBColor}{r, g, b}
A color.
\end{classdesc}

\begin{classdesc}{Type}{type}
An OSA type value with the given 4-character name.
\end{classdesc}

\begin{classdesc}{Keyword}{name}
An OSA keyword with the given 4-character name.
\end{classdesc}

\begin{classdesc}{Range}{start, stop}
A range.
\end{classdesc}

\begin{classdesc}{Ordinal}{abso}
Non-numeric absolute positions, such as \code{"firs"}, first, or \code{"midd"},
middle.
\end{classdesc}

\begin{classdesc}{Logical}{logc, term}
The logical expression of applying operator \code{logc} to
\code{term}.
\end{classdesc}

\begin{classdesc}{Comparison}{obj1, relo, obj2}
The comparison \code{relo} of \code{obj1} to \code{obj2}.
\end{classdesc}

The following classes are used as base classes by the generated stub
packages to represent AppleScript classes and properties in Python:

\begin{classdesc}{ComponentItem}{which\optional{, fr}}
Abstract baseclass for an OSA class. The subclass should set the class
attribute \code{want} to the 4-character OSA class code. Instances of
subclasses of this class are equivalent to AppleScript Object
Specifiers. Upon instantiation you should pass a selector in
\code{which}, and optionally a parent object in \code{fr}.
\end{classdesc}

\begin{classdesc}{NProperty}{fr}
Abstract baseclass for an OSA property. The subclass should set the class
attributes \code{want} and \code{which} to designate which property we
are talking about. Instances of subclasses of this class are Object
Specifiers.
\end{classdesc}

\begin{classdesc}{ObjectSpecifier}{want, form, seld\optional{, fr}}
Base class of \code{ComponentItem} and \code{NProperty}, a general
OSA Object Specifier. See the Apple Open Scripting Architecture
documentation for the parameters. Note that this class is not abstract.
\end{classdesc}



\chapter{MacOS Toolbox Modules \label{toolbox}}

There are a set of modules that provide interfaces to various MacOS
toolboxes.  If applicable the module will define a number of Python
objects for the various structures declared by the toolbox, and
operations will be implemented as methods of the object.  Other
operations will be implemented as functions in the module.  Not all
operations possible in C will also be possible in Python (callbacks
are often a problem), and parameters will occasionally be different in
Python (input and output buffers, especially).  All methods and
functions have a \member{__doc__} string describing their arguments
and return values, and for additional description you are referred to
\citetitle[http://developer.apple.com/documentation/macos8/mac8.html]{Inside
Macintosh} or similar works.

These modules all live in a package called \module{Carbon}. Despite that name
they are not all part of the Carbon framework: CF is really in the CoreFoundation
framework and Qt is in the QuickTime framework.
The normal use pattern is

\begin{verbatim}
from Carbon import AE
\end{verbatim}

\strong{Warning!}  These modules are not yet documented.  If you
wish to contribute documentation of any of these modules, please get
in touch with \email{docs@python.org}.

\localmoduletable


%\section{Argument Handling for Toolbox Modules}


\section{\module{Carbon.AE} --- Apple Events}
\declaremodule{standard}{Carbon.AE}
  \platform{Mac}
\modulesynopsis{Interface to the Apple Events toolbox.}

\section{\module{Carbon.AH} --- Apple Help}
\declaremodule{standard}{Carbon.AH}
  \platform{Mac}
\modulesynopsis{Interface to the Apple Help manager.}


\section{\module{Carbon.App} --- Appearance Manager}
\declaremodule{standard}{Carbon.App}
  \platform{Mac}
\modulesynopsis{Interface to the Appearance Manager.}


\section{\module{Carbon.CF} --- Core Foundation}
\declaremodule{standard}{Carbon.CF}
  \platform{Mac}
\modulesynopsis{Interface to the Core Foundation.}

The
\code{CFBase}, \code{CFArray}, \code{CFData}, \code{CFDictionary},
\code{CFString} and \code{CFURL} objects are supported, some
only partially.

\section{\module{Carbon.CG} --- Core Graphics}
\declaremodule{standard}{Carbon.CG}
  \platform{Mac}
\modulesynopsis{Interface to the Component Manager.}

\section{\module{Carbon.CarbonEvt} --- Carbon Event Manager}
\declaremodule{standard}{Carbon.CaronEvt}
  \platform{Mac}
\modulesynopsis{Interface to the Carbon Event Manager.}

\section{\module{Carbon.Cm} --- Component Manager}
\declaremodule{standard}{Carbon.Cm}
  \platform{Mac}
\modulesynopsis{Interface to the Component Manager.}


\section{\module{Carbon.Ctl} --- Control Manager}
\declaremodule{standard}{Carbon.Ctl}
  \platform{Mac}
\modulesynopsis{Interface to the Control Manager.}


\section{\module{Carbon.Dlg} --- Dialog Manager}
\declaremodule{standard}{Carbon.Dlg}
  \platform{Mac}
\modulesynopsis{Interface to the Dialog Manager.}


\section{\module{Carbon.Evt} --- Event Manager}
\declaremodule{standard}{Carbon.Evt}
  \platform{Mac}
\modulesynopsis{Interface to the classic Event Manager.}


\section{\module{Carbon.Fm} --- Font Manager}
\declaremodule{standard}{Carbon.Fm}
  \platform{Mac}
\modulesynopsis{Interface to the Font Manager.}

\section{\module{Carbon.Folder} --- Folder Manager}
\declaremodule{standard}{Carbon.Folder}
  \platform{Mac}
\modulesynopsis{Interface to the Folder Manager.}


\section{\module{Carbon.Help} --- Help Manager}
\declaremodule{standard}{Carbon.Help}
  \platform{Mac}
\modulesynopsis{Interface to the Carbon Help Manager.}

\section{\module{Carbon.List} --- List Manager}
\declaremodule{standard}{Carbon.List}
  \platform{Mac}
\modulesynopsis{Interface to the List Manager.}


\section{\module{Carbon.Menu} --- Menu Manager}
\declaremodule{standard}{Carbon.Menu}
  \platform{Mac}
\modulesynopsis{Interface to the Menu Manager.}


\section{\module{Carbon.Mlte} --- MultiLingual Text Editor}
\declaremodule{standard}{Carbon.Mlte}
  \platform{Mac}
\modulesynopsis{Interface to the MultiLingual Text Editor.}


\section{\module{Carbon.Qd} --- QuickDraw}
\declaremodule{builtin}{Carbon.Qd}
  \platform{Mac}
\modulesynopsis{Interface to the QuickDraw toolbox.}


\section{\module{Carbon.Qdoffs} --- QuickDraw Offscreen}
\declaremodule{builtin}{Carbon.Qdoffs}
  \platform{Mac}
\modulesynopsis{Interface to the QuickDraw Offscreen APIs.}


\section{\module{Carbon.Qt} --- QuickTime}
\declaremodule{standard}{Carbon.Qt}
  \platform{Mac}
\modulesynopsis{Interface to the QuickTime toolbox.}


\section{\module{Carbon.Res} --- Resource Manager and Handles}
\declaremodule{standard}{Carbon.Res}
  \platform{Mac}
\modulesynopsis{Interface to the Resource Manager and Handles.}

\section{\module{Carbon.Scrap} --- Scrap Manager}
\declaremodule{standard}{Carbon.Scrap}
  \platform{Mac}
\modulesynopsis{Interface to the Carbon Scrap Manager.}

\section{\module{Carbon.Snd} --- Sound Manager}
\declaremodule{standard}{Carbon.Snd}
  \platform{Mac}
\modulesynopsis{Interface to the Sound Manager.}


\section{\module{Carbon.TE} --- TextEdit}
\declaremodule{standard}{Carbon.TE}
  \platform{Mac}
\modulesynopsis{Interface to TextEdit.}


\section{\module{Carbon.Win} --- Window Manager}
\declaremodule{standard}{Carbon.Win}
  \platform{Mac}
\modulesynopsis{Interface to the Window Manager.}
                         % MacOS Toolbox Modules
\section{\module{ColorPicker} ---
         Color selection dialog}

\declaremodule{extension}{ColorPicker}
  \platform{Mac}
\modulesynopsis{}
\moduleauthor{Just van Rossum}{just@letterror.com}
\sectionauthor{Fred L. Drake, Jr.}{fdrake@acm.org}


The \module{ColorPicker} module provides access to the standard color
picker dialog.


\begin{funcdesc}{GetColor}{prompt, rgb}
  Show a standard color selection dialog and allow the user to select
  a color.  The user is given instruction by the \var{prompt} string,
  and the default color is set to \var{rgb}.  \var{rgb} must be a
  tuple giving the red, green, and blue components of the color.
  \function{GetColor()} returns a tuple giving the user's selected
  color and a flag indicating whether they accepted the selection of
  cancelled.
\end{funcdesc}


\chapter{Undocumented Modules \label{undocumented-modules}}


The modules in this chapter are poorly documented (if at all).  If you
wish to contribute documentation of any of these modules, please get in
touch with
\ulink{\email{docs@python.org}}{mailto:docs@python.org}.

\localmoduletable


\section{\module{applesingle} --- AppleSingle decoder}
\declaremodule{standard}{applesingle}
  \platform{Mac}
\modulesynopsis{Rudimentary decoder for AppleSingle format files.}


\section{\module{buildtools} --- Helper module for BuildApplet and Friends}
\declaremodule{standard}{buildtools}
  \platform{Mac}
\modulesynopsis{Helper module for BuildApplet, BuildApplication and
                macfreeze.}

\deprecated{2.4}{}

\section{\module{cfmfile} --- Code Fragment Resource module}
\declaremodule{standard}{cfmfile}
  \platform{Mac}
\modulesynopsis{Code Fragment Resource module.}

\module{cfmfile} is a module that understands Code Fragments and the
accompanying ``cfrg'' resources. It can parse them and merge them, and is
used by BuildApplication to combine all plugin modules to a single
executable.

\deprecated{2.4}{}

\section{\module{icopen} --- Internet Config replacement for \method{open()}}
\declaremodule{standard}{icopen}
  \platform{Mac}
\modulesynopsis{Internet Config replacement for \method{open()}.}

Importing \module{icopen} will replace the builtin \method{open()}
with a version that uses Internet Config to set file type and creator
for new files.


\section{\module{macerrors} --- Mac OS Errors}
\declaremodule{standard}{macerrors}
  \platform{Mac}
\modulesynopsis{Constant definitions for many Mac OS error codes.}

\module{macerrors} contains constant definitions for many Mac OS error
codes.


\section{\module{macresource} --- Locate script resources}
\declaremodule{standard}{macresource}
  \platform{Mac}
\modulesynopsis{Locate script resources.}

\module{macresource} helps scripts finding their resources, such as
dialogs and menus, without requiring special case code for when the
script is run under MacPython, as a MacPython applet or under OSX Python.

\section{\module{Nav} --- NavServices calls}
\declaremodule{standard}{Nav}
  \platform{Mac}
\modulesynopsis{Interface to Navigation Services.}

A low-level interface to Navigation Services. 

\section{\module{PixMapWrapper} --- Wrapper for PixMap objects}
\declaremodule{standard}{PixMapWrapper}
  \platform{Mac}
\modulesynopsis{Wrapper for PixMap objects.}

\module{PixMapWrapper} wraps a PixMap object with a Python object that
allows access to the fields by name. It also has methods to convert
to and from \module{PIL} images.

\section{\module{videoreader} --- Read QuickTime movies}
\declaremodule{standard}{videoreader}
  \platform{Mac}
\modulesynopsis{Read QuickTime movies frame by frame for further processing.}

\module{videoreader} reads and decodes QuickTime movies and passes
a stream of images to your program. It also provides some support for
audio tracks.

\section{\module{W} --- Widgets built on \module{FrameWork}}
\declaremodule{standard}{W}
  \platform{Mac}
\modulesynopsis{Widgets for the Mac, built on top of \refmodule{FrameWork}.}

The \module{W} widgets are used extensively in the \program{IDE}.

                           % Undocumented Modules

\appendix
\chapter{History and License}
\section{History of the software}

Python was created in the early 1990s by Guido van Rossum at Stichting
Mathematisch Centrum (CWI, see \url{http://www.cwi.nl/}) in the Netherlands
as a successor of a language called ABC.  Guido remains Python's
principal author, although it includes many contributions from others.

In 1995, Guido continued his work on Python at the Corporation for
National Research Initiatives (CNRI, see \url{http://www.cnri.reston.va.us/})
in Reston, Virginia where he released several versions of the
software.

In May 2000, Guido and the Python core development team moved to
BeOpen.com to form the BeOpen PythonLabs team.  In October of the same
year, the PythonLabs team moved to Digital Creations (now Zope
Corporation; see \url{http://www.zope.com/}).  In 2001, the Python
Software Foundation (PSF, see \url{http://www.python.org/psf/}) was
formed, a non-profit organization created specifically to own
Python-related Intellectual Property.  Zope Corporation is a
sponsoring member of the PSF.

All Python releases are Open Source (see
\url{http://www.opensource.org/} for the Open Source Definition).
Historically, most, but not all, Python releases have also been
GPL-compatible; the table below summarizes the various releases.

\begin{tablev}{c|c|c|c|c}{textrm}%
  {Release}{Derived from}{Year}{Owner}{GPL compatible?}
  \linev{0.9.0 thru 1.2}{n/a}{1991-1995}{CWI}{yes}
  \linev{1.3 thru 1.5.2}{1.2}{1995-1999}{CNRI}{yes}
  \linev{1.6}{1.5.2}{2000}{CNRI}{no}
  \linev{2.0}{1.6}{2000}{BeOpen.com}{no}
  \linev{1.6.1}{1.6}{2001}{CNRI}{no}
  \linev{2.1}{2.0+1.6.1}{2001}{PSF}{no}
  \linev{2.0.1}{2.0+1.6.1}{2001}{PSF}{yes}
  \linev{2.1.1}{2.1+2.0.1}{2001}{PSF}{yes}
  \linev{2.2}{2.1.1}{2001}{PSF}{yes}
  \linev{2.1.2}{2.1.1}{2002}{PSF}{yes}
  \linev{2.1.3}{2.1.2}{2002}{PSF}{yes}
  \linev{2.2.1}{2.2}{2002}{PSF}{yes}
  \linev{2.2.2}{2.2.1}{2002}{PSF}{yes}
  \linev{2.2.3}{2.2.2}{2002-2003}{PSF}{yes}
  \linev{2.3}{2.2.2}{2002-2003}{PSF}{yes}
  \linev{2.3.1}{2.3}{2002-2003}{PSF}{yes}
  \linev{2.3.2}{2.3.1}{2003}{PSF}{yes}
  \linev{2.3.3}{2.3.2}{2003}{PSF}{yes}
  \linev{2.3.4}{2.3.3}{2004}{PSF}{yes}
  \linev{2.3.5}{2.3.4}{2005}{PSF}{yes}
  \linev{2.4}{2.3}{2004}{PSF}{yes}
  \linev{2.4.1}{2.4}{2005}{PSF}{yes}
  \linev{2.4.2}{2.4.1}{2005}{PSF}{yes}
  \linev{2.4.3}{2.4.2}{2006}{PSF}{yes}
  \linev{2.5}{2.4}{2006}{PSF}{yes}
\end{tablev}

\note{GPL-compatible doesn't mean that we're distributing
Python under the GPL.  All Python licenses, unlike the GPL, let you
distribute a modified version without making your changes open source.
The GPL-compatible licenses make it possible to combine Python with
other software that is released under the GPL; the others don't.}

Thanks to the many outside volunteers who have worked under Guido's
direction to make these releases possible.


\section{Terms and conditions for accessing or otherwise using Python}

\centerline{\strong{PSF LICENSE AGREEMENT FOR PYTHON \version}}

\begin{enumerate}
\item
This LICENSE AGREEMENT is between the Python Software Foundation
(``PSF''), and the Individual or Organization (``Licensee'') accessing
and otherwise using Python \version{} software in source or binary
form and its associated documentation.

\item
Subject to the terms and conditions of this License Agreement, PSF
hereby grants Licensee a nonexclusive, royalty-free, world-wide
license to reproduce, analyze, test, perform and/or display publicly,
prepare derivative works, distribute, and otherwise use Python
\version{} alone or in any derivative version, provided, however, that
PSF's License Agreement and PSF's notice of copyright, i.e.,
``Copyright \copyright{} 2001-2006 Python Software Foundation; All
Rights Reserved'' are retained in Python \version{} alone or in any
derivative version prepared by Licensee.

\item
In the event Licensee prepares a derivative work that is based on
or incorporates Python \version{} or any part thereof, and wants to
make the derivative work available to others as provided herein, then
Licensee hereby agrees to include in any such work a brief summary of
the changes made to Python \version.

\item
PSF is making Python \version{} available to Licensee on an ``AS IS''
basis.  PSF MAKES NO REPRESENTATIONS OR WARRANTIES, EXPRESS OR
IMPLIED.  BY WAY OF EXAMPLE, BUT NOT LIMITATION, PSF MAKES NO AND
DISCLAIMS ANY REPRESENTATION OR WARRANTY OF MERCHANTABILITY OR FITNESS
FOR ANY PARTICULAR PURPOSE OR THAT THE USE OF PYTHON \version{} WILL
NOT INFRINGE ANY THIRD PARTY RIGHTS.

\item
PSF SHALL NOT BE LIABLE TO LICENSEE OR ANY OTHER USERS OF PYTHON
\version{} FOR ANY INCIDENTAL, SPECIAL, OR CONSEQUENTIAL DAMAGES OR
LOSS AS A RESULT OF MODIFYING, DISTRIBUTING, OR OTHERWISE USING PYTHON
\version, OR ANY DERIVATIVE THEREOF, EVEN IF ADVISED OF THE
POSSIBILITY THEREOF.

\item
This License Agreement will automatically terminate upon a material
breach of its terms and conditions.

\item
Nothing in this License Agreement shall be deemed to create any
relationship of agency, partnership, or joint venture between PSF and
Licensee.  This License Agreement does not grant permission to use PSF
trademarks or trade name in a trademark sense to endorse or promote
products or services of Licensee, or any third party.

\item
By copying, installing or otherwise using Python \version, Licensee
agrees to be bound by the terms and conditions of this License
Agreement.
\end{enumerate}


\centerline{\strong{BEOPEN.COM LICENSE AGREEMENT FOR PYTHON 2.0}}

\centerline{\strong{BEOPEN PYTHON OPEN SOURCE LICENSE AGREEMENT VERSION 1}}

\begin{enumerate}
\item
This LICENSE AGREEMENT is between BeOpen.com (``BeOpen''), having an
office at 160 Saratoga Avenue, Santa Clara, CA 95051, and the
Individual or Organization (``Licensee'') accessing and otherwise
using this software in source or binary form and its associated
documentation (``the Software'').

\item
Subject to the terms and conditions of this BeOpen Python License
Agreement, BeOpen hereby grants Licensee a non-exclusive,
royalty-free, world-wide license to reproduce, analyze, test, perform
and/or display publicly, prepare derivative works, distribute, and
otherwise use the Software alone or in any derivative version,
provided, however, that the BeOpen Python License is retained in the
Software, alone or in any derivative version prepared by Licensee.

\item
BeOpen is making the Software available to Licensee on an ``AS IS''
basis.  BEOPEN MAKES NO REPRESENTATIONS OR WARRANTIES, EXPRESS OR
IMPLIED.  BY WAY OF EXAMPLE, BUT NOT LIMITATION, BEOPEN MAKES NO AND
DISCLAIMS ANY REPRESENTATION OR WARRANTY OF MERCHANTABILITY OR FITNESS
FOR ANY PARTICULAR PURPOSE OR THAT THE USE OF THE SOFTWARE WILL NOT
INFRINGE ANY THIRD PARTY RIGHTS.

\item
BEOPEN SHALL NOT BE LIABLE TO LICENSEE OR ANY OTHER USERS OF THE
SOFTWARE FOR ANY INCIDENTAL, SPECIAL, OR CONSEQUENTIAL DAMAGES OR LOSS
AS A RESULT OF USING, MODIFYING OR DISTRIBUTING THE SOFTWARE, OR ANY
DERIVATIVE THEREOF, EVEN IF ADVISED OF THE POSSIBILITY THEREOF.

\item
This License Agreement will automatically terminate upon a material
breach of its terms and conditions.

\item
This License Agreement shall be governed by and interpreted in all
respects by the law of the State of California, excluding conflict of
law provisions.  Nothing in this License Agreement shall be deemed to
create any relationship of agency, partnership, or joint venture
between BeOpen and Licensee.  This License Agreement does not grant
permission to use BeOpen trademarks or trade names in a trademark
sense to endorse or promote products or services of Licensee, or any
third party.  As an exception, the ``BeOpen Python'' logos available
at http://www.pythonlabs.com/logos.html may be used according to the
permissions granted on that web page.

\item
By copying, installing or otherwise using the software, Licensee
agrees to be bound by the terms and conditions of this License
Agreement.
\end{enumerate}


\centerline{\strong{CNRI LICENSE AGREEMENT FOR PYTHON 1.6.1}}

\begin{enumerate}
\item
This LICENSE AGREEMENT is between the Corporation for National
Research Initiatives, having an office at 1895 Preston White Drive,
Reston, VA 20191 (``CNRI''), and the Individual or Organization
(``Licensee'') accessing and otherwise using Python 1.6.1 software in
source or binary form and its associated documentation.

\item
Subject to the terms and conditions of this License Agreement, CNRI
hereby grants Licensee a nonexclusive, royalty-free, world-wide
license to reproduce, analyze, test, perform and/or display publicly,
prepare derivative works, distribute, and otherwise use Python 1.6.1
alone or in any derivative version, provided, however, that CNRI's
License Agreement and CNRI's notice of copyright, i.e., ``Copyright
\copyright{} 1995-2001 Corporation for National Research Initiatives;
All Rights Reserved'' are retained in Python 1.6.1 alone or in any
derivative version prepared by Licensee.  Alternately, in lieu of
CNRI's License Agreement, Licensee may substitute the following text
(omitting the quotes): ``Python 1.6.1 is made available subject to the
terms and conditions in CNRI's License Agreement.  This Agreement
together with Python 1.6.1 may be located on the Internet using the
following unique, persistent identifier (known as a handle):
1895.22/1013.  This Agreement may also be obtained from a proxy server
on the Internet using the following URL:
\url{http://hdl.handle.net/1895.22/1013}.''

\item
In the event Licensee prepares a derivative work that is based on
or incorporates Python 1.6.1 or any part thereof, and wants to make
the derivative work available to others as provided herein, then
Licensee hereby agrees to include in any such work a brief summary of
the changes made to Python 1.6.1.

\item
CNRI is making Python 1.6.1 available to Licensee on an ``AS IS''
basis.  CNRI MAKES NO REPRESENTATIONS OR WARRANTIES, EXPRESS OR
IMPLIED.  BY WAY OF EXAMPLE, BUT NOT LIMITATION, CNRI MAKES NO AND
DISCLAIMS ANY REPRESENTATION OR WARRANTY OF MERCHANTABILITY OR FITNESS
FOR ANY PARTICULAR PURPOSE OR THAT THE USE OF PYTHON 1.6.1 WILL NOT
INFRINGE ANY THIRD PARTY RIGHTS.

\item
CNRI SHALL NOT BE LIABLE TO LICENSEE OR ANY OTHER USERS OF PYTHON
1.6.1 FOR ANY INCIDENTAL, SPECIAL, OR CONSEQUENTIAL DAMAGES OR LOSS AS
A RESULT OF MODIFYING, DISTRIBUTING, OR OTHERWISE USING PYTHON 1.6.1,
OR ANY DERIVATIVE THEREOF, EVEN IF ADVISED OF THE POSSIBILITY THEREOF.

\item
This License Agreement will automatically terminate upon a material
breach of its terms and conditions.

\item
This License Agreement shall be governed by the federal
intellectual property law of the United States, including without
limitation the federal copyright law, and, to the extent such
U.S. federal law does not apply, by the law of the Commonwealth of
Virginia, excluding Virginia's conflict of law provisions.
Notwithstanding the foregoing, with regard to derivative works based
on Python 1.6.1 that incorporate non-separable material that was
previously distributed under the GNU General Public License (GPL), the
law of the Commonwealth of Virginia shall govern this License
Agreement only as to issues arising under or with respect to
Paragraphs 4, 5, and 7 of this License Agreement.  Nothing in this
License Agreement shall be deemed to create any relationship of
agency, partnership, or joint venture between CNRI and Licensee.  This
License Agreement does not grant permission to use CNRI trademarks or
trade name in a trademark sense to endorse or promote products or
services of Licensee, or any third party.

\item
By clicking on the ``ACCEPT'' button where indicated, or by copying,
installing or otherwise using Python 1.6.1, Licensee agrees to be
bound by the terms and conditions of this License Agreement.
\end{enumerate}

\centerline{ACCEPT}



\centerline{\strong{CWI LICENSE AGREEMENT FOR PYTHON 0.9.0 THROUGH 1.2}}

Copyright \copyright{} 1991 - 1995, Stichting Mathematisch Centrum
Amsterdam, The Netherlands.  All rights reserved.

Permission to use, copy, modify, and distribute this software and its
documentation for any purpose and without fee is hereby granted,
provided that the above copyright notice appear in all copies and that
both that copyright notice and this permission notice appear in
supporting documentation, and that the name of Stichting Mathematisch
Centrum or CWI not be used in advertising or publicity pertaining to
distribution of the software without specific, written prior
permission.

STICHTING MATHEMATISCH CENTRUM DISCLAIMS ALL WARRANTIES WITH REGARD TO
THIS SOFTWARE, INCLUDING ALL IMPLIED WARRANTIES OF MERCHANTABILITY AND
FITNESS, IN NO EVENT SHALL STICHTING MATHEMATISCH CENTRUM BE LIABLE
FOR ANY SPECIAL, INDIRECT OR CONSEQUENTIAL DAMAGES OR ANY DAMAGES
WHATSOEVER RESULTING FROM LOSS OF USE, DATA OR PROFITS, WHETHER IN AN
ACTION OF CONTRACT, NEGLIGENCE OR OTHER TORTIOUS ACTION, ARISING OUT
OF OR IN CONNECTION WITH THE USE OR PERFORMANCE OF THIS SOFTWARE.


\section{Licenses and Acknowledgements for Incorporated Software}

This section is an incomplete, but growing list of licenses and
acknowledgements for third-party software incorporated in the
Python distribution.


\subsection{Mersenne Twister}

The \module{_random} module includes code based on a download from
\url{http://www.math.keio.ac.jp/~matumoto/MT2002/emt19937ar.html}.
The following are the verbatim comments from the original code:

\begin{verbatim}
A C-program for MT19937, with initialization improved 2002/1/26.
Coded by Takuji Nishimura and Makoto Matsumoto.

Before using, initialize the state by using init_genrand(seed)
or init_by_array(init_key, key_length).

Copyright (C) 1997 - 2002, Makoto Matsumoto and Takuji Nishimura,
All rights reserved.

Redistribution and use in source and binary forms, with or without
modification, are permitted provided that the following conditions
are met:

 1. Redistributions of source code must retain the above copyright
    notice, this list of conditions and the following disclaimer.

 2. Redistributions in binary form must reproduce the above copyright
    notice, this list of conditions and the following disclaimer in the
    documentation and/or other materials provided with the distribution.

 3. The names of its contributors may not be used to endorse or promote
    products derived from this software without specific prior written
    permission.

THIS SOFTWARE IS PROVIDED BY THE COPYRIGHT HOLDERS AND CONTRIBUTORS
"AS IS" AND ANY EXPRESS OR IMPLIED WARRANTIES, INCLUDING, BUT NOT
LIMITED TO, THE IMPLIED WARRANTIES OF MERCHANTABILITY AND FITNESS FOR
A PARTICULAR PURPOSE ARE DISCLAIMED.  IN NO EVENT SHALL THE COPYRIGHT OWNER OR
CONTRIBUTORS BE LIABLE FOR ANY DIRECT, INDIRECT, INCIDENTAL, SPECIAL,
EXEMPLARY, OR CONSEQUENTIAL DAMAGES (INCLUDING, BUT NOT LIMITED TO,
PROCUREMENT OF SUBSTITUTE GOODS OR SERVICES; LOSS OF USE, DATA, OR
PROFITS; OR BUSINESS INTERRUPTION) HOWEVER CAUSED AND ON ANY THEORY OF
LIABILITY, WHETHER IN CONTRACT, STRICT LIABILITY, OR TORT (INCLUDING
NEGLIGENCE OR OTHERWISE) ARISING IN ANY WAY OUT OF THE USE OF THIS
SOFTWARE, EVEN IF ADVISED OF THE POSSIBILITY OF SUCH DAMAGE.


Any feedback is very welcome.
http://www.math.keio.ac.jp/matumoto/emt.html
email: matumoto@math.keio.ac.jp
\end{verbatim}



\subsection{Sockets}

The \module{socket} module uses the functions, \function{getaddrinfo},
and \function{getnameinfo}, which are coded in separate source files
from the WIDE Project, \url{http://www.wide.ad.jp/about/index.html}.

\begin{verbatim}      
Copyright (C) 1995, 1996, 1997, and 1998 WIDE Project.
All rights reserved.
 
Redistribution and use in source and binary forms, with or without
modification, are permitted provided that the following conditions
are met:
1. Redistributions of source code must retain the above copyright
   notice, this list of conditions and the following disclaimer.
2. Redistributions in binary form must reproduce the above copyright
   notice, this list of conditions and the following disclaimer in the
   documentation and/or other materials provided with the distribution.
3. Neither the name of the project nor the names of its contributors
   may be used to endorse or promote products derived from this software
   without specific prior written permission.

THIS SOFTWARE IS PROVIDED BY THE PROJECT AND CONTRIBUTORS ``AS IS'' AND
GAI_ANY EXPRESS OR IMPLIED WARRANTIES, INCLUDING, BUT NOT LIMITED TO, THE
IMPLIED WARRANTIES OF MERCHANTABILITY AND FITNESS FOR A PARTICULAR PURPOSE
ARE DISCLAIMED.  IN NO EVENT SHALL THE PROJECT OR CONTRIBUTORS BE LIABLE
FOR GAI_ANY DIRECT, INDIRECT, INCIDENTAL, SPECIAL, EXEMPLARY, OR CONSEQUENTIAL
DAMAGES (INCLUDING, BUT NOT LIMITED TO, PROCUREMENT OF SUBSTITUTE GOODS
OR SERVICES; LOSS OF USE, DATA, OR PROFITS; OR BUSINESS INTERRUPTION)
HOWEVER CAUSED AND ON GAI_ANY THEORY OF LIABILITY, WHETHER IN CONTRACT, STRICT
LIABILITY, OR TORT (INCLUDING NEGLIGENCE OR OTHERWISE) ARISING IN GAI_ANY WAY
OUT OF THE USE OF THIS SOFTWARE, EVEN IF ADVISED OF THE POSSIBILITY OF
SUCH DAMAGE.
\end{verbatim}



\subsection{Floating point exception control}

The source for the \module{fpectl} module includes the following notice:

\begin{verbatim}
     ---------------------------------------------------------------------  
    /                       Copyright (c) 1996.                           \ 
   |          The Regents of the University of California.                 |
   |                        All rights reserved.                           |
   |                                                                       |
   |   Permission to use, copy, modify, and distribute this software for   |
   |   any purpose without fee is hereby granted, provided that this en-   |
   |   tire notice is included in all copies of any software which is or   |
   |   includes  a  copy  or  modification  of  this software and in all   |
   |   copies of the supporting documentation for such software.           |
   |                                                                       |
   |   This  work was produced at the University of California, Lawrence   |
   |   Livermore National Laboratory under  contract  no.  W-7405-ENG-48   |
   |   between  the  U.S.  Department  of  Energy and The Regents of the   |
   |   University of California for the operation of UC LLNL.              |
   |                                                                       |
   |                              DISCLAIMER                               |
   |                                                                       |
   |   This  software was prepared as an account of work sponsored by an   |
   |   agency of the United States Government. Neither the United States   |
   |   Government  nor the University of California nor any of their em-   |
   |   ployees, makes any warranty, express or implied, or  assumes  any   |
   |   liability  or  responsibility  for the accuracy, completeness, or   |
   |   usefulness of any information,  apparatus,  product,  or  process   |
   |   disclosed,   or  represents  that  its  use  would  not  infringe   |
   |   privately-owned rights. Reference herein to any specific  commer-   |
   |   cial  products,  process,  or  service  by trade name, trademark,   |
   |   manufacturer, or otherwise, does not  necessarily  constitute  or   |
   |   imply  its endorsement, recommendation, or favoring by the United   |
   |   States Government or the University of California. The views  and   |
   |   opinions  of authors expressed herein do not necessarily state or   |
   |   reflect those of the United States Government or  the  University   |
   |   of  California,  and shall not be used for advertising or product   |
    \  endorsement purposes.                                              / 
     ---------------------------------------------------------------------
\end{verbatim}



\subsection{MD5 message digest algorithm}

The source code for the \module{md5} module contains the following notice:

\begin{verbatim}
  Copyright (C) 1999, 2002 Aladdin Enterprises.  All rights reserved.

  This software is provided 'as-is', without any express or implied
  warranty.  In no event will the authors be held liable for any damages
  arising from the use of this software.

  Permission is granted to anyone to use this software for any purpose,
  including commercial applications, and to alter it and redistribute it
  freely, subject to the following restrictions:

  1. The origin of this software must not be misrepresented; you must not
     claim that you wrote the original software. If you use this software
     in a product, an acknowledgment in the product documentation would be
     appreciated but is not required.
  2. Altered source versions must be plainly marked as such, and must not be
     misrepresented as being the original software.
  3. This notice may not be removed or altered from any source distribution.

  L. Peter Deutsch
  ghost@aladdin.com

  Independent implementation of MD5 (RFC 1321).

  This code implements the MD5 Algorithm defined in RFC 1321, whose
  text is available at
	http://www.ietf.org/rfc/rfc1321.txt
  The code is derived from the text of the RFC, including the test suite
  (section A.5) but excluding the rest of Appendix A.  It does not include
  any code or documentation that is identified in the RFC as being
  copyrighted.

  The original and principal author of md5.h is L. Peter Deutsch
  <ghost@aladdin.com>.  Other authors are noted in the change history
  that follows (in reverse chronological order):

  2002-04-13 lpd Removed support for non-ANSI compilers; removed
	references to Ghostscript; clarified derivation from RFC 1321;
	now handles byte order either statically or dynamically.
  1999-11-04 lpd Edited comments slightly for automatic TOC extraction.
  1999-10-18 lpd Fixed typo in header comment (ansi2knr rather than md5);
	added conditionalization for C++ compilation from Martin
	Purschke <purschke@bnl.gov>.
  1999-05-03 lpd Original version.
\end{verbatim}



\subsection{Asynchronous socket services}

The \module{asynchat} and \module{asyncore} modules contain the
following notice:

\begin{verbatim}      
 Copyright 1996 by Sam Rushing

                         All Rights Reserved

 Permission to use, copy, modify, and distribute this software and
 its documentation for any purpose and without fee is hereby
 granted, provided that the above copyright notice appear in all
 copies and that both that copyright notice and this permission
 notice appear in supporting documentation, and that the name of Sam
 Rushing not be used in advertising or publicity pertaining to
 distribution of the software without specific, written prior
 permission.

 SAM RUSHING DISCLAIMS ALL WARRANTIES WITH REGARD TO THIS SOFTWARE,
 INCLUDING ALL IMPLIED WARRANTIES OF MERCHANTABILITY AND FITNESS, IN
 NO EVENT SHALL SAM RUSHING BE LIABLE FOR ANY SPECIAL, INDIRECT OR
 CONSEQUENTIAL DAMAGES OR ANY DAMAGES WHATSOEVER RESULTING FROM LOSS
 OF USE, DATA OR PROFITS, WHETHER IN AN ACTION OF CONTRACT,
 NEGLIGENCE OR OTHER TORTIOUS ACTION, ARISING OUT OF OR IN
 CONNECTION WITH THE USE OR PERFORMANCE OF THIS SOFTWARE.
\end{verbatim}


\subsection{Cookie management}

The \module{Cookie} module contains the following notice:

\begin{verbatim}
 Copyright 2000 by Timothy O'Malley <timo@alum.mit.edu>

                All Rights Reserved

 Permission to use, copy, modify, and distribute this software
 and its documentation for any purpose and without fee is hereby
 granted, provided that the above copyright notice appear in all
 copies and that both that copyright notice and this permission
 notice appear in supporting documentation, and that the name of
 Timothy O'Malley  not be used in advertising or publicity
 pertaining to distribution of the software without specific, written
 prior permission.

 Timothy O'Malley DISCLAIMS ALL WARRANTIES WITH REGARD TO THIS
 SOFTWARE, INCLUDING ALL IMPLIED WARRANTIES OF MERCHANTABILITY
 AND FITNESS, IN NO EVENT SHALL Timothy O'Malley BE LIABLE FOR
 ANY SPECIAL, INDIRECT OR CONSEQUENTIAL DAMAGES OR ANY DAMAGES
 WHATSOEVER RESULTING FROM LOSS OF USE, DATA OR PROFITS,
 WHETHER IN AN ACTION OF CONTRACT, NEGLIGENCE OR OTHER TORTIOUS
 ACTION, ARISING OUT OF OR IN CONNECTION WITH THE USE OR
 PERFORMANCE OF THIS SOFTWARE.
\end{verbatim}      



\subsection{Profiling}

The \module{profile} and \module{pstats} modules contain
the following notice:

\begin{verbatim}
 Copyright 1994, by InfoSeek Corporation, all rights reserved.
 Written by James Roskind

 Permission to use, copy, modify, and distribute this Python software
 and its associated documentation for any purpose (subject to the
 restriction in the following sentence) without fee is hereby granted,
 provided that the above copyright notice appears in all copies, and
 that both that copyright notice and this permission notice appear in
 supporting documentation, and that the name of InfoSeek not be used in
 advertising or publicity pertaining to distribution of the software
 without specific, written prior permission.  This permission is
 explicitly restricted to the copying and modification of the software
 to remain in Python, compiled Python, or other languages (such as C)
 wherein the modified or derived code is exclusively imported into a
 Python module.

 INFOSEEK CORPORATION DISCLAIMS ALL WARRANTIES WITH REGARD TO THIS
 SOFTWARE, INCLUDING ALL IMPLIED WARRANTIES OF MERCHANTABILITY AND
 FITNESS. IN NO EVENT SHALL INFOSEEK CORPORATION BE LIABLE FOR ANY
 SPECIAL, INDIRECT OR CONSEQUENTIAL DAMAGES OR ANY DAMAGES WHATSOEVER
 RESULTING FROM LOSS OF USE, DATA OR PROFITS, WHETHER IN AN ACTION OF
 CONTRACT, NEGLIGENCE OR OTHER TORTIOUS ACTION, ARISING OUT OF OR IN
 CONNECTION WITH THE USE OR PERFORMANCE OF THIS SOFTWARE.
\end{verbatim}



\subsection{Execution tracing}

The \module{trace} module contains the following notice:

\begin{verbatim}
 portions copyright 2001, Autonomous Zones Industries, Inc., all rights...
 err...  reserved and offered to the public under the terms of the
 Python 2.2 license.
 Author: Zooko O'Whielacronx
 http://zooko.com/
 mailto:zooko@zooko.com

 Copyright 2000, Mojam Media, Inc., all rights reserved.
 Author: Skip Montanaro

 Copyright 1999, Bioreason, Inc., all rights reserved.
 Author: Andrew Dalke

 Copyright 1995-1997, Automatrix, Inc., all rights reserved.
 Author: Skip Montanaro

 Copyright 1991-1995, Stichting Mathematisch Centrum, all rights reserved.


 Permission to use, copy, modify, and distribute this Python software and
 its associated documentation for any purpose without fee is hereby
 granted, provided that the above copyright notice appears in all copies,
 and that both that copyright notice and this permission notice appear in
 supporting documentation, and that the name of neither Automatrix,
 Bioreason or Mojam Media be used in advertising or publicity pertaining to
 distribution of the software without specific, written prior permission.
\end{verbatim} 



\subsection{UUencode and UUdecode functions}

The \module{uu} module contains the following notice:

\begin{verbatim}
 Copyright 1994 by Lance Ellinghouse
 Cathedral City, California Republic, United States of America.
                        All Rights Reserved
 Permission to use, copy, modify, and distribute this software and its
 documentation for any purpose and without fee is hereby granted,
 provided that the above copyright notice appear in all copies and that
 both that copyright notice and this permission notice appear in
 supporting documentation, and that the name of Lance Ellinghouse
 not be used in advertising or publicity pertaining to distribution
 of the software without specific, written prior permission.
 LANCE ELLINGHOUSE DISCLAIMS ALL WARRANTIES WITH REGARD TO
 THIS SOFTWARE, INCLUDING ALL IMPLIED WARRANTIES OF MERCHANTABILITY AND
 FITNESS, IN NO EVENT SHALL LANCE ELLINGHOUSE CENTRUM BE LIABLE
 FOR ANY SPECIAL, INDIRECT OR CONSEQUENTIAL DAMAGES OR ANY DAMAGES
 WHATSOEVER RESULTING FROM LOSS OF USE, DATA OR PROFITS, WHETHER IN AN
 ACTION OF CONTRACT, NEGLIGENCE OR OTHER TORTIOUS ACTION, ARISING OUT
 OF OR IN CONNECTION WITH THE USE OR PERFORMANCE OF THIS SOFTWARE.

 Modified by Jack Jansen, CWI, July 1995:
 - Use binascii module to do the actual line-by-line conversion
   between ascii and binary. This results in a 1000-fold speedup. The C
   version is still 5 times faster, though.
 - Arguments more compliant with python standard
\end{verbatim}



\subsection{XML Remote Procedure Calls}

The \module{xmlrpclib} module contains the following notice:

\begin{verbatim}
     The XML-RPC client interface is

 Copyright (c) 1999-2002 by Secret Labs AB
 Copyright (c) 1999-2002 by Fredrik Lundh

 By obtaining, using, and/or copying this software and/or its
 associated documentation, you agree that you have read, understood,
 and will comply with the following terms and conditions:

 Permission to use, copy, modify, and distribute this software and
 its associated documentation for any purpose and without fee is
 hereby granted, provided that the above copyright notice appears in
 all copies, and that both that copyright notice and this permission
 notice appear in supporting documentation, and that the name of
 Secret Labs AB or the author not be used in advertising or publicity
 pertaining to distribution of the software without specific, written
 prior permission.

 SECRET LABS AB AND THE AUTHOR DISCLAIMS ALL WARRANTIES WITH REGARD
 TO THIS SOFTWARE, INCLUDING ALL IMPLIED WARRANTIES OF MERCHANT-
 ABILITY AND FITNESS.  IN NO EVENT SHALL SECRET LABS AB OR THE AUTHOR
 BE LIABLE FOR ANY SPECIAL, INDIRECT OR CONSEQUENTIAL DAMAGES OR ANY
 DAMAGES WHATSOEVER RESULTING FROM LOSS OF USE, DATA OR PROFITS,
 WHETHER IN AN ACTION OF CONTRACT, NEGLIGENCE OR OTHER TORTIOUS
 ACTION, ARISING OUT OF OR IN CONNECTION WITH THE USE OR PERFORMANCE
 OF THIS SOFTWARE.
\end{verbatim}


%
%  The ugly "%begin{latexonly}" pseudo-environments are really just to
%  keep LaTeX2HTML quiet during the \renewcommand{} macros; they're
%  not really valuable.
%

%begin{latexonly}
\renewcommand{\indexname}{Module Index}
%end{latexonly}
\input{modmac.ind}      % Module Index

%begin{latexonly}
\renewcommand{\indexname}{Index}
%end{latexonly}
\documentclass{manual}

\title{Macintosh Library Modules}

\input{boilerplate}

\makeindex              % tell \index to actually write the .idx file
\makemodindex           % ... and the module index as well.


\begin{document}

\maketitle

\ifhtml
\chapter*{Front Matter\label{front}}
\fi

\input{copyright}

\begin{abstract}

\noindent
This library reference manual documents Python's extensions for the
Macintosh.  It should be used in conjunction with the
\citetitle[../lib/lib.html]{Python Library Reference}, which documents
the standard library and built-in types.

This manual assumes basic knowledge about the Python language.  For an
informal introduction to Python, see the
\citetitle[../tut/tut.html]{Python Tutorial}; the
\citetitle[../ref/ref.html]{Python Reference Manual} remains the
highest authority on syntactic and semantic questions.  Finally, the
manual entitled \citetitle[../ext/ext.html]{Extending and Embedding
the Python Interpreter} describes how to add new extensions to Python
and how to embed it in other applications.

\end{abstract}

\tableofcontents


\input{using.tex}                       % Using Python on the Macintosh


\chapter{MacPython Modules \label{macpython-modules}}

The following modules are only available on the Macintosh, and are
documented here:

\localmoduletable

\input{libmac}
\input{libctb}
%\input{libmacconsole}
\input{libmacdnr}
\input{libmacfs}
\input{libmacic}
\input{libmacos}
\input{libmacostools}
\input{libmactcp}
\input{libmacspeech}
\input{libmacui}
\input{libframework}
\input{libminiae}
\input{libaepack}
\input{libaetypes}

\input{toolbox}                         % MacOS Toolbox Modules
\input{libcolorpicker}

\input{undoc}                           % Undocumented Modules

\appendix
\chapter{History and License}
\input{license}

%
%  The ugly "%begin{latexonly}" pseudo-environments are really just to
%  keep LaTeX2HTML quiet during the \renewcommand{} macros; they're
%  not really valuable.
%

%begin{latexonly}
\renewcommand{\indexname}{Module Index}
%end{latexonly}
\input{modmac.ind}      % Module Index

%begin{latexonly}
\renewcommand{\indexname}{Index}
%end{latexonly}
\input{mac.ind}         % Index

\end{document}
         % Index

\end{document}
         % Index

\end{document}
         % Index

\end{document}
