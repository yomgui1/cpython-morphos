\section{Built-in Functions}

The Python interpreter has a number of functions built into it that
are always available.  They are listed here in alphabetical order.


\renewcommand{\indexsubitem}{(built-in function)}
\begin{funcdesc}{abs}{x}
  Return the absolute value of a number.  The argument may be a plain
  or long integer or a floating point number.
\end{funcdesc}

\begin{funcdesc}{apply}{function\, args}
The \var{function} argument must be a callable object (a user-defined or
built-in function or method, or a class object) and the \var{args}
argument must be a tuple.  The \var{function} is called with
\var{args} as argument list; the number of arguments is the the length
of the tuple.  (This is different from just calling
\code{\var{func}(\var{args})}, since in that case there is always
exactly one argument.)
\end{funcdesc}

\begin{funcdesc}{chr}{i}
  Return a string of one character whose \ASCII{} code is the integer
  \var{i}, e.g., \code{chr(97)} returns the string \code{'a'}.  This is the
  inverse of \code{ord()}.  The argument must be in the range [0..255],
  inclusive.
\end{funcdesc}

\begin{funcdesc}{cmp}{x\, y}
  Compare the two objects \var{x} and \var{y} and return an integer
  according to the outcome.  The return value is negative if \code{\var{x}
  < \var{y}}, zero if \code{\var{x} == \var{y}} and strictly positive if
  \code{\var{x} > \var{y}}.
\end{funcdesc}

\begin{funcdesc}{coerce}{x\, y}
  Return a tuple consisting of the two numeric arguments converted to
  a common type, using the same rules as used by arithmetic
  operations.
\end{funcdesc}

\begin{funcdesc}{compile}{string\, filename\, kind}
  Compile the \var{string} into a code object.  Code objects can be
  executed by a \code{exec()} statement or evaluated by a call to
  \code{eval()}.  The \var{filename} argument should
  give the file from which the code was read; pass e.g. \code{'<string>'}
  if it wasn't read from a file.  The \var{kind} argument specifies
  what kind of code must be compiled; it can be \code{'exec'} if
  \var{string} consists of a sequence of statements, or \code{'eval'}
  if it consists of a single expression.
\end{funcdesc}

\begin{funcdesc}{dir}{}
  Without arguments, return the list of names in the current local
  symbol table.  With a module, class or class instance object as
  argument (or anything else that has a \code{__dict__} attribute),
  returns the list of names in that object's attribute dictionary.
  The resulting list is sorted.  For example:

\bcode\begin{verbatim}
>>> import sys
>>> dir()
['sys']
>>> dir(sys)
['argv', 'exit', 'modules', 'path', 'stderr', 'stdin', 'stdout']
>>> 
\end{verbatim}\ecode
\end{funcdesc}

\begin{funcdesc}{divmod}{a\, b}
  Take two numbers as arguments and return a pair of integers
  consisting of their integer quotient and remainder.  With mixed
  operand types, the rules for binary arithmetic operators apply.  For
  plain and long integers, the result is the same as
  \code{(\var{a} / \var{b}, \var{a} \%{} \var{b})}.
  For floating point numbers the result is the same as
  \code{(math.floor(\var{a} / \var{b}), \var{a} \%{} \var{b})}.
\end{funcdesc}

\begin{funcdesc}{eval}{s\optional{\, globals\optional{\, locals}}}
  The arguments are a string and two optional dictionaries.  The
  string argument is parsed and evaluated as a Python expression
  (technically speaking, a condition list) using the dictionaries as
  global and local name space.  The string must not contain null bytes
  or newline characters.  The return value is the
  result of the expression.  If the third argument is omitted it
  defaults to the second.  If both dictionaries are omitted, the
  expression is executed in the environment where \code{eval} is
  called.  Syntax errors are reported as exceptions.  Example:

\bcode\begin{verbatim}
>>> x = 1
>>> print eval('x+1')
2
>>> 
\end{verbatim}\ecode

  This function can also be used to execute arbitrary code objects
  (e.g. created by \code{compile()}).  In this case pass a code
  object instead of a string.  The code object must have been compiled
  passing \code{'eval'} to the \var{kind} argument.

  Note: dynamic execution of statements is supported by the
  \code{exec} statement.

\end{funcdesc}

\begin{funcdesc}{filter}{function\, list}
Construct a list from those elements of \var{list} for which
\var{function} returns true.  If \var{list} is a string or a tuple,
the result also has that type; otherwise it is always a list.  If
\var{function} is \code{None}, the identity function is assumed,
i.e. all elements of \var{list} that are false (zero or empty) are
removed.
\end{funcdesc}

\begin{funcdesc}{float}{x}
  Convert a number to floating point.  The argument may be a plain or
  long integer or a floating point number.
\end{funcdesc}

\begin{funcdesc}{getattr}{object\, name}
  The arguments are an object and a string.  The string must be the
  name
  of one of the object's attributes.  The result is the value of that
  attribute.  For example, \code{getattr(\var{x}, '\var{foobar}')} is equivalent to
  \code{\var{x}.\var{foobar}}.
\end{funcdesc}

\begin{funcdesc}{hasattr}{object\, name}
  The arguments are an object and a string.  The result is 1 if the
  string is the name of one of the object's attributes, 0 if not.
  (This is implemented by calling \code{getattr(object, name)} and
  seeing whether it raises an exception or not.)
\end{funcdesc}

\begin{funcdesc}{hash}{object}
  Return the hash value of the object (if it has one).  Hash values
  are 32-bit integers.  They are used to quickly compare dictionary
  keys during a dictionary lookup.  Numeric values that compare equal
  have the same hash value (even if they are of different types, e.g.
  1 and 1.0).
\end{funcdesc}

\begin{funcdesc}{hex}{x}
  Convert a number to a hexadecimal string.  The result is a valid
  Python expression.
\end{funcdesc}

\begin{funcdesc}{id}{object}
  Return the `identity' of an object.  This is an integer which is
  guaranteed to be unique and constant for this object during its
  lifetime.  (Two objects whose lifetimes are disjunct may have the
  same id() value.)  (Implementation note: this is the address of the
  object.)
\end{funcdesc}

\begin{funcdesc}{input}{\optional{prompt}}
  Almost equivalent to \code{eval(raw_input(\var{prompt}))}.  Like
  \code{raw_input()}, the \var{prompt} argument is optional.  The difference
  is that a long input expression may be broken over multiple lines using
  the backslash convention.
\end{funcdesc}

\begin{funcdesc}{int}{x}
  Convert a number to a plain integer.  The argument may be a plain or
  long integer or a floating point number.
\end{funcdesc}

\begin{funcdesc}{len}{s}
  Return the length (the number of items) of an object.  The argument
  may be a sequence (string, tuple or list) or a mapping (dictionary).
\end{funcdesc}

\begin{funcdesc}{long}{x}
  Convert a number to a long integer.  The argument may be a plain or
  long integer or a floating point number.
\end{funcdesc}

\begin{funcdesc}{map}{function\, list\, ...}
Apply \var{function} to every item of \var{list} and return a list
of the results.  If additional \var{list} arguments are passed, 
\var{function} must take that many arguments and is applied to
the items of all lists in parallel; if a list is shorter than another
it is assumed to be extended with \code{None} items.  If
\var{function} is \code{None}, the identity function is assumed; if
there are multiple list arguments, \code{map} returns a list
consisting of tuples containing the corresponding items from all lists
(i.e. a kind of transpose operation).  The \var{list} arguments may be
any kind of sequence; the result is always a list.
\end{funcdesc}

\begin{funcdesc}{max}{s}
  Return the largest item of a non-empty sequence (string, tuple or
  list).
\end{funcdesc}

\begin{funcdesc}{min}{s}
  Return the smallest item of a non-empty sequence (string, tuple or
  list).
\end{funcdesc}

\begin{funcdesc}{oct}{x}
  Convert a number to an octal string.  The result is a valid Python
  expression.
\end{funcdesc}

\begin{funcdesc}{open}{filename\, \optional{mode\optional{\, bufsize}}}
  Return a new file object (described earlier under Built-in Types).
  The first two arguments are the same as for \code{stdio}'s
  \code{fopen()}: \var{filename} is the file name to be opened,
  \var{mode} indicates how the file is to be opened: \code{'r'} for
  reading, \code{'w'} for writing (truncating an existing file), and
  \code{'a'} opens it for appending.  Modes \code{'r+'}, \code{'w+'} and
  \code{'a+'} open the file for updating, provided the underlying
  \code{stdio} library understands this.  On systems that differentiate
  between binary and text files, \code{'b'} appended to the mode opens
  the file in binary mode.  If the file cannot be opened, \code{IOError}
  is raised.
If \var{mode} is omitted, it defaults to \code{'r'}.
The optional \var{bufsize} argument specifies the file's desired
buffer size: 0 means unbuffered, 1 means line buffered, any other
positive value means use a buffer of (approximately) that size.  A
negative \var{bufsize} means to use the system default, which is
usually line buffered for for tty devices and fully buffered for other
files.%
\footnote{Specifying a buffer size currently has no effect on systems
that don't have \code{setvbuf()}.  The interface to specify the buffer
size is not done using a method that calls \code{setvbuf()}, because
that may dump core when called after any I/O has been performed, and
there's no reliable way to determine whether this is the case.}
\end{funcdesc}

\begin{funcdesc}{ord}{c}
  Return the \ASCII{} value of a string of one character.  E.g.,
  \code{ord('a')} returns the integer \code{97}.  This is the inverse of
  \code{chr()}.
\end{funcdesc}

\begin{funcdesc}{pow}{x\, y\optional{\, z}}
  Return \var{x} to the power \var{y}; if \var{z} is present, $x^y \bmod z$
  is returned.  The arguments must have
  numeric types.  With mixed operand types, the rules for binary
  arithmetic operators apply.  The effective operand type is also the
  type of the result; if the result is not expressible in this type, the
  function raises an exception; e.g., \code{pow(2, -1)} or \code{pow(2,
  35000)} is not allowed.
\end{funcdesc}

\begin{funcdesc}{range}{\optional{start\,} end\optional{\, step}}
  This is a versatile function to create lists containing arithmetic
  progressions.  It is most often used in \code{for} loops.  The
  arguments must be plain integers.  If the \var{step} argument is
  omitted, it defaults to \code{1}.  If the \var{start} argument is
  omitted, it defaults to \code{0}.  The full form returns a list of
  plain integers \code{[\var{start}, \var{start} + \var{step},
  \var{start} + 2 * \var{step}, \ldots]}.  If \var{step} is positive,
  the last element is the largest \code{\var{start} + \var{i} *
  \var{step}} less than \var{end}; if \var{step} is negative, the last
  element is the largest \code{\var{start} + \var{i} * \var{step}}
  greater than \var{end}.  \var{step} must not be zero.  Example:

\bcode\begin{verbatim}
>>> range(10)
[0, 1, 2, 3, 4, 5, 6, 7, 8, 9]
>>> range(1, 11)
[1, 2, 3, 4, 5, 6, 7, 8, 9, 10]
>>> range(0, 30, 5)
[0, 5, 10, 15, 20, 25]
>>> range(0, 10, 3)
[0, 3, 6, 9]
>>> range(0, -10, -1)
[0, -1, -2, -3, -4, -5, -6, -7, -8, -9]
>>> range(0)
[]
>>> range(1, 0)
[]
>>> 
\end{verbatim}\ecode
\end{funcdesc}

\begin{funcdesc}{raw_input}{\optional{prompt}}
  If the \var{prompt} argument is present, it is written to standard output
  without a trailing newline.  The function then reads a line from input,
  converts it to a string (stripping a trailing newline), and returns that.
  When \EOF{} is read, \code{EOFError} is raised. Example:

\bcode\begin{verbatim}
>>> s = raw_input('--> ')
--> Monty Python's Flying Circus
>>> s
'Monty Python\'s Flying Circus'
>>> 
\end{verbatim}\ecode
\end{funcdesc}

\begin{funcdesc}{reduce}{function\, list\optional{\, initializer}}
Apply the binary \var{function} to the items of \var{list} so as to
reduce the list to a single value.  E.g.,
\code{reduce(lambda x, y: x*y, \var{list}, 1)} returns the product of
the elements of \var{list}.  The optional \var{initializer} can be
thought of as being prepended to \var{list} so as to allow reduction
of an empty \var{list}.  The \var{list} arguments may be any kind of
sequence.
\end{funcdesc}

\begin{funcdesc}{reload}{module}
  Re-parse and re-initialize an already imported \var{module}.  The
  argument must be a module object, so it must have been successfully
  imported before.  This is useful if you have edited the module source
  file using an external editor and want to try out the new version
  without leaving the Python interpreter.  Note that if a module is
  syntactically correct but its initialization fails, the first
  \code{import} statement for it does not import the name, but does
  create a (partially initialized) module object; to reload the module
  you must first \code{import} it again (this will just make the
  partially initialized module object available) before you can
  \code{reload()} it.
\end{funcdesc}

\begin{funcdesc}{repr}{object}
Return a string containing a printable representation of an object.
This is the same value yielded by conversions (reverse quotes).
It is sometimes useful to be able to access this operation as an
ordinary function.  For many types, this function makes an attempt
to return a string that would yield an object with the same value
when passed to \code{eval()}.
\end{funcdesc}

\begin{funcdesc}{round}{x\, n}
  Return the floating point value \var{x} rounded to \var{n} digits
  after the decimal point.  If \var{n} is omitted, it defaults to zero.
  The result is a floating point number.  Values are rounded to the
  closest multiple of 10 to the power minus \var{n}; if two multiples
  are equally close, rounding is done away from 0 (so e.g.
  \code{round(0.5)} is \code{1.0} and \code{round(-0.5)} is \code{-1.0}).
\end{funcdesc}

\begin{funcdesc}{setattr}{object\, name\, value}
  This is the counterpart of \code{getattr}.  The arguments are an
  object, a string and an arbitrary value.  The string must be the name
  of one of the object's attributes.  The function assigns the value to
  the attribute, provided the object allows it.  For example,
  \code{setattr(\var{x}, '\var{foobar}', 123)} is equivalent to
  \code{\var{x}.\var{foobar} = 123}.
\end{funcdesc}

\begin{funcdesc}{str}{object}
Return a string containing a nicely printable representation of an
object.  For strings, this returns the string itself.  The difference
with \code{repr(\var{object}} is that \code{str(\var{object}} does not
always attempt to return a string that is acceptable to \code{eval()};
its goal is to return a printable string.
\end{funcdesc}

\begin{funcdesc}{type}{object}
% XXXJH xref to buil-in objects here?
  Return the type of an \var{object}.  The return value is a type
  object.  There is not much you can do with type objects except compare
  them to other type objects; e.g., the following checks if a variable
  is a string:

\bcode\begin{verbatim}
>>> if type(x) == type(''): print 'It is a string'
\end{verbatim}\ecode
\end{funcdesc}

\begin{funcdesc}{vars}{}
Without arguments, return a dictionary corresponding to the current
local symbol table.  With a module, class or class instance object as
argument (or anything else that has a \code{__dict__} attribute),
returns a dictionary corresponding to the object's symbol table.
The returned dictionary should not be modified: the effects on the
corresponding symbol table are undefined.%
\footnote{In the current implementation, local variable bindings
cannot normally be affected this way, but variables retrieved from
other scopes can be.  This may change.}
\end{funcdesc}

\begin{funcdesc}{xrange}{\optional{start\,} end\optional{\, step}}
This function is very similar to \code{range()}, but returns an
``xrange object'' instead of a list.  This is an opaque sequence type
which yields the same values as the corresponding list, without
actually storing them all simultaneously.  The advantage of
\code{xrange()} over \code{range()} is minimal (since \code{xrange()}
still has to create the values when asked for them) except when a very
large range is used on a memory-starved machine (e.g. DOS) or when all
of the range's elements are never used (e.g. when the loop is usually
terminated with \code{break}).
\end{funcdesc}
