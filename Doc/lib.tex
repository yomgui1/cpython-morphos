% Format this file with latex.

\documentstyle[myformat]{report}		% To preview with xdvi

% Underscores are not magic throughout this document
\catcode`_=12

% Dummy \cbstart and \cbend so I can leave the changebars in...
\newcommand{\cbstart}{}
\newcommand{\cbend}{}

% Define \itembreak: force the text after an item to start on a new line
\newcommand{\itembreak}{
\mbox{}
\\*[0mm]
}

% Define \itemjoin: some negative vspace to join two items together
\newcommand{\itemjoin}{
\mbox{}
\vspace{-\itemsep}
\vspace{-\parsep}
}

% Define \funcitem{func}{args}: define a function item
\newcommand{\funcitem}[2]{
\index{#1@{\tt#1}}
\item[{\tt #1(#2)}]
\ 
}

% Define \dataitem{name}: define a data item
\newcommand{\dataitem}[1]{
\index{#1@{\tt#1}}
\item[{\tt #1}]
\ 
}

% Define \excitem{name}{string}: define an exception item
\newcommand{\excitem}[2]{
\index{#1@{\tt#1}}
\item[{\tt #1 = '#2'}]
\itembreak
}

\title{\bf
	Python Library Reference
}

\author{
	Guido van Rossum \\
	Dept. CST, CWI, Kruislaan 413 \\
	1098 SJ Amsterdam, The Netherlands \\
	E-mail: {\tt guido@cwi.nl}
}

\makeindex

\begin{document}

\pagenumbering{roman}

\maketitle

\begin{abstract}

\noindent
This document describes the built-in types, exceptions and functions
and the standard modules that come with the Python system.  It assumes
basic knowledge about the Python language.  For an informal
introduction to the language, see the {\em Python Tutorial}.  The {\em
Python Reference Manual} gives a more formal definition of the
language.

\end{abstract}

\pagebreak

\tableofcontents

\pagebreak

\pagenumbering{arabic}

\input{lib1.tex}	% intro; built-in types, functions and exceptions
\input{lib2.tex}	% built-in modules
\input{lib3.tex}	% standard modules
\input{lib4.tex}	% OS-dependent chapters
\input{lib5.tex}	% Graphics chapters
\input{libindex.tex}	% The index

\end{document}
