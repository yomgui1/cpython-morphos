\section{Standard Module \sectcode{Bastion}}
\label{module-Bastion}
\stmodindex{Bastion}
\setindexsubitem{(in module Bastion)}

% I'm concerned that the word 'bastion' won't be understood by people
% for whom English is a second language, making the module name
% somewhat mysterious.  Thus, the brief definition... --amk

According to the dictionary, a bastion is ``a fortified area or
position'', or ``something that is considered a stronghold.''  It's a
suitable name for this module, which provides a way to forbid access
to certain attributes of an object.  It must always be used with the
\module{rexec} module, in order to allow restricted-mode programs access
to certain safe attributes of an object, while denying access to
other, unsafe attributes.

% I've punted on the issue of documenting keyword arguments for now.

\begin{funcdesc}{Bastion}{object\optional{, filter, name, class}}
Protect the object \var{object}, returning a bastion for the
object.  Any attempt to access one of the object's attributes will
have to be approved by the \var{filter} function; if the access is
denied an \exception{AttributeError} exception will be raised.

If present, \var{filter} must be a function that accepts a string
containing an attribute name, and returns true if access to that
attribute will be permitted; if \var{filter} returns false, the access
is denied.  The default filter denies access to any function beginning
with an underscore (\samp{_}).  The bastion's string representation
will be \samp{<Bastion for \var{name}>} if a value for
\var{name} is provided; otherwise, \samp{repr(\var{object})} will be
used.

\var{class}, if present, would be a subclass of \class{BastionClass};
see the code in \file{bastion.py} for the details.  Overriding the
default \class{BastionClass} will rarely be required.

\end{funcdesc}
