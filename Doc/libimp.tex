\section{Built-in Module \sectcode{imp}}
\label{module-imp}
\bimodindex{imp}
\index{import}

This module provides an interface to the mechanisms used to implement
the \keyword{import} statement.  It defines the following constants and
functions:


\begin{funcdesc}{get_magic}{}
Return the magic string value used to recognize byte-compiled code
files (``\code{.pyc} files'').  (This value may be different for each
Python version.)
\end{funcdesc}

\begin{funcdesc}{get_suffixes}{}
Return a list of triples, each describing a particular type of module.
Each triple has the form \code{(\var{suffix}, \var{mode},
\var{type})}, where \var{suffix} is a string to be appended to the
module name to form the filename to search for, \var{mode} is the mode
string to pass to the built-in \code{open} function to open the file
(this can be \code{'r'} for text files or \code{'rb'} for binary
files), and \var{type} is the file type, which has one of the values
\constant{PY_SOURCE}, \constant{PY_COMPILED}, or
\constant{C_EXTENSION}, described below.
\end{funcdesc}

\begin{funcdesc}{find_module}{name\optional{, path}}
Try to find the module \var{name} on the search path \var{path}.  If
\var{path} is a list of directory names, each directory is searched
for files with any of the suffixes returned by \function{get_suffixes()}
above.  Invalid names in the list are silently ignored (but all list
items must be strings).  If \var{path} is omitted or \code{None}, the
list of directory names given by \code{sys.path} is searched, but
first it searches a few special places: it tries to find a built-in
module with the given name (\constant{C_BUILTIN}), then a frozen module
(\constant{PY_FROZEN}), and on some systems some other places are looked
in as well (on the Mac, it looks for a resource (\constant{PY_RESOURCE});
on Windows, it looks in the registry which may point to a specific
file).

If search is successful, the return value is a triple
\code{(\var{file}, \var{pathname}, \var{description})} where
\var{file} is an open file object positioned at the beginning,
\var{pathname} is the pathname of the
file found, and \var{description} is a triple as contained in the list
returned by \function{get_suffixes()} describing the kind of module found.
If the module does not live in a file, the returned \var{file} is
\code{None}, \var{filename} is the empty string, and the
\var{description} tuple contains empty strings for its suffix and
mode; the module type is as indicate in parentheses dabove.  If the
search is unsuccessful, \exception{ImportError} is raised.  Other
exceptions indicate problems with the arguments or environment.

This function does not handle hierarchical module names (names
containing dots).  In order to find \var{P}.\var{M}, i.e., submodule
\var{M} of package \var{P}, use \function{find_module()} and
\function{load_module()} to find and load package \var{P}, and then use
\function{find_module()} with the \var{path} argument set to
\code{\var{P}.__path__}.  When \var{P} itself has a dotted name, apply
this recipe recursively.
\end{funcdesc}

\begin{funcdesc}{load_module}{name, file, filename, description}
Load a module that was previously found by \function{find_module()} (or by
an otherwise conducted search yielding compatible results).  This
function does more than importing the module: if the module was
already imported, it is equivalent to a
\function{reload()}\bifuncindex{reload}!  The
\var{name} argument indicates the full module name (including the
package name, if this is a submodule of a package).  The \var{file}
argument is an open file, and \var{filename} is the corresponding
file name; these can be \code{None} and \code{''}, respectively, when
the module is not being loaded from a file.  The \var{description}
argument is a tuple as returned by \function{find_module()} describing
what kind of module must be loaded.

If the load is successful, the return value is the module object;
otherwise, an exception (usually \exception{ImportError}) is raised.

\strong{Important:} the caller is responsible for closing the
\var{file} argument, if it was not \code{None}, even when an exception
is raised.  This is best done using a \keyword{try}
... \keyword{finally} statement.
\end{funcdesc}

\begin{funcdesc}{new_module}{name}
Return a new empty module object called \var{name}.  This object is
\emph{not} inserted in \code{sys.modules}.
\end{funcdesc}

The following constants with integer values, defined in this module,
are used to indicate the search result of \function{find_module()}.

\begin{datadesc}{PY_SOURCE}
The module was found as a source file.
\end{datadesc}

\begin{datadesc}{PY_COMPILED}
The module was found as a compiled code object file.
\end{datadesc}

\begin{datadesc}{C_EXTENSION}
The module was found as dynamically loadable shared library.
\end{datadesc}

\begin{datadesc}{PY_RESOURCE}
The module was found as a Macintosh resource.  This value can only be
returned on a Macintosh.
\end{datadesc}

\begin{datadesc}{PKG_DIRECTORY}
The module was found as a package directory.
\end{datadesc}

\begin{datadesc}{C_BUILTIN}
The module was found as a built-in module.
\end{datadesc}

\begin{datadesc}{PY_FROZEN}
The module was found as a frozen module (see \function{init_frozen()}).
\end{datadesc}

The following constant and functions are obsolete; their functionality
is available through \function{find_module()} or \function{load_module()}.
They are kept around for backward compatibility:

\begin{datadesc}{SEARCH_ERROR}
Unused.
\end{datadesc}

\begin{funcdesc}{init_builtin}{name}
Initialize the built-in module called \var{name} and return its module
object.  If the module was already initialized, it will be initialized
\emph{again}.  A few modules cannot be initialized twice --- attempting
to initialize these again will raise an \exception{ImportError}
exception.  If there is no
built-in module called \var{name}, \code{None} is returned.
\end{funcdesc}

\begin{funcdesc}{init_frozen}{name}
Initialize the frozen module called \var{name} and return its module
object.  If the module was already initialized, it will be initialized
\emph{again}.  If there is no frozen module called \var{name},
\code{None} is returned.  (Frozen modules are modules written in
Python whose compiled byte-code object is incorporated into a
custom-built Python interpreter by Python's \program{freeze} utility.
See \file{Tools/freeze/} for now.)
\end{funcdesc}

\begin{funcdesc}{is_builtin}{name}
Return \code{1} if there is a built-in module called \var{name} which
can be initialized again.  Return \code{-1} if there is a built-in
module called \var{name} which cannot be initialized again (see
\function{init_builtin()}).  Return \code{0} if there is no built-in
module called \var{name}.
\end{funcdesc}

\begin{funcdesc}{is_frozen}{name}
Return \code{1} if there is a frozen module (see
\function{init_frozen()}) called \var{name}, or \code{0} if there is
no such module.
\end{funcdesc}

\begin{funcdesc}{load_compiled}{name, pathname, file}
Load and initialize a module implemented as a byte-compiled code file
and return its module object.  If the module was already initialized,
it will be initialized \emph{again}.  The \var{name} argument is used
to create or access a module object.  The \var{pathname} argument
points to the byte-compiled code file.  The \var{file}
argument is the byte-compiled code file, open for reading in binary
mode, from the beginning.
It must currently be a real file object, not a
user-defined class emulating a file.
\end{funcdesc}

\begin{funcdesc}{load_dynamic}{name, pathname\optional{, file}}
Load and initialize a module implemented as a dynamically loadable
shared library and return its module object.  If the module was
already initialized, it will be initialized \emph{again}.  Some modules
don't like that and may raise an exception.  The \var{pathname}
argument must point to the shared library.  The \var{name} argument is
used to construct the name of the initialization function: an external
C function called \samp{init\var{name}()} in the shared library is
called.  The optional \var{file} argment is ignored.  (Note: using
shared libraries is highly system dependent, and not all systems
support it.)
\end{funcdesc}

\begin{funcdesc}{load_source}{name, pathname, file}
Load and initialize a module implemented as a Python source file and
return its module object.  If the module was already initialized, it
will be initialized \emph{again}.  The \var{name} argument is used to
create or access a module object.  The \var{pathname} argument points
to the source file.  The \var{file} argument is the source
file, open for reading as text, from the beginning.
It must currently be a real file
object, not a user-defined class emulating a file.  Note that if a
properly matching byte-compiled file (with suffix \file{.pyc}) exists,
it will be used instead of parsing the given source file.
\end{funcdesc}


\subsection{Examples}
\label{examples-imp}

The following function emulates what was the standard import statement
up to Python 1.4 (i.e., no hierarchical module names).  (This
\emph{implementation} wouldn't work in that version, since
\function{find_module()} has been extended and
\function{load_module()} has been added in 1.4.)

\begin{verbatim}
import imp import sys

def __import__(name, globals=None, locals=None, fromlist=None):
    # Fast path: see if the module has already been imported.
    try:
        return sys.modules[name]
    except KeyError:
        pass

    # If any of the following calls raises an exception,
    # there's a problem we can't handle -- let the caller handle it.

    fp, pathname, description = imp.find_module(name)
    
    try:
        return imp.load_module(name, fp, pathname, description)
    finally:
        # Since we may exit via an exception, close fp explicitly.
        if fp:
            fp.close()
\end{verbatim}

A more complete example that implements hierarchical module names and
includes a \function{reload()}\bifuncindex{reload} function can be
found in the standard module \module{knee}\refstmodindex{knee} (which
is intended as an example only --- don't rely on any part of it being
a standard interface).
