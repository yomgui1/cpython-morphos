\chapter{The Very High Level Layer \label{veryhigh}}


The functions in this chapter will let you execute Python source code
given in a file or a buffer, but they will not let you interact in a
more detailed way with the interpreter.

Several of these functions accept a start symbol from the grammar as a 
parameter.  The available start symbols are \constant{Py_eval_input},
\constant{Py_file_input}, and \constant{Py_single_input}.  These are
described following the functions which accept them as parameters.

Note also that several of these functions take \ctype{FILE*}
parameters.  On particular issue which needs to be handled carefully
is that the \ctype{FILE} structure for different C libraries can be
different and incompatible.  Under Windows (at least), it is possible
for dynamically linked extensions to actually use different libraries,
so care should be taken that \ctype{FILE*} parameters are only passed
to these functions if it is certain that they were created by the same
library that the Python runtime is using.


\begin{cfuncdesc}{int}{Py_Main}{int argc, char **argv}
  The main program for the standard interpreter.  This is made
  available for programs which embed Python.  The \var{argc} and
  \var{argv} parameters should be prepared exactly as those which are
  passed to a C program's \cfunction{main()} function.  It is
  important to note that the argument list may be modified (but the
  contents of the strings pointed to by the argument list are not).
  The return value will be the integer passed to the
  \function{sys.exit()} function, \code{1} if the interpreter exits
  due to an exception, or \code{2} if the parameter list does not
  represent a valid Python command line.
\end{cfuncdesc}

\begin{cfuncdesc}{int}{PyRun_AnyFile}{FILE *fp, const char *filename}
  This is a simplified interface to \cfunction{PyRun_AnyFileExFlags()}
  below, leaving \var{closeit} set to \code{0} and \var{flags} set to \NULL.
\end{cfuncdesc}

\begin{cfuncdesc}{int}{PyRun_AnyFileFlags}{FILE *fp, const char *filename,
                                           PyCompilerFlags *flags}
  This is a simplified interface to \cfunction{PyRun_AnyFileExFlags()}
  below, leaving the \var{closeit} argument set to \code{0}.
\end{cfuncdesc}

\begin{cfuncdesc}{int}{PyRun_AnyFileEx}{FILE *fp, const char *filename,
                                        int closeit}
  This is a simplified interface to \cfunction{PyRun_AnyFileExFlags()}
  below, leaving the \var{flags} argument set to \NULL.
\end{cfuncdesc}

\begin{cfuncdesc}{int}{PyRun_AnyFileExFlags}{FILE *fp, const char *filename,
                                             int closeit,
                                             PyCompilerFlags *flags}
  If \var{fp} refers to a file associated with an interactive device
  (console or terminal input or \UNIX{} pseudo-terminal), return the
  value of \cfunction{PyRun_InteractiveLoop()}, otherwise return the
  result of \cfunction{PyRun_SimpleFile()}.  If \var{filename} is
  \NULL, this function uses \code{"???"} as the filename.
\end{cfuncdesc}

\begin{cfuncdesc}{int}{PyRun_SimpleString}{const char *command}
  This is a simplified interface to \cfunction{PyRun_SimpleStringFlags()}
  below, leaving the \var{PyCompilerFlags*} argument set to NULL.
\end{cfuncdesc}

\begin{cfuncdesc}{int}{PyRun_SimpleStringFlags}{const char *command,
                                                PyCompilerFlags *flags}
  Executes the Python source code from \var{command} in the
  \module{__main__} module according to the \var{flags} argument.
  If \module{__main__} does not already exist, it is created.  Returns
  \code{0} on success or \code{-1} if an exception was raised.  If there
  was an error, there is no way to get the exception information.
  For the meaning of \var{flags}, see below.
\end{cfuncdesc}

\begin{cfuncdesc}{int}{PyRun_SimpleFile}{FILE *fp, const char *filename}
  This is a simplified interface to \cfunction{PyRun_SimpleFileExFlags()}
  below, leaving \var{closeit} set to \code{0} and \var{flags} set to
  \NULL.
\end{cfuncdesc}

\begin{cfuncdesc}{int}{PyRun_SimpleFileFlags}{FILE *fp, const char *filename,
                                              PyCompilerFlags *flags}
  This is a simplified interface to \cfunction{PyRun_SimpleFileExFlags()}
  below, leaving \var{closeit} set to \code{0}.
\end{cfuncdesc}

\begin{cfuncdesc}{int}{PyRun_SimpleFileEx}{FILE *fp, const char *filename,
                                           int closeit}
  This is a simplified interface to \cfunction{PyRun_SimpleFileExFlags()}
  below, leaving \var{flags} set to \NULL.
\end{cfuncdesc}

\begin{cfuncdesc}{int}{PyRun_SimpleFileExFlags}{FILE *fp, const char *filename,
                                                int closeit,
                                                PyCompilerFlags *flags}
  Similar to \cfunction{PyRun_SimpleStringFlags()}, but the Python source
  code is read from \var{fp} instead of an in-memory string.
  \var{filename} should be the name of the file.  If \var{closeit} is
  true, the file is closed before PyRun_SimpleFileExFlags returns.
\end{cfuncdesc}

\begin{cfuncdesc}{int}{PyRun_InteractiveOne}{FILE *fp, const char *filename}
  This is a simplified interface to \cfunction{PyRun_InteractiveOneFlags()}
  below, leaving \var{flags} set to \NULL.
\end{cfuncdesc}

\begin{cfuncdesc}{int}{PyRun_InteractiveOneFlags}{FILE *fp,
                                                  const char *filename,
                                                  PyCompilerFlags *flags}
  Read and execute a single statement from a file associated with an
  interactive device according to the \var{flags} argument.  If
  \var{filename} is \NULL, \code{"???"} is used instead.  The user will
  be prompted using \code{sys.ps1} and \code{sys.ps2}.  Returns \code{0}
  when the input was executed successfully, \code{-1} if there was an
  exception, or an error code from the \file{errcode.h} include file
  distributed as part of Python if there was a parse error.  (Note that
  \file{errcode.h} is not included by \file{Python.h}, so must be included
  specifically if needed.)
\end{cfuncdesc}

\begin{cfuncdesc}{int}{PyRun_InteractiveLoop}{FILE *fp, const char *filename}
  This is a simplified interface to \cfunction{PyRun_InteractiveLoopFlags()}
  below, leaving \var{flags} set to \NULL.
\end{cfuncdesc}

\begin{cfuncdesc}{int}{PyRun_InteractiveLoopFlags}{FILE *fp, 
                                                   const char *filename,
                                                   PyCompilerFlags *flags}
  Read and execute statements from a file associated with an
  interactive device until \EOF{} is reached.  If \var{filename} is
  \NULL, \code{"???"} is used instead.  The user will be prompted
  using \code{sys.ps1} and \code{sys.ps2}.  Returns \code{0} at \EOF.
\end{cfuncdesc}

\begin{cfuncdesc}{struct _node*}{PyParser_SimpleParseString}{const char *str,
                                                             int start}
  This is a simplified interface to
  \cfunction{PyParser_SimpleParseStringFlagsFilename()} below, leaving 
  \var{filename} set to \NULL{} and \var{flags} set to \code{0}.
\end{cfuncdesc}

\begin{cfuncdesc}{struct _node*}{PyParser_SimpleParseStringFlags}{
                                 const char *str, int start, int flags}
  This is a simplified interface to
  \cfunction{PyParser_SimpleParseStringFlagsFilename()} below, leaving 
  \var{filename} set to \NULL.
\end{cfuncdesc}

\begin{cfuncdesc}{struct _node*}{PyParser_SimpleParseStringFlagsFilename}{
                                 const char *str, const char *filename,
                                 int start, int flags}
  Parse Python source code from \var{str} using the start token
  \var{start} according to the \var{flags} argument.  The result can
  be used to create a code object which can be evaluated efficiently.
  This is useful if a code fragment must be evaluated many times.
\end{cfuncdesc}

\begin{cfuncdesc}{struct _node*}{PyParser_SimpleParseFile}{FILE *fp,
                                 const char *filename, int start}
  This is a simplified interface to \cfunction{PyParser_SimpleParseFileFlags()}
  below, leaving \var{flags} set to \code{0}
\end{cfuncdesc}

\begin{cfuncdesc}{struct _node*}{PyParser_SimpleParseFileFlags}{FILE *fp,
                                 const char *filename, int start, int flags}
  Similar to \cfunction{PyParser_SimpleParseStringFlagsFilename()}, but
  the Python source code is read from \var{fp} instead of an in-memory
  string.
\end{cfuncdesc}

\begin{cfuncdesc}{PyObject*}{PyRun_String}{const char *str, int start,
                                           PyObject *globals,
                                           PyObject *locals}
  This is a simplified interface to \cfunction{PyRun_StringFlags()} below,
  leaving \var{flags} set to \NULL.
\end{cfuncdesc}

\begin{cfuncdesc}{PyObject*}{PyRun_StringFlags}{const char *str, int start,
                                                PyObject *globals,
                                                PyObject *locals,
                                                PyCompilerFlags *flags}
  Execute Python source code from \var{str} in the context specified
  by the dictionaries \var{globals} and \var{locals} with the compiler
  flags specified by \var{flags}.  The parameter \var{start} specifies
  the start token that should be used to parse the source code.

  Returns the result of executing the code as a Python object, or
  \NULL{} if an exception was raised.
\end{cfuncdesc}

\begin{cfuncdesc}{PyObject*}{PyRun_File}{FILE *fp, const char *filename,
                                         int start, PyObject *globals,
                                         PyObject *locals}
  This is a simplified interface to \cfunction{PyRun_FileExFlags()} below,
  leaving \var{closeit} set to \code{0} and \var{flags} set to \NULL.
\end{cfuncdesc}

\begin{cfuncdesc}{PyObject*}{PyRun_FileEx}{FILE *fp, const char *filename,
                                         int start, PyObject *globals,
                                         PyObject *locals, int closeit}
  This is a simplified interface to \cfunction{PyRun_FileExFlags()} below,
  leaving \var{flags} set to \NULL.
\end{cfuncdesc}

\begin{cfuncdesc}{PyObject*}{PyRun_FileFlags}{FILE *fp, const char *filename,
                                         int start, PyObject *globals,
                                         PyObject *locals,
                                         PyCompilerFlags *flags}
  This is a simplified interface to \cfunction{PyRun_FileExFlags()} below,
  leaving \var{closeit} set to \code{0}.
\end{cfuncdesc}

\begin{cfuncdesc}{PyObject*}{PyRun_FileExFlags}{FILE *fp, const char *filename,
                                                int start, PyObject *globals,
                                                PyObject *locals, int closeit,
                                                PyCompilerFlags *flags}
  Similar to \cfunction{PyRun_StringFlags()}, but the Python source code is
  read from \var{fp} instead of an in-memory string.
  \var{filename} should be the name of the file.
  If \var{closeit} is true, the file is closed before
  \cfunction{PyRun_FileExFlags()} returns.
\end{cfuncdesc}

\begin{cfuncdesc}{PyObject*}{Py_CompileString}{const char *str,
                                               const char *filename,
                                               int start}
  This is a simplified interface to \cfunction{Py_CompileStringFlags()} below,
  leaving \var{flags} set to \NULL.
\end{cfuncdesc}

\begin{cfuncdesc}{PyObject*}{Py_CompileStringFlags}{const char *str,
                                                    const char *filename,
                                                    int start,
                                                    PyCompilerFlags *flags}
  Parse and compile the Python source code in \var{str}, returning the
  resulting code object.  The start token is given by \var{start};
  this can be used to constrain the code which can be compiled and should
  be \constant{Py_eval_input}, \constant{Py_file_input}, or
  \constant{Py_single_input}.  The filename specified by
  \var{filename} is used to construct the code object and may appear
  in tracebacks or \exception{SyntaxError} exception messages.  This
  returns \NULL{} if the code cannot be parsed or compiled.
\end{cfuncdesc}

\begin{cvardesc}{int}{Py_eval_input}
  The start symbol from the Python grammar for isolated expressions;
  for use with
  \cfunction{Py_CompileString()}\ttindex{Py_CompileString()}.
\end{cvardesc}

\begin{cvardesc}{int}{Py_file_input}
  The start symbol from the Python grammar for sequences of statements
  as read from a file or other source; for use with
  \cfunction{Py_CompileString()}\ttindex{Py_CompileString()}.  This is
  the symbol to use when compiling arbitrarily long Python source code.
\end{cvardesc}

\begin{cvardesc}{int}{Py_single_input}
  The start symbol from the Python grammar for a single statement; for
  use with \cfunction{Py_CompileString()}\ttindex{Py_CompileString()}.
  This is the symbol used for the interactive interpreter loop.
\end{cvardesc}

\begin{ctypedesc}[PyCompilerFlags]{struct PyCompilerFlags}
  This is the structure used to hold compiler flags.  In cases where
  code is only being compiled, it is passed as \code{int flags}, and in
  cases where code is being executed, it is passed as
  \code{PyCompilerFlags *flags}.  In this case, \code{from __future__
  import} can modify \var{flags}.

  Whenever \code{PyCompilerFlags *flags} is \NULL, \member{cf_flags}
  is treated as equal to \code{0}, and any modification due to
  \code{from __future__ import} is discarded.
\begin{verbatim}
struct PyCompilerFlags {
    int cf_flags;
}
\end{verbatim}
\end{ctypedesc}

\begin{cvardesc}{int}{CO_FUTURE_DIVISION}
  This bit can be set in \var{flags} to cause division operator \code{/}
  to be interpreted as ``true division'' according to \pep{238}.
\end{cvardesc}
