\chapter{Concrete Objects Layer \label{concrete}}


The functions in this chapter are specific to certain Python object
types.  Passing them an object of the wrong type is not a good idea;
if you receive an object from a Python program and you are not sure
that it has the right type, you must perform a type check first;
for example, to check that an object is a dictionary, use
\cfunction{PyDict_Check()}.  The chapter is structured like the
``family tree'' of Python object types.

\warning{While the functions described in this chapter carefully check
the type of the objects which are passed in, many of them do not check
for \NULL{} being passed instead of a valid object.  Allowing \NULL{}
to be passed in can cause memory access violations and immediate
termination of the interpreter.}


\section{Fundamental Objects \label{fundamental}}

This section describes Python type objects and the singleton object
\code{None}.


\subsection{Type Objects \label{typeObjects}}

\obindex{type}
\begin{ctypedesc}{PyTypeObject}
  The C structure of the objects used to describe built-in types.
\end{ctypedesc}

\begin{cvardesc}{PyObject*}{PyType_Type}
  This is the type object for type objects; it is the same object as
  \code{types.TypeType} in the Python layer.
  \withsubitem{(in module types)}{\ttindex{TypeType}}
\end{cvardesc}

\begin{cfuncdesc}{int}{PyType_Check}{PyObject *o}
  Return true if the object \var{o} is a type object, including
  instances of types derived from the standard type object.  Return
  false in all other cases.
\end{cfuncdesc}

\begin{cfuncdesc}{int}{PyType_CheckExact}{PyObject *o}
  Return true if the object \var{o} is a type object, but not a
  subtype of the standard type object.  Return false in all other
  cases.
  \versionadded{2.2}
\end{cfuncdesc}

\begin{cfuncdesc}{int}{PyType_HasFeature}{PyObject *o, int feature}
  Return true if the type object \var{o} sets the feature
  \var{feature}.  Type features are denoted by single bit flags.
\end{cfuncdesc}

\begin{cfuncdesc}{int}{PyType_IS_GC}{PyObject *o}
  Return true if the type object includes support for the cycle
  detector; this tests the type flag \constant{Py_TPFLAGS_HAVE_GC}.
  \versionadded{2.0}
\end{cfuncdesc}

\begin{cfuncdesc}{int}{PyType_IsSubtype}{PyTypeObject *a, PyTypeObject *b}
  Return true if \var{a} is a subtype of \var{b}.
  \versionadded{2.2}
\end{cfuncdesc}

\begin{cfuncdesc}{PyObject*}{PyType_GenericAlloc}{PyTypeObject *type,
                                                  Py_ssize_t nitems}
  \versionadded{2.2}
\end{cfuncdesc}

\begin{cfuncdesc}{PyObject*}{PyType_GenericNew}{PyTypeObject *type,
                                            PyObject *args, PyObject *kwds}
  \versionadded{2.2}
\end{cfuncdesc}

\begin{cfuncdesc}{int}{PyType_Ready}{PyTypeObject *type}
  Finalize a type object.  This should be called on all type objects
  to finish their initialization.  This function is responsible for
  adding inherited slots from a type's base class.  Return \code{0}
  on success, or return \code{-1} and sets an exception on error.
  \versionadded{2.2}
\end{cfuncdesc}


\subsection{The None Object \label{noneObject}}

\obindex{None}
Note that the \ctype{PyTypeObject} for \code{None} is not directly
exposed in the Python/C API.  Since \code{None} is a singleton,
testing for object identity (using \samp{==} in C) is sufficient.
There is no \cfunction{PyNone_Check()} function for the same reason.

\begin{cvardesc}{PyObject*}{Py_None}
  The Python \code{None} object, denoting lack of value.  This object
  has no methods.  It needs to be treated just like any other object
  with respect to reference counts.
\end{cvardesc}

\begin{csimplemacrodesc}{Py_RETURN_NONE}
  Properly handle returning \cdata{Py_None} from within a C function.
\end{csimplemacrodesc}


\section{Numeric Objects \label{numericObjects}}

\obindex{numeric}


\subsection{Plain Integer Objects \label{intObjects}}

\obindex{integer}
\begin{ctypedesc}{PyIntObject}
  This subtype of \ctype{PyObject} represents a Python integer
  object.
\end{ctypedesc}

\begin{cvardesc}{PyTypeObject}{PyInt_Type}
  This instance of \ctype{PyTypeObject} represents the Python plain
  integer type.  This is the same object as \code{types.IntType}.
  \withsubitem{(in modules types)}{\ttindex{IntType}}
\end{cvardesc}

\begin{cfuncdesc}{int}{PyInt_Check}{PyObject *o}
  Return true if \var{o} is of type \cdata{PyInt_Type} or a subtype
  of \cdata{PyInt_Type}.
  \versionchanged[Allowed subtypes to be accepted]{2.2}
\end{cfuncdesc}

\begin{cfuncdesc}{int}{PyInt_CheckExact}{PyObject *o}
  Return true if \var{o} is of type \cdata{PyInt_Type}, but not a
  subtype of \cdata{PyInt_Type}.
  \versionadded{2.2}
\end{cfuncdesc}

\begin{cfuncdesc}{PyObject*}{PyInt_FromString}{char *str, char **pend,
                                               int base}
  Return a new \ctype{PyIntObject} or \ctype{PyLongObject} based on the
  string value in \var{str}, which is interpreted according to the radix in
  \var{base}.  If \var{pend} is non-\NULL{}, \code{*\var{pend}} will point to
  the first character in \var{str} which follows the representation of the
  number.  If \var{base} is \code{0}, the radix will be determined based on
  the leading characters of \var{str}: if \var{str} starts with \code{'0x'}
  or \code{'0X'}, radix 16 will be used; if \var{str} starts with
  \code{'0'}, radix 8 will be used; otherwise radix 10 will be used.  If
  \var{base} is not \code{0}, it must be between \code{2} and \code{36},
  inclusive.  Leading spaces are ignored.  If there are no digits,
  \exception{ValueError} will be raised.  If the string represents a number
  too large to be contained within the machine's \ctype{long int} type and
  overflow warnings are being suppressed, a \ctype{PyLongObject} will be
  returned.  If overflow warnings are not being suppressed, \NULL{} will be
  returned in this case.
\end{cfuncdesc}

\begin{cfuncdesc}{PyObject*}{PyInt_FromLong}{long ival}
  Create a new integer object with a value of \var{ival}.

  The current implementation keeps an array of integer objects for all
  integers between \code{-5} and \code{256}, when you create an int in
  that range you actually just get back a reference to the existing
  object. So it should be possible to change the value of \code{1}.  I
  suspect the behaviour of Python in this case is undefined. :-)
\end{cfuncdesc}

\begin{cfuncdesc}{PyObject*}{PyInt_FromSsize_t}{Py_ssize_t ival}
  Create a new integer object with a value of \var{ival}.
  If the value exceeds \code{LONG_MAX}, a long integer object is
  returned.

 \versionadded{2.5}
\end{cfuncdesc}

\begin{cfuncdesc}{long}{PyInt_AsLong}{PyObject *io}
  Will first attempt to cast the object to a \ctype{PyIntObject}, if
  it is not already one, and then return its value. If there is an
  error, \code{-1} is returned, and the caller should check
  \code{PyErr_Occurred()} to find out whether there was an error, or
  whether the value just happened to be -1.
\end{cfuncdesc}

\begin{cfuncdesc}{long}{PyInt_AS_LONG}{PyObject *io}
  Return the value of the object \var{io}.  No error checking is
  performed.
\end{cfuncdesc}

\begin{cfuncdesc}{unsigned long}{PyInt_AsUnsignedLongMask}{PyObject *io}
  Will first attempt to cast the object to a \ctype{PyIntObject} or
  \ctype{PyLongObject}, if it is not already one, and then return its
  value as unsigned long.  This function does not check for overflow.
  \versionadded{2.3}
\end{cfuncdesc}

\begin{cfuncdesc}{unsigned PY_LONG_LONG}{PyInt_AsUnsignedLongLongMask}{PyObject *io}
  Will first attempt to cast the object to a \ctype{PyIntObject} or
  \ctype{PyLongObject}, if it is not already one, and then return its
  value as unsigned long long, without checking for overflow.
  \versionadded{2.3}
\end{cfuncdesc}

\begin{cfuncdesc}{Py_ssize_t}{PyInt_AsSsize_t}{PyObject *io}
  Will first attempt to cast the object to a \ctype{PyIntObject} or
  \ctype{PyLongObject}, if it is not already one, and then return its
  value as \ctype{Py_ssize_t}.
  \versionadded{2.5}
\end{cfuncdesc}

\begin{cfuncdesc}{long}{PyInt_GetMax}{}
  Return the system's idea of the largest integer it can handle
  (\constant{LONG_MAX}\ttindex{LONG_MAX}, as defined in the system
  header files).
\end{cfuncdesc}

\subsection{Boolean Objects \label{boolObjects}}

Booleans in Python are implemented as a subclass of integers.  There
are only two booleans, \constant{Py_False} and \constant{Py_True}.  As
such, the normal creation and deletion functions don't apply to
booleans.  The following macros are available, however.

\begin{cfuncdesc}{int}{PyBool_Check}{PyObject *o}
  Return true if \var{o} is of type \cdata{PyBool_Type}.
  \versionadded{2.3}
\end{cfuncdesc}

\begin{cvardesc}{PyObject*}{Py_False}
  The Python \code{False} object.  This object has no methods.  It needs to
  be treated just like any other object with respect to reference counts.
\end{cvardesc}

\begin{cvardesc}{PyObject*}{Py_True}
  The Python \code{True} object.  This object has no methods.  It needs to
  be treated just like any other object with respect to reference counts.
\end{cvardesc}

\begin{csimplemacrodesc}{Py_RETURN_FALSE}
  Return \constant{Py_False} from a function, properly incrementing its
  reference count.
\versionadded{2.4}
\end{csimplemacrodesc}

\begin{csimplemacrodesc}{Py_RETURN_TRUE}
  Return \constant{Py_True} from a function, properly incrementing its
  reference count.
\versionadded{2.4}
\end{csimplemacrodesc}

\begin{cfuncdesc}{PyObject*}{PyBool_FromLong}{long v}
  Return a new reference to \constant{Py_True} or \constant{Py_False} 
  depending on the truth value of \var{v}.
\versionadded{2.3}
\end{cfuncdesc}

\subsection{Long Integer Objects \label{longObjects}}

\obindex{long integer}
\begin{ctypedesc}{PyLongObject}
  This subtype of \ctype{PyObject} represents a Python long integer
  object.
\end{ctypedesc}

\begin{cvardesc}{PyTypeObject}{PyLong_Type}
  This instance of \ctype{PyTypeObject} represents the Python long
  integer type.  This is the same object as \code{types.LongType}.
  \withsubitem{(in modules types)}{\ttindex{LongType}}
\end{cvardesc}

\begin{cfuncdesc}{int}{PyLong_Check}{PyObject *p}
  Return true if its argument is a \ctype{PyLongObject} or a subtype
  of \ctype{PyLongObject}.
  \versionchanged[Allowed subtypes to be accepted]{2.2}
\end{cfuncdesc}

\begin{cfuncdesc}{int}{PyLong_CheckExact}{PyObject *p}
  Return true if its argument is a \ctype{PyLongObject}, but not a
  subtype of \ctype{PyLongObject}.
  \versionadded{2.2}
\end{cfuncdesc}

\begin{cfuncdesc}{PyObject*}{PyLong_FromLong}{long v}
  Return a new \ctype{PyLongObject} object from \var{v}, or \NULL{}
  on failure.
\end{cfuncdesc}

\begin{cfuncdesc}{PyObject*}{PyLong_FromUnsignedLong}{unsigned long v}
  Return a new \ctype{PyLongObject} object from a C \ctype{unsigned
  long}, or \NULL{} on failure.
\end{cfuncdesc}

\begin{cfuncdesc}{PyObject*}{PyLong_FromLongLong}{PY_LONG_LONG v}
  Return a new \ctype{PyLongObject} object from a C \ctype{long long},
  or \NULL{} on failure.
\end{cfuncdesc}

\begin{cfuncdesc}{PyObject*}{PyLong_FromUnsignedLongLong}{unsigned PY_LONG_LONG v}
  Return a new \ctype{PyLongObject} object from a C \ctype{unsigned
  long long}, or \NULL{} on failure.
\end{cfuncdesc}

\begin{cfuncdesc}{PyObject*}{PyLong_FromDouble}{double v}
  Return a new \ctype{PyLongObject} object from the integer part of
  \var{v}, or \NULL{} on failure.
\end{cfuncdesc}

\begin{cfuncdesc}{PyObject*}{PyLong_FromString}{char *str, char **pend,
                                                int base}
  Return a new \ctype{PyLongObject} based on the string value in
  \var{str}, which is interpreted according to the radix in
  \var{base}.  If \var{pend} is non-\NULL{}, \code{*\var{pend}} will
  point to the first character in \var{str} which follows the
  representation of the number.  If \var{base} is \code{0}, the radix
  will be determined based on the leading characters of \var{str}: if
  \var{str} starts with \code{'0x'} or \code{'0X'}, radix 16 will be
  used; if \var{str} starts with \code{'0'}, radix 8 will be used;
  otherwise radix 10 will be used.  If \var{base} is not \code{0}, it
  must be between \code{2} and \code{36}, inclusive.  Leading spaces
  are ignored.  If there are no digits, \exception{ValueError} will be
  raised.
\end{cfuncdesc}

\begin{cfuncdesc}{PyObject*}{PyLong_FromUnicode}{Py_UNICODE *u,
                                                 Py_ssize_t length, int base}
  Convert a sequence of Unicode digits to a Python long integer
  value.  The first parameter, \var{u}, points to the first character
  of the Unicode string, \var{length} gives the number of characters,
  and \var{base} is the radix for the conversion.  The radix must be
  in the range [2, 36]; if it is out of range, \exception{ValueError}
  will be raised.
  \versionadded{1.6}
\end{cfuncdesc}

\begin{cfuncdesc}{PyObject*}{PyLong_FromVoidPtr}{void *p}
  Create a Python integer or long integer from the pointer \var{p}.
  The pointer value can be retrieved from the resulting value using
  \cfunction{PyLong_AsVoidPtr()}.
  \versionadded{1.5.2}
\end{cfuncdesc}

\begin{cfuncdesc}{long}{PyLong_AsLong}{PyObject *pylong}
  Return a C \ctype{long} representation of the contents of
  \var{pylong}.  If \var{pylong} is greater than
  \constant{LONG_MAX}\ttindex{LONG_MAX}, an \exception{OverflowError}
  is raised.
  \withsubitem{(built-in exception)}{\ttindex{OverflowError}}
\end{cfuncdesc}

\begin{cfuncdesc}{unsigned long}{PyLong_AsUnsignedLong}{PyObject *pylong}
  Return a C \ctype{unsigned long} representation of the contents of
  \var{pylong}.  If \var{pylong} is greater than
  \constant{ULONG_MAX}\ttindex{ULONG_MAX}, an
  \exception{OverflowError} is raised.
  \withsubitem{(built-in exception)}{\ttindex{OverflowError}}
\end{cfuncdesc}

\begin{cfuncdesc}{PY_LONG_LONG}{PyLong_AsLongLong}{PyObject *pylong}
  Return a C \ctype{long long} from a Python long integer.  If
  \var{pylong} cannot be represented as a \ctype{long long}, an
  \exception{OverflowError} will be raised.
  \versionadded{2.2}
\end{cfuncdesc}

\begin{cfuncdesc}{unsigned PY_LONG_LONG}{PyLong_AsUnsignedLongLong}{PyObject
                                                                 *pylong}
  Return a C \ctype{unsigned long long} from a Python long integer.
  If \var{pylong} cannot be represented as an \ctype{unsigned long
  long}, an \exception{OverflowError} will be raised if the value is
  positive, or a \exception{TypeError} will be raised if the value is
  negative.
  \versionadded{2.2}
\end{cfuncdesc}

\begin{cfuncdesc}{unsigned long}{PyLong_AsUnsignedLongMask}{PyObject *io}
  Return a C \ctype{unsigned long} from a Python long integer, without
  checking for overflow.
  \versionadded{2.3}
\end{cfuncdesc}

\begin{cfuncdesc}{unsigned long}{PyLong_AsUnsignedLongLongMask}{PyObject *io}
  Return a C \ctype{unsigned long long} from a Python long integer, without
  checking for overflow.
  \versionadded{2.3}
\end{cfuncdesc}

\begin{cfuncdesc}{double}{PyLong_AsDouble}{PyObject *pylong}
  Return a C \ctype{double} representation of the contents of
  \var{pylong}.  If \var{pylong} cannot be approximately represented
  as a \ctype{double}, an \exception{OverflowError} exception is
  raised and \code{-1.0} will be returned.
\end{cfuncdesc}

\begin{cfuncdesc}{void*}{PyLong_AsVoidPtr}{PyObject *pylong}
  Convert a Python integer or long integer \var{pylong} to a C
  \ctype{void} pointer.  If \var{pylong} cannot be converted, an
  \exception{OverflowError} will be raised.  This is only assured to
  produce a usable \ctype{void} pointer for values created with
  \cfunction{PyLong_FromVoidPtr()}.
  \versionadded{1.5.2}
\end{cfuncdesc}


\subsection{Floating Point Objects \label{floatObjects}}

\obindex{floating point}
\begin{ctypedesc}{PyFloatObject}
  This subtype of \ctype{PyObject} represents a Python floating point
  object.
\end{ctypedesc}

\begin{cvardesc}{PyTypeObject}{PyFloat_Type}
  This instance of \ctype{PyTypeObject} represents the Python floating
  point type.  This is the same object as \code{types.FloatType}.
  \withsubitem{(in modules types)}{\ttindex{FloatType}}
\end{cvardesc}

\begin{cfuncdesc}{int}{PyFloat_Check}{PyObject *p}
  Return true if its argument is a \ctype{PyFloatObject} or a subtype
  of \ctype{PyFloatObject}.
  \versionchanged[Allowed subtypes to be accepted]{2.2}
\end{cfuncdesc}

\begin{cfuncdesc}{int}{PyFloat_CheckExact}{PyObject *p}
  Return true if its argument is a \ctype{PyFloatObject}, but not a
  subtype of \ctype{PyFloatObject}.
  \versionadded{2.2}
\end{cfuncdesc}

\begin{cfuncdesc}{PyObject*}{PyFloat_FromString}{PyObject *str, char **pend}
  Create a \ctype{PyFloatObject} object based on the string value in
  \var{str}, or \NULL{} on failure.  The \var{pend} argument is ignored.  It
  remains only for backward compatibility.
\end{cfuncdesc}

\begin{cfuncdesc}{PyObject*}{PyFloat_FromDouble}{double v}
  Create a \ctype{PyFloatObject} object from \var{v}, or \NULL{} on
  failure.
\end{cfuncdesc}

\begin{cfuncdesc}{double}{PyFloat_AsDouble}{PyObject *pyfloat}
  Return a C \ctype{double} representation of the contents of
  \var{pyfloat}.
\end{cfuncdesc}

\begin{cfuncdesc}{double}{PyFloat_AS_DOUBLE}{PyObject *pyfloat}
  Return a C \ctype{double} representation of the contents of
  \var{pyfloat}, but without error checking.
\end{cfuncdesc}


\subsection{Complex Number Objects \label{complexObjects}}

\obindex{complex number}
Python's complex number objects are implemented as two distinct types
when viewed from the C API:  one is the Python object exposed to
Python programs, and the other is a C structure which represents the
actual complex number value.  The API provides functions for working
with both.

\subsubsection{Complex Numbers as C Structures}

Note that the functions which accept these structures as parameters
and return them as results do so \emph{by value} rather than
dereferencing them through pointers.  This is consistent throughout
the API.

\begin{ctypedesc}{Py_complex}
  The C structure which corresponds to the value portion of a Python
  complex number object.  Most of the functions for dealing with
  complex number objects use structures of this type as input or
  output values, as appropriate.  It is defined as:

\begin{verbatim}
typedef struct {
   double real;
   double imag;
} Py_complex;
\end{verbatim}
\end{ctypedesc}

\begin{cfuncdesc}{Py_complex}{_Py_c_sum}{Py_complex left, Py_complex right}
  Return the sum of two complex numbers, using the C
  \ctype{Py_complex} representation.
\end{cfuncdesc}

\begin{cfuncdesc}{Py_complex}{_Py_c_diff}{Py_complex left, Py_complex right}
  Return the difference between two complex numbers, using the C
  \ctype{Py_complex} representation.
\end{cfuncdesc}

\begin{cfuncdesc}{Py_complex}{_Py_c_neg}{Py_complex complex}
  Return the negation of the complex number \var{complex}, using the C
  \ctype{Py_complex} representation.
\end{cfuncdesc}

\begin{cfuncdesc}{Py_complex}{_Py_c_prod}{Py_complex left, Py_complex right}
  Return the product of two complex numbers, using the C
  \ctype{Py_complex} representation.
\end{cfuncdesc}

\begin{cfuncdesc}{Py_complex}{_Py_c_quot}{Py_complex dividend,
                                          Py_complex divisor}
  Return the quotient of two complex numbers, using the C
  \ctype{Py_complex} representation.
\end{cfuncdesc}

\begin{cfuncdesc}{Py_complex}{_Py_c_pow}{Py_complex num, Py_complex exp}
  Return the exponentiation of \var{num} by \var{exp}, using the C
  \ctype{Py_complex} representation.
\end{cfuncdesc}


\subsubsection{Complex Numbers as Python Objects}

\begin{ctypedesc}{PyComplexObject}
  This subtype of \ctype{PyObject} represents a Python complex number
  object.
\end{ctypedesc}

\begin{cvardesc}{PyTypeObject}{PyComplex_Type}
  This instance of \ctype{PyTypeObject} represents the Python complex
  number type.
\end{cvardesc}

\begin{cfuncdesc}{int}{PyComplex_Check}{PyObject *p}
  Return true if its argument is a \ctype{PyComplexObject} or a
  subtype of \ctype{PyComplexObject}.
  \versionchanged[Allowed subtypes to be accepted]{2.2}
\end{cfuncdesc}

\begin{cfuncdesc}{int}{PyComplex_CheckExact}{PyObject *p}
  Return true if its argument is a \ctype{PyComplexObject}, but not a
  subtype of \ctype{PyComplexObject}.
  \versionadded{2.2}
\end{cfuncdesc}

\begin{cfuncdesc}{PyObject*}{PyComplex_FromCComplex}{Py_complex v}
  Create a new Python complex number object from a C
  \ctype{Py_complex} value.
\end{cfuncdesc}

\begin{cfuncdesc}{PyObject*}{PyComplex_FromDoubles}{double real, double imag}
  Return a new \ctype{PyComplexObject} object from \var{real} and
  \var{imag}.
\end{cfuncdesc}

\begin{cfuncdesc}{double}{PyComplex_RealAsDouble}{PyObject *op}
  Return the real part of \var{op} as a C \ctype{double}.
\end{cfuncdesc}

\begin{cfuncdesc}{double}{PyComplex_ImagAsDouble}{PyObject *op}
  Return the imaginary part of \var{op} as a C \ctype{double}.
\end{cfuncdesc}

\begin{cfuncdesc}{Py_complex}{PyComplex_AsCComplex}{PyObject *op}
  Return the \ctype{Py_complex} value of the complex number
  \var{op}.
\end{cfuncdesc}



\section{Sequence Objects \label{sequenceObjects}}

\obindex{sequence}
Generic operations on sequence objects were discussed in the previous
chapter; this section deals with the specific kinds of sequence
objects that are intrinsic to the Python language.


\subsection{String Objects \label{stringObjects}}

These functions raise \exception{TypeError} when expecting a string
parameter and are called with a non-string parameter.

\obindex{string}
\begin{ctypedesc}{PyStringObject}
  This subtype of \ctype{PyObject} represents a Python string object.
\end{ctypedesc}

\begin{cvardesc}{PyTypeObject}{PyString_Type}
  This instance of \ctype{PyTypeObject} represents the Python string
  type; it is the same object as \code{types.TypeType} in the Python
  layer.
  \withsubitem{(in module types)}{\ttindex{StringType}}.
\end{cvardesc}

\begin{cfuncdesc}{int}{PyString_Check}{PyObject *o}
  Return true if the object \var{o} is a string object or an instance
  of a subtype of the string type.
  \versionchanged[Allowed subtypes to be accepted]{2.2}
\end{cfuncdesc}

\begin{cfuncdesc}{int}{PyString_CheckExact}{PyObject *o}
  Return true if the object \var{o} is a string object, but not an
  instance of a subtype of the string type.
  \versionadded{2.2}
\end{cfuncdesc}

\begin{cfuncdesc}{PyObject*}{PyString_FromString}{const char *v}
  Return a new string object with the value \var{v} on success, and
  \NULL{} on failure.  The parameter \var{v} must not be \NULL{}; it
  will not be checked.
\end{cfuncdesc}

\begin{cfuncdesc}{PyObject*}{PyString_FromStringAndSize}{const char *v,
                                                         Py_ssize_t len}
  Return a new string object with the value \var{v} and length
  \var{len} on success, and \NULL{} on failure.  If \var{v} is
  \NULL{}, the contents of the string are uninitialized.
\end{cfuncdesc}

\begin{cfuncdesc}{PyObject*}{PyString_FromFormat}{const char *format, ...}
  Take a C \cfunction{printf()}-style \var{format} string and a
  variable number of arguments, calculate the size of the resulting
  Python string and return a string with the values formatted into
  it.  The variable arguments must be C types and must correspond
  exactly to the format characters in the \var{format} string.  The
  following format characters are allowed:

  \begin{tableiii}{l|l|l}{member}{Format Characters}{Type}{Comment}
    \lineiii{\%\%}{\emph{n/a}}{The literal \% character.}
    \lineiii{\%c}{int}{A single character, represented as an C int.}
    \lineiii{\%d}{int}{Exactly equivalent to \code{printf("\%d")}.}
    \lineiii{\%ld}{long}{Exactly equivalent to \code{printf("\%ld")}.}
    \lineiii{\%zd}{long}{Exactly equivalent to \code{printf("\%zd")}.}
    \lineiii{\%i}{int}{Exactly equivalent to \code{printf("\%i")}.}
    \lineiii{\%x}{int}{Exactly equivalent to \code{printf("\%x")}.}
    \lineiii{\%s}{char*}{A null-terminated C character array.}
    \lineiii{\%p}{void*}{The hex representation of a C pointer.
	Mostly equivalent to \code{printf("\%p")} except that it is
	guaranteed to start with the literal \code{0x} regardless of
	what the platform's \code{printf} yields.}
  \end{tableiii}
\end{cfuncdesc}

\begin{cfuncdesc}{PyObject*}{PyString_FromFormatV}{const char *format,
                                                   va_list vargs}
  Identical to \function{PyString_FromFormat()} except that it takes
  exactly two arguments.
\end{cfuncdesc}

\begin{cfuncdesc}{Py_ssize_t}{PyString_Size}{PyObject *string}
  Return the length of the string in string object \var{string}.
\end{cfuncdesc}

\begin{cfuncdesc}{Py_ssize_t}{PyString_GET_SIZE}{PyObject *string}
  Macro form of \cfunction{PyString_Size()} but without error
  checking.
\end{cfuncdesc}

\begin{cfuncdesc}{char*}{PyString_AsString}{PyObject *string}
  Return a NUL-terminated representation of the contents of
  \var{string}.  The pointer refers to the internal buffer of
  \var{string}, not a copy.  The data must not be modified in any way,
  unless the string was just created using
  \code{PyString_FromStringAndSize(NULL, \var{size})}.
  It must not be deallocated.  If \var{string} is a Unicode object,
  this function computes the default encoding of \var{string} and
  operates on that.  If \var{string} is not a string object at all,
  \cfunction{PyString_AsString()} returns \NULL{} and raises
  \exception{TypeError}.
\end{cfuncdesc}

\begin{cfuncdesc}{char*}{PyString_AS_STRING}{PyObject *string}
  Macro form of \cfunction{PyString_AsString()} but without error
  checking.  Only string objects are supported; no Unicode objects
  should be passed.
\end{cfuncdesc}

\begin{cfuncdesc}{int}{PyString_AsStringAndSize}{PyObject *obj,
                                                 char **buffer,
                                                 Py_ssize_t *length}
  Return a NUL-terminated representation of the contents of the
  object \var{obj} through the output variables \var{buffer} and
  \var{length}.

  The function accepts both string and Unicode objects as input. For
  Unicode objects it returns the default encoded version of the
  object.  If \var{length} is \NULL{}, the resulting buffer may not
  contain NUL characters; if it does, the function returns \code{-1}
  and a \exception{TypeError} is raised.

  The buffer refers to an internal string buffer of \var{obj}, not a
  copy. The data must not be modified in any way, unless the string
  was just created using \code{PyString_FromStringAndSize(NULL,
  \var{size})}.  It must not be deallocated.  If \var{string} is a
  Unicode object, this function computes the default encoding of
  \var{string} and operates on that.  If \var{string} is not a string
  object at all, \cfunction{PyString_AsStringAndSize()} returns 
  \code{-1} and raises \exception{TypeError}.
\end{cfuncdesc}

\begin{cfuncdesc}{void}{PyString_Concat}{PyObject **string,
                                         PyObject *newpart}
  Create a new string object in \var{*string} containing the contents
  of \var{newpart} appended to \var{string}; the caller will own the
  new reference.  The reference to the old value of \var{string} will
  be stolen.  If the new string cannot be created, the old reference
  to \var{string} will still be discarded and the value of
  \var{*string} will be set to \NULL{}; the appropriate exception will
  be set.
\end{cfuncdesc}

\begin{cfuncdesc}{void}{PyString_ConcatAndDel}{PyObject **string,
                                               PyObject *newpart}
  Create a new string object in \var{*string} containing the contents
  of \var{newpart} appended to \var{string}.  This version decrements
  the reference count of \var{newpart}.
\end{cfuncdesc}

\begin{cfuncdesc}{int}{_PyString_Resize}{PyObject **string, Py_ssize_t newsize}
  A way to resize a string object even though it is ``immutable''.
  Only use this to build up a brand new string object; don't use this
  if the string may already be known in other parts of the code.  It
  is an error to call this function if the refcount on the input string
  object is not one.
  Pass the address of an existing string object as an lvalue (it may
  be written into), and the new size desired.  On success, \var{*string}
  holds the resized string object and \code{0} is returned; the address in
  \var{*string} may differ from its input value.  If the
  reallocation fails, the original string object at \var{*string} is
  deallocated, \var{*string} is set to \NULL{}, a memory exception is set,
  and \code{-1} is returned.
\end{cfuncdesc}

\begin{cfuncdesc}{PyObject*}{PyString_Format}{PyObject *format,
                                              PyObject *args}
  Return a new string object from \var{format} and \var{args}.
  Analogous to \code{\var{format} \%\ \var{args}}.  The \var{args}
  argument must be a tuple.
\end{cfuncdesc}

\begin{cfuncdesc}{void}{PyString_InternInPlace}{PyObject **string}
  Intern the argument \var{*string} in place.  The argument must be
  the address of a pointer variable pointing to a Python string
  object.  If there is an existing interned string that is the same as
  \var{*string}, it sets \var{*string} to it (decrementing the
  reference count of the old string object and incrementing the
  reference count of the interned string object), otherwise it leaves
  \var{*string} alone and interns it (incrementing its reference
  count).  (Clarification: even though there is a lot of talk about
  reference counts, think of this function as reference-count-neutral;
  you own the object after the call if and only if you owned it before
  the call.)
\end{cfuncdesc}

\begin{cfuncdesc}{PyObject*}{PyString_InternFromString}{const char *v}
  A combination of \cfunction{PyString_FromString()} and
  \cfunction{PyString_InternInPlace()}, returning either a new string
  object that has been interned, or a new (``owned'') reference to an
  earlier interned string object with the same value.
\end{cfuncdesc}

\begin{cfuncdesc}{PyObject*}{PyString_Decode}{const char *s,
                                               Py_ssize_t size,
                                               const char *encoding,
                                               const char *errors}
  Create an object by decoding \var{size} bytes of the encoded
  buffer \var{s} using the codec registered for
  \var{encoding}.  \var{encoding} and \var{errors} have the same
  meaning as the parameters of the same name in the
  \function{unicode()} built-in function.  The codec to be used is
  looked up using the Python codec registry.  Return \NULL{} if
  an exception was raised by the codec.
\end{cfuncdesc}

\begin{cfuncdesc}{PyObject*}{PyString_AsDecodedObject}{PyObject *str,
                                               const char *encoding,
                                               const char *errors}
  Decode a string object by passing it to the codec registered for
  \var{encoding} and return the result as Python
  object. \var{encoding} and \var{errors} have the same meaning as the
  parameters of the same name in the string \method{encode()} method.
  The codec to be used is looked up using the Python codec registry.
  Return \NULL{} if an exception was raised by the codec.
\end{cfuncdesc}

\begin{cfuncdesc}{PyObject*}{PyString_Encode}{const char *s,
                                               Py_ssize_t size,
                                               const char *encoding,
                                               const char *errors}
  Encode the \ctype{char} buffer of the given size by passing it to
  the codec registered for \var{encoding} and return a Python object.
  \var{encoding} and \var{errors} have the same meaning as the
  parameters of the same name in the string \method{encode()} method.
  The codec to be used is looked up using the Python codec
  registry.  Return \NULL{} if an exception was raised by the
  codec.
\end{cfuncdesc}

\begin{cfuncdesc}{PyObject*}{PyString_AsEncodedObject}{PyObject *str,
                                               const char *encoding,
                                               const char *errors}
  Encode a string object using the codec registered for
  \var{encoding} and return the result as Python object.
  \var{encoding} and \var{errors} have the same meaning as the
  parameters of the same name in the string \method{encode()} method.
  The codec to be used is looked up using the Python codec registry.
  Return \NULL{} if an exception was raised by the codec.
\end{cfuncdesc}


\subsection{Unicode Objects \label{unicodeObjects}}
\sectionauthor{Marc-Andre Lemburg}{mal@lemburg.com}

%--- Unicode Type -------------------------------------------------------

These are the basic Unicode object types used for the Unicode
implementation in Python:

\begin{ctypedesc}{Py_UNICODE}
  This type represents the storage type which is used by Python
  internally as basis for holding Unicode ordinals.  Python's default
  builds use a 16-bit type for \ctype{Py_UNICODE} and store Unicode
  values internally as UCS2. It is also possible to build a UCS4
  version of Python (most recent Linux distributions come with UCS4
  builds of Python). These builds then use a 32-bit type for
  \ctype{Py_UNICODE} and store Unicode data internally as UCS4. On
  platforms where \ctype{wchar_t} is available and compatible with the
  chosen Python Unicode build variant, \ctype{Py_UNICODE} is a typedef
  alias for \ctype{wchar_t} to enhance native platform compatibility.
  On all other platforms, \ctype{Py_UNICODE} is a typedef alias for
  either \ctype{unsigned short} (UCS2) or \ctype{unsigned long}
  (UCS4).
\end{ctypedesc}

Note that UCS2 and UCS4 Python builds are not binary compatible.
Please keep this in mind when writing extensions or interfaces.

\begin{ctypedesc}{PyUnicodeObject}
  This subtype of \ctype{PyObject} represents a Python Unicode object.
\end{ctypedesc}

\begin{cvardesc}{PyTypeObject}{PyUnicode_Type}
  This instance of \ctype{PyTypeObject} represents the Python Unicode
  type.
\end{cvardesc}

The following APIs are really C macros and can be used to do fast
checks and to access internal read-only data of Unicode objects:

\begin{cfuncdesc}{int}{PyUnicode_Check}{PyObject *o}
  Return true if the object \var{o} is a Unicode object or an
  instance of a Unicode subtype.
  \versionchanged[Allowed subtypes to be accepted]{2.2}
\end{cfuncdesc}

\begin{cfuncdesc}{int}{PyUnicode_CheckExact}{PyObject *o}
  Return true if the object \var{o} is a Unicode object, but not an
  instance of a subtype.
  \versionadded{2.2}
\end{cfuncdesc}

\begin{cfuncdesc}{Py_ssize_t}{PyUnicode_GET_SIZE}{PyObject *o}
  Return the size of the object.  \var{o} has to be a
  \ctype{PyUnicodeObject} (not checked).
\end{cfuncdesc}

\begin{cfuncdesc}{Py_ssize_t}{PyUnicode_GET_DATA_SIZE}{PyObject *o}
  Return the size of the object's internal buffer in bytes.  \var{o}
  has to be a \ctype{PyUnicodeObject} (not checked).
\end{cfuncdesc}

\begin{cfuncdesc}{Py_UNICODE*}{PyUnicode_AS_UNICODE}{PyObject *o}
  Return a pointer to the internal \ctype{Py_UNICODE} buffer of the
  object.  \var{o} has to be a \ctype{PyUnicodeObject} (not checked).
\end{cfuncdesc}

\begin{cfuncdesc}{const char*}{PyUnicode_AS_DATA}{PyObject *o}
  Return a pointer to the internal buffer of the object.
  \var{o} has to be a \ctype{PyUnicodeObject} (not checked).
\end{cfuncdesc}

% --- Unicode character properties ---------------------------------------

Unicode provides many different character properties. The most often
needed ones are available through these macros which are mapped to C
functions depending on the Python configuration.

\begin{cfuncdesc}{int}{Py_UNICODE_ISSPACE}{Py_UNICODE ch}
  Return 1 or 0 depending on whether \var{ch} is a whitespace
  character.
\end{cfuncdesc}

\begin{cfuncdesc}{int}{Py_UNICODE_ISLOWER}{Py_UNICODE ch}
  Return 1 or 0 depending on whether \var{ch} is a lowercase character.
\end{cfuncdesc}

\begin{cfuncdesc}{int}{Py_UNICODE_ISUPPER}{Py_UNICODE ch}
  Return 1 or 0 depending on whether \var{ch} is an uppercase
  character.
\end{cfuncdesc}

\begin{cfuncdesc}{int}{Py_UNICODE_ISTITLE}{Py_UNICODE ch}
  Return 1 or 0 depending on whether \var{ch} is a titlecase character.
\end{cfuncdesc}

\begin{cfuncdesc}{int}{Py_UNICODE_ISLINEBREAK}{Py_UNICODE ch}
  Return 1 or 0 depending on whether \var{ch} is a linebreak character.
\end{cfuncdesc}

\begin{cfuncdesc}{int}{Py_UNICODE_ISDECIMAL}{Py_UNICODE ch}
  Return 1 or 0 depending on whether \var{ch} is a decimal character.
\end{cfuncdesc}

\begin{cfuncdesc}{int}{Py_UNICODE_ISDIGIT}{Py_UNICODE ch}
  Return 1 or 0 depending on whether \var{ch} is a digit character.
\end{cfuncdesc}

\begin{cfuncdesc}{int}{Py_UNICODE_ISNUMERIC}{Py_UNICODE ch}
  Return 1 or 0 depending on whether \var{ch} is a numeric character.
\end{cfuncdesc}

\begin{cfuncdesc}{int}{Py_UNICODE_ISALPHA}{Py_UNICODE ch}
  Return 1 or 0 depending on whether \var{ch} is an alphabetic
  character.
\end{cfuncdesc}

\begin{cfuncdesc}{int}{Py_UNICODE_ISALNUM}{Py_UNICODE ch}
  Return 1 or 0 depending on whether \var{ch} is an alphanumeric
  character.
\end{cfuncdesc}

These APIs can be used for fast direct character conversions:

\begin{cfuncdesc}{Py_UNICODE}{Py_UNICODE_TOLOWER}{Py_UNICODE ch}
  Return the character \var{ch} converted to lower case.
\end{cfuncdesc}

\begin{cfuncdesc}{Py_UNICODE}{Py_UNICODE_TOUPPER}{Py_UNICODE ch}
  Return the character \var{ch} converted to upper case.
\end{cfuncdesc}

\begin{cfuncdesc}{Py_UNICODE}{Py_UNICODE_TOTITLE}{Py_UNICODE ch}
  Return the character \var{ch} converted to title case.
\end{cfuncdesc}

\begin{cfuncdesc}{int}{Py_UNICODE_TODECIMAL}{Py_UNICODE ch}
  Return the character \var{ch} converted to a decimal positive
  integer.  Return \code{-1} if this is not possible.  This macro
  does not raise exceptions.
\end{cfuncdesc}

\begin{cfuncdesc}{int}{Py_UNICODE_TODIGIT}{Py_UNICODE ch}
  Return the character \var{ch} converted to a single digit integer.
  Return \code{-1} if this is not possible.  This macro does not raise
  exceptions.
\end{cfuncdesc}

\begin{cfuncdesc}{double}{Py_UNICODE_TONUMERIC}{Py_UNICODE ch}
  Return the character \var{ch} converted to a (positive) double.
  Return \code{-1.0} if this is not possible.  This macro does not raise
  exceptions.
\end{cfuncdesc}

% --- Plain Py_UNICODE ---------------------------------------------------

To create Unicode objects and access their basic sequence properties,
use these APIs:

\begin{cfuncdesc}{PyObject*}{PyUnicode_FromUnicode}{const Py_UNICODE *u,
                                                    Py_ssize_t size}
  Create a Unicode Object from the Py_UNICODE buffer \var{u} of the
  given size. \var{u} may be \NULL{} which causes the contents to be
  undefined. It is the user's responsibility to fill in the needed
  data.  The buffer is copied into the new object. If the buffer is
  not \NULL{}, the return value might be a shared object. Therefore,
  modification of the resulting Unicode object is only allowed when
  \var{u} is \NULL{}.
\end{cfuncdesc}

\begin{cfuncdesc}{Py_UNICODE*}{PyUnicode_AsUnicode}{PyObject *unicode}
  Return a read-only pointer to the Unicode object's internal
  \ctype{Py_UNICODE} buffer, \NULL{} if \var{unicode} is not a Unicode
  object.
\end{cfuncdesc}

\begin{cfuncdesc}{Py_ssize_t}{PyUnicode_GetSize}{PyObject *unicode}
  Return the length of the Unicode object.
\end{cfuncdesc}

\begin{cfuncdesc}{PyObject*}{PyUnicode_FromEncodedObject}{PyObject *obj,
                                                      const char *encoding,
                                                      const char *errors}
  Coerce an encoded object \var{obj} to an Unicode object and return a
  reference with incremented refcount.

  Coercion is done in the following way:

\begin{enumerate}
\item  Unicode objects are passed back as-is with incremented
       refcount. \note{These cannot be decoded; passing a non-\NULL{}
       value for encoding will result in a \exception{TypeError}.}

\item String and other char buffer compatible objects are decoded
      according to the given encoding and using the error handling
      defined by errors.  Both can be \NULL{} to have the interface
      use the default values (see the next section for details).

\item All other objects cause an exception.
\end{enumerate}

  The API returns \NULL{} if there was an error.  The caller is
  responsible for decref'ing the returned objects.
\end{cfuncdesc}

\begin{cfuncdesc}{PyObject*}{PyUnicode_FromObject}{PyObject *obj}
  Shortcut for \code{PyUnicode_FromEncodedObject(obj, NULL, "strict")}
  which is used throughout the interpreter whenever coercion to
  Unicode is needed.
\end{cfuncdesc}

% --- wchar_t support for platforms which support it ---------------------

If the platform supports \ctype{wchar_t} and provides a header file
wchar.h, Python can interface directly to this type using the
following functions. Support is optimized if Python's own
\ctype{Py_UNICODE} type is identical to the system's \ctype{wchar_t}.

\begin{cfuncdesc}{PyObject*}{PyUnicode_FromWideChar}{const wchar_t *w,
                                                     Py_ssize_t size}
  Create a Unicode object from the \ctype{wchar_t} buffer \var{w} of
  the given size.  Return \NULL{} on failure.
\end{cfuncdesc}

\begin{cfuncdesc}{Py_ssize_t}{PyUnicode_AsWideChar}{PyUnicodeObject *unicode,
                                             wchar_t *w,
                                             Py_ssize_t size}
  Copy the Unicode object contents into the \ctype{wchar_t} buffer
  \var{w}.  At most \var{size} \ctype{wchar_t} characters are copied
  (excluding a possibly trailing 0-termination character).  Return
  the number of \ctype{wchar_t} characters copied or -1 in case of an
  error.  Note that the resulting \ctype{wchar_t} string may or may
  not be 0-terminated.  It is the responsibility of the caller to make
  sure that the \ctype{wchar_t} string is 0-terminated in case this is
  required by the application.
\end{cfuncdesc}


\subsubsection{Built-in Codecs \label{builtinCodecs}}

Python provides a set of builtin codecs which are written in C
for speed. All of these codecs are directly usable via the
following functions.

Many of the following APIs take two arguments encoding and
errors. These parameters encoding and errors have the same semantics
as the ones of the builtin unicode() Unicode object constructor.

Setting encoding to \NULL{} causes the default encoding to be used
which is \ASCII.  The file system calls should use
\cdata{Py_FileSystemDefaultEncoding} as the encoding for file
names. This variable should be treated as read-only: On some systems,
it will be a pointer to a static string, on others, it will change at
run-time (such as when the application invokes setlocale).

Error handling is set by errors which may also be set to \NULL{}
meaning to use the default handling defined for the codec.  Default
error handling for all builtin codecs is ``strict''
(\exception{ValueError} is raised).

The codecs all use a similar interface.  Only deviation from the
following generic ones are documented for simplicity.

% --- Generic Codecs -----------------------------------------------------

These are the generic codec APIs:

\begin{cfuncdesc}{PyObject*}{PyUnicode_Decode}{const char *s,
                                               Py_ssize_t size,
                                               const char *encoding,
                                               const char *errors}
  Create a Unicode object by decoding \var{size} bytes of the encoded
  string \var{s}.  \var{encoding} and \var{errors} have the same
  meaning as the parameters of the same name in the
  \function{unicode()} builtin function.  The codec to be used is
  looked up using the Python codec registry.  Return \NULL{} if an
  exception was raised by the codec.
\end{cfuncdesc}

\begin{cfuncdesc}{PyObject*}{PyUnicode_Encode}{const Py_UNICODE *s,
                                               Py_ssize_t size,
                                               const char *encoding,
                                               const char *errors}
  Encode the \ctype{Py_UNICODE} buffer of the given size and return
  a Python string object.  \var{encoding} and \var{errors} have the
  same meaning as the parameters of the same name in the Unicode
  \method{encode()} method.  The codec to be used is looked up using
  the Python codec registry.  Return \NULL{} if an exception was
  raised by the codec.
\end{cfuncdesc}

\begin{cfuncdesc}{PyObject*}{PyUnicode_AsEncodedString}{PyObject *unicode,
                                               const char *encoding,
                                               const char *errors}
  Encode a Unicode object and return the result as Python string
  object. \var{encoding} and \var{errors} have the same meaning as the
  parameters of the same name in the Unicode \method{encode()} method.
  The codec to be used is looked up using the Python codec registry.
  Return \NULL{} if an exception was raised by the codec.
\end{cfuncdesc}

% --- UTF-8 Codecs -------------------------------------------------------

These are the UTF-8 codec APIs:

\begin{cfuncdesc}{PyObject*}{PyUnicode_DecodeUTF8}{const char *s,
                                               Py_ssize_t size,
                                               const char *errors}
  Create a Unicode object by decoding \var{size} bytes of the UTF-8
  encoded string \var{s}. Return \NULL{} if an exception was raised
  by the codec.
\end{cfuncdesc}

\begin{cfuncdesc}{PyObject*}{PyUnicode_DecodeUTF8Stateful}{const char *s,
                                               Py_ssize_t size,
                                               const char *errors,
                                               Py_ssize_t *consumed}
  If \var{consumed} is \NULL{}, behave like \cfunction{PyUnicode_DecodeUTF8()}.
  If \var{consumed} is not \NULL{}, trailing incomplete UTF-8 byte sequences
  will not be treated as an error. Those bytes will not be decoded and the
  number of bytes that have been decoded will be stored in \var{consumed}.
  \versionadded{2.4}
\end{cfuncdesc}

\begin{cfuncdesc}{PyObject*}{PyUnicode_EncodeUTF8}{const Py_UNICODE *s,
                                               Py_ssize_t size,
                                               const char *errors}
  Encode the \ctype{Py_UNICODE} buffer of the given size using UTF-8
  and return a Python string object.  Return \NULL{} if an exception
  was raised by the codec.
\end{cfuncdesc}

\begin{cfuncdesc}{PyObject*}{PyUnicode_AsUTF8String}{PyObject *unicode}
  Encode a Unicode objects using UTF-8 and return the result as
  Python string object.  Error handling is ``strict''.  Return
  \NULL{} if an exception was raised by the codec.
\end{cfuncdesc}

% --- UTF-16 Codecs ------------------------------------------------------ */

These are the UTF-16 codec APIs:

\begin{cfuncdesc}{PyObject*}{PyUnicode_DecodeUTF16}{const char *s,
                                               Py_ssize_t size,
                                               const char *errors,
                                               int *byteorder}
  Decode \var{length} bytes from a UTF-16 encoded buffer string and
  return the corresponding Unicode object.  \var{errors} (if
  non-\NULL{}) defines the error handling. It defaults to ``strict''.

  If \var{byteorder} is non-\NULL{}, the decoder starts decoding using
  the given byte order:

\begin{verbatim}
   *byteorder == -1: little endian
   *byteorder == 0:  native order
   *byteorder == 1:  big endian
\end{verbatim}

  and then switches according to all byte order marks (BOM) it finds
  in the input data.  BOMs are not copied into the resulting Unicode
  string.  After completion, \var{*byteorder} is set to the current
  byte order at the end of input data.

  If \var{byteorder} is \NULL{}, the codec starts in native order mode.

  Return \NULL{} if an exception was raised by the codec.
\end{cfuncdesc}

\begin{cfuncdesc}{PyObject*}{PyUnicode_DecodeUTF16Stateful}{const char *s,
                                               Py_ssize_t size,
                                               const char *errors,
                                               int *byteorder,
                                               Py_ssize_t *consumed}
  If \var{consumed} is \NULL{}, behave like
  \cfunction{PyUnicode_DecodeUTF16()}. If \var{consumed} is not \NULL{},
  \cfunction{PyUnicode_DecodeUTF16Stateful()} will not treat trailing incomplete
  UTF-16 byte sequences (such as an odd number of bytes or a split surrogate pair)
  as an error. Those bytes will not be decoded and the number of bytes that
  have been decoded will be stored in \var{consumed}.
  \versionadded{2.4}
\end{cfuncdesc}

\begin{cfuncdesc}{PyObject*}{PyUnicode_EncodeUTF16}{const Py_UNICODE *s,
                                               Py_ssize_t size,
                                               const char *errors,
                                               int byteorder}
  Return a Python string object holding the UTF-16 encoded value of
  the Unicode data in \var{s}.  If \var{byteorder} is not \code{0},
  output is written according to the following byte order:

\begin{verbatim}
   byteorder == -1: little endian
   byteorder == 0:  native byte order (writes a BOM mark)
   byteorder == 1:  big endian
\end{verbatim}

  If byteorder is \code{0}, the output string will always start with
  the Unicode BOM mark (U+FEFF). In the other two modes, no BOM mark
  is prepended.

  If \var{Py_UNICODE_WIDE} is defined, a single \ctype{Py_UNICODE}
  value may get represented as a surrogate pair. If it is not
  defined, each \ctype{Py_UNICODE} values is interpreted as an
  UCS-2 character.

  Return \NULL{} if an exception was raised by the codec.
\end{cfuncdesc}

\begin{cfuncdesc}{PyObject*}{PyUnicode_AsUTF16String}{PyObject *unicode}
  Return a Python string using the UTF-16 encoding in native byte
  order. The string always starts with a BOM mark.  Error handling is
  ``strict''.  Return \NULL{} if an exception was raised by the
  codec.
\end{cfuncdesc}

% --- Unicode-Escape Codecs ----------------------------------------------

These are the ``Unicode Escape'' codec APIs:

\begin{cfuncdesc}{PyObject*}{PyUnicode_DecodeUnicodeEscape}{const char *s,
                                               Py_ssize_t size,
                                               const char *errors}
  Create a Unicode object by decoding \var{size} bytes of the
  Unicode-Escape encoded string \var{s}.  Return \NULL{} if an
  exception was raised by the codec.
\end{cfuncdesc}

\begin{cfuncdesc}{PyObject*}{PyUnicode_EncodeUnicodeEscape}{const Py_UNICODE *s,
                                               Py_ssize_t size}
  Encode the \ctype{Py_UNICODE} buffer of the given size using
  Unicode-Escape and return a Python string object.  Return \NULL{}
  if an exception was raised by the codec.
\end{cfuncdesc}

\begin{cfuncdesc}{PyObject*}{PyUnicode_AsUnicodeEscapeString}{PyObject *unicode}
  Encode a Unicode objects using Unicode-Escape and return the
  result as Python string object.  Error handling is ``strict''.
  Return \NULL{} if an exception was raised by the codec.
\end{cfuncdesc}

% --- Raw-Unicode-Escape Codecs ------------------------------------------

These are the ``Raw Unicode Escape'' codec APIs:

\begin{cfuncdesc}{PyObject*}{PyUnicode_DecodeRawUnicodeEscape}{const char *s,
                                               Py_ssize_t size,
                                               const char *errors}
  Create a Unicode object by decoding \var{size} bytes of the
  Raw-Unicode-Escape encoded string \var{s}.  Return \NULL{} if an
  exception was raised by the codec.
\end{cfuncdesc}

\begin{cfuncdesc}{PyObject*}{PyUnicode_EncodeRawUnicodeEscape}{const Py_UNICODE *s,
                                               Py_ssize_t size,
                                               const char *errors}
  Encode the \ctype{Py_UNICODE} buffer of the given size using
  Raw-Unicode-Escape and return a Python string object.  Return
  \NULL{} if an exception was raised by the codec.
\end{cfuncdesc}

\begin{cfuncdesc}{PyObject*}{PyUnicode_AsRawUnicodeEscapeString}{PyObject *unicode}
  Encode a Unicode objects using Raw-Unicode-Escape and return the
  result as Python string object. Error handling is ``strict''.
  Return \NULL{} if an exception was raised by the codec.
\end{cfuncdesc}

% --- Latin-1 Codecs -----------------------------------------------------

These are the Latin-1 codec APIs:
Latin-1 corresponds to the first 256 Unicode ordinals and only these
are accepted by the codecs during encoding.

\begin{cfuncdesc}{PyObject*}{PyUnicode_DecodeLatin1}{const char *s,
                                                     Py_ssize_t size,
                                                     const char *errors}
  Create a Unicode object by decoding \var{size} bytes of the Latin-1
  encoded string \var{s}.  Return \NULL{} if an exception was raised
  by the codec.
\end{cfuncdesc}

\begin{cfuncdesc}{PyObject*}{PyUnicode_EncodeLatin1}{const Py_UNICODE *s,
                                                     Py_ssize_t size,
                                                     const char *errors}
  Encode the \ctype{Py_UNICODE} buffer of the given size using
  Latin-1 and return a Python string object.  Return \NULL{} if an
  exception was raised by the codec.
\end{cfuncdesc}

\begin{cfuncdesc}{PyObject*}{PyUnicode_AsLatin1String}{PyObject *unicode}
  Encode a Unicode objects using Latin-1 and return the result as
  Python string object.  Error handling is ``strict''.  Return
  \NULL{} if an exception was raised by the codec.
\end{cfuncdesc}

% --- ASCII Codecs -------------------------------------------------------

These are the \ASCII{} codec APIs.  Only 7-bit \ASCII{} data is
accepted. All other codes generate errors.

\begin{cfuncdesc}{PyObject*}{PyUnicode_DecodeASCII}{const char *s,
                                                    Py_ssize_t size,
                                                    const char *errors}
  Create a Unicode object by decoding \var{size} bytes of the
  \ASCII{} encoded string \var{s}.  Return \NULL{} if an exception
  was raised by the codec.
\end{cfuncdesc}

\begin{cfuncdesc}{PyObject*}{PyUnicode_EncodeASCII}{const Py_UNICODE *s,
                                                    Py_ssize_t size,
                                                    const char *errors}
  Encode the \ctype{Py_UNICODE} buffer of the given size using
  \ASCII{} and return a Python string object.  Return \NULL{} if an
  exception was raised by the codec.
\end{cfuncdesc}

\begin{cfuncdesc}{PyObject*}{PyUnicode_AsASCIIString}{PyObject *unicode}
  Encode a Unicode objects using \ASCII{} and return the result as
  Python string object.  Error handling is ``strict''.  Return
  \NULL{} if an exception was raised by the codec.
\end{cfuncdesc}

% --- Character Map Codecs -----------------------------------------------

These are the mapping codec APIs:

This codec is special in that it can be used to implement many
different codecs (and this is in fact what was done to obtain most of
the standard codecs included in the \module{encodings} package). The
codec uses mapping to encode and decode characters.

Decoding mappings must map single string characters to single Unicode
characters, integers (which are then interpreted as Unicode ordinals)
or None (meaning "undefined mapping" and causing an error).

Encoding mappings must map single Unicode characters to single string
characters, integers (which are then interpreted as Latin-1 ordinals)
or None (meaning "undefined mapping" and causing an error).

The mapping objects provided must only support the __getitem__ mapping
interface.

If a character lookup fails with a LookupError, the character is
copied as-is meaning that its ordinal value will be interpreted as
Unicode or Latin-1 ordinal resp. Because of this, mappings only need
to contain those mappings which map characters to different code
points.

\begin{cfuncdesc}{PyObject*}{PyUnicode_DecodeCharmap}{const char *s,
                                               Py_ssize_t size,
                                               PyObject *mapping,
                                               const char *errors}
  Create a Unicode object by decoding \var{size} bytes of the encoded
  string \var{s} using the given \var{mapping} object.  Return
  \NULL{} if an exception was raised by the codec. If \var{mapping} is \NULL{}
  latin-1 decoding will be done. Else it can be a dictionary mapping byte or a
  unicode string, which is treated as a lookup table. Byte values greater
  that the length of the string and U+FFFE "characters" are treated as
  "undefined mapping".
  \versionchanged[Allowed unicode string as mapping argument]{2.4}
\end{cfuncdesc}

\begin{cfuncdesc}{PyObject*}{PyUnicode_EncodeCharmap}{const Py_UNICODE *s,
                                               Py_ssize_t size,
                                               PyObject *mapping,
                                               const char *errors}
  Encode the \ctype{Py_UNICODE} buffer of the given size using the
  given \var{mapping} object and return a Python string object.
  Return \NULL{} if an exception was raised by the codec.
\end{cfuncdesc}

\begin{cfuncdesc}{PyObject*}{PyUnicode_AsCharmapString}{PyObject *unicode,
                                                        PyObject *mapping}
  Encode a Unicode objects using the given \var{mapping} object and
  return the result as Python string object.  Error handling is
  ``strict''.  Return \NULL{} if an exception was raised by the
  codec.
\end{cfuncdesc}

The following codec API is special in that maps Unicode to Unicode.

\begin{cfuncdesc}{PyObject*}{PyUnicode_TranslateCharmap}{const Py_UNICODE *s,
                                               Py_ssize_t size,
                                               PyObject *table,
                                               const char *errors}
  Translate a \ctype{Py_UNICODE} buffer of the given length by
  applying a character mapping \var{table} to it and return the
  resulting Unicode object.  Return \NULL{} when an exception was
  raised by the codec.

  The \var{mapping} table must map Unicode ordinal integers to Unicode
  ordinal integers or None (causing deletion of the character).

  Mapping tables need only provide the method{__getitem__()}
  interface; dictionaries and sequences work well.  Unmapped character
  ordinals (ones which cause a \exception{LookupError}) are left
  untouched and are copied as-is.
\end{cfuncdesc}

% --- MBCS codecs for Windows --------------------------------------------

These are the MBCS codec APIs. They are currently only available on
Windows and use the Win32 MBCS converters to implement the
conversions.  Note that MBCS (or DBCS) is a class of encodings, not
just one.  The target encoding is defined by the user settings on the
machine running the codec.

\begin{cfuncdesc}{PyObject*}{PyUnicode_DecodeMBCS}{const char *s,
                                               Py_ssize_t size,
                                               const char *errors}
  Create a Unicode object by decoding \var{size} bytes of the MBCS
  encoded string \var{s}.  Return \NULL{} if an exception was
  raised by the codec.
\end{cfuncdesc}

\begin{cfuncdesc}{PyObject*}{PyUnicode_EncodeMBCS}{const Py_UNICODE *s,
                                               Py_ssize_t size,
                                               const char *errors}
  Encode the \ctype{Py_UNICODE} buffer of the given size using MBCS
  and return a Python string object.  Return \NULL{} if an exception
  was raised by the codec.
\end{cfuncdesc}

\begin{cfuncdesc}{PyObject*}{PyUnicode_AsMBCSString}{PyObject *unicode}
  Encode a Unicode objects using MBCS and return the result as
  Python string object.  Error handling is ``strict''.  Return
  \NULL{} if an exception was raised by the codec.
\end{cfuncdesc}

% --- Methods & Slots ----------------------------------------------------

\subsubsection{Methods and Slot Functions \label{unicodeMethodsAndSlots}}

The following APIs are capable of handling Unicode objects and strings
on input (we refer to them as strings in the descriptions) and return
Unicode objects or integers as appropriate.

They all return \NULL{} or \code{-1} if an exception occurs.

\begin{cfuncdesc}{PyObject*}{PyUnicode_Concat}{PyObject *left,
                                               PyObject *right}
  Concat two strings giving a new Unicode string.
\end{cfuncdesc}

\begin{cfuncdesc}{PyObject*}{PyUnicode_Split}{PyObject *s,
                                              PyObject *sep,
                                              Py_ssize_t maxsplit}
  Split a string giving a list of Unicode strings.  If sep is \NULL{},
  splitting will be done at all whitespace substrings.  Otherwise,
  splits occur at the given separator.  At most \var{maxsplit} splits
  will be done.  If negative, no limit is set.  Separators are not
  included in the resulting list.
\end{cfuncdesc}

\begin{cfuncdesc}{PyObject*}{PyUnicode_Splitlines}{PyObject *s,
                                                   int keepend}
  Split a Unicode string at line breaks, returning a list of Unicode
  strings.  CRLF is considered to be one line break.  If \var{keepend}
  is 0, the Line break characters are not included in the resulting
  strings.
\end{cfuncdesc}

\begin{cfuncdesc}{PyObject*}{PyUnicode_Translate}{PyObject *str,
                                                  PyObject *table,
                                                  const char *errors}
  Translate a string by applying a character mapping table to it and
  return the resulting Unicode object.

  The mapping table must map Unicode ordinal integers to Unicode
  ordinal integers or None (causing deletion of the character).

  Mapping tables need only provide the \method{__getitem__()}
  interface; dictionaries and sequences work well.  Unmapped character
  ordinals (ones which cause a \exception{LookupError}) are left
  untouched and are copied as-is.

  \var{errors} has the usual meaning for codecs. It may be \NULL{}
  which indicates to use the default error handling.
\end{cfuncdesc}

\begin{cfuncdesc}{PyObject*}{PyUnicode_Join}{PyObject *separator,
                                             PyObject *seq}
  Join a sequence of strings using the given separator and return the
  resulting Unicode string.
\end{cfuncdesc}

\begin{cfuncdesc}{int}{PyUnicode_Tailmatch}{PyObject *str,
                                                  PyObject *substr,
                                                  Py_ssize_t start,
                                                  Py_ssize_t end,
                                                  int direction}
  Return 1 if \var{substr} matches \var{str}[\var{start}:\var{end}] at
  the given tail end (\var{direction} == -1 means to do a prefix
  match, \var{direction} == 1 a suffix match), 0 otherwise.
  Return \code{-1} if an error occurred.                         
\end{cfuncdesc}

\begin{cfuncdesc}{Py_ssize_t}{PyUnicode_Find}{PyObject *str,
                                       PyObject *substr,
                                       Py_ssize_t start,
                                       Py_ssize_t end,
                                       int direction}
  Return the first position of \var{substr} in
  \var{str}[\var{start}:\var{end}] using the given \var{direction}
  (\var{direction} == 1 means to do a forward search,
  \var{direction} == -1 a backward search).  The return value is the
  index of the first match; a value of \code{-1} indicates that no
  match was found, and \code{-2} indicates that an error occurred and
  an exception has been set.
\end{cfuncdesc}

\begin{cfuncdesc}{Py_ssize_t}{PyUnicode_Count}{PyObject *str,
                                        PyObject *substr,
                                        Py_ssize_t start,
                                        Py_ssize_t end}
  Return the number of non-overlapping occurrences of \var{substr} in
  \code{\var{str}[\var{start}:\var{end}]}.  Return \code{-1} if an
  error occurred.
\end{cfuncdesc}

\begin{cfuncdesc}{PyObject*}{PyUnicode_Replace}{PyObject *str,
                                                PyObject *substr,
                                                PyObject *replstr,
                                                Py_ssize_t maxcount}
  Replace at most \var{maxcount} occurrences of \var{substr} in
  \var{str} with \var{replstr} and return the resulting Unicode object.
  \var{maxcount} == -1 means replace all occurrences.
\end{cfuncdesc}

\begin{cfuncdesc}{int}{PyUnicode_Compare}{PyObject *left, PyObject *right}
  Compare two strings and return -1, 0, 1 for less than, equal, and
  greater than, respectively.
\end{cfuncdesc}

\begin{cfuncdesc}{PyObject*}{PyUnicode_Format}{PyObject *format,
                                              PyObject *args}
  Return a new string object from \var{format} and \var{args}; this
  is analogous to \code{\var{format} \%\ \var{args}}.  The
  \var{args} argument must be a tuple.
\end{cfuncdesc}

\begin{cfuncdesc}{int}{PyUnicode_Contains}{PyObject *container,
                                           PyObject *element}
  Check whether \var{element} is contained in \var{container} and
  return true or false accordingly.

  \var{element} has to coerce to a one element Unicode
  string. \code{-1} is returned if there was an error.
\end{cfuncdesc}


\subsection{Buffer Objects \label{bufferObjects}}
\sectionauthor{Greg Stein}{gstein@lyra.org}

\obindex{buffer}
Python objects implemented in C can export a group of functions called
the ``buffer\index{buffer interface} interface.''  These functions can
be used by an object to expose its data in a raw, byte-oriented
format. Clients of the object can use the buffer interface to access
the object data directly, without needing to copy it first.

Two examples of objects that support
the buffer interface are strings and arrays. The string object exposes
the character contents in the buffer interface's byte-oriented
form. An array can also expose its contents, but it should be noted
that array elements may be multi-byte values.

An example user of the buffer interface is the file object's
\method{write()} method. Any object that can export a series of bytes
through the buffer interface can be written to a file. There are a
number of format codes to \cfunction{PyArg_ParseTuple()} that operate
against an object's buffer interface, returning data from the target
object.

More information on the buffer interface is provided in the section
``Buffer Object Structures'' (section~\ref{buffer-structs}), under
the description for \ctype{PyBufferProcs}\ttindex{PyBufferProcs}.

A ``buffer object'' is defined in the \file{bufferobject.h} header
(included by \file{Python.h}). These objects look very similar to
string objects at the Python programming level: they support slicing,
indexing, concatenation, and some other standard string
operations. However, their data can come from one of two sources: from
a block of memory, or from another object which exports the buffer
interface.

Buffer objects are useful as a way to expose the data from another
object's buffer interface to the Python programmer. They can also be
used as a zero-copy slicing mechanism. Using their ability to
reference a block of memory, it is possible to expose any data to the
Python programmer quite easily. The memory could be a large, constant
array in a C extension, it could be a raw block of memory for
manipulation before passing to an operating system library, or it
could be used to pass around structured data in its native, in-memory
format.

\begin{ctypedesc}{PyBufferObject}
  This subtype of \ctype{PyObject} represents a buffer object.
\end{ctypedesc}

\begin{cvardesc}{PyTypeObject}{PyBuffer_Type}
  The instance of \ctype{PyTypeObject} which represents the Python
  buffer type; it is the same object as \code{types.BufferType} in the
  Python layer.\withsubitem{(in module types)}{\ttindex{BufferType}}.
\end{cvardesc}

\begin{cvardesc}{int}{Py_END_OF_BUFFER}
  This constant may be passed as the \var{size} parameter to
  \cfunction{PyBuffer_FromObject()} or
  \cfunction{PyBuffer_FromReadWriteObject()}.  It indicates that the
  new \ctype{PyBufferObject} should refer to \var{base} object from
  the specified \var{offset} to the end of its exported buffer.  Using
  this enables the caller to avoid querying the \var{base} object for
  its length.
\end{cvardesc}

\begin{cfuncdesc}{int}{PyBuffer_Check}{PyObject *p}
  Return true if the argument has type \cdata{PyBuffer_Type}.
\end{cfuncdesc}

\begin{cfuncdesc}{PyObject*}{PyBuffer_FromObject}{PyObject *base,
                                                  Py_ssize_t offset, Py_ssize_t size}
  Return a new read-only buffer object.  This raises
  \exception{TypeError} if \var{base} doesn't support the read-only
  buffer protocol or doesn't provide exactly one buffer segment, or it
  raises \exception{ValueError} if \var{offset} is less than zero. The
  buffer will hold a reference to the \var{base} object, and the
  buffer's contents will refer to the \var{base} object's buffer
  interface, starting as position \var{offset} and extending for
  \var{size} bytes. If \var{size} is \constant{Py_END_OF_BUFFER}, then
  the new buffer's contents extend to the length of the \var{base}
  object's exported buffer data.
\end{cfuncdesc}

\begin{cfuncdesc}{PyObject*}{PyBuffer_FromReadWriteObject}{PyObject *base,
                                                           Py_ssize_t offset,
                                                           Py_ssize_t size}
  Return a new writable buffer object.  Parameters and exceptions are
  similar to those for \cfunction{PyBuffer_FromObject()}.  If the
  \var{base} object does not export the writeable buffer protocol,
  then \exception{TypeError} is raised.
\end{cfuncdesc}

\begin{cfuncdesc}{PyObject*}{PyBuffer_FromMemory}{void *ptr, Py_ssize_t size}
  Return a new read-only buffer object that reads from a specified
  location in memory, with a specified size.  The caller is
  responsible for ensuring that the memory buffer, passed in as
  \var{ptr}, is not deallocated while the returned buffer object
  exists.  Raises \exception{ValueError} if \var{size} is less than
  zero.  Note that \constant{Py_END_OF_BUFFER} may \emph{not} be
  passed for the \var{size} parameter; \exception{ValueError} will be
  raised in that case.
\end{cfuncdesc}

\begin{cfuncdesc}{PyObject*}{PyBuffer_FromReadWriteMemory}{void *ptr, Py_ssize_t size}
  Similar to \cfunction{PyBuffer_FromMemory()}, but the returned
  buffer is writable.
\end{cfuncdesc}

\begin{cfuncdesc}{PyObject*}{PyBuffer_New}{Py_ssize_t size}
  Return a new writable buffer object that maintains its own memory
  buffer of \var{size} bytes.  \exception{ValueError} is returned if
  \var{size} is not zero or positive.  Note that the memory buffer (as
  returned by \cfunction{PyObject_AsWriteBuffer()}) is not specifically
  aligned.
\end{cfuncdesc}


\subsection{Tuple Objects \label{tupleObjects}}

\obindex{tuple}
\begin{ctypedesc}{PyTupleObject}
  This subtype of \ctype{PyObject} represents a Python tuple object.
\end{ctypedesc}

\begin{cvardesc}{PyTypeObject}{PyTuple_Type}
  This instance of \ctype{PyTypeObject} represents the Python tuple
  type; it is the same object as \code{types.TupleType} in the Python
  layer.\withsubitem{(in module types)}{\ttindex{TupleType}}.
\end{cvardesc}

\begin{cfuncdesc}{int}{PyTuple_Check}{PyObject *p}
  Return true if \var{p} is a tuple object or an instance of a subtype
  of the tuple type.
  \versionchanged[Allowed subtypes to be accepted]{2.2}
\end{cfuncdesc}

\begin{cfuncdesc}{int}{PyTuple_CheckExact}{PyObject *p}
  Return true if \var{p} is a tuple object, but not an instance of a
  subtype of the tuple type.
  \versionadded{2.2}
\end{cfuncdesc}

\begin{cfuncdesc}{PyObject*}{PyTuple_New}{Py_ssize_t len}
  Return a new tuple object of size \var{len}, or \NULL{} on failure.
\end{cfuncdesc}

\begin{cfuncdesc}{PyObject*}{PyTuple_Pack}{Py_ssize_t n, \moreargs}
  Return a new tuple object of size \var{n}, or \NULL{} on failure.
  The tuple values are initialized to the subsequent \var{n} C arguments
  pointing to Python objects.  \samp{PyTuple_Pack(2, \var{a}, \var{b})}
  is equivalent to \samp{Py_BuildValue("(OO)", \var{a}, \var{b})}.
  \versionadded{2.4}
\end{cfuncdesc}

\begin{cfuncdesc}{int}{PyTuple_Size}{PyObject *p}
  Take a pointer to a tuple object, and return the size of that
  tuple.
\end{cfuncdesc}

\begin{cfuncdesc}{int}{PyTuple_GET_SIZE}{PyObject *p}
  Return the size of the tuple \var{p}, which must be non-\NULL{} and
  point to a tuple; no error checking is performed.
\end{cfuncdesc}

\begin{cfuncdesc}{PyObject*}{PyTuple_GetItem}{PyObject *p, Py_ssize_t pos}
  Return the object at position \var{pos} in the tuple pointed to by
  \var{p}.  If \var{pos} is out of bounds, return \NULL{} and sets an
  \exception{IndexError} exception.
\end{cfuncdesc}

\begin{cfuncdesc}{PyObject*}{PyTuple_GET_ITEM}{PyObject *p, Py_ssize_t pos}
  Like \cfunction{PyTuple_GetItem()}, but does no checking of its
  arguments.
\end{cfuncdesc}

\begin{cfuncdesc}{PyObject*}{PyTuple_GetSlice}{PyObject *p,
                                               Py_ssize_t low, Py_ssize_t high}
  Take a slice of the tuple pointed to by \var{p} from \var{low} to
  \var{high} and return it as a new tuple.
\end{cfuncdesc}

\begin{cfuncdesc}{int}{PyTuple_SetItem}{PyObject *p,
                                        Py_ssize_t pos, PyObject *o}
  Insert a reference to object \var{o} at position \var{pos} of the
  tuple pointed to by \var{p}. Return \code{0} on success.
  \note{This function ``steals'' a reference to \var{o}.}
\end{cfuncdesc}

\begin{cfuncdesc}{void}{PyTuple_SET_ITEM}{PyObject *p,
                                          Py_ssize_t pos, PyObject *o}
  Like \cfunction{PyTuple_SetItem()}, but does no error checking, and
  should \emph{only} be used to fill in brand new tuples.  \note{This
  function ``steals'' a reference to \var{o}.}
\end{cfuncdesc}

\begin{cfuncdesc}{int}{_PyTuple_Resize}{PyObject **p, Py_ssize_t newsize}
  Can be used to resize a tuple.  \var{newsize} will be the new length
  of the tuple.  Because tuples are \emph{supposed} to be immutable,
  this should only be used if there is only one reference to the
  object.  Do \emph{not} use this if the tuple may already be known to
  some other part of the code.  The tuple will always grow or shrink
  at the end.  Think of this as destroying the old tuple and creating
  a new one, only more efficiently.  Returns \code{0} on success.
  Client code should never assume that the resulting value of
  \code{*\var{p}} will be the same as before calling this function.
  If the object referenced by \code{*\var{p}} is replaced, the
  original \code{*\var{p}} is destroyed.  On failure, returns
  \code{-1} and sets \code{*\var{p}} to \NULL{}, and raises
  \exception{MemoryError} or
  \exception{SystemError}.
  \versionchanged[Removed unused third parameter, \var{last_is_sticky}]{2.2}
\end{cfuncdesc}


\subsection{List Objects \label{listObjects}}

\obindex{list}
\begin{ctypedesc}{PyListObject}
  This subtype of \ctype{PyObject} represents a Python list object.
\end{ctypedesc}

\begin{cvardesc}{PyTypeObject}{PyList_Type}
  This instance of \ctype{PyTypeObject} represents the Python list
  type.  This is the same object as \code{types.ListType}.
  \withsubitem{(in module types)}{\ttindex{ListType}}
\end{cvardesc}

\begin{cfuncdesc}{int}{PyList_Check}{PyObject *p}
  Return true if \var{p} is a list object or an instance of a
  subtype of the list type.
  \versionchanged[Allowed subtypes to be accepted]{2.2}
\end{cfuncdesc}

\begin{cfuncdesc}{int}{PyList_CheckExact}{PyObject *p}
  Return true if \var{p} is a list object, but not an instance of a
  subtype of the list type.
  \versionadded{2.2}
\end{cfuncdesc}

\begin{cfuncdesc}{PyObject*}{PyList_New}{Py_ssize_t len}
  Return a new list of length \var{len} on success, or \NULL{} on
  failure.
\end{cfuncdesc}

\begin{cfuncdesc}{Py_ssize_t}{PyList_Size}{PyObject *list}
  Return the length of the list object in \var{list}; this is
  equivalent to \samp{len(\var{list})} on a list object.
  \bifuncindex{len}
\end{cfuncdesc}

\begin{cfuncdesc}{Py_ssize_t}{PyList_GET_SIZE}{PyObject *list}
  Macro form of \cfunction{PyList_Size()} without error checking.
\end{cfuncdesc}

\begin{cfuncdesc}{PyObject*}{PyList_GetItem}{PyObject *list, Py_ssize_t index}
  Return the object at position \var{pos} in the list pointed to by
  \var{p}.  If \var{pos} is out of bounds, return \NULL{} and set an
  \exception{IndexError} exception.
\end{cfuncdesc}

\begin{cfuncdesc}{PyObject*}{PyList_GET_ITEM}{PyObject *list, Py_ssize_t i}
  Macro form of \cfunction{PyList_GetItem()} without error checking.
\end{cfuncdesc}

\begin{cfuncdesc}{int}{PyList_SetItem}{PyObject *list, Py_ssize_t index,
                                       PyObject *item}
  Set the item at index \var{index} in list to \var{item}.  Return
  \code{0} on success or \code{-1} on failure.  \note{This function
  ``steals'' a reference to \var{item} and discards a reference to an
  item already in the list at the affected position.}
\end{cfuncdesc}

\begin{cfuncdesc}{void}{PyList_SET_ITEM}{PyObject *list, Py_ssize_t i,
                                              PyObject *o}
  Macro form of \cfunction{PyList_SetItem()} without error checking.
  This is normally only used to fill in new lists where there is no
  previous content.
  \note{This function ``steals'' a reference to \var{item}, and,
  unlike \cfunction{PyList_SetItem()}, does \emph{not} discard a
  reference to any item that it being replaced; any reference in
  \var{list} at position \var{i} will be leaked.}
\end{cfuncdesc}

\begin{cfuncdesc}{int}{PyList_Insert}{PyObject *list, Py_ssize_t index,
                                      PyObject *item}
  Insert the item \var{item} into list \var{list} in front of index
  \var{index}.  Return \code{0} if successful; return \code{-1} and
  set an exception if unsuccessful.  Analogous to
  \code{\var{list}.insert(\var{index}, \var{item})}.
\end{cfuncdesc}

\begin{cfuncdesc}{int}{PyList_Append}{PyObject *list, PyObject *item}
  Append the object \var{item} at the end of list \var{list}.
  Return \code{0} if successful; return \code{-1} and set an
  exception if unsuccessful.  Analogous to
  \code{\var{list}.append(\var{item})}.
\end{cfuncdesc}

\begin{cfuncdesc}{PyObject*}{PyList_GetSlice}{PyObject *list,
                                              Py_ssize_t low, Py_ssize_t high}
  Return a list of the objects in \var{list} containing the objects
  \emph{between} \var{low} and \var{high}.  Return \NULL{} and set
  an exception if unsuccessful.
  Analogous to \code{\var{list}[\var{low}:\var{high}]}.
\end{cfuncdesc}

\begin{cfuncdesc}{int}{PyList_SetSlice}{PyObject *list,
                                        Py_ssize_t low, Py_ssize_t high,
                                        PyObject *itemlist}
  Set the slice of \var{list} between \var{low} and \var{high} to the
  contents of \var{itemlist}.  Analogous to
  \code{\var{list}[\var{low}:\var{high}] = \var{itemlist}}.
  The \var{itemlist} may be \NULL{}, indicating the assignment
  of an empty list (slice deletion).
  Return \code{0} on success, \code{-1} on failure.
\end{cfuncdesc}

\begin{cfuncdesc}{int}{PyList_Sort}{PyObject *list}
  Sort the items of \var{list} in place.  Return \code{0} on
  success, \code{-1} on failure.  This is equivalent to
  \samp{\var{list}.sort()}.
\end{cfuncdesc}

\begin{cfuncdesc}{int}{PyList_Reverse}{PyObject *list}
  Reverse the items of \var{list} in place.  Return \code{0} on
  success, \code{-1} on failure.  This is the equivalent of
  \samp{\var{list}.reverse()}.
\end{cfuncdesc}

\begin{cfuncdesc}{PyObject*}{PyList_AsTuple}{PyObject *list}
  Return a new tuple object containing the contents of \var{list};
  equivalent to \samp{tuple(\var{list})}.\bifuncindex{tuple}
\end{cfuncdesc}


\section{Mapping Objects \label{mapObjects}}

\obindex{mapping}


\subsection{Dictionary Objects \label{dictObjects}}

\obindex{dictionary}
\begin{ctypedesc}{PyDictObject}
  This subtype of \ctype{PyObject} represents a Python dictionary
  object.
\end{ctypedesc}

\begin{cvardesc}{PyTypeObject}{PyDict_Type}
  This instance of \ctype{PyTypeObject} represents the Python
  dictionary type.  This is exposed to Python programs as
  \code{types.DictType} and \code{types.DictionaryType}.
  \withsubitem{(in module types)}{\ttindex{DictType}\ttindex{DictionaryType}}
\end{cvardesc}

\begin{cfuncdesc}{int}{PyDict_Check}{PyObject *p}
  Return true if \var{p} is a dict object or an instance of a
  subtype of the dict type.
  \versionchanged[Allowed subtypes to be accepted]{2.2}
\end{cfuncdesc}

\begin{cfuncdesc}{int}{PyDict_CheckExact}{PyObject *p}
  Return true if \var{p} is a dict object, but not an instance of a
  subtype of the dict type.
  \versionadded{2.4}
\end{cfuncdesc}

\begin{cfuncdesc}{PyObject*}{PyDict_New}{}
  Return a new empty dictionary, or \NULL{} on failure.
\end{cfuncdesc}

\begin{cfuncdesc}{PyObject*}{PyDictProxy_New}{PyObject *dict}
  Return a proxy object for a mapping which enforces read-only
  behavior.  This is normally used to create a proxy to prevent
  modification of the dictionary for non-dynamic class types.
  \versionadded{2.2}
\end{cfuncdesc}

\begin{cfuncdesc}{void}{PyDict_Clear}{PyObject *p}
  Empty an existing dictionary of all key-value pairs.
\end{cfuncdesc}

\begin{cfuncdesc}{int}{PyDict_Contains}{PyObject *p, PyObject *key}
  Determine if dictionary \var{p} contains \var{key}.  If an item
  in \var{p} is matches \var{key}, return \code{1}, otherwise return
  \code{0}.  On error, return \code{-1}.  This is equivalent to the
  Python expression \samp{\var{key} in \var{p}}.
  \versionadded{2.4}
\end{cfuncdesc}

\begin{cfuncdesc}{PyObject*}{PyDict_Copy}{PyObject *p}
  Return a new dictionary that contains the same key-value pairs as
  \var{p}.
  \versionadded{1.6}
\end{cfuncdesc}

\begin{cfuncdesc}{int}{PyDict_SetItem}{PyObject *p, PyObject *key,
                                       PyObject *val}
  Insert \var{value} into the dictionary \var{p} with a key of
  \var{key}.  \var{key} must be hashable; if it isn't,
  \exception{TypeError} will be raised.
  Return \code{0} on success or \code{-1} on failure.
\end{cfuncdesc}

\begin{cfuncdesc}{int}{PyDict_SetItemString}{PyObject *p,
            const char *key,
            PyObject *val}
  Insert \var{value} into the dictionary \var{p} using \var{key} as a
  key. \var{key} should be a \ctype{char*}.  The key object is created
  using \code{PyString_FromString(\var{key})}. Return \code{0} on
  success or \code{-1} on failure.
  \ttindex{PyString_FromString()}
\end{cfuncdesc}

\begin{cfuncdesc}{int}{PyDict_DelItem}{PyObject *p, PyObject *key}
  Remove the entry in dictionary \var{p} with key \var{key}.
  \var{key} must be hashable; if it isn't, \exception{TypeError} is
  raised.  Return \code{0} on success or \code{-1} on failure.
\end{cfuncdesc}

\begin{cfuncdesc}{int}{PyDict_DelItemString}{PyObject *p, char *key}
  Remove the entry in dictionary \var{p} which has a key specified by
  the string \var{key}.  Return \code{0} on success or \code{-1} on
  failure.
\end{cfuncdesc}

\begin{cfuncdesc}{PyObject*}{PyDict_GetItem}{PyObject *p, PyObject *key}
  Return the object from dictionary \var{p} which has a key
  \var{key}.  Return \NULL{} if the key \var{key} is not present, but
  \emph{without} setting an exception.
\end{cfuncdesc}

\begin{cfuncdesc}{PyObject*}{PyDict_GetItemString}{PyObject *p, const char *key}
  This is the same as \cfunction{PyDict_GetItem()}, but \var{key} is
  specified as a \ctype{char*}, rather than a \ctype{PyObject*}.
\end{cfuncdesc}

\begin{cfuncdesc}{PyObject*}{PyDict_Items}{PyObject *p}
  Return a \ctype{PyListObject} containing all the items from the
  dictionary, as in the dictionary method \method{items()} (see the
  \citetitle[../lib/lib.html]{Python Library Reference}).
\end{cfuncdesc}

\begin{cfuncdesc}{PyObject*}{PyDict_Keys}{PyObject *p}
  Return a \ctype{PyListObject} containing all the keys from the
  dictionary, as in the dictionary method \method{keys()} (see the
  \citetitle[../lib/lib.html]{Python Library Reference}).
\end{cfuncdesc}

\begin{cfuncdesc}{PyObject*}{PyDict_Values}{PyObject *p}
  Return a \ctype{PyListObject} containing all the values from the
  dictionary \var{p}, as in the dictionary method \method{values()}
  (see the \citetitle[../lib/lib.html]{Python Library Reference}).
\end{cfuncdesc}

\begin{cfuncdesc}{Py_ssize_t}{PyDict_Size}{PyObject *p}
  Return the number of items in the dictionary.  This is equivalent
  to \samp{len(\var{p})} on a dictionary.\bifuncindex{len}
\end{cfuncdesc}

\begin{cfuncdesc}{int}{PyDict_Next}{PyObject *p, Py_ssize_t *ppos,
                                    PyObject **pkey, PyObject **pvalue}
  Iterate over all key-value pairs in the dictionary \var{p}.  The
  \ctype{int} referred to by \var{ppos} must be initialized to
  \code{0} prior to the first call to this function to start the
  iteration; the function returns true for each pair in the
  dictionary, and false once all pairs have been reported.  The
  parameters \var{pkey} and \var{pvalue} should either point to
  \ctype{PyObject*} variables that will be filled in with each key and
  value, respectively, or may be \NULL{}.  Any references returned through
  them are borrowed.  \var{ppos} should not be altered during iteration.
  Its value represents offsets within the internal dictionary structure,
  and since the structure is sparse, the offsets are not consecutive.

  For example:

\begin{verbatim}
PyObject *key, *value;
int pos = 0;

while (PyDict_Next(self->dict, &pos, &key, &value)) {
    /* do something interesting with the values... */
    ...
}
\end{verbatim}

  The dictionary \var{p} should not be mutated during iteration.  It
  is safe (since Python 2.1) to modify the values of the keys as you
  iterate over the dictionary, but only so long as the set of keys
  does not change.  For example:

\begin{verbatim}
PyObject *key, *value;
int pos = 0;

while (PyDict_Next(self->dict, &pos, &key, &value)) {
    int i = PyInt_AS_LONG(value) + 1;
    PyObject *o = PyInt_FromLong(i);
    if (o == NULL)
        return -1;
    if (PyDict_SetItem(self->dict, key, o) < 0) {
        Py_DECREF(o);
        return -1;
    }
    Py_DECREF(o);
}
\end{verbatim}
\end{cfuncdesc}

\begin{cfuncdesc}{int}{PyDict_Merge}{PyObject *a, PyObject *b, int override}
  Iterate over mapping object \var{b} adding key-value pairs to dictionary
  \var{a}.
  \var{b} may be a dictionary, or any object supporting
  \function{PyMapping_Keys()} and \function{PyObject_GetItem()}.
  If \var{override} is true, existing pairs in \var{a} will
  be replaced if a matching key is found in \var{b}, otherwise pairs
  will only be added if there is not a matching key in \var{a}.
  Return \code{0} on success or \code{-1} if an exception was
  raised.
\versionadded{2.2}
\end{cfuncdesc}

\begin{cfuncdesc}{int}{PyDict_Update}{PyObject *a, PyObject *b}
  This is the same as \code{PyDict_Merge(\var{a}, \var{b}, 1)} in C,
  or \code{\var{a}.update(\var{b})} in Python.  Return \code{0} on
  success or \code{-1} if an exception was raised.
  \versionadded{2.2}
\end{cfuncdesc}

\begin{cfuncdesc}{int}{PyDict_MergeFromSeq2}{PyObject *a, PyObject *seq2,
                                             int override}
  Update or merge into dictionary \var{a}, from the key-value pairs in
  \var{seq2}.  \var{seq2} must be an iterable object producing
  iterable objects of length 2, viewed as key-value pairs.  In case of
  duplicate keys, the last wins if \var{override} is true, else the
  first wins.
  Return \code{0} on success or \code{-1} if an exception
  was raised.
  Equivalent Python (except for the return value):

\begin{verbatim}
def PyDict_MergeFromSeq2(a, seq2, override):
    for key, value in seq2:
        if override or key not in a:
            a[key] = value
\end{verbatim}

  \versionadded{2.2}
\end{cfuncdesc}


\section{Other Objects \label{otherObjects}}

\subsection{File Objects \label{fileObjects}}

\obindex{file}
Python's built-in file objects are implemented entirely on the
\ctype{FILE*} support from the C standard library.  This is an
implementation detail and may change in future releases of Python.

\begin{ctypedesc}{PyFileObject}
  This subtype of \ctype{PyObject} represents a Python file object.
\end{ctypedesc}

\begin{cvardesc}{PyTypeObject}{PyFile_Type}
  This instance of \ctype{PyTypeObject} represents the Python file
  type.  This is exposed to Python programs as \code{types.FileType}.
  \withsubitem{(in module types)}{\ttindex{FileType}}
\end{cvardesc}

\begin{cfuncdesc}{int}{PyFile_Check}{PyObject *p}
  Return true if its argument is a \ctype{PyFileObject} or a subtype
  of \ctype{PyFileObject}.
  \versionchanged[Allowed subtypes to be accepted]{2.2}
\end{cfuncdesc}

\begin{cfuncdesc}{int}{PyFile_CheckExact}{PyObject *p}
  Return true if its argument is a \ctype{PyFileObject}, but not a
  subtype of \ctype{PyFileObject}.
  \versionadded{2.2}
\end{cfuncdesc}

\begin{cfuncdesc}{PyObject*}{PyFile_FromString}{char *filename, char *mode}
  On success, return a new file object that is opened on the file
  given by \var{filename}, with a file mode given by \var{mode}, where
  \var{mode} has the same semantics as the standard C routine
  \cfunction{fopen()}\ttindex{fopen()}.  On failure, return \NULL{}.
\end{cfuncdesc}

\begin{cfuncdesc}{PyObject*}{PyFile_FromFile}{FILE *fp,
                                              char *name, char *mode,
                                              int (*close)(FILE*)}
  Create a new \ctype{PyFileObject} from the already-open standard C
  file pointer, \var{fp}.  The function \var{close} will be called
  when the file should be closed.  Return \NULL{} on failure.
\end{cfuncdesc}

\begin{cfuncdesc}{FILE*}{PyFile_AsFile}{PyObject *p}
  Return the file object associated with \var{p} as a \ctype{FILE*}.
\end{cfuncdesc}

\begin{cfuncdesc}{PyObject*}{PyFile_GetLine}{PyObject *p, int n}
  Equivalent to \code{\var{p}.readline(\optional{\var{n}})}, this
  function reads one line from the object \var{p}.  \var{p} may be a
  file object or any object with a \method{readline()} method.  If
  \var{n} is \code{0}, exactly one line is read, regardless of the
  length of the line.  If \var{n} is greater than \code{0}, no more
  than \var{n} bytes will be read from the file; a partial line can be
  returned.  In both cases, an empty string is returned if the end of
  the file is reached immediately.  If \var{n} is less than \code{0},
  however, one line is read regardless of length, but
  \exception{EOFError} is raised if the end of the file is reached
  immediately.
  \withsubitem{(built-in exception)}{\ttindex{EOFError}}
\end{cfuncdesc}

\begin{cfuncdesc}{PyObject*}{PyFile_Name}{PyObject *p}
  Return the name of the file specified by \var{p} as a string
  object.
\end{cfuncdesc}

\begin{cfuncdesc}{void}{PyFile_SetBufSize}{PyFileObject *p, int n}
  Available on systems with \cfunction{setvbuf()}\ttindex{setvbuf()}
  only.  This should only be called immediately after file object
  creation.
\end{cfuncdesc}

\begin{cfuncdesc}{int}{PyFile_Encoding}{PyFileObject *p, char *enc}
  Set the file's encoding for Unicode output to \var{enc}. Return
  1 on success and 0 on failure.
  \versionadded{2.3}
\end{cfuncdesc}

\begin{cfuncdesc}{int}{PyFile_SoftSpace}{PyObject *p, int newflag}
  This function exists for internal use by the interpreter.  Set the
  \member{softspace} attribute of \var{p} to \var{newflag} and
  \withsubitem{(file attribute)}{\ttindex{softspace}}return the
  previous value.  \var{p} does not have to be a file object for this
  function to work properly; any object is supported (thought its only
  interesting if the \member{softspace} attribute can be set).  This
  function clears any errors, and will return \code{0} as the previous
  value if the attribute either does not exist or if there were errors
  in retrieving it.  There is no way to detect errors from this
  function, but doing so should not be needed.
\end{cfuncdesc}

\begin{cfuncdesc}{int}{PyFile_WriteObject}{PyObject *obj, PyObject *p,
                                           int flags}
  Write object \var{obj} to file object \var{p}.  The only supported
  flag for \var{flags} is
  \constant{Py_PRINT_RAW}\ttindex{Py_PRINT_RAW}; if given, the
  \function{str()} of the object is written instead of the
  \function{repr()}.  Return \code{0} on success or \code{-1} on
  failure; the appropriate exception will be set.
\end{cfuncdesc}

\begin{cfuncdesc}{int}{PyFile_WriteString}{const char *s, PyObject *p}
  Write string \var{s} to file object \var{p}.  Return \code{0} on
  success or \code{-1} on failure; the appropriate exception will be
  set.
\end{cfuncdesc}


\subsection{Instance Objects \label{instanceObjects}}

\obindex{instance}
There are very few functions specific to instance objects.

\begin{cvardesc}{PyTypeObject}{PyInstance_Type}
  Type object for class instances.
\end{cvardesc}

\begin{cfuncdesc}{int}{PyInstance_Check}{PyObject *obj}
  Return true if \var{obj} is an instance.
\end{cfuncdesc}

\begin{cfuncdesc}{PyObject*}{PyInstance_New}{PyObject *class,
                                             PyObject *arg,
                                             PyObject *kw}
  Create a new instance of a specific class.  The parameters \var{arg}
  and \var{kw} are used as the positional and keyword parameters to
  the object's constructor.
\end{cfuncdesc}

\begin{cfuncdesc}{PyObject*}{PyInstance_NewRaw}{PyObject *class,
                                                PyObject *dict}
  Create a new instance of a specific class without calling its
  constructor.  \var{class} is the class of new object.  The
  \var{dict} parameter will be used as the object's \member{__dict__};
  if \NULL{}, a new dictionary will be created for the instance.
\end{cfuncdesc}


\subsection{Function Objects \label{function-objects}}

\obindex{function}
There are a few functions specific to Python functions.

\begin{ctypedesc}{PyFunctionObject}
  The C structure used for functions.
\end{ctypedesc}

\begin{cvardesc}{PyTypeObject}{PyFunction_Type}
  This is an instance of \ctype{PyTypeObject} and represents the
  Python function type.  It is exposed to Python programmers as
  \code{types.FunctionType}.
  \withsubitem{(in module types)}{\ttindex{MethodType}}
\end{cvardesc}

\begin{cfuncdesc}{int}{PyFunction_Check}{PyObject *o}
  Return true if \var{o} is a function object (has type
  \cdata{PyFunction_Type}).  The parameter must not be \NULL{}.
\end{cfuncdesc}

\begin{cfuncdesc}{PyObject*}{PyFunction_New}{PyObject *code,
                                             PyObject *globals}
  Return a new function object associated with the code object
  \var{code}. \var{globals} must be a dictionary with the global
  variables accessible to the function.

  The function's docstring, name and \var{__module__} are retrieved
  from the code object, the argument defaults and closure are set to
  \NULL{}.
\end{cfuncdesc}

\begin{cfuncdesc}{PyObject*}{PyFunction_GetCode}{PyObject *op}
  Return the code object associated with the function object \var{op}.
\end{cfuncdesc}

\begin{cfuncdesc}{PyObject*}{PyFunction_GetGlobals}{PyObject *op}
  Return the globals dictionary associated with the function object
  \var{op}.
\end{cfuncdesc}

\begin{cfuncdesc}{PyObject*}{PyFunction_GetModule}{PyObject *op}
  Return the \var{__module__} attribute of the function object \var{op}.
  This is normally a string containing the module name, but can be set
  to any other object by Python code.
\end{cfuncdesc}

\begin{cfuncdesc}{PyObject*}{PyFunction_GetDefaults}{PyObject *op}
  Return the argument default values of the function object \var{op}.
  This can be a tuple of arguments or \NULL{}.
\end{cfuncdesc}

\begin{cfuncdesc}{int}{PyFunction_SetDefaults}{PyObject *op,
                                               PyObject *defaults}
  Set the argument default values for the function object \var{op}.
  \var{defaults} must be \var{Py_None} or a tuple.

  Raises \exception{SystemError} and returns \code{-1} on failure.
\end{cfuncdesc}

\begin{cfuncdesc}{PyObject*}{PyFunction_GetClosure}{PyObject *op}
  Return the closure associated with the function object \var{op}.
  This can be \NULL{} or a tuple of cell objects.
\end{cfuncdesc}

\begin{cfuncdesc}{int}{PyFunction_SetClosure}{PyObject *op,
                                              PyObject *closure}
  Set the closure associated with the function object \var{op}.
  \var{closure} must be \var{Py_None} or a tuple of cell objects.

  Raises \exception{SystemError} and returns \code{-1} on failure.
\end{cfuncdesc}


\subsection{Method Objects \label{method-objects}}

\obindex{method}
There are some useful functions that are useful for working with
method objects.

\begin{cvardesc}{PyTypeObject}{PyMethod_Type}
  This instance of \ctype{PyTypeObject} represents the Python method
  type.  This is exposed to Python programs as \code{types.MethodType}.
  \withsubitem{(in module types)}{\ttindex{MethodType}}
\end{cvardesc}

\begin{cfuncdesc}{int}{PyMethod_Check}{PyObject *o}
  Return true if \var{o} is a method object (has type
  \cdata{PyMethod_Type}).  The parameter must not be \NULL{}.
\end{cfuncdesc}

\begin{cfuncdesc}{PyObject*}{PyMethod_New}{PyObject *func,
                                           PyObject *self, PyObject *class}
  Return a new method object, with \var{func} being any callable
  object; this is the function that will be called when the method is
  called.  If this method should be bound to an instance, \var{self}
  should be the instance and \var{class} should be the class of
  \var{self}, otherwise \var{self} should be \NULL{} and \var{class}
  should be the class which provides the unbound method..
\end{cfuncdesc}

\begin{cfuncdesc}{PyObject*}{PyMethod_Class}{PyObject *meth}
  Return the class object from which the method \var{meth} was
  created; if this was created from an instance, it will be the class
  of the instance.
\end{cfuncdesc}

\begin{cfuncdesc}{PyObject*}{PyMethod_GET_CLASS}{PyObject *meth}
  Macro version of \cfunction{PyMethod_Class()} which avoids error
  checking.
\end{cfuncdesc}

\begin{cfuncdesc}{PyObject*}{PyMethod_Function}{PyObject *meth}
  Return the function object associated with the method \var{meth}.
\end{cfuncdesc}

\begin{cfuncdesc}{PyObject*}{PyMethod_GET_FUNCTION}{PyObject *meth}
  Macro version of \cfunction{PyMethod_Function()} which avoids error
  checking.
\end{cfuncdesc}

\begin{cfuncdesc}{PyObject*}{PyMethod_Self}{PyObject *meth}
  Return the instance associated with the method \var{meth} if it is
  bound, otherwise return \NULL{}.
\end{cfuncdesc}

\begin{cfuncdesc}{PyObject*}{PyMethod_GET_SELF}{PyObject *meth}
  Macro version of \cfunction{PyMethod_Self()} which avoids error
  checking.
\end{cfuncdesc}


\subsection{Module Objects \label{moduleObjects}}

\obindex{module}
There are only a few functions special to module objects.

\begin{cvardesc}{PyTypeObject}{PyModule_Type}
  This instance of \ctype{PyTypeObject} represents the Python module
  type.  This is exposed to Python programs as
  \code{types.ModuleType}.
  \withsubitem{(in module types)}{\ttindex{ModuleType}}
\end{cvardesc}

\begin{cfuncdesc}{int}{PyModule_Check}{PyObject *p}
  Return true if \var{p} is a module object, or a subtype of a module
  object.
  \versionchanged[Allowed subtypes to be accepted]{2.2}
\end{cfuncdesc}

\begin{cfuncdesc}{int}{PyModule_CheckExact}{PyObject *p}
  Return true if \var{p} is a module object, but not a subtype of
  \cdata{PyModule_Type}.
  \versionadded{2.2}
\end{cfuncdesc}

\begin{cfuncdesc}{PyObject*}{PyModule_New}{const char *name}
  Return a new module object with the \member{__name__} attribute set
  to \var{name}.  Only the module's \member{__doc__} and
  \member{__name__} attributes are filled in; the caller is
  responsible for providing a \member{__file__} attribute.
  \withsubitem{(module attribute)}{
    \ttindex{__name__}\ttindex{__doc__}\ttindex{__file__}}
\end{cfuncdesc}

\begin{cfuncdesc}{PyObject*}{PyModule_GetDict}{PyObject *module}
  Return the dictionary object that implements \var{module}'s
  namespace; this object is the same as the \member{__dict__}
  attribute of the module object.  This function never fails.
  \withsubitem{(module attribute)}{\ttindex{__dict__}}
  It is recommended extensions use other \cfunction{PyModule_*()}
  and \cfunction{PyObject_*()} functions rather than directly
  manipulate a module's \member{__dict__}.
\end{cfuncdesc}

\begin{cfuncdesc}{char*}{PyModule_GetName}{PyObject *module}
  Return \var{module}'s \member{__name__} value.  If the module does
  not provide one, or if it is not a string, \exception{SystemError}
  is raised and \NULL{} is returned.
  \withsubitem{(module attribute)}{\ttindex{__name__}}
  \withsubitem{(built-in exception)}{\ttindex{SystemError}}
\end{cfuncdesc}

\begin{cfuncdesc}{char*}{PyModule_GetFilename}{PyObject *module}
  Return the name of the file from which \var{module} was loaded using
  \var{module}'s \member{__file__} attribute.  If this is not defined,
  or if it is not a string, raise \exception{SystemError} and return
  \NULL{}.
  \withsubitem{(module attribute)}{\ttindex{__file__}}
  \withsubitem{(built-in exception)}{\ttindex{SystemError}}
\end{cfuncdesc}

\begin{cfuncdesc}{int}{PyModule_AddObject}{PyObject *module,
                                           const char *name, PyObject *value}
  Add an object to \var{module} as \var{name}.  This is a convenience
  function which can be used from the module's initialization
  function.  This steals a reference to \var{value}.  Return
  \code{-1} on error, \code{0} on success.
  \versionadded{2.0}
\end{cfuncdesc}

\begin{cfuncdesc}{int}{PyModule_AddIntConstant}{PyObject *module,
                                                const char *name, long value}
  Add an integer constant to \var{module} as \var{name}.  This
  convenience function can be used from the module's initialization
  function. Return \code{-1} on error, \code{0} on success.
  \versionadded{2.0}
\end{cfuncdesc}

\begin{cfuncdesc}{int}{PyModule_AddStringConstant}{PyObject *module,
                                                   const char *name, const char *value}
  Add a string constant to \var{module} as \var{name}.  This
  convenience function can be used from the module's initialization
  function.  The string \var{value} must be null-terminated.  Return
  \code{-1} on error, \code{0} on success.
  \versionadded{2.0}
\end{cfuncdesc}


\subsection{Iterator Objects \label{iterator-objects}}

Python provides two general-purpose iterator objects.  The first, a
sequence iterator, works with an arbitrary sequence supporting the
\method{__getitem__()} method.  The second works with a callable
object and a sentinel value, calling the callable for each item in the
sequence, and ending the iteration when the sentinel value is
returned.

\begin{cvardesc}{PyTypeObject}{PySeqIter_Type}
  Type object for iterator objects returned by
  \cfunction{PySeqIter_New()} and the one-argument form of the
  \function{iter()} built-in function for built-in sequence types.
  \versionadded{2.2}
\end{cvardesc}

\begin{cfuncdesc}{int}{PySeqIter_Check}{op}
  Return true if the type of \var{op} is \cdata{PySeqIter_Type}.
  \versionadded{2.2}
\end{cfuncdesc}

\begin{cfuncdesc}{PyObject*}{PySeqIter_New}{PyObject *seq}
  Return an iterator that works with a general sequence object,
  \var{seq}.  The iteration ends when the sequence raises
  \exception{IndexError} for the subscripting operation.
  \versionadded{2.2}
\end{cfuncdesc}

\begin{cvardesc}{PyTypeObject}{PyCallIter_Type}
  Type object for iterator objects returned by
  \cfunction{PyCallIter_New()} and the two-argument form of the
  \function{iter()} built-in function.
  \versionadded{2.2}
\end{cvardesc}

\begin{cfuncdesc}{int}{PyCallIter_Check}{op}
  Return true if the type of \var{op} is \cdata{PyCallIter_Type}.
  \versionadded{2.2}
\end{cfuncdesc}

\begin{cfuncdesc}{PyObject*}{PyCallIter_New}{PyObject *callable,
                                             PyObject *sentinel}
  Return a new iterator.  The first parameter, \var{callable}, can be
  any Python callable object that can be called with no parameters;
  each call to it should return the next item in the iteration.  When
  \var{callable} returns a value equal to \var{sentinel}, the
  iteration will be terminated.
  \versionadded{2.2}
\end{cfuncdesc}


\subsection{Descriptor Objects \label{descriptor-objects}}

``Descriptors'' are objects that describe some attribute of an object.
They are found in the dictionary of type objects.

\begin{cvardesc}{PyTypeObject}{PyProperty_Type}
  The type object for the built-in descriptor types.
  \versionadded{2.2}
\end{cvardesc}

\begin{cfuncdesc}{PyObject*}{PyDescr_NewGetSet}{PyTypeObject *type,
					        struct PyGetSetDef *getset}
  \versionadded{2.2}
\end{cfuncdesc}

\begin{cfuncdesc}{PyObject*}{PyDescr_NewMember}{PyTypeObject *type,
					        struct PyMemberDef *meth}
  \versionadded{2.2}
\end{cfuncdesc}

\begin{cfuncdesc}{PyObject*}{PyDescr_NewMethod}{PyTypeObject *type,
                                                struct PyMethodDef *meth}
  \versionadded{2.2}
\end{cfuncdesc}

\begin{cfuncdesc}{PyObject*}{PyDescr_NewWrapper}{PyTypeObject *type,
						 struct wrapperbase *wrapper,
                                                 void *wrapped}
  \versionadded{2.2}
\end{cfuncdesc}

\begin{cfuncdesc}{PyObject*}{PyDescr_NewClassMethod}{PyTypeObject *type,
						     PyMethodDef *method}
  \versionadded{2.3}
\end{cfuncdesc}

\begin{cfuncdesc}{int}{PyDescr_IsData}{PyObject *descr}
  Return true if the descriptor objects \var{descr} describes a data
  attribute, or false if it describes a method.  \var{descr} must be a
  descriptor object; there is no error checking.
  \versionadded{2.2}
\end{cfuncdesc}

\begin{cfuncdesc}{PyObject*}{PyWrapper_New}{PyObject *, PyObject *}
  \versionadded{2.2}
\end{cfuncdesc}


\subsection{Slice Objects \label{slice-objects}}

\begin{cvardesc}{PyTypeObject}{PySlice_Type}
  The type object for slice objects.  This is the same as
  \code{types.SliceType}.
  \withsubitem{(in module types)}{\ttindex{SliceType}}
\end{cvardesc}

\begin{cfuncdesc}{int}{PySlice_Check}{PyObject *ob}
  Return true if \var{ob} is a slice object; \var{ob} must not be
  \NULL{}.
\end{cfuncdesc}

\begin{cfuncdesc}{PyObject*}{PySlice_New}{PyObject *start, PyObject *stop,
                                          PyObject *step}
  Return a new slice object with the given values.  The \var{start},
  \var{stop}, and \var{step} parameters are used as the values of the
  slice object attributes of the same names.  Any of the values may be
  \NULL{}, in which case the \code{None} will be used for the
  corresponding attribute.  Return \NULL{} if the new object could
  not be allocated.
\end{cfuncdesc}

\begin{cfuncdesc}{int}{PySlice_GetIndices}{PySliceObject *slice, Py_ssize_t length,
                                           Py_ssize_t *start, Py_ssize_t *stop, Py_ssize_t *step}
Retrieve the start, stop and step indices from the slice object
\var{slice}, assuming a sequence of length \var{length}. Treats
indices greater than \var{length} as errors.

Returns 0 on success and -1 on error with no exception set (unless one
of the indices was not \constant{None} and failed to be converted to
an integer, in which case -1 is returned with an exception set).

You probably do not want to use this function.  If you want to use
slice objects in versions of Python prior to 2.3, you would probably
do well to incorporate the source of \cfunction{PySlice_GetIndicesEx},
suitably renamed, in the source of your extension.
\end{cfuncdesc}

\begin{cfuncdesc}{int}{PySlice_GetIndicesEx}{PySliceObject *slice, Py_ssize_t length,
                                             Py_ssize_t *start, Py_ssize_t *stop, Py_ssize_t *step,
                                             Py_ssize_t *slicelength}
Usable replacement for \cfunction{PySlice_GetIndices}.  Retrieve the
start, stop, and step indices from the slice object \var{slice}
assuming a sequence of length \var{length}, and store the length of
the slice in \var{slicelength}.  Out of bounds indices are clipped in
a manner consistent with the handling of normal slices.

Returns 0 on success and -1 on error with exception set.

\versionadded{2.3}
\end{cfuncdesc}


\subsection{Weak Reference Objects \label{weakref-objects}}

Python supports \emph{weak references} as first-class objects.  There
are two specific object types which directly implement weak
references.  The first is a simple reference object, and the second
acts as a proxy for the original object as much as it can.

\begin{cfuncdesc}{int}{PyWeakref_Check}{ob}
  Return true if \var{ob} is either a reference or proxy object.
  \versionadded{2.2}
\end{cfuncdesc}

\begin{cfuncdesc}{int}{PyWeakref_CheckRef}{ob}
  Return true if \var{ob} is a reference object.
  \versionadded{2.2}
\end{cfuncdesc}

\begin{cfuncdesc}{int}{PyWeakref_CheckProxy}{ob}
  Return true if \var{ob} is a proxy object.
  \versionadded{2.2}
\end{cfuncdesc}

\begin{cfuncdesc}{PyObject*}{PyWeakref_NewRef}{PyObject *ob,
                                               PyObject *callback}
  Return a weak reference object for the object \var{ob}.  This will
  always return a new reference, but is not guaranteed to create a new
  object; an existing reference object may be returned.  The second
  parameter, \var{callback}, can be a callable object that receives
  notification when \var{ob} is garbage collected; it should accept a
  single parameter, which will be the weak reference object itself.
  \var{callback} may also be \code{None} or \NULL{}.  If \var{ob}
  is not a weakly-referencable object, or if \var{callback} is not
  callable, \code{None}, or \NULL{}, this will return \NULL{} and
  raise \exception{TypeError}.
  \versionadded{2.2}
\end{cfuncdesc}

\begin{cfuncdesc}{PyObject*}{PyWeakref_NewProxy}{PyObject *ob,
                                                 PyObject *callback}
  Return a weak reference proxy object for the object \var{ob}.  This
  will always return a new reference, but is not guaranteed to create
  a new object; an existing proxy object may be returned.  The second
  parameter, \var{callback}, can be a callable object that receives
  notification when \var{ob} is garbage collected; it should accept a
  single parameter, which will be the weak reference object itself.
  \var{callback} may also be \code{None} or \NULL{}.  If \var{ob} is not
  a weakly-referencable object, or if \var{callback} is not callable,
  \code{None}, or \NULL{}, this will return \NULL{} and raise
  \exception{TypeError}.
  \versionadded{2.2}
\end{cfuncdesc}

\begin{cfuncdesc}{PyObject*}{PyWeakref_GetObject}{PyObject *ref}
  Return the referenced object from a weak reference, \var{ref}.  If
  the referent is no longer live, returns \code{None}.
  \versionadded{2.2}
\end{cfuncdesc}

\begin{cfuncdesc}{PyObject*}{PyWeakref_GET_OBJECT}{PyObject *ref}
  Similar to \cfunction{PyWeakref_GetObject()}, but implemented as a
  macro that does no error checking.
  \versionadded{2.2}
\end{cfuncdesc}


\subsection{CObjects \label{cObjects}}

\obindex{CObject}
Refer to \emph{Extending and Embedding the Python Interpreter},
section~1.12, ``Providing a C API for an Extension Module,'' for more
information on using these objects.


\begin{ctypedesc}{PyCObject}
  This subtype of \ctype{PyObject} represents an opaque value, useful
  for C extension modules who need to pass an opaque value (as a
  \ctype{void*} pointer) through Python code to other C code.  It is
  often used to make a C function pointer defined in one module
  available to other modules, so the regular import mechanism can be
  used to access C APIs defined in dynamically loaded modules.
\end{ctypedesc}

\begin{cfuncdesc}{int}{PyCObject_Check}{PyObject *p}
  Return true if its argument is a \ctype{PyCObject}.
\end{cfuncdesc}

\begin{cfuncdesc}{PyObject*}{PyCObject_FromVoidPtr}{void* cobj,
                                                    void (*destr)(void *)}
  Create a \ctype{PyCObject} from the \code{void *}\var{cobj}.  The
  \var{destr} function will be called when the object is reclaimed,
  unless it is \NULL{}.
\end{cfuncdesc}

\begin{cfuncdesc}{PyObject*}{PyCObject_FromVoidPtrAndDesc}{void* cobj,
	                          void* desc, void (*destr)(void *, void *)}
  Create a \ctype{PyCObject} from the \ctype{void *}\var{cobj}.  The
  \var{destr} function will be called when the object is reclaimed.
  The \var{desc} argument can be used to pass extra callback data for
  the destructor function.
\end{cfuncdesc}

\begin{cfuncdesc}{void*}{PyCObject_AsVoidPtr}{PyObject* self}
  Return the object \ctype{void *} that the \ctype{PyCObject}
  \var{self} was created with.
\end{cfuncdesc}

\begin{cfuncdesc}{void*}{PyCObject_GetDesc}{PyObject* self}
  Return the description \ctype{void *} that the \ctype{PyCObject}
  \var{self} was created with.
\end{cfuncdesc}

\begin{cfuncdesc}{int}{PyCObject_SetVoidPtr}{PyObject* self, void* cobj}
  Set the void pointer inside \var{self} to \var{cobj}.
  The \ctype{PyCObject} must not have an associated destructor.
  Return true on success, false on failure.
\end{cfuncdesc}


\subsection{Cell Objects \label{cell-objects}}

``Cell'' objects are used to implement variables referenced by
multiple scopes.  For each such variable, a cell object is created to
store the value; the local variables of each stack frame that
references the value contains a reference to the cells from outer
scopes which also use that variable.  When the value is accessed, the
value contained in the cell is used instead of the cell object
itself.  This de-referencing of the cell object requires support from
the generated byte-code; these are not automatically de-referenced
when accessed.  Cell objects are not likely to be useful elsewhere.

\begin{ctypedesc}{PyCellObject}
  The C structure used for cell objects.
\end{ctypedesc}

\begin{cvardesc}{PyTypeObject}{PyCell_Type}
  The type object corresponding to cell objects.
\end{cvardesc}

\begin{cfuncdesc}{int}{PyCell_Check}{ob}
  Return true if \var{ob} is a cell object; \var{ob} must not be
  \NULL{}.
\end{cfuncdesc}

\begin{cfuncdesc}{PyObject*}{PyCell_New}{PyObject *ob}
  Create and return a new cell object containing the value \var{ob}.
  The parameter may be \NULL{}.
\end{cfuncdesc}

\begin{cfuncdesc}{PyObject*}{PyCell_Get}{PyObject *cell}
  Return the contents of the cell \var{cell}.
\end{cfuncdesc}

\begin{cfuncdesc}{PyObject*}{PyCell_GET}{PyObject *cell}
  Return the contents of the cell \var{cell}, but without checking
  that \var{cell} is non-\NULL{} and a cell object.
\end{cfuncdesc}

\begin{cfuncdesc}{int}{PyCell_Set}{PyObject *cell, PyObject *value}
  Set the contents of the cell object \var{cell} to \var{value}.  This
  releases the reference to any current content of the cell.
  \var{value} may be \NULL{}.  \var{cell} must be non-\NULL{}; if it is
  not a cell object, \code{-1} will be returned.  On success, \code{0}
  will be returned.
\end{cfuncdesc}

\begin{cfuncdesc}{void}{PyCell_SET}{PyObject *cell, PyObject *value}
  Sets the value of the cell object \var{cell} to \var{value}.  No
  reference counts are adjusted, and no checks are made for safety;
  \var{cell} must be non-\NULL{} and must be a cell object.
\end{cfuncdesc}


\subsection{Generator Objects \label{gen-objects}}

Generator objects are what Python uses to implement generator iterators.
They are normally created by iterating over a function that yields values,
rather than explicitly calling \cfunction{PyGen_New}.

\begin{ctypedesc}{PyGenObject}
  The C structure used for generator objects.
\end{ctypedesc}

\begin{cvardesc}{PyTypeObject}{PyGen_Type}
  The type object corresponding to generator objects
\end{cvardesc}

\begin{cfuncdesc}{int}{PyGen_Check}{ob}
  Return true if \var{ob} is a generator object; \var{ob} must not be
  \NULL{}.
\end{cfuncdesc}

\begin{cfuncdesc}{int}{PyGen_CheckExact}{ob}
  Return true if \var{ob}'s type is \var{PyGen_Type}
  is a generator object; \var{ob} must not be
  \NULL{}.
\end{cfuncdesc}

\begin{cfuncdesc}{PyObject*}{PyGen_New}{PyFrameObject *frame}
  Create and return a new generator object based on the \var{frame} object.
  A reference to \var{frame} is stolen by this function.
  The parameter must not be \NULL{}.
\end{cfuncdesc}


\subsection{DateTime Objects \label{datetime-objects}}

Various date and time objects are supplied by the \module{datetime}
module.  Before using any of these functions, the header file
\file{datetime.h} must be included in your source (note that this is
not include by \file{Python.h}), and macro \cfunction{PyDateTime_IMPORT()}
must be invoked.  The macro arranges to put a pointer to a C structure
in a static variable \code{PyDateTimeAPI}, which is used by the following
macros.

Type-check macros:

\begin{cfuncdesc}{int}{PyDate_Check}{PyObject *ob}
  Return true if \var{ob} is of type \cdata{PyDateTime_DateType} or
  a subtype of \cdata{PyDateTime_DateType}.  \var{ob} must not be
  \NULL{}.
  \versionadded{2.4}
\end{cfuncdesc}

\begin{cfuncdesc}{int}{PyDate_CheckExact}{PyObject *ob}
  Return true if \var{ob} is of type \cdata{PyDateTime_DateType}.
  \var{ob} must not be \NULL{}.
  \versionadded{2.4}
\end{cfuncdesc}

\begin{cfuncdesc}{int}{PyDateTime_Check}{PyObject *ob}
  Return true if \var{ob} is of type \cdata{PyDateTime_DateTimeType} or
  a subtype of \cdata{PyDateTime_DateTimeType}.  \var{ob} must not be
  \NULL{}.
  \versionadded{2.4}
\end{cfuncdesc}

\begin{cfuncdesc}{int}{PyDateTime_CheckExact}{PyObject *ob}
  Return true if \var{ob} is of type \cdata{PyDateTime_DateTimeType}.
  \var{ob} must not be \NULL{}.
  \versionadded{2.4}
\end{cfuncdesc}

\begin{cfuncdesc}{int}{PyTime_Check}{PyObject *ob}
  Return true if \var{ob} is of type \cdata{PyDateTime_TimeType} or
  a subtype of \cdata{PyDateTime_TimeType}.  \var{ob} must not be
  \NULL{}.
  \versionadded{2.4}
\end{cfuncdesc}

\begin{cfuncdesc}{int}{PyTime_CheckExact}{PyObject *ob}
  Return true if \var{ob} is of type \cdata{PyDateTime_TimeType}.
  \var{ob} must not be \NULL{}.
  \versionadded{2.4}
\end{cfuncdesc}

\begin{cfuncdesc}{int}{PyDelta_Check}{PyObject *ob}
  Return true if \var{ob} is of type \cdata{PyDateTime_DeltaType} or
  a subtype of \cdata{PyDateTime_DeltaType}.  \var{ob} must not be
  \NULL{}.
  \versionadded{2.4}
\end{cfuncdesc}

\begin{cfuncdesc}{int}{PyDelta_CheckExact}{PyObject *ob}
  Return true if \var{ob} is of type \cdata{PyDateTime_DeltaType}.
  \var{ob} must not be \NULL{}.
  \versionadded{2.4}
\end{cfuncdesc}

\begin{cfuncdesc}{int}{PyTZInfo_Check}{PyObject *ob}
  Return true if \var{ob} is of type \cdata{PyDateTime_TZInfoType} or
  a subtype of \cdata{PyDateTime_TZInfoType}.  \var{ob} must not be
  \NULL{}.
  \versionadded{2.4}
\end{cfuncdesc}

\begin{cfuncdesc}{int}{PyTZInfo_CheckExact}{PyObject *ob}
  Return true if \var{ob} is of type \cdata{PyDateTime_TZInfoType}.
  \var{ob} must not be \NULL{}.
  \versionadded{2.4}
\end{cfuncdesc}

Macros to create objects:

\begin{cfuncdesc}{PyObject*}{PyDate_FromDate}{int year, int month, int day}
  Return a \code{datetime.date} object with the specified year, month
  and day.
  \versionadded{2.4}
\end{cfuncdesc}

\begin{cfuncdesc}{PyObject*}{PyDateTime_FromDateAndTime}{int year, int month,
        int day, int hour, int minute, int second, int usecond}
  Return a \code{datetime.datetime} object with the specified year, month,
  day, hour, minute, second and microsecond.
  \versionadded{2.4}
\end{cfuncdesc}

\begin{cfuncdesc}{PyObject*}{PyTime_FromTime}{int hour, int minute,
        int second, int usecond}
  Return a \code{datetime.time} object with the specified hour, minute,
  second and microsecond.
  \versionadded{2.4}
\end{cfuncdesc}

\begin{cfuncdesc}{PyObject*}{PyDelta_FromDSU}{int days, int seconds,
        int useconds}
  Return a \code{datetime.timedelta} object representing the given number
  of days, seconds and microseconds.  Normalization is performed so that
  the resulting number of microseconds and seconds lie in the ranges
  documented for \code{datetime.timedelta} objects.
  \versionadded{2.4}
\end{cfuncdesc}

Macros to extract fields from date objects.  The argument must be an
instance of \cdata{PyDateTime_Date}, including subclasses (such as
\cdata{PyDateTime_DateTime}).  The argument must not be \NULL{}, and
the type is not checked:

\begin{cfuncdesc}{int}{PyDateTime_GET_YEAR}{PyDateTime_Date *o}
  Return the year, as a positive int.
  \versionadded{2.4}
\end{cfuncdesc}

\begin{cfuncdesc}{int}{PyDateTime_GET_MONTH}{PyDateTime_Date *o}
  Return the month, as an int from 1 through 12.
  \versionadded{2.4}
\end{cfuncdesc}

\begin{cfuncdesc}{int}{PyDateTime_GET_DAY}{PyDateTime_Date *o}
  Return the day, as an int from 1 through 31.
  \versionadded{2.4}
\end{cfuncdesc}

Macros to extract fields from datetime objects.  The argument must be an
instance of \cdata{PyDateTime_DateTime}, including subclasses.
The argument must not be \NULL{}, and the type is not checked:

\begin{cfuncdesc}{int}{PyDateTime_DATE_GET_HOUR}{PyDateTime_DateTime *o}
  Return the hour, as an int from 0 through 23.
  \versionadded{2.4}
\end{cfuncdesc}

\begin{cfuncdesc}{int}{PyDateTime_DATE_GET_MINUTE}{PyDateTime_DateTime *o}
  Return the minute, as an int from 0 through 59.
  \versionadded{2.4}
\end{cfuncdesc}

\begin{cfuncdesc}{int}{PyDateTime_DATE_GET_SECOND}{PyDateTime_DateTime *o}
  Return the second, as an int from 0 through 59.
  \versionadded{2.4}
\end{cfuncdesc}

\begin{cfuncdesc}{int}{PyDateTime_DATE_GET_MICROSECOND}{PyDateTime_DateTime *o}
  Return the microsecond, as an int from 0 through 999999.
  \versionadded{2.4}
\end{cfuncdesc}

Macros to extract fields from time objects.  The argument must be an
instance of \cdata{PyDateTime_Time}, including subclasses.
The argument must not be \NULL{}, and the type is not checked:

\begin{cfuncdesc}{int}{PyDateTime_TIME_GET_HOUR}{PyDateTime_Time *o}
  Return the hour, as an int from 0 through 23.
  \versionadded{2.4}
\end{cfuncdesc}

\begin{cfuncdesc}{int}{PyDateTime_TIME_GET_MINUTE}{PyDateTime_Time *o}
  Return the minute, as an int from 0 through 59.
  \versionadded{2.4}
\end{cfuncdesc}

\begin{cfuncdesc}{int}{PyDateTime_TIME_GET_SECOND}{PyDateTime_Time *o}
  Return the second, as an int from 0 through 59.
  \versionadded{2.4}
\end{cfuncdesc}

\begin{cfuncdesc}{int}{PyDateTime_TIME_GET_MICROSECOND}{PyDateTime_Time *o}
  Return the microsecond, as an int from 0 through 999999.
  \versionadded{2.4}
\end{cfuncdesc}

Macros for the convenience of modules implementing the DB API:

\begin{cfuncdesc}{PyObject*}{PyDateTime_FromTimestamp}{PyObject *args}
  Create and return a new \code{datetime.datetime} object given an argument
  tuple suitable for passing to \code{datetime.datetime.fromtimestamp()}.
  \versionadded{2.4}
\end{cfuncdesc}

\begin{cfuncdesc}{PyObject*}{PyDate_FromTimestamp}{PyObject *args}
  Create and return a new \code{datetime.date} object given an argument
  tuple suitable for passing to \code{datetime.date.fromtimestamp()}.
  \versionadded{2.4}
\end{cfuncdesc}


\subsection{Set Objects \label{setObjects}}
\sectionauthor{Raymond D. Hettinger}{python@rcn.com}                     

\obindex{set}
\obindex{frozenset}
\versionadded{2.5}

This section details the public API for \class{set} and \class{frozenset}
objects.  Any functionality not listed below is best accessed using the
either the abstract object protocol (including
\cfunction{PyObject_CallMethod()}, \cfunction{PyObject_RichCompareBool()}, 
\cfunction{PyObject_Hash()}, \cfunction{PyObject_Repr()}, 
\cfunction{PyObject_IsTrue()}, \cfunction{PyObject_Print()}, and
\cfunction{PyObject_GetIter()})
or the abstract number protocol (including
\cfunction{PyNumber_Add()}, \cfunction{PyNumber_Subtract()},
\cfunction{PyNumber_Or()}, \cfunction{PyNumber_Xor()},
\cfunction{PyNumber_InPlaceAdd()}, \cfunction{PyNumber_InPlaceSubtract()},
\cfunction{PyNumber_InPlaceOr()}, and \cfunction{PyNumber_InPlaceXor()}).

\begin{ctypedesc}{PySetObject}
  This subtype of \ctype{PyObject} is used to hold the internal data for
  both \class{set} and \class{frozenset} objects.  It is like a
  \ctype{PyDictObject} in that it is a fixed size for small sets
  (much like tuple storage) and will point to a separate, variable sized
  block of memory for medium and large sized sets (much like list storage).
  None of the fields of this structure should be considered public and
  are subject to change.  All access should be done through the
  documented API rather than by manipulating the values in the structure. 

\end{ctypedesc}

\begin{cvardesc}{PyTypeObject}{PySet_Type}
  This is an instance of \ctype{PyTypeObject} representing the Python
  \class{set} type.
\end{cvardesc}

\begin{cvardesc}{PyTypeObject}{PyFrozenSet_Type}
  This is an instance of \ctype{PyTypeObject} representing the Python
  \class{frozenset} type.
\end{cvardesc}


The following type check macros work on pointers to any Python object.
Likewise, the constructor functions work with any iterable Python object.

\begin{cfuncdesc}{int}{PyAnySet_Check}{PyObject *p}
  Return true if \var{p} is a \class{set} object, a \class{frozenset} 
  object, or an instance of a subtype.
\end{cfuncdesc}

\begin{cfuncdesc}{int}{PyAnySet_CheckExact}{PyObject *p}
  Return true if \var{p} is a \class{set} object or a \class{frozenset}
  object but not an instance of a subtype.
\end{cfuncdesc}

\begin{cfuncdesc}{int}{PyFrozenSet_CheckExact}{PyObject *p}
  Return true if \var{p} is a \class{frozenset} object
  but not an instance of a subtype.
\end{cfuncdesc}

\begin{cfuncdesc}{PyObject*}{PySet_New}{PyObject *iterable}
  Return a new \class{set} containing objects returned by the
  \var{iterable}.  The \var{iterable} may be \NULL{} to create a
  new empty set.  Return the new set on success or \NULL{} on
  failure.  Raise \exception{TypeError} if \var{iterable} is
  not actually iterable.  The constructor is also useful for
  copying a set (\code{c=set(s)}).
\end{cfuncdesc}

\begin{cfuncdesc}{PyObject*}{PyFrozenSet_New}{PyObject *iterable}
  Return a new \class{frozenset} containing objects returned by the
  \var{iterable}.  The \var{iterable} may be \NULL{} to create a
  new empty frozenset.  Return the new set on success or \NULL{} on
  failure.  Raise \exception{TypeError} if \var{iterable} is
  not actually iterable.
\end{cfuncdesc}


The following functions and macros are available for instances of
\class{set} or \class{frozenset} or instances of their subtypes.

\begin{cfuncdesc}{int}{PySet_Size}{PyObject *anyset}
  Return the length of a \class{set} or \class{frozenset} object.
  Equivalent to \samp{len(\var{anyset})}.  Raises a
  \exception{PyExc_SystemError} if \var{anyset} is not a \class{set},
  \class{frozenset}, or an instance of a subtype.
  \bifuncindex{len}
\end{cfuncdesc}

\begin{cfuncdesc}{int}{PySet_GET_SIZE}{PyObject *anyset}
  Macro form of \cfunction{PySet_Size()} without error checking.
\end{cfuncdesc}

\begin{cfuncdesc}{int}{PySet_Contains}{PyObject *anyset, PyObject *key}
  Return 1 if found, 0 if not found, and -1 if an error is
  encountered.  Unlike the Python \method{__contains__()} method, this
  function does not automatically convert unhashable sets into temporary
  frozensets.  Raise a \exception{TypeError} if the \var{key} is unhashable.
  Raise \exception{PyExc_SystemError} if \var{anyset} is not a \class{set},
  \class{frozenset}, or an instance of a subtype.                         
\end{cfuncdesc}

The following functions are available for instances of \class{set} or
its subtypes but not for instances of \class{frozenset} or its subtypes.

\begin{cfuncdesc}{int}{PySet_Add}{PyObject *set, PyObject *key}
  Add \var{key} to a \class{set} instance.  Does not apply to
  \class{frozenset} instances.  Return 0 on success or -1 on failure.
  Raise a \exception{TypeError} if the \var{key} is unhashable.
  Raise a \exception{MemoryError} if there is no room to grow.
  Raise a \exception{SystemError} if \var{set} is an not an instance
  of \class{set} or its subtype.
\end{cfuncdesc}

\begin{cfuncdesc}{int}{PySet_Discard}{PyObject *set, PyObject *key}
  Return 1 if found and removed, 0 if not found (no action taken),
  and -1 if an error is encountered.  Does not raise \exception{KeyError}
  for missing keys.  Raise a \exception{TypeError} if the \var{key} is
  unhashable.  Unlike the Python \method{discard()} method, this function
  does not automatically convert unhashable sets into temporary frozensets.
  Raise \exception{PyExc_SystemError} if \var{set} is an not an instance
  of \class{set} or its subtype.                         
\end{cfuncdesc}

\begin{cfuncdesc}{PyObject*}{PySet_Pop}{PyObject *set}
  Return a new reference to an arbitrary object in the \var{set},
  and removes the object from the \var{set}.  Return \NULL{} on
  failure.  Raise \exception{KeyError} if the set is empty.
  Raise a \exception{SystemError} if \var{set} is an not an instance
  of \class{set} or its subtype.                        
\end{cfuncdesc}

\begin{cfuncdesc}{int}{PySet_Clear}{PyObject *set}
  Empty an existing set of all elements.
\end{cfuncdesc}
