\chapter{Object Implementation Support \label{newTypes}}


This chapter describes the functions, types, and macros used when
defining new object types.


\section{Allocating Objects on the Heap
         \label{allocating-objects}}

\begin{cfuncdesc}{PyObject*}{_PyObject_New}{PyTypeObject *type}
\end{cfuncdesc}

\begin{cfuncdesc}{PyVarObject*}{_PyObject_NewVar}{PyTypeObject *type, int size}
\end{cfuncdesc}

\begin{cfuncdesc}{void}{_PyObject_Del}{PyObject *op}
\end{cfuncdesc}

\begin{cfuncdesc}{PyObject*}{PyObject_Init}{PyObject *op,
					    PyTypeObject *type}
  Initialize a newly-allocated object \var{op} with its type and
  initial reference.  Returns the initialized object.  If \var{type}
  indicates that the object participates in the cyclic garbage
  detector, it is added to the detector's set of observed objects.
  Other fields of the object are not affected.
\end{cfuncdesc}

\begin{cfuncdesc}{PyVarObject*}{PyObject_InitVar}{PyVarObject *op,
						  PyTypeObject *type, int size}
  This does everything \cfunction{PyObject_Init()} does, and also
  initializes the length information for a variable-size object.
\end{cfuncdesc}

\begin{cfuncdesc}{\var{TYPE}*}{PyObject_New}{TYPE, PyTypeObject *type}
  Allocate a new Python object using the C structure type \var{TYPE}
  and the Python type object \var{type}.  Fields not defined by the
  Python object header are not initialized; the object's reference
  count will be one.  The size of the memory
  allocation is determined from the \member{tp_basicsize} field of the
  type object.
\end{cfuncdesc}

\begin{cfuncdesc}{\var{TYPE}*}{PyObject_NewVar}{TYPE, PyTypeObject *type,
                                                int size}
  Allocate a new Python object using the C structure type \var{TYPE}
  and the Python type object \var{type}.  Fields not defined by the
  Python object header are not initialized.  The allocated memory
  allows for the \var{TYPE} structure plus \var{size} fields of the
  size given by the \member{tp_itemsize} field of \var{type}.  This is
  useful for implementing objects like tuples, which are able to
  determine their size at construction time.  Embedding the array of
  fields into the same allocation decreases the number of allocations,
  improving the memory management efficiency.
\end{cfuncdesc}

\begin{cfuncdesc}{void}{PyObject_Del}{PyObject *op}
  Releases memory allocated to an object using
  \cfunction{PyObject_New()} or \cfunction{PyObject_NewVar()}.  This
  is normally called from the \member{tp_dealloc} handler specified in
  the object's type.  The fields of the object should not be accessed
  after this call as the memory is no longer a valid Python object.
\end{cfuncdesc}

\begin{cfuncdesc}{\var{TYPE}*}{PyObject_NEW}{TYPE, PyTypeObject *type}
  Macro version of \cfunction{PyObject_New()}, to gain performance at
  the expense of safety.  This does not check \var{type} for a \NULL{}
  value.
\end{cfuncdesc}

\begin{cfuncdesc}{\var{TYPE}*}{PyObject_NEW_VAR}{TYPE, PyTypeObject *type,
                                                int size}
  Macro version of \cfunction{PyObject_NewVar()}, to gain performance
  at the expense of safety.  This does not check \var{type} for a
  \NULL{} value.
\end{cfuncdesc}

\begin{cfuncdesc}{void}{PyObject_DEL}{PyObject *op}
  Macro version of \cfunction{PyObject_Del()}.
\end{cfuncdesc}

\begin{cfuncdesc}{PyObject*}{Py_InitModule}{char *name,
                                            PyMethodDef *methods}
  Create a new module object based on a name and table of functions,
  returning the new module object.

  \versionchanged[Older versions of Python did not support \NULL{} as
                  the value for the \var{methods} argument]{2.3}
\end{cfuncdesc}

\begin{cfuncdesc}{PyObject*}{Py_InitModule3}{char *name,
                                             PyMethodDef *methods,
                                             char *doc}
  Create a new module object based on a name and table of functions,
  returning the new module object.  If \var{doc} is non-\NULL, it will
  be used to define the docstring for the module.

  \versionchanged[Older versions of Python did not support \NULL{} as
                  the value for the \var{methods} argument]{2.3}
\end{cfuncdesc}

\begin{cfuncdesc}{PyObject*}{Py_InitModule4}{char *name,
                                             PyMethodDef *methods,
                                             char *doc, PyObject *self,
                                             int apiver}
  Create a new module object based on a name and table of functions,
  returning the new module object.  If \var{doc} is non-\NULL, it will
  be used to define the docstring for the module.  If \var{self} is
  non-\NULL, it will passed to the functions of the module as their
  (otherwise \NULL) first parameter.  (This was added as an
  experimental feature, and there are no known uses in the current
  version of Python.)  For \var{apiver}, the only value which should
  be passed is defined by the constant \constant{PYTHON_API_VERSION}.

  \note{Most uses of this function should probably be using
  the \cfunction{Py_InitModule3()} instead; only use this if you are
  sure you need it.}

  \versionchanged[Older versions of Python did not support \NULL{} as
                  the value for the \var{methods} argument]{2.3}
\end{cfuncdesc}

DL_IMPORT

\begin{cvardesc}{PyObject}{_Py_NoneStruct}
  Object which is visible in Python as \code{None}.  This should only
  be accessed using the \code{Py_None} macro, which evaluates to a
  pointer to this object.
\end{cvardesc}


\section{Common Object Structures \label{common-structs}}

There are a large number of structures which are used in the
definition of object types for Python.  This section describes these
structures and how they are used.

All Python objects ultimately share a small number of fields at the
beginning of the object's representation in memory.  These are
represented by the \ctype{PyObject} and \ctype{PyVarObject} types,
which are defined, in turn, by the expansions of some macros also
used, whether directly or indirectly, in the definition of all other
Python objects.

\begin{ctypedesc}{PyObject}
  All object types are extensions of this type.  This is a type which
  contains the information Python needs to treat a pointer to an
  object as an object.  In a normal ``release'' build, it contains
  only the objects reference count and a pointer to the corresponding
  type object.  It corresponds to the fields defined by the
  expansion of the \code{PyObject_HEAD} macro.
\end{ctypedesc}

\begin{ctypedesc}{PyVarObject}
  This is an extension of \ctype{PyObject} that adds the
  \member{ob_size} field.  This is only used for objects that have
  some notion of \emph{length}.  This type does not often appear in
  the Python/C API.  It corresponds to the fields defined by the
  expansion of the \code{PyObject_VAR_HEAD} macro.
\end{ctypedesc}

These macros are used in the definition of \ctype{PyObject} and
\ctype{PyVarObject}:

\begin{csimplemacrodesc}{PyObject_HEAD}
  This is a macro which expands to the declarations of the fields of
  the \ctype{PyObject} type; it is used when declaring new types which
  represent objects without a varying length.  The specific fields it
  expands to depend on the definition of
  \csimplemacro{Py_TRACE_REFS}.  By default, that macro is not
  defined, and \csimplemacro{PyObject_HEAD} expands to:
  \begin{verbatim}
    int ob_refcnt;
    PyTypeObject *ob_type;
  \end{verbatim}
  When \csimplemacro{Py_TRACE_REFS} is defined, it expands to:
  \begin{verbatim}
    PyObject *_ob_next, *_ob_prev;
    int ob_refcnt;
    PyTypeObject *ob_type;
  \end{verbatim}
\end{csimplemacrodesc}

\begin{csimplemacrodesc}{PyObject_VAR_HEAD}
  This is a macro which expands to the declarations of the fields of
  the \ctype{PyVarObject} type; it is used when declaring new types which
  represent objects with a length that varies from instance to
  instance.  This macro always expands to:
  \begin{verbatim}
    PyObject_HEAD
    int ob_size;
  \end{verbatim}
  Note that \csimplemacro{PyObject_HEAD} is part of the expansion, and
  that it's own expansion varies depending on the definition of
  \csimplemacro{Py_TRACE_REFS}.
\end{csimplemacrodesc}

PyObject_HEAD_INIT

\begin{ctypedesc}{PyCFunction}
  Type of the functions used to implement most Python callables in C.
  Functions of this type take two \ctype{PyObject*} parameters and
  return one such value.  If the return value is \NULL, an exception
  shall have been set.  If not \NULL, the return value is interpreted
  as the return value of the function as exposed in Python.  The
  function must return a new reference.
\end{ctypedesc}

\begin{ctypedesc}{PyMethodDef}
  Structure used to describe a method of an extension type.  This
  structure has four fields:

  \begin{tableiii}{l|l|l}{member}{Field}{C Type}{Meaning}
    \lineiii{ml_name}{char *}{name of the method}
    \lineiii{ml_meth}{PyCFunction}{pointer to the C implementation}
    \lineiii{ml_flags}{int}{flag bits indicating how the call should be
                            constructed}
    \lineiii{ml_doc}{char *}{points to the contents of the docstring}
  \end{tableiii}
\end{ctypedesc}

The \member{ml_meth} is a C function pointer.  The functions may be of
different types, but they always return \ctype{PyObject*}.  If the
function is not of the \ctype{PyCFunction}, the compiler will require
a cast in the method table.  Even though \ctype{PyCFunction} defines
the first parameter as \ctype{PyObject*}, it is common that the method
implementation uses a the specific C type of the \var{self} object.

The \member{ml_flags} field is a bitfield which can include the
following flags.  The individual flags indicate either a calling
convention or a binding convention.  Of the calling convention flags,
only \constant{METH_VARARGS} and \constant{METH_KEYWORDS} can be
combined (but note that \constant{METH_KEYWORDS} alone is equivalent
to \code{\constant{METH_VARARGS} | \constant{METH_KEYWORDS}}).
Any of the calling convention flags can be combined with a
binding flag.

\begin{datadesc}{METH_VARARGS}
  This is the typical calling convention, where the methods have the
  type \ctype{PyCFunction}. The function expects two
  \ctype{PyObject*} values.  The first one is the \var{self} object for
  methods; for module functions, it has the value given to
  \cfunction{Py_InitModule4()} (or \NULL{} if
  \cfunction{Py_InitModule()} was used).  The second parameter
  (often called \var{args}) is a tuple object representing all
  arguments. This parameter is typically processed using
  \cfunction{PyArg_ParseTuple()} or \cfunction{PyArg_UnpackTuple}.
\end{datadesc}

\begin{datadesc}{METH_KEYWORDS}
  Methods with these flags must be of type
  \ctype{PyCFunctionWithKeywords}.  The function expects three
  parameters: \var{self}, \var{args}, and a dictionary of all the
  keyword arguments.  The flag is typically combined with
  \constant{METH_VARARGS}, and the parameters are typically processed
  using \cfunction{PyArg_ParseTupleAndKeywords()}.
\end{datadesc}

\begin{datadesc}{METH_NOARGS}
  Methods without parameters don't need to check whether arguments are
  given if they are listed with the \constant{METH_NOARGS} flag.  They
  need to be of type \ctype{PyCFunction}.  When used with object
  methods, the first parameter is typically named \code{self} and will
  hold a reference to the object instance.  In all cases the second
  parameter will be \NULL.
\end{datadesc}

\begin{datadesc}{METH_O}
  Methods with a single object argument can be listed with the
  \constant{METH_O} flag, instead of invoking
  \cfunction{PyArg_ParseTuple()} with a \code{"O"} argument. They have
  the type \ctype{PyCFunction}, with the \var{self} parameter, and a
  \ctype{PyObject*} parameter representing the single argument.
\end{datadesc}

\begin{datadesc}{METH_OLDARGS}
  This calling convention is deprecated.  The method must be of type
  \ctype{PyCFunction}.  The second argument is \NULL{} if no arguments
  are given, a single object if exactly one argument is given, and a
  tuple of objects if more than one argument is given.  There is no
  way for a function using this convention to distinguish between a
  call with multiple arguments and a call with a tuple as the only
  argument.
\end{datadesc}

These two constants are not used to indicate the calling convention
but the binding when use with methods of classes.  These may not be
used for functions defined for modules.  At most one of these flags
may be set for any given method.

\begin{datadesc}{METH_CLASS}
  The method will be passed the type object as the first parameter
  rather than an instance of the type.  This is used to create
  \emph{class methods}, similar to what is created when using the
  \function{classmethod()}\bifuncindex{classmethod} built-in
  function.
  \versionadded{2.3}
\end{datadesc}

\begin{datadesc}{METH_STATIC}
  The method will be passed \NULL{} as the first parameter rather than
  an instance of the type.  This is used to create \emph{static
  methods}, similar to what is created when using the
  \function{staticmethod()}\bifuncindex{staticmethod} built-in
  function.
  \versionadded{2.3}
\end{datadesc}

One other constant controls whether a method is loaded in place of
another definition with the same method name.

\begin{datadesc}{METH_COEXIST}
  The method will be loaded in place of existing definitions.  Without
  \var{METH_COEXIST}, the default is to skip repeated definitions.  Since
  slot wrappers are loaded before the method table, the existence of a
  \var{sq_contains} slot, for example, would generate a wrapped method
  named \method{__contains__()} and preclude the loading of a
  corresponding PyCFunction with the same name.  With the flag defined,
  the PyCFunction will be loaded in place of the wrapper object and will
  co-exist with the slot.  This is helpful because calls to PyCFunctions
  are optimized more than wrapper object calls.
  \versionadded{2.4}
\end{datadesc}

\begin{cfuncdesc}{PyObject*}{Py_FindMethod}{PyMethodDef table[],
                                            PyObject *ob, char *name}
  Return a bound method object for an extension type implemented in
  C.  This can be useful in the implementation of a
  \member{tp_getattro} or \member{tp_getattr} handler that does not
  use the \cfunction{PyObject_GenericGetAttr()} function.
\end{cfuncdesc}


\section{Type Objects \label{type-structs}}

Perhaps one of the most important structures of the Python object
system is the structure that defines a new type: the
\ctype{PyTypeObject} structure.  Type objects can be handled using any
of the \cfunction{PyObject_*()} or \cfunction{PyType_*()} functions,
but do not offer much that's interesting to most Python applications.
These objects are fundamental to how objects behave, so they are very
important to the interpreter itself and to any extension module that
implements new types.

Type objects are fairly large compared to most of the standard types.
The reason for the size is that each type object stores a large number
of values, mostly C function pointers, each of which implements a
small part of the type's functionality.  The fields of the type object
are examined in detail in this section.  The fields will be described
in the order in which they occur in the structure.

Typedefs:
unaryfunc, binaryfunc, ternaryfunc, inquiry, coercion, intargfunc,
intintargfunc, intobjargproc, intintobjargproc, objobjargproc,
destructor, freefunc, printfunc, getattrfunc, getattrofunc, setattrfunc,
setattrofunc, cmpfunc, reprfunc, hashfunc

The structure definition for \ctype{PyTypeObject} can be found in
\file{Include/object.h}.  For convenience of reference, this repeats
the definition found there:

\verbatiminput{typestruct.h}

The type object structure extends the \ctype{PyVarObject} structure.
The \member{ob_size} field is used for dynamic types (created
by  \function{type_new()}, usually called from a class statement).
Note that \cdata{PyType_Type} (the metatype) initializes
\member{tp_itemsize}, which means that its instances (i.e. type
objects) \emph{must} have the \member{ob_size} field.

\begin{cmemberdesc}{PyObject}{PyObject*}{_ob_next}
\cmemberline{PyObject}{PyObject*}{_ob_prev}
  These fields are only present when the macro \code{Py_TRACE_REFS} is
  defined.  Their initialization to \NULL{} is taken care of by the
  \code{PyObject_HEAD_INIT} macro.  For statically allocated objects,
  these fields always remain \NULL.  For dynamically allocated
  objects, these two fields are used to link the object into a
  doubly-linked list of \emph{all} live objects on the heap.  This
  could be used for various debugging purposes; currently the only use
  is to print the objects that are still alive at the end of a run
  when the environment variable \envvar{PYTHONDUMPREFS} is set.

  These fields are not inherited by subtypes.
\end{cmemberdesc}

\begin{cmemberdesc}{PyObject}{int}{ob_refcnt}
  This is the type object's reference count, initialized to \code{1}
  by the \code{PyObject_HEAD_INIT} macro.  Note that for statically
  allocated type objects, the type's instances (objects whose
  \member{ob_type} points back to the type) do \emph{not} count as
  references.  But for dynamically allocated type objects, the
  instances \emph{do} count as references.

  This field is not inherited by subtypes.
\end{cmemberdesc}

\begin{cmemberdesc}{PyObject}{PyTypeObject*}{ob_type}
  This is the type's type, in other words its metatype.  It is
  initialized by the argument to the \code{PyObject_HEAD_INIT} macro,
  and its value should normally be \code{\&PyType_Type}.  However, for
  dynamically loadable extension modules that must be usable on
  Windows (at least), the compiler complains that this is not a valid
  initializer.  Therefore, the convention is to pass \NULL{} to the
  \code{PyObject_HEAD_INIT} macro and to initialize this field
  explicitly at the start of the module's initialization function,
  before doing anything else.  This is typically done like this:

\begin{verbatim}
Foo_Type.ob_type = &PyType_Type;
\end{verbatim}

  This should be done before any instances of the type are created.
  \cfunction{PyType_Ready()} checks if \member{ob_type} is \NULL, and
  if so, initializes it: in Python 2.2, it is set to
  \code{\&PyType_Type}; in Python 2.2.1 and later it is
  initialized to the \member{ob_type} field of the base class.
  \cfunction{PyType_Ready()} will not change this field if it is
  non-zero.

  In Python 2.2, this field is not inherited by subtypes.  In 2.2.1,
  and in 2.3 and beyond, it is inherited by subtypes.
\end{cmemberdesc}

\begin{cmemberdesc}{PyVarObject}{int}{ob_size}
  For statically allocated type objects, this should be initialized
  to zero.  For dynamically allocated type objects, this field has a
  special internal meaning.

  This field is not inherited by subtypes.
\end{cmemberdesc}

\begin{cmemberdesc}{PyTypeObject}{char*}{tp_name}
  Pointer to a NUL-terminated string containing the name of the type.
  For types that are accessible as module globals, the string should
  be the full module name, followed by a dot, followed by the type
  name; for built-in types, it should be just the type name.  If the
  module is a submodule of a package, the full package name is part of
  the full module name.  For example, a type named \class{T} defined
  in module \module{M} in subpackage \module{Q} in package \module{P}
  should have the \member{tp_name} initializer \code{"P.Q.M.T"}.

  For dynamically allocated type objects, this should just be the type
  name, and the module name explicitly stored in the type dict as the
  value for key \code{'__module__'}.

  For statically allocated type objects, the tp_name field should
  contain a dot.  Everything before the last dot is made accessible as
  the \member{__module__} attribute, and everything after the last dot
  is made accessible as the \member{__name__} attribute.

  If no dot is present, the entire \member{tp_name} field is made
  accessible as the \member{__name__} attribute, and the
  \member{__module__} attribute is undefined (unless explicitly set in
  the dictionary, as explained above).  This means your type will be
  impossible to pickle.

  This field is not inherited by subtypes.
\end{cmemberdesc}

\begin{cmemberdesc}{PyTypeObject}{int}{tp_basicsize}
\cmemberline{PyTypeObject}{int}{tp_itemsize}
  These fields allow calculating the size in bytes of instances of
  the type.

  There are two kinds of types: types with fixed-length instances have
  a zero \member{tp_itemsize} field, types with variable-length
  instances have a non-zero \member{tp_itemsize} field.  For a type
  with fixed-length instances, all instances have the same size,
  given in \member{tp_basicsize}.

  For a type with variable-length instances, the instances must have
  an \member{ob_size} field, and the instance size is
  \member{tp_basicsize} plus N times \member{tp_itemsize}, where N is
  the ``length'' of the object.  The value of N is typically stored in
  the instance's \member{ob_size} field.  There are exceptions:  for
  example, long ints use a negative \member{ob_size} to indicate a
  negative number, and N is \code{abs(\member{ob_size})} there.  Also,
  the presence of an \member{ob_size} field in the instance layout
  doesn't mean that the instance structure is variable-length (for
  example, the structure for the list type has fixed-length instances,
  yet those instances have a meaningful \member{ob_size} field).

  The basic size includes the fields in the instance declared by the
  macro \csimplemacro{PyObject_HEAD} or
  \csimplemacro{PyObject_VAR_HEAD} (whichever is used to declare the
  instance struct) and this in turn includes the \member{_ob_prev} and
  \member{_ob_next} fields if they are present.  This means that the
  only correct way to get an initializer for the \member{tp_basicsize}
  is to use the \keyword{sizeof} operator on the struct used to
  declare the instance layout.  The basic size does not include the GC
  header size (this is new in Python 2.2; in 2.1 and 2.0, the GC
  header size was included in \member{tp_basicsize}).

  These fields are inherited separately by subtypes.  If the base type
  has a non-zero \member{tp_itemsize}, it is generally not safe to set
  \member{tp_itemsize} to a different non-zero value in a subtype
  (though this depends on the implementation of the base type).

  A note about alignment: if the variable items require a particular
  alignment, this should be taken care of by the value of
  \member{tp_basicsize}.  Example: suppose a type implements an array
  of \code{double}. \member{tp_itemsize} is \code{sizeof(double)}.
  It is the programmer's responsibility that \member{tp_basicsize} is
  a multiple of \code{sizeof(double)} (assuming this is the alignment
  requirement for \code{double}).
\end{cmemberdesc}

\begin{cmemberdesc}{PyTypeObject}{destructor}{tp_dealloc}
  A pointer to the instance destructor function.  This function must
  be defined unless the type guarantees that its instances will never
  be deallocated (as is the case for the singletons \code{None} and
  \code{Ellipsis}).

  The destructor function is called by the \cfunction{Py_DECREF()} and
  \cfunction{Py_XDECREF()} macros when the new reference count is
  zero.  At this point, the instance is still in existence, but there
  are no references to it.  The destructor function should free all
  references which the instance owns, free all memory buffers owned by
  the instance (using the freeing function corresponding to the
  allocation function used to allocate the buffer), and finally (as
  its last action) call the type's \member{tp_free} function.  If the
  type is not subtypable (doesn't have the
  \constant{Py_TPFLAGS_BASETYPE} flag bit set), it is permissible to
  call the object deallocator directly instead of via
  \member{tp_free}.  The object deallocator should be the one used to
  allocate the instance; this is normally \cfunction{PyObject_Del()}
  if the instance was allocated using \cfunction{PyObject_New()} or
  \cfunction{PyOject_VarNew()}, or \cfunction{PyObject_GC_Del()} if
  the instance was allocated using \cfunction{PyObject_GC_New()} or
  \cfunction{PyObject_GC_VarNew()}.

  This field is inherited by subtypes.
\end{cmemberdesc}

\begin{cmemberdesc}{PyTypeObject}{printfunc}{tp_print}
  An optional pointer to the instance print function.

  The print function is only called when the instance is printed to a
  \emph{real} file; when it is printed to a pseudo-file (like a
  \class{StringIO} instance), the instance's \member{tp_repr} or
  \member{tp_str} function is called to convert it to a string.  These
  are also called when the type's \member{tp_print} field is \NULL.  A
  type should never implement \member{tp_print} in a way that produces
  different output than \member{tp_repr} or \member{tp_str} would.

  The print function is called with the same signature as
  \cfunction{PyObject_Print()}: \code{int tp_print(PyObject *self, FILE
  *file, int flags)}.  The \var{self} argument is the instance to be
  printed.  The \var{file} argument is the stdio file to which it is
  to be printed.  The \var{flags} argument is composed of flag bits.
  The only flag bit currently defined is \constant{Py_PRINT_RAW}.
  When the \constant{Py_PRINT_RAW} flag bit is set, the instance
  should be printed the same way as \member{tp_str} would format it;
  when the \constant{Py_PRINT_RAW} flag bit is clear, the instance
  should be printed the same was as \member{tp_repr} would format it.
  It should return \code{-1} and set an exception condition when an
  error occurred during the comparison.

  It is possible that the \member{tp_print} field will be deprecated.
  In any case, it is recommended not to define \member{tp_print}, but
  instead to rely on \member{tp_repr} and \member{tp_str} for
  printing.

  This field is inherited by subtypes.
\end{cmemberdesc}

\begin{cmemberdesc}{PyTypeObject}{getattrfunc}{tp_getattr}
  An optional pointer to the get-attribute-string function.

  This field is deprecated.  When it is defined, it should point to a
  function that acts the same as the \member{tp_getattro} function,
  but taking a C string instead of a Python string object to give the
  attribute name.  The signature is the same as for
  \cfunction{PyObject_GetAttrString()}.

  This field is inherited by subtypes together with
  \member{tp_getattro}: a subtype inherits both \member{tp_getattr}
  and \member{tp_getattro} from its base type when the subtype's
  \member{tp_getattr} and \member{tp_getattro} are both \NULL.
\end{cmemberdesc}

\begin{cmemberdesc}{PyTypeObject}{setattrfunc}{tp_setattr}
  An optional pointer to the set-attribute-string function.

  This field is deprecated.  When it is defined, it should point to a
  function that acts the same as the \member{tp_setattro} function,
  but taking a C string instead of a Python string object to give the
  attribute name.  The signature is the same as for
  \cfunction{PyObject_SetAttrString()}.

  This field is inherited by subtypes together with
  \member{tp_setattro}: a subtype inherits both \member{tp_setattr}
  and \member{tp_setattro} from its base type when the subtype's
  \member{tp_setattr} and \member{tp_setattro} are both \NULL.
\end{cmemberdesc}

\begin{cmemberdesc}{PyTypeObject}{cmpfunc}{tp_compare}
  An optional pointer to the three-way comparison function.

  The signature is the same as for \cfunction{PyObject_Compare()}.
  The function should return \code{1} if \var{self} greater than
  \var{other}, \code{0} if \var{self} is equal to \var{other}, and
  \code{-1} if \var{self} less than \var{other}.  It should return
  \code{-1} and set an exception condition when an error occurred
  during the comparison.

  This field is inherited by subtypes together with
  \member{tp_richcompare} and \member{tp_hash}: a subtypes inherits
  all three of \member{tp_compare}, \member{tp_richcompare}, and
  \member{tp_hash} when the subtype's \member{tp_compare},
  \member{tp_richcompare}, and \member{tp_hash} are all \NULL.
\end{cmemberdesc}

\begin{cmemberdesc}{PyTypeObject}{reprfunc}{tp_repr}
  An optional pointer to a function that implements the built-in
  function \function{repr()}.\bifuncindex{repr}

  The signature is the same as for \cfunction{PyObject_Repr()}; it
  must return a string or a Unicode object.  Ideally, this function
  should return a string that, when passed to \function{eval()}, given
  a suitable environment, returns an object with the same value.  If
  this is not feasible, it should return a string starting with
  \character{\textless} and ending with \character{\textgreater} from
  which both the type and the value of the object can be deduced.

  When this field is not set, a string of the form \samp{<\%s object
  at \%p>} is returned, where \code{\%s} is replaced by the type name,
  and \code{\%p} by the object's memory address.

  This field is inherited by subtypes.
\end{cmemberdesc}

PyNumberMethods *tp_as_number;

    XXX

PySequenceMethods *tp_as_sequence;

    XXX

PyMappingMethods *tp_as_mapping;

    XXX

\begin{cmemberdesc}{PyTypeObject}{hashfunc}{tp_hash}
  An optional pointer to a function that implements the built-in
  function \function{hash()}.\bifuncindex{hash}

  The signature is the same as for \cfunction{PyObject_Hash()}; it
  must return a C long.  The value \code{-1} should not be returned as
  a normal return value; when an error occurs during the computation
  of the hash value, the function should set an exception and return
  \code{-1}.

  When this field is not set, two possibilities exist: if the
  \member{tp_compare} and \member{tp_richcompare} fields are both
  \NULL, a default hash value based on the object's address is
  returned; otherwise, a \exception{TypeError} is raised.

  This field is inherited by subtypes together with
  \member{tp_richcompare} and \member{tp_compare}: a subtypes inherits
  all three of \member{tp_compare}, \member{tp_richcompare}, and
  \member{tp_hash}, when the subtype's \member{tp_compare},
  \member{tp_richcompare} and \member{tp_hash} are all \NULL.
\end{cmemberdesc}

\begin{cmemberdesc}{PyTypeObject}{ternaryfunc}{tp_call}
  An optional pointer to a function that implements calling the
  object.  This should be \NULL{} if the object is not callable.  The
  signature is the same as for \cfunction{PyObject_Call()}.

  This field is inherited by subtypes.
\end{cmemberdesc}

\begin{cmemberdesc}{PyTypeObject}{reprfunc}{tp_str}
  An optional pointer to a function that implements the built-in
  operation \function{str()}.  (Note that \class{str} is a type now,
  and \function{str()} calls the constructor for that type.  This
  constructor calls \cfunction{PyObject_Str()} to do the actual work,
  and \cfunction{PyObject_Str()} will call this handler.)

  The signature is the same as for \cfunction{PyObject_Str()}; it must
  return a string or a Unicode object.  This function should return a
  ``friendly'' string representation of the object, as this is the
  representation that will be used by the print statement.

  When this field is not set, \cfunction{PyObject_Repr()} is called to
  return a string representation.

  This field is inherited by subtypes.
\end{cmemberdesc}

\begin{cmemberdesc}{PyTypeObject}{getattrofunc}{tp_getattro}
  An optional pointer to the get-attribute function.

  The signature is the same as for \cfunction{PyObject_GetAttr()}.  It
  is usually convenient to set this field to
  \cfunction{PyObject_GenericGetAttr()}, which implements the normal
  way of looking for object attributes.

  This field is inherited by subtypes together with
  \member{tp_getattr}: a subtype inherits both \member{tp_getattr} and
  \member{tp_getattro} from its base type when the subtype's
  \member{tp_getattr} and \member{tp_getattro} are both \NULL.
\end{cmemberdesc}

\begin{cmemberdesc}{PyTypeObject}{setattrofunc}{tp_setattro}
  An optional pointer to the set-attribute function.

  The signature is the same as for \cfunction{PyObject_SetAttr()}.  It
  is usually convenient to set this field to
  \cfunction{PyObject_GenericSetAttr()}, which implements the normal
  way of setting object attributes.

  This field is inherited by subtypes together with
  \member{tp_setattr}: a subtype inherits both \member{tp_setattr} and
  \member{tp_setattro} from its base type when the subtype's
  \member{tp_setattr} and \member{tp_setattro} are both \NULL.
\end{cmemberdesc}

\begin{cmemberdesc}{PyTypeObject}{PyBufferProcs*}{tp_as_buffer}
  Pointer to an additional structure that contains fields relevant only to
  objects which implement the buffer interface.  These fields are
  documented in ``Buffer Object Structures'' (section
  \ref{buffer-structs}).

  The \member{tp_as_buffer} field is not inherited, but the contained
  fields are inherited individually.
\end{cmemberdesc}

\begin{cmemberdesc}{PyTypeObject}{long}{tp_flags}
  This field is a bit mask of various flags.  Some flags indicate
  variant semantics for certain situations; others are used to
  indicate that certain fields in the type object (or in the extension
  structures referenced via \member{tp_as_number},
  \member{tp_as_sequence}, \member{tp_as_mapping}, and
  \member{tp_as_buffer}) that were historically not always present are
  valid; if such a flag bit is clear, the type fields it guards must
  not be accessed and must be considered to have a zero or \NULL{}
  value instead.

  Inheritance of this field is complicated.  Most flag bits are
  inherited individually, i.e. if the base type has a flag bit set,
  the subtype inherits this flag bit.  The flag bits that pertain to
  extension structures are strictly inherited if the extension
  structure is inherited, i.e. the base type's value of the flag bit
  is copied into the subtype together with a pointer to the extension
  structure.  The \constant{Py_TPFLAGS_HAVE_GC} flag bit is inherited
  together with the \member{tp_traverse} and \member{tp_clear} fields,
  i.e. if the \constant{Py_TPFLAGS_HAVE_GC} flag bit is clear in the
  subtype and the \member{tp_traverse} and \member{tp_clear} fields in
  the subtype exist (as indicated by the
  \constant{Py_TPFLAGS_HAVE_RICHCOMPARE} flag bit) and have \NULL{}
  values.

  The following bit masks are currently defined; these can be or-ed
  together using the \code{|} operator to form the value of the
  \member{tp_flags} field.  The macro \cfunction{PyType_HasFeature()}
  takes a type and a flags value, \var{tp} and \var{f}, and checks
  whether \code{\var{tp}->tp_flags \& \var{f}} is non-zero.

  \begin{datadesc}{Py_TPFLAGS_HAVE_GETCHARBUFFER}
    If this bit is set, the \ctype{PyBufferProcs} struct referenced by
    \member{tp_as_buffer} has the \member{bf_getcharbuffer} field.
  \end{datadesc}

  \begin{datadesc}{Py_TPFLAGS_HAVE_SEQUENCE_IN}
    If this bit is set, the \ctype{PySequenceMethods} struct
    referenced by \member{tp_as_sequence} has the \member{sq_contains}
    field.
  \end{datadesc}

  \begin{datadesc}{Py_TPFLAGS_GC}
    This bit is obsolete.  The bit it used to name is no longer in
    use.  The symbol is now defined as zero.
  \end{datadesc}

  \begin{datadesc}{Py_TPFLAGS_HAVE_INPLACEOPS}
    If this bit is set, the \ctype{PySequenceMethods} struct
    referenced by \member{tp_as_sequence} and the
    \ctype{PyNumberMethods} structure referenced by
    \member{tp_as_number} contain the fields for in-place operators.
    In particular, this means that the \ctype{PyNumberMethods}
    structure has the fields \member{nb_inplace_add},
    \member{nb_inplace_subtract}, \member{nb_inplace_multiply},
    \member{nb_inplace_divide}, \member{nb_inplace_remainder},
    \member{nb_inplace_power}, \member{nb_inplace_lshift},
    \member{nb_inplace_rshift}, \member{nb_inplace_and},
    \member{nb_inplace_xor}, and \member{nb_inplace_or}; and the
    \ctype{PySequenceMethods} struct has the fields
    \member{sq_inplace_concat} and \member{sq_inplace_repeat}.
  \end{datadesc}

  \begin{datadesc}{Py_TPFLAGS_CHECKTYPES}
    If this bit is set, the binary and ternary operations in the
    \ctype{PyNumberMethods} structure referenced by
    \member{tp_as_number} accept arguments of arbitrary object types,
    and do their own type conversions if needed.  If this bit is
    clear, those operations require that all arguments have the
    current type as their type, and the caller is supposed to perform
    a coercion operation first.  This applies to \member{nb_add},
    \member{nb_subtract}, \member{nb_multiply}, \member{nb_divide},
    \member{nb_remainder}, \member{nb_divmod}, \member{nb_power},
    \member{nb_lshift}, \member{nb_rshift}, \member{nb_and},
    \member{nb_xor}, and \member{nb_or}.
  \end{datadesc}

  \begin{datadesc}{Py_TPFLAGS_HAVE_RICHCOMPARE}
    If this bit is set, the type object has the
    \member{tp_richcompare} field, as well as the \member{tp_traverse}
    and the \member{tp_clear} fields.
  \end{datadesc}

  \begin{datadesc}{Py_TPFLAGS_HAVE_WEAKREFS}
    If this bit is set, the \member{tp_weaklistoffset} field is
    defined.  Instances of a type are weakly referenceable if the
    type's \member{tp_weaklistoffset} field has a value greater than
    zero.
  \end{datadesc}

  \begin{datadesc}{Py_TPFLAGS_HAVE_ITER}
    If this bit is set, the type object has the \member{tp_iter} and
    \member{tp_iternext} fields.
  \end{datadesc}

  \begin{datadesc}{Py_TPFLAGS_HAVE_CLASS}
    If this bit is set, the type object has several new fields defined
    starting in Python 2.2: \member{tp_methods}, \member{tp_members},
    \member{tp_getset}, \member{tp_base}, \member{tp_dict},
    \member{tp_descr_get}, \member{tp_descr_set},
    \member{tp_dictoffset}, \member{tp_init}, \member{tp_alloc},
    \member{tp_new}, \member{tp_free}, \member{tp_is_gc},
    \member{tp_bases}, \member{tp_mro}, \member{tp_cache},
    \member{tp_subclasses}, and \member{tp_weaklist}.
  \end{datadesc}

  \begin{datadesc}{Py_TPFLAGS_HEAPTYPE}
    This bit is set when the type object itself is allocated on the
    heap.  In this case, the \member{ob_type} field of its instances
    is considered a reference to the type, and the type object is
    INCREF'ed when a new instance is created, and DECREF'ed when an
    instance is destroyed (this does not apply to instances of
    subtypes; only the type referenced by the instance's ob_type gets
    INCREF'ed or DECREF'ed).
  \end{datadesc}

  \begin{datadesc}{Py_TPFLAGS_BASETYPE}
    This bit is set when the type can be used as the base type of
    another type.  If this bit is clear, the type cannot be subtyped
    (similar to a "final" class in Java).
  \end{datadesc}

  \begin{datadesc}{Py_TPFLAGS_READY}
    This bit is set when the type object has been fully initialized by
    \cfunction{PyType_Ready()}.
  \end{datadesc}

  \begin{datadesc}{Py_TPFLAGS_READYING}
    This bit is set while \cfunction{PyType_Ready()} is in the process
    of initializing the type object.
  \end{datadesc}

  \begin{datadesc}{Py_TPFLAGS_HAVE_GC}
    This bit is set when the object supports garbage collection.  If
    this bit is set, instances must be created using
    \cfunction{PyObject_GC_New()} and destroyed using
    \cfunction{PyObject_GC_Del()}.  More information in section XXX
    about garbage collection.  This bit also implies that the
    GC-related fields \member{tp_traverse} and \member{tp_clear} are
    present in the type object; but those fields also exist when
    \constant{Py_TPFLAGS_HAVE_GC} is clear but
    \constant{Py_TPFLAGS_HAVE_RICHCOMPARE} is set.
  \end{datadesc}

  \begin{datadesc}{Py_TPFLAGS_DEFAULT}
    This is a bitmask of all the bits that pertain to the existence of
    certain fields in the type object and its extension structures.
    Currently, it includes the following bits:
    \constant{Py_TPFLAGS_HAVE_GETCHARBUFFER},
    \constant{Py_TPFLAGS_HAVE_SEQUENCE_IN},
    \constant{Py_TPFLAGS_HAVE_INPLACEOPS},
    \constant{Py_TPFLAGS_HAVE_RICHCOMPARE},
    \constant{Py_TPFLAGS_HAVE_WEAKREFS},
    \constant{Py_TPFLAGS_HAVE_ITER}, and
    \constant{Py_TPFLAGS_HAVE_CLASS}.
  \end{datadesc}
\end{cmemberdesc}

\begin{cmemberdesc}{PyTypeObject}{char*}{tp_doc}
  An optional pointer to a NUL-terminated C string giving the
  docstring for this type object.  This is exposed as the
  \member{__doc__} attribute on the type and instances of the type.

  This field is \emph{not} inherited by subtypes.
\end{cmemberdesc}

The following three fields only exist if the
\constant{Py_TPFLAGS_HAVE_RICHCOMPARE} flag bit is set.

\begin{cmemberdesc}{PyTypeObject}{traverseproc}{tp_traverse}
  An optional pointer to a traversal function for the garbage
  collector.  This is only used if the \constant{Py_TPFLAGS_HAVE_GC}
  flag bit is set.  More information in section
  \ref{supporting-cycle-detection} about garbage collection.

  This field is inherited by subtypes together with \member{tp_clear}
  and the \constant{Py_TPFLAGS_HAVE_GC} flag bit: the flag bit,
  \member{tp_traverse}, and \member{tp_clear} are all inherited from
  the base type if they are all zero in the subtype \emph{and} the
  subtype has the \constant{Py_TPFLAGS_HAVE_RICHCOMPARE} flag bit set.
\end{cmemberdesc}

\begin{cmemberdesc}{PyTypeObject}{inquiry}{tp_clear}
  An optional pointer to a clear function for the garbage collector.
  This is only used if the \constant{Py_TPFLAGS_HAVE_GC} flag bit is
  set.  More information in section
  \ref{supporting-cycle-detection} about garbage collection.

  This field is inherited by subtypes together with \member{tp_clear}
  and the \constant{Py_TPFLAGS_HAVE_GC} flag bit: the flag bit,
  \member{tp_traverse}, and \member{tp_clear} are all inherited from
  the base type if they are all zero in the subtype \emph{and} the
  subtype has the \constant{Py_TPFLAGS_HAVE_RICHCOMPARE} flag bit set.
\end{cmemberdesc}

\begin{cmemberdesc}{PyTypeObject}{richcmpfunc}{tp_richcompare}
  An optional pointer to the rich comparison function.

  The signature is the same as for \cfunction{PyObject_RichCompare()}.
  The function should return \code{1} if the requested comparison
  returns true, \code{0} if it returns false.  It should return
  \code{-1} and set an exception condition when an error occurred
  during the comparison.

  This field is inherited by subtypes together with
  \member{tp_compare} and \member{tp_hash}: a subtype inherits all
  three of \member{tp_compare}, \member{tp_richcompare}, and
  \member{tp_hash}, when the subtype's \member{tp_compare},
  \member{tp_richcompare}, and \member{tp_hash} are all \NULL.

  The following constants are defined to be used as the third argument
  for \member{tp_richcompare} and for \cfunction{PyObject_RichCompare()}:

  \begin{tableii}{l|c}{constant}{Constant}{Comparison}
    \lineii{Py_LT}{\code{<}}
    \lineii{Py_LE}{\code{<=}}
    \lineii{Py_EQ}{\code{==}}
    \lineii{Py_NE}{\code{!=}}
    \lineii{Py_GT}{\code{>}}
    \lineii{Py_GE}{\code{>=}}
  \end{tableii}
\end{cmemberdesc}

The next field only exists if the \constant{Py_TPFLAGS_HAVE_WEAKREFS}
flag bit is set.

\begin{cmemberdesc}{PyTypeObject}{long}{tp_weaklistoffset}
  If the instances of this type are weakly referenceable, this field
  is greater than zero and contains the offset in the instance
  structure of the weak reference list head (ignoring the GC header,
  if present); this offset is used by
  \cfunction{PyObject_ClearWeakRefs()} and the
  \cfunction{PyWeakref_*()} functions.  The instance structure needs
  to include a field of type \ctype{PyObject*} which is initialized to
  \NULL.

  Do not confuse this field with \member{tp_weaklist}; that is the
  list head for weak references to the type object itself.

  This field is inherited by subtypes, but see the rules listed below.
  A subtype may override this offset; this means that the subtype uses
  a different weak reference list head than the base type.  Since the
  list head is always found via \member{tp_weaklistoffset}, this
  should not be a problem.

  When a type defined by a class statement has no \member{__slots__}
  declaration, and none of its base types are weakly referenceable,
  the type is made weakly referenceable by adding a weak reference
  list head slot to the instance layout and setting the
  \member{tp_weaklistoffset} of that slot's offset.

  When a type's \member{__slots__} declaration contains a slot named
  \member{__weakref__}, that slot becomes the weak reference list head
  for instances of the type, and the slot's offset is stored in the
  type's \member{tp_weaklistoffset}.

  When a type's \member{__slots__} declaration does not contain a slot
  named \member{__weakref__}, the type inherits its
  \member{tp_weaklistoffset} from its base type.
\end{cmemberdesc}

The next two fields only exist if the
\constant{Py_TPFLAGS_HAVE_CLASS} flag bit is set.

\begin{cmemberdesc}{PyTypeObject}{getiterfunc}{tp_iter}
  An optional pointer to a function that returns an iterator for the
  object.  Its presence normally signals that the instances of this
  type are iterable (although sequences may be iterable without this
  function, and classic instances always have this function, even if
  they don't define an \method{__iter__()} method).

  This function has the same signature as
  \cfunction{PyObject_GetIter()}.

  This field is inherited by subtypes.
\end{cmemberdesc}

\begin{cmemberdesc}{PyTypeObject}{iternextfunc}{tp_iternext}
  An optional pointer to a function that returns the next item in an
  iterator, or raises \exception{StopIteration} when the iterator is
  exhausted.  Its presence normally signals that the instances of this
  type are iterators (although classic instances always have this
  function, even if they don't define a \method{next()} method).

  Iterator types should also define the \member{tp_iter} function, and
  that function should return the iterator instance itself (not a new
  iterator instance).

  This function has the same signature as \cfunction{PyIter_Next()}.

  This field is inherited by subtypes.
\end{cmemberdesc}

The next fields, up to and including \member{tp_weaklist}, only exist
if the \constant{Py_TPFLAGS_HAVE_CLASS} flag bit is set.

\begin{cmemberdesc}{PyTypeObject}{struct PyMethodDef*}{tp_methods}
  An optional pointer to a static \NULL-terminated array of
  \ctype{PyMethodDef} structures, declaring regular methods of this
  type.

  For each entry in the array, an entry is added to the type's
  dictionary (see \member{tp_dict} below) containing a method
  descriptor.

  This field is not inherited by subtypes (methods are
  inherited through a different mechanism).
\end{cmemberdesc}

\begin{cmemberdesc}{PyTypeObject}{struct PyMemberDef*}{tp_members}
  An optional pointer to a static \NULL-terminated array of
  \ctype{PyMemberDef} structures, declaring regular data members
  (fields or slots) of instances of this type.

  For each entry in the array, an entry is added to the type's
  dictionary (see \member{tp_dict} below) containing a member
  descriptor.

  This field is not inherited by subtypes (members are inherited
  through a different mechanism).
\end{cmemberdesc}

\begin{cmemberdesc}{PyTypeObject}{struct PyGetSetDef*}{tp_getset}
  An optional pointer to a static \NULL-terminated array of
  \ctype{PyGetSetDef} structures, declaring computed attributes of
  instances of this type.

  For each entry in the array, an entry is added to the type's
  dictionary (see \member{tp_dict} below) containing a getset
  descriptor.

  This field is not inherited by subtypes (computed attributes are
  inherited through a different mechanism).

  Docs for PyGetSetDef (XXX belong elsewhere):

\begin{verbatim}
typedef PyObject *(*getter)(PyObject *, void *);
typedef int (*setter)(PyObject *, PyObject *, void *);

typedef struct PyGetSetDef {
    char *name;    /* attribute name */
    getter get;    /* C function to get the attribute */
    setter set;    /* C function to set the attribute */
    char *doc;     /* optional doc string */
    void *closure; /* optional additional data for getter and setter */
} PyGetSetDef;
\end{verbatim}
\end{cmemberdesc}

\begin{cmemberdesc}{PyTypeObject}{PyTypeObject*}{tp_base}
  An optional pointer to a base type from which type properties are
  inherited.  At this level, only single inheritance is supported;
  multiple inheritance require dynamically creating a type object by
  calling the metatype.

  This field is not inherited by subtypes (obviously), but it defaults
  to \code{\&PyBaseObject_Type} (which to Python programmers is known
  as the type \class{object}).
\end{cmemberdesc}

\begin{cmemberdesc}{PyTypeObject}{PyObject*}{tp_dict}
  The type's dictionary is stored here by \cfunction{PyType_Ready()}.

  This field should normally be initialized to \NULL{} before
  PyType_Ready is called; it may also be initialized to a dictionary
  containing initial attributes for the type.  Once
  \cfunction{PyType_Ready()} has initialized the type, extra
  attributes for the type may be added to this dictionary only if they
  don't correspond to overloaded operations (like \method{__add__()}).

  This field is not inherited by subtypes (though the attributes
  defined in here are inherited through a different mechanism).
\end{cmemberdesc}

\begin{cmemberdesc}{PyTypeObject}{descrgetfunc}{tp_descr_get}
  An optional pointer to a "descriptor get" function.

  XXX blah, blah.

  This field is inherited by subtypes.
\end{cmemberdesc}

\begin{cmemberdesc}{PyTypeObject}{descrsetfunc}{tp_descr_set}
  An optional pointer to a "descriptor set" function.

  XXX blah, blah.

  This field is inherited by subtypes.
\end{cmemberdesc}

\begin{cmemberdesc}{PyTypeObject}{long}{tp_dictoffset}
  If the instances of this type have a dictionary containing instance
  variables, this field is non-zero and contains the offset in the
  instances of the type of the instance variable dictionary; this
  offset is used by \cfunction{PyObject_GenericGetAttr()}.

  Do not confuse this field with \member{tp_dict}; that is the
  dictionary for attributes of the type object itself.

  If the value of this field is greater than zero, it specifies the
  offset from the start of the instance structure.  If the value is
  less than zero, it specifies the offset from the *end* of the
  instance structure.  A negative offset is more expensive to use, and
  should only be used when the instance structure contains a
  variable-length part.  This is used for example to add an instance
  variable dictionary to subtypes of \class{str} or \class{tuple}.
  Note that the \member{tp_basicsize} field should account for the
  dictionary added to the end in that case, even though the dictionary
  is not included in the basic object layout.  On a system with a
  pointer size of 4 bytes, \member{tp_dictoffset} should be set to
  \code{-4} to indicate that the dictionary is at the very end of the
  structure.

  The real dictionary offset in an instance can be computed from a
  negative \member{tp_dictoffset} as follows:

\begin{verbatim}
dictoffset = tp_basicsize + abs(ob_size)*tp_itemsize + tp_dictoffset
if dictoffset is not aligned on sizeof(void*):
    round up to sizeof(void*)
\end{verbatim}

  where \member{tp_basicsize}, \member{tp_itemsize} and
  \member{tp_dictoffset} are taken from the type object, and
  \member{ob_size} is taken from the instance.  The absolute value is
  taken because long ints use the sign of \member{ob_size} to store
  the sign of the number.  (There's never a need to do this
  calculation yourself; it is done for you by
  \cfunction{_PyObject_GetDictPtr()}.)

  This field is inherited by subtypes, but see the rules listed below.
  A subtype may override this offset; this means that the subtype
  instances store the dictionary at a difference offset than the base
  type.  Since the dictionary is always found via
  \member{tp_dictoffset}, this should not be a problem.

  When a type defined by a class statement has no \member{__slots__}
  declaration, and none of its base types has an instance variable
  dictionary, a dictionary slot is added to the instance layout and
  the \member{tp_dictoffset} is set to that slot's offset.

  When a type defined by a class statement has a \member{__slots__}
  declaration, the type inherits its \member{tp_dictoffset} from its
  base type.

  (Adding a slot named \member{__dict__} to the \member{__slots__}
  declaration does not have the expected effect, it just causes
  confusion.  Maybe this should be added as a feature just like
  \member{__weakref__} though.)
\end{cmemberdesc}

\begin{cmemberdesc}{PyTypeObject}{initproc}{tp_init}
  An optional pointer to an instance initialization function.

  This function corresponds to the \method{__init__()} method of
  classes.  Like \method{__init__()}, it is possible to create an
  instance without calling \method{__init__()}, and it is possible to
  reinitialize an instance by calling its \method{__init__()} method
  again.

  The function signature is

\begin{verbatim}
int tp_init(PyObject *self, PyObject *args, PyObject *kwds)
\end{verbatim}

  The self argument is the instance to be initialized; the \var{args}
  and \var{kwds} arguments represent positional and keyword arguments
  of the call to \method{__init__()}.

  The \member{tp_init} function, if not \NULL, is called when an
  instance is created normally by calling its type, after the type's
  \member{tp_new} function has returned an instance of the type.  If
  the \member{tp_new} function returns an instance of some other type
  that is not a subtype of the original type, no \member{tp_init}
  function is called; if \member{tp_new} returns an instance of a
  subtype of the original type, the subtype's \member{tp_init} is
  called.  (VERSION NOTE: described here is what is implemented in
  Python 2.2.1 and later.  In Python 2.2, the \member{tp_init} of the
  type of the object returned by \member{tp_new} was always called, if
  not \NULL.)

  This field is inherited by subtypes.
\end{cmemberdesc}

\begin{cmemberdesc}{PyTypeObject}{allocfunc}{tp_alloc}
  An optional pointer to an instance allocation function.

  The function signature is

\begin{verbatim}
PyObject *tp_alloc(PyTypeObject *self, int nitems)
\end{verbatim}

  The purpose of this function is to separate memory allocation from
  memory initialization.  It should return a pointer to a block of
  memory of adequate length for the instance, suitably aligned, and
  initialized to zeros, but with \member{ob_refcnt} set to \code{1}
  and \member{ob_type} set to the type argument.  If the type's
  \member{tp_itemsize} is non-zero, the object's \member{ob_size} field
  should be initialized to \var{nitems} and the length of the
  allocated memory block should be \code{tp_basicsize +
  \var{nitems}*tp_itemsize}, rounded up to a multiple of
  \code{sizeof(void*)}; otherwise, \var{nitems} is not used and the
  length of the block should be \member{tp_basicsize}.

  Do not use this function to do any other instance initialization,
  not even to allocate additional memory; that should be done by
  \member{tp_new}.

  This field is inherited by static subtypes, but not by dynamic
  subtypes (subtypes created by a class statement); in the latter,
  this field is always set to \cfunction{PyType_GenericAlloc()}, to
  force a standard heap allocation strategy.  That is also the
  recommended value for statically defined types.
\end{cmemberdesc}

\begin{cmemberdesc}{PyTypeObject}{newfunc}{tp_new}
  An optional pointer to an instance creation function.

  If this function is \NULL{} for a particular type, that type cannot
  be called to create new instances; presumably there is some other
  way to create instances, like a factory function.

  The function signature is

\begin{verbatim}
PyObject *tp_new(PyTypeObject *subtype, PyObject *args, PyObject *kwds)
\end{verbatim}

  The subtype argument is the type of the object being created; the
  \var{args} and \var{kwds} arguments represent positional and keyword
  arguments of the call to the type.  Note that subtype doesn't have
  to equal the type whose \member{tp_new} function is called; it may
  be a subtype of that type (but not an unrelated type).

  The \member{tp_new} function should call
  \code{\var{subtype}->tp_alloc(\var{subtype}, \var{nitems})} to
  allocate space for the object, and then do only as much further
  initialization as is absolutely necessary.  Initialization that can
  safely be ignored or repeated should be placed in the
  \member{tp_init} handler.  A good rule of thumb is that for
  immutable types, all initialization should take place in
  \member{tp_new}, while for mutable types, most initialization should
  be deferred to \member{tp_init}.

  This field is inherited by subtypes, except it is not inherited by
  static types whose \member{tp_base} is \NULL{} or
  \code{\&PyBaseObject_Type}.  The latter exception is a precaution so
  that old extension types don't become callable simply by being
  linked with Python 2.2.
\end{cmemberdesc}

\begin{cmemberdesc}{PyTypeObject}{destructor}{tp_free}
  An optional pointer to an instance deallocation function.

  The signature of this function has changed slightly: in Python
  2.2 and 2.2.1, its signature is \ctype{destructor}:

\begin{verbatim}
void tp_free(PyObject *)
\end{verbatim}

  In Python 2.3 and beyond, its signature is \ctype{freefunc}:

\begin{verbatim}
void tp_free(void *)
\end{verbatim}

  The only initializer that is compatible with both versions is
  \code{_PyObject_Del}, whose definition has suitably adapted in
  Python 2.3.

  This field is inherited by static subtypes, but not by dynamic
  subtypes (subtypes created by a class statement); in the latter,
  this field is set to a deallocator suitable to match
  \cfunction{PyType_GenericAlloc()} and the value of the
  \constant{Py_TPFLAGS_HAVE_GC} flag bit.
\end{cmemberdesc}

\begin{cmemberdesc}{PyTypeObject}{inquiry}{tp_is_gc}
  An optional pointer to a function called by the garbage collector.

  The garbage collector needs to know whether a particular object is
  collectible or not.  Normally, it is sufficient to look at the
  object's type's \member{tp_flags} field, and check the
  \constant{Py_TPFLAGS_HAVE_GC} flag bit.  But some types have a
  mixture of statically and dynamically allocated instances, and the
  statically allocated instances are not collectible.  Such types
  should define this function; it should return \code{1} for a
  collectible instance, and \code{0} for a non-collectible instance.
  The signature is

\begin{verbatim}
int tp_is_gc(PyObject *self)
\end{verbatim}

  (The only example of this are types themselves.  The metatype,
  \cdata{PyType_Type}, defines this function to distinguish between
  statically and dynamically allocated types.)

  This field is inherited by subtypes.  (VERSION NOTE: in Python
  2.2, it was not inherited.  It is inherited in 2.2.1 and later
  versions.)
\end{cmemberdesc}

\begin{cmemberdesc}{PyTypeObject}{PyObject*}{tp_bases}
  Tuple of base types.

  This is set for types created by a class statement.  It should be
  \NULL{} for statically defined types.

  This field is not inherited.
\end{cmemberdesc}

\begin{cmemberdesc}{PyTypeObject}{PyObject*}{tp_mro}
  Tuple containing the expanded set of base types, starting with the
  type itself and ending with \class{object}, in Method Resolution
  Order.

  This field is not inherited; it is calculated fresh by
  \cfunction{PyType_Ready()}.
\end{cmemberdesc}

\begin{cmemberdesc}{PyTypeObject}{PyObject*}{tp_cache}
  Unused.  Not inherited.  Internal use only.
\end{cmemberdesc}

\begin{cmemberdesc}{PyTypeObject}{PyObject*}{tp_subclasses}
  List of weak references to subclasses.  Not inherited.  Internal
  use only.
\end{cmemberdesc}

\begin{cmemberdesc}{PyTypeObject}{PyObject*}{tp_weaklist}
  Weak reference list head, for weak references to this type
  object.  Not inherited.  Internal use only.
\end{cmemberdesc}

The remaining fields are only defined if the feature test macro
\constant{COUNT_ALLOCS} is defined, and are for internal use only.
They are documented here for completeness.  None of these fields are
inherited by subtypes.

\begin{cmemberdesc}{PyTypeObject}{int}{tp_allocs}
  Number of allocations.
\end{cmemberdesc}

\begin{cmemberdesc}{PyTypeObject}{int}{tp_frees}
  Number of frees.
\end{cmemberdesc}

\begin{cmemberdesc}{PyTypeObject}{int}{tp_maxalloc}
  Maximum simultaneously allocated objects.
\end{cmemberdesc}

\begin{cmemberdesc}{PyTypeObject}{PyTypeObject*}{tp_next}
  Pointer to the next type object with a non-zero \member{tp_allocs}
  field.
\end{cmemberdesc}

Also, note that, in a garbage collected Python, tp_dealloc may be
called from any Python thread, not just the thread which created the
object (if the object becomes part of a refcount cycle, that cycle
might be collected by a garbage collection on any thread).  This is
not a problem for Python API calls, since the thread on which
tp_dealloc is called will own the Global Interpreter Lock (GIL).
However, if the object being destroyed in turn destroys objects from
some other C or C++ library, care should be taken to ensure that
destroying those objects on the thread which called tp_dealloc will
not violate any assumptions of the library.

\section{Mapping Object Structures \label{mapping-structs}}

\begin{ctypedesc}{PyMappingMethods}
  Structure used to hold pointers to the functions used to implement
  the mapping protocol for an extension type.
\end{ctypedesc}


\section{Number Object Structures \label{number-structs}}

\begin{ctypedesc}{PyNumberMethods}
  Structure used to hold pointers to the functions an extension type
  uses to implement the number protocol.
\end{ctypedesc}


\section{Sequence Object Structures \label{sequence-structs}}

\begin{ctypedesc}{PySequenceMethods}
  Structure used to hold pointers to the functions which an object
  uses to implement the sequence protocol.
\end{ctypedesc}


\section{Buffer Object Structures \label{buffer-structs}}
\sectionauthor{Greg J. Stein}{greg@lyra.org}

The buffer interface exports a model where an object can expose its
internal data as a set of chunks of data, where each chunk is
specified as a pointer/length pair.  These chunks are called
\dfn{segments} and are presumed to be non-contiguous in memory.

If an object does not export the buffer interface, then its
\member{tp_as_buffer} member in the \ctype{PyTypeObject} structure
should be \NULL.  Otherwise, the \member{tp_as_buffer} will point to
a \ctype{PyBufferProcs} structure.

\note{It is very important that your \ctype{PyTypeObject} structure
uses \constant{Py_TPFLAGS_DEFAULT} for the value of the
\member{tp_flags} member rather than \code{0}.  This tells the Python
runtime that your \ctype{PyBufferProcs} structure contains the
\member{bf_getcharbuffer} slot. Older versions of Python did not have
this member, so a new Python interpreter using an old extension needs
to be able to test for its presence before using it.}

\begin{ctypedesc}{PyBufferProcs}
  Structure used to hold the function pointers which define an
  implementation of the buffer protocol.

  The first slot is \member{bf_getreadbuffer}, of type
  \ctype{getreadbufferproc}.  If this slot is \NULL, then the object
  does not support reading from the internal data.  This is
  non-sensical, so implementors should fill this in, but callers
  should test that the slot contains a non-\NULL{} value.

  The next slot is \member{bf_getwritebuffer} having type
  \ctype{getwritebufferproc}.  This slot may be \NULL{} if the object
  does not allow writing into its returned buffers.

  The third slot is \member{bf_getsegcount}, with type
  \ctype{getsegcountproc}.  This slot must not be \NULL{} and is used
  to inform the caller how many segments the object contains.  Simple
  objects such as \ctype{PyString_Type} and \ctype{PyBuffer_Type}
  objects contain a single segment.

  The last slot is \member{bf_getcharbuffer}, of type
  \ctype{getcharbufferproc}.  This slot will only be present if the
  \constant{Py_TPFLAGS_HAVE_GETCHARBUFFER} flag is present in the
  \member{tp_flags} field of the object's \ctype{PyTypeObject}.
  Before using this slot, the caller should test whether it is present
  by using the
  \cfunction{PyType_HasFeature()}\ttindex{PyType_HasFeature()}
  function.  If present, it may be \NULL, indicating that the object's
  contents cannot be used as \emph{8-bit characters}.
  The slot function may also raise an error if the object's contents
  cannot be interpreted as 8-bit characters.  For example, if the
  object is an array which is configured to hold floating point
  values, an exception may be raised if a caller attempts to use
  \member{bf_getcharbuffer} to fetch a sequence of 8-bit characters.
  This notion of exporting the internal buffers as ``text'' is used to
  distinguish between objects that are binary in nature, and those
  which have character-based content.

  \note{The current policy seems to state that these characters
  may be multi-byte characters. This implies that a buffer size of
  \var{N} does not mean there are \var{N} characters present.}
\end{ctypedesc}

\begin{datadesc}{Py_TPFLAGS_HAVE_GETCHARBUFFER}
  Flag bit set in the type structure to indicate that the
  \member{bf_getcharbuffer} slot is known.  This being set does not
  indicate that the object supports the buffer interface or that the
  \member{bf_getcharbuffer} slot is non-\NULL.
\end{datadesc}

\begin{ctypedesc}[getreadbufferproc]{int (*getreadbufferproc)
                            (PyObject *self, int segment, void **ptrptr)}
  Return a pointer to a readable segment of the buffer.  This function
  is allowed to raise an exception, in which case it must return
  \code{-1}.  The \var{segment} which is passed must be zero or
  positive, and strictly less than the number of segments returned by
  the \member{bf_getsegcount} slot function.  On success, it returns
  the length of the buffer memory, and sets \code{*\var{ptrptr}} to a
  pointer to that memory.
\end{ctypedesc}

\begin{ctypedesc}[getwritebufferproc]{int (*getwritebufferproc)
                            (PyObject *self, int segment, void **ptrptr)}
  Return a pointer to a writable memory buffer in
  \code{*\var{ptrptr}}, and the length of that segment as the function
  return value.  The memory buffer must correspond to buffer segment
  \var{segment}.  Must return \code{-1} and set an exception on
  error.  \exception{TypeError} should be raised if the object only
  supports read-only buffers, and \exception{SystemError} should be
  raised when \var{segment} specifies a segment that doesn't exist.
% Why doesn't it raise ValueError for this one?
% GJS: because you shouldn't be calling it with an invalid
%      segment. That indicates a blatant programming error in the C
%      code.
\end{ctypedesc}

\begin{ctypedesc}[getsegcountproc]{int (*getsegcountproc)
                            (PyObject *self, int *lenp)}
  Return the number of memory segments which comprise the buffer.  If
  \var{lenp} is not \NULL, the implementation must report the sum of
  the sizes (in bytes) of all segments in \code{*\var{lenp}}.
  The function cannot fail.
\end{ctypedesc}

\begin{ctypedesc}[getcharbufferproc]{int (*getcharbufferproc)
                            (PyObject *self, int segment, const char **ptrptr)}
  Return the size of the memory buffer in \var{ptrptr} for segment
  \var{segment}.  \code{*\var{ptrptr}} is set to the memory buffer.
\end{ctypedesc}


\section{Supporting the Iterator Protocol
         \label{supporting-iteration}}


\section{Supporting Cyclic Garbage Collection
         \label{supporting-cycle-detection}}

Python's support for detecting and collecting garbage which involves
circular references requires support from object types which are
``containers'' for other objects which may also be containers.  Types
which do not store references to other objects, or which only store
references to atomic types (such as numbers or strings), do not need
to provide any explicit support for garbage collection.

An example showing the use of these interfaces can be found in
``\ulink{Supporting the Cycle
Collector}{../ext/example-cycle-support.html}'' in
\citetitle[../ext/ext.html]{Extending and Embedding the Python
Interpreter}.

To create a container type, the \member{tp_flags} field of the type
object must include the \constant{Py_TPFLAGS_HAVE_GC} and provide an
implementation of the \member{tp_traverse} handler.  If instances of the
type are mutable, a \member{tp_clear} implementation must also be
provided.

\begin{datadesc}{Py_TPFLAGS_HAVE_GC}
  Objects with a type with this flag set must conform with the rules
  documented here.  For convenience these objects will be referred to
  as container objects.
\end{datadesc}

Constructors for container types must conform to two rules:

\begin{enumerate}
\item  The memory for the object must be allocated using
       \cfunction{PyObject_GC_New()} or \cfunction{PyObject_GC_VarNew()}.

\item  Once all the fields which may contain references to other
       containers are initialized, it must call
       \cfunction{PyObject_GC_Track()}.
\end{enumerate}

\begin{cfuncdesc}{\var{TYPE}*}{PyObject_GC_New}{TYPE, PyTypeObject *type}
  Analogous to \cfunction{PyObject_New()} but for container objects with
  the \constant{Py_TPFLAGS_HAVE_GC} flag set.
\end{cfuncdesc}

\begin{cfuncdesc}{\var{TYPE}*}{PyObject_GC_NewVar}{TYPE, PyTypeObject *type,
                                                   int size}
  Analogous to \cfunction{PyObject_NewVar()} but for container objects
  with the \constant{Py_TPFLAGS_HAVE_GC} flag set.
\end{cfuncdesc}

\begin{cfuncdesc}{PyVarObject *}{PyObject_GC_Resize}{PyVarObject *op, int}
  Resize an object allocated by \cfunction{PyObject_NewVar()}.  Returns
  the resized object or \NULL{} on failure.
\end{cfuncdesc}

\begin{cfuncdesc}{void}{PyObject_GC_Track}{PyObject *op}
  Adds the object \var{op} to the set of container objects tracked by
  the collector.  The collector can run at unexpected times so objects
  must be valid while being tracked.  This should be called once all
  the fields followed by the \member{tp_traverse} handler become valid,
  usually near the end of the constructor.
\end{cfuncdesc}

\begin{cfuncdesc}{void}{_PyObject_GC_TRACK}{PyObject *op}
  A macro version of \cfunction{PyObject_GC_Track()}.  It should not be
  used for extension modules.
\end{cfuncdesc}

Similarly, the deallocator for the object must conform to a similar
pair of rules:

\begin{enumerate}
\item  Before fields which refer to other containers are invalidated,
       \cfunction{PyObject_GC_UnTrack()} must be called.

\item  The object's memory must be deallocated using
       \cfunction{PyObject_GC_Del()}.
\end{enumerate}

\begin{cfuncdesc}{void}{PyObject_GC_Del}{PyObject *op}
  Releases memory allocated to an object using
  \cfunction{PyObject_GC_New()} or \cfunction{PyObject_GC_NewVar()}.
\end{cfuncdesc}

\begin{cfuncdesc}{void}{PyObject_GC_UnTrack}{PyObject *op}
  Remove the object \var{op} from the set of container objects tracked
  by the collector.  Note that \cfunction{PyObject_GC_Track()} can be
  called again on this object to add it back to the set of tracked
  objects.  The deallocator (\member{tp_dealloc} handler) should call
  this for the object before any of the fields used by the
  \member{tp_traverse} handler become invalid.
\end{cfuncdesc}

\begin{cfuncdesc}{void}{_PyObject_GC_UNTRACK}{PyObject *op}
  A macro version of \cfunction{PyObject_GC_UnTrack()}.  It should not be
  used for extension modules.
\end{cfuncdesc}

The \member{tp_traverse} handler accepts a function parameter of this
type:

\begin{ctypedesc}[visitproc]{int (*visitproc)(PyObject *object, void *arg)}
  Type of the visitor function passed to the \member{tp_traverse}
  handler.  The function should be called with an object to traverse
  as \var{object} and the third parameter to the \member{tp_traverse}
  handler as \var{arg}.
\end{ctypedesc}

The \member{tp_traverse} handler must have the following type:

\begin{ctypedesc}[traverseproc]{int (*traverseproc)(PyObject *self,
                                visitproc visit, void *arg)}
  Traversal function for a container object.  Implementations must
  call the \var{visit} function for each object directly contained by
  \var{self}, with the parameters to \var{visit} being the contained
  object and the \var{arg} value passed to the handler.  If
  \var{visit} returns a non-zero value then an error has occurred and
  that value should be returned immediately.
\end{ctypedesc}

The \member{tp_clear} handler must be of the \ctype{inquiry} type, or
\NULL{} if the object is immutable.

\begin{ctypedesc}[inquiry]{int (*inquiry)(PyObject *self)}
  Drop references that may have created reference cycles.  Immutable
  objects do not have to define this method since they can never
  directly create reference cycles.  Note that the object must still
  be valid after calling this method (don't just call
  \cfunction{Py_DECREF()} on a reference).  The collector will call
  this method if it detects that this object is involved in a
  reference cycle.
\end{ctypedesc}
