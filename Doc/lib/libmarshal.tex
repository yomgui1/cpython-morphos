\section{\module{marshal} ---
         Internal Python object serialization}

\declaremodule{builtin}{marshal}
\modulesynopsis{Convert Python objects to streams of bytes and back
                (with different constraints).}


This module contains functions that can read and write Python
values in a binary format.  The format is specific to Python, but
independent of machine architecture issues (e.g., you can write a
Python value to a file on a PC, transport the file to a Sun, and read
it back there).  Details of the format are undocumented on purpose;
it may change between Python versions (although it rarely
does).\footnote{The name of this module stems from a bit of
  terminology used by the designers of Modula-3 (amongst others), who
  use the term ``marshalling'' for shipping of data around in a
  self-contained form. Strictly speaking, ``to marshal'' means to
  convert some data from internal to external form (in an RPC buffer for
  instance) and ``unmarshalling'' for the reverse process.}

This is not a general ``persistence'' module.  For general persistence
and transfer of Python objects through RPC calls, see the modules
\refmodule{pickle} and \refmodule{shelve}.  The \module{marshal} module exists
mainly to support reading and writing the ``pseudo-compiled'' code for
Python modules of \file{.pyc} files.  Therefore, the Python
maintainers reserve the right to modify the marshal format in backward
incompatible ways should the need arise.  If you're serializing and
de-serializing Python objects, use the \module{pickle} module instead.  
\refstmodindex{pickle}
\refstmodindex{shelve}
\obindex{code}

\begin{notice}[warning]
The \module{marshal} module is not intended to be secure against
erroneous or maliciously constructed data.  Never unmarshal data
received from an untrusted or unauthenticated source.
\end{notice}

Not all Python object types are supported; in general, only objects
whose value is independent from a particular invocation of Python can
be written and read by this module.  The following types are supported:
\code{None}, integers, long integers, floating point numbers,
strings, Unicode objects, tuples, lists, dictionaries, and code
objects, where it should be understood that tuples, lists and
dictionaries are only supported as long as the values contained
therein are themselves supported; and recursive lists and dictionaries
should not be written (they will cause infinite loops).

\strong{Caveat:} On machines where C's \code{long int} type has more than
32 bits (such as the DEC Alpha), it is possible to create plain Python
integers that are longer than 32 bits.
If such an integer is marshaled and read back in on a machine where
C's \code{long int} type has only 32 bits, a Python long integer object
is returned instead.  While of a different type, the numeric value is
the same.  (This behavior is new in Python 2.2.  In earlier versions,
all but the least-significant 32 bits of the value were lost, and a
warning message was printed.)

There are functions that read/write files as well as functions
operating on strings.

The module defines these functions:

\begin{funcdesc}{dump}{value, file}
  Write the value on the open file.  The value must be a supported
  type.  The file must be an open file object such as
  \code{sys.stdout} or returned by \function{open()} or
  \function{posix.popen()}.  It must be opened in binary mode
  (\code{'wb'} or \code{'w+b'}).

  If the value has (or contains an object that has) an unsupported type,
  a \exception{ValueError} exception is raised --- but garbage data
  will also be written to the file.  The object will not be properly
  read back by \function{load()}.
\end{funcdesc}

\begin{funcdesc}{load}{file}
  Read one value from the open file and return it.  If no valid value
  is read, raise \exception{EOFError}, \exception{ValueError} or
  \exception{TypeError}.  The file must be an open file object opened
  in binary mode (\code{'rb'} or \code{'r+b'}).

  \warning{If an object containing an unsupported type was
  marshalled with \function{dump()}, \function{load()} will substitute
  \code{None} for the unmarshallable type.}
\end{funcdesc}

\begin{funcdesc}{dumps}{value}
  Return the string that would be written to a file by
  \code{dump(\var{value}, \var{file})}.  The value must be a supported
  type.  Raise a \exception{ValueError} exception if value has (or
  contains an object that has) an unsupported type.
\end{funcdesc}

\begin{funcdesc}{loads}{string}
  Convert the string to a value.  If no valid value is found, raise
  \exception{EOFError}, \exception{ValueError} or
  \exception{TypeError}.  Extra characters in the string are ignored.
\end{funcdesc}
