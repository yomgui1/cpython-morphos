\section{\module{shlex} ---
         Simple lexical analysis}

\declaremodule{standard}{shlex}
\modulesynopsis{Simple lexical analysis for \UNIX\ shell-like languages.}
\moduleauthor{Eric S. Raymond}{esr@snark.thyrsus.com}
\moduleauthor{Gustavo Niemeyer}{niemeyer@conectiva.com}
\sectionauthor{Eric S. Raymond}{esr@snark.thyrsus.com}
\sectionauthor{Gustavo Niemeyer}{niemeyer@conectiva.com}

\versionadded{1.5.2}

The \class{shlex} class makes it easy to write lexical analyzers for
simple syntaxes resembling that of the \UNIX{} shell.  This will often
be useful for writing minilanguages, (e.g. in run control files for
Python applications) or for parsing quoted strings.

\begin{seealso}
  \seemodule{ConfigParser}{Parser for configuration files similar to the
                           Windows \file{.ini} files.}
\end{seealso}


\subsection{Module Contents}

The \module{shlex} module defines the following functions:

\begin{funcdesc}{split}{s\optional{, comments=\code{False}}}
Split the string \var{s} using shell-like syntax. If \var{comments} is
\code{False}, the parsing of comments in the given string will be
disabled (setting the \member{commenters} member of the \class{shlex}
instance to the empty string). This function operates in \POSIX{} mode.
\versionadded{2.3}
\end{funcdesc}

The \module{shlex} module defines the following classes:

\begin{classdesc}{shlex}{\optional{instream=\code{sys.stdin}\optional{,
			 infile=\code{None}\optional{,
			 posix=\code{False}}}}}
A \class{shlex} instance or subclass instance is a lexical analyzer
object.  The initialization argument, if present, specifies where to
read characters from. It must be a file-/stream-like object with
\method{read()} and \method{readline()} methods, or a string (strings
are accepted since Python 2.3). If no argument is given, input will be
taken from \code{sys.stdin}.  The second optional argument is a filename
string, which sets the initial value of the \member{infile} member.  If
the \var{instream} argument is omitted or equal to \code{sys.stdin},
this second argument defaults to ``stdin''.  The \var{posix} argument
was introduced in Python 2.3, and defines the operational mode.  When
\var{posix} is not true (default), the \class{shlex} instance will
operate in compatibility mode.  When operating in \POSIX{} mode,
\class{shlex} will try to be as close as possible to the \POSIX{} shell
parsing rules.  See~\ref{shlex-objects}.
\end{classdesc}

\subsection{shlex Objects \label{shlex-objects}}

A \class{shlex} instance has the following methods:

\begin{methoddesc}{get_token}{}
Return a token.  If tokens have been stacked using
\method{push_token()}, pop a token off the stack.  Otherwise, read one
from the input stream.  If reading encounters an immediate
end-of-file, \member{self.eof} is returned (the empty string (\code{''})
in non-\POSIX{} mode, and \code{None} in \POSIX{} mode).
\end{methoddesc}

\begin{methoddesc}{push_token}{str}
Push the argument onto the token stack.
\end{methoddesc}

\begin{methoddesc}{read_token}{}
Read a raw token.  Ignore the pushback stack, and do not interpret source
requests.  (This is not ordinarily a useful entry point, and is
documented here only for the sake of completeness.)
\end{methoddesc}

\begin{methoddesc}{sourcehook}{filename}
When \class{shlex} detects a source request (see
\member{source} below) this method is given the following token as
argument, and expected to return a tuple consisting of a filename and
an open file-like object.

Normally, this method first strips any quotes off the argument.  If
the result is an absolute pathname, or there was no previous source
request in effect, or the previous source was a stream
(e.g. \code{sys.stdin}), the result is left alone.  Otherwise, if the
result is a relative pathname, the directory part of the name of the
file immediately before it on the source inclusion stack is prepended
(this behavior is like the way the C preprocessor handles
\code{\#include "file.h"}).

The result of the manipulations is treated as a filename, and returned
as the first component of the tuple, with
\function{open()} called on it to yield the second component. (Note:
this is the reverse of the order of arguments in instance initialization!)

This hook is exposed so that you can use it to implement directory
search paths, addition of file extensions, and other namespace hacks.
There is no corresponding `close' hook, but a shlex instance will call
the \method{close()} method of the sourced input stream when it
returns \EOF.

For more explicit control of source stacking, use the
\method{push_source()} and \method{pop_source()} methods. 
\end{methoddesc}

\begin{methoddesc}{push_source}{stream\optional{, filename}}
Push an input source stream onto the input stack.  If the filename
argument is specified it will later be available for use in error
messages.  This is the same method used internally by the
\method{sourcehook} method.
\versionadded{2.1}
\end{methoddesc}

\begin{methoddesc}{pop_source}{}
Pop the last-pushed input source from the input stack.
This is the same method used internally when the lexer reaches
\EOF{} on a stacked input stream.
\versionadded{2.1}
\end{methoddesc}

\begin{methoddesc}{error_leader}{\optional{file\optional{, line}}}
This method generates an error message leader in the format of a
\UNIX{} C compiler error label; the format is \code{'"\%s", line \%d: '},
where the \samp{\%s} is replaced with the name of the current source
file and the \samp{\%d} with the current input line number (the
optional arguments can be used to override these).

This convenience is provided to encourage \module{shlex} users to
generate error messages in the standard, parseable format understood
by Emacs and other \UNIX{} tools.
\end{methoddesc}

Instances of \class{shlex} subclasses have some public instance
variables which either control lexical analysis or can be used for
debugging:

\begin{memberdesc}{commenters}
The string of characters that are recognized as comment beginners.
All characters from the comment beginner to end of line are ignored.
Includes just \character{\#} by default.   
\end{memberdesc}

\begin{memberdesc}{wordchars}
The string of characters that will accumulate into multi-character
tokens.  By default, includes all \ASCII{} alphanumerics and
underscore.
\end{memberdesc}

\begin{memberdesc}{whitespace}
Characters that will be considered whitespace and skipped.  Whitespace
bounds tokens.  By default, includes space, tab, linefeed and
carriage-return.
\end{memberdesc}

\begin{memberdesc}{escape}
Characters that will be considered as escape. This will be only used
in \POSIX{} mode, and includes just \character{\textbackslash} by default.
\versionadded{2.3}
\end{memberdesc}

\begin{memberdesc}{quotes}
Characters that will be considered string quotes.  The token
accumulates until the same quote is encountered again (thus, different
quote types protect each other as in the shell.)  By default, includes
\ASCII{} single and double quotes.
\end{memberdesc}

\begin{memberdesc}{escapedquotes}
Characters in \member{quotes} that will interpret escape characters
defined in \member{escape}.  This is only used in \POSIX{} mode, and
includes just \character{"} by default.
\versionadded{2.3}
\end{memberdesc}

\begin{memberdesc}{whitespace_split}
If \code{True}, tokens will only be split in whitespaces. This is useful, for
example, for parsing command lines with \class{shlex}, getting tokens
in a similar way to shell arguments.
\versionadded{2.3}
\end{memberdesc}

\begin{memberdesc}{infile}
The name of the current input file, as initially set at class
instantiation time or stacked by later source requests.  It may
be useful to examine this when constructing error messages.
\end{memberdesc}

\begin{memberdesc}{instream}
The input stream from which this \class{shlex} instance is reading
characters.
\end{memberdesc}

\begin{memberdesc}{source}
This member is \code{None} by default.  If you assign a string to it,
that string will be recognized as a lexical-level inclusion request
similar to the \samp{source} keyword in various shells.  That is, the
immediately following token will opened as a filename and input taken
from that stream until \EOF, at which point the \method{close()}
method of that stream will be called and the input source will again
become the original input stream. Source requests may be stacked any
number of levels deep.
\end{memberdesc}

\begin{memberdesc}{debug}
If this member is numeric and \code{1} or more, a \class{shlex}
instance will print verbose progress output on its behavior.  If you
need to use this, you can read the module source code to learn the
details.
\end{memberdesc}

\begin{memberdesc}{lineno}
Source line number (count of newlines seen so far plus one).
\end{memberdesc}

\begin{memberdesc}{token}
The token buffer.  It may be useful to examine this when catching
exceptions.
\end{memberdesc}

\begin{memberdesc}{eof}
Token used to determine end of file. This will be set to the empty
string (\code{''}), in non-\POSIX{} mode, and to \code{None} in
\POSIX{} mode.
\versionadded{2.3}
\end{memberdesc}

\subsection{Parsing Rules\label{shlex-parsing-rules}}

When operating in non-\POSIX{} mode, \class{shlex} will try to obey to
the following rules.

\begin{itemize}
\item Quote characters are not recognized within words
      (\code{Do"Not"Separate} is parsed as the single word
      \code{Do"Not"Separate});
\item Escape characters are not recognized;
\item Enclosing characters in quotes preserve the literal value of
      all characters within the quotes;
\item Closing quotes separate words (\code{"Do"Separate} is parsed
      as \code{"Do"} and \code{Separate});
\item If \member{whitespace_split} is \code{False}, any character not
      declared to be a word character, whitespace, or a quote will be
      returned as a single-character token. If it is \code{True},
      \class{shlex} will only split words in whitespaces;
\item EOF is signaled with an empty string (\code{''});
\item It's not possible to parse empty strings, even if quoted.
\end{itemize}

When operating in \POSIX{} mode, \class{shlex} will try to obey to the
following parsing rules.

\begin{itemize}
\item Quotes are stripped out, and do not separate words
      (\code{"Do"Not"Separate"} is parsed as the single word
      \code{DoNotSeparate});
\item Non-quoted escape characters (e.g. \character{\textbackslash})
      preserve the literal value of the next character that follows;
\item Enclosing characters in quotes which are not part of
      \member{escapedquotes} (e.g. \character{'}) preserve the literal
      value of all characters within the quotes;
\item Enclosing characters in quotes which are part of
      \member{escapedquotes} (e.g. \character{"}) preserves the literal
      value of all characters within the quotes, with the exception of
      the characters mentioned in \member{escape}. The escape characters
      retain its special meaning only when followed by the quote in use,
      or the escape character itself. Otherwise the escape character
      will be considered a normal character.
\item EOF is signaled with a \code{None} value;
\item Quoted empty strings (\code{''}) are allowed;
\end{itemize}

