\section{Standard Module \sectcode{pickle}}
\label{module-pickle}
\stmodindex{pickle}
\index{persistency}
\indexii{persistent}{objects}
\indexii{serializing}{objects}
\indexii{marshalling}{objects}
\indexii{flattening}{objects}
\indexii{pickling}{objects}


The \module{pickle} module implements a basic but powerful algorithm for
``pickling'' (a.k.a.\ serializing, marshalling or flattening) nearly
arbitrary Python objects.  This is the act of converting objects to a
stream of bytes (and back: ``unpickling'').
This is a more primitive notion than
persistency --- although \module{pickle} reads and writes file objects,
it does not handle the issue of naming persistent objects, nor the
(even more complicated) area of concurrent access to persistent
objects.  The \module{pickle} module can transform a complex object into
a byte stream and it can transform the byte stream into an object with
the same internal structure.  The most obvious thing to do with these
byte streams is to write them onto a file, but it is also conceivable
to send them across a network or store them in a database.  The module
\module{shelve} provides a simple interface to pickle and unpickle
objects on ``dbm''-style database files.
\refstmodindex{shelve}

\strong{Note:} The \module{pickle} module is rather slow.  A
reimplementation of the same algorithm in \C{}, which is up to 1000 times
faster, is available as the \module{cPickle}\refbimodindex{cPickle}
module.  This has the same interface except that \code{Pickler} and
\code{Unpickler} are factory functions, not classes (so they cannot be
used as base classes for inheritance).

Unlike the built-in module \module{marshal}, \module{pickle} handles
the following correctly:
\refbimodindex{marshal}

\begin{itemize}

\item recursive objects (objects containing references to themselves)

\item object sharing (references to the same object in different places)

\item user-defined classes and their instances

\end{itemize}

The data format used by \module{pickle} is Python-specific.  This has
the advantage that there are no restrictions imposed by external
standards such as XDR%
\index{XDR}
\index{External Data Representation}
(which can't represent pointer sharing); however
it means that non-Python programs may not be able to reconstruct
pickled Python objects.

By default, the \module{pickle} data format uses a printable \ASCII{}
representation.  This is slightly more voluminous than a binary
representation.  The big advantage of using printable \ASCII{} (and of
some other characteristics of \module{pickle}'s representation) is that
for debugging or recovery purposes it is possible for a human to read
the pickled file with a standard text editor.

A binary format, which is slightly more efficient, can be chosen by
specifying a nonzero (true) value for the \var{bin} argument to the
\class{Pickler} constructor or the \function{dump()} and \function{dumps()}
functions.  The binary format is not the default because of backwards
compatibility with the Python 1.4 pickle module.  In a future version,
the default may change to binary.

The \module{pickle} module doesn't handle code objects, which the
\module{marshal} module does.  I suppose \module{pickle} could, and maybe
it should, but there's probably no great need for it right now (as
long as \module{marshal} continues to be used for reading and writing
code objects), and at least this avoids the possibility of smuggling
Trojan horses into a program.
\refbimodindex{marshal}

For the benefit of persistency modules written using \module{pickle}, it
supports the notion of a reference to an object outside the pickled
data stream.  Such objects are referenced by a name, which is an
arbitrary string of printable \ASCII{} characters.  The resolution of
such names is not defined by the \module{pickle} module --- the
persistent object module will have to implement a method
\method{persistent_load()}.  To write references to persistent objects,
the persistent module must define a method \method{persistent_id()} which
returns either \code{None} or the persistent ID of the object.

There are some restrictions on the pickling of class instances.

First of all, the class must be defined at the top level in a module.
Furthermore, all its instance variables must be picklable.

\setindexsubitem{(pickle protocol)}

When a pickled class instance is unpickled, its \method{__init__()} method
is normally \emph{not} invoked.  \strong{Note:} This is a deviation
from previous versions of this module; the change was introduced in
Python 1.5b2.  The reason for the change is that in many cases it is
desirable to have a constructor that requires arguments; it is a
(minor) nuisance to have to provide a \method{__getinitargs__()} method.

If it is desirable that the \method{__init__()} method be called on
unpickling, a class can define a method \method{__getinitargs__()},
which should return a \emph{tuple} containing the arguments to be
passed to the class constructor (\method{__init__()}).  This method is
called at pickle time; the tuple it returns is incorporated in the
pickle for the instance.
\ttindex{__getinitargs__()}
\ttindex{__init__()}

Classes can further influence how their instances are pickled --- if the class
defines the method \method{__getstate__()}, it is called and the return
state is pickled as the contents for the instance, and if the class
defines the method \method{__setstate__()}, it is called with the
unpickled state.  (Note that these methods can also be used to
implement copying class instances.)  If there is no
\method{__getstate__()} method, the instance's \member{__dict__} is
pickled.  If there is no \method{__setstate__()} method, the pickled
object must be a dictionary and its items are assigned to the new
instance's dictionary.  (If a class defines both \method{__getstate__()}
and \method{__setstate__()}, the state object needn't be a dictionary
--- these methods can do what they want.)  This protocol is also used
by the shallow and deep copying operations defined in the \module{copy}
module.\refstmodindex{copy}
\ttindex{__getstate__()}
\ttindex{__setstate__()}
\ttindex{__dict__}

Note that when class instances are pickled, their class's code and
data are not pickled along with them.  Only the instance data are
pickled.  This is done on purpose, so you can fix bugs in a class or
add methods and still load objects that were created with an earlier
version of the class.  If you plan to have long-lived objects that
will see many versions of a class, it may be worthwhile to put a version
number in the objects so that suitable conversions can be made by the
class's \method{__setstate__()} method.

When a class itself is pickled, only its name is pickled --- the class
definition is not pickled, but re-imported by the unpickling process.
Therefore, the restriction that the class must be defined at the top
level in a module applies to pickled classes as well.

\setindexsubitem{(in module pickle)}

The interface can be summarized as follows.

To pickle an object \code{x} onto a file \code{f}, open for writing:

\begin{verbatim}
p = pickle.Pickler(f)
p.dump(x)
\end{verbatim}

A shorthand for this is:

\begin{verbatim}
pickle.dump(x, f)
\end{verbatim}

To unpickle an object \code{x} from a file \code{f}, open for reading:

\begin{verbatim}
u = pickle.Unpickler(f)
x = u.load()
\end{verbatim}

A shorthand is:

\begin{verbatim}
x = pickle.load(f)
\end{verbatim}

The \class{Pickler} class only calls the method \code{f.write()} with a
string argument.  The \class{Unpickler} calls the methods \code{f.read()}
(with an integer argument) and \code{f.readline()} (without argument),
both returning a string.  It is explicitly allowed to pass non-file
objects here, as long as they have the right methods.
\ttindex{Unpickler}
\ttindex{Pickler}

The constructor for the \class{Pickler} class has an optional second
argument, \var{bin}.  If this is present and nonzero, the binary
pickle format is used; if it is zero or absent, the (less efficient,
but backwards compatible) text pickle format is used.  The
\class{Unpickler} class does not have an argument to distinguish
between binary and text pickle formats; it accepts either format.

The following types can be pickled:
\begin{itemize}

\item \code{None}

\item integers, long integers, floating point numbers

\item strings

\item tuples, lists and dictionaries containing only picklable objects

\item classes that are defined at the top level in a module

\item instances of such classes whose \member{__dict__} or
\method{__setstate__()} is picklable

\end{itemize}

Attempts to pickle unpicklable objects will raise the
\exception{PicklingError} exception; when this happens, an unspecified
number of bytes may have been written to the file.

It is possible to make multiple calls to the \method{dump()} method of
the same \class{Pickler} instance.  These must then be matched to the
same number of calls to the \method{load()} method of the
corresponding \class{Unpickler} instance.  If the same object is
pickled by multiple \method{dump()} calls, the \method{load()} will all
yield references to the same object.  \emph{Warning}: this is intended
for pickling multiple objects without intervening modifications to the
objects or their parts.  If you modify an object and then pickle it
again using the same \class{Pickler} instance, the object is not
pickled again --- a reference to it is pickled and the
\class{Unpickler} will return the old value, not the modified one.
(There are two problems here: (a) detecting changes, and (b)
marshalling a minimal set of changes.  I have no answers.  Garbage
Collection may also become a problem here.)

Apart from the \class{Pickler} and \class{Unpickler} classes, the
module defines the following functions, and an exception:

\begin{funcdesc}{dump}{object, file\optional{, bin}}
Write a pickled representation of \var{obect} to the open file object
\var{file}.  This is equivalent to
\samp{Pickler(\var{file}, \var{bin}).dump(\var{object})}.
If the optional \var{bin} argument is present and nonzero, the binary
pickle format is used; if it is zero or absent, the (less efficient)
text pickle format is used.
\end{funcdesc}

\begin{funcdesc}{load}{file}
Read a pickled object from the open file object \var{file}.  This is
equivalent to \samp{Unpickler(\var{file}).load()}.
\end{funcdesc}

\begin{funcdesc}{dumps}{object\optional{, bin}}
Return the pickled representation of the object as a string, instead
of writing it to a file.  If the optional \var{bin} argument is
present and nonzero, the binary pickle format is used; if it is zero
or absent, the (less efficient) text pickle format is used.
\end{funcdesc}

\begin{funcdesc}{loads}{string}
Read a pickled object from a string instead of a file.  Characters in
the string past the pickled object's representation are ignored.
\end{funcdesc}

\begin{excdesc}{PicklingError}
This exception is raised when an unpicklable object is passed to
\code{Pickler.dump()}.
\end{excdesc}


\begin{seealso}
\seemodule{copy}{shallow and deep object copying}

\seemodule[copyreg]{copy_reg}{pickle interface constructor
registration}

\seemodule{marshal}{high-performance serialization of built-in types}

\seemodule{shelve}{indexed databases of objects; uses \module{pickle}}
\end{seealso}
