\section{Standard Module \module{traceback}}
\label{module-traceback}
\stmodindex{traceback}


This module provides a standard interface to format and print stack
traces of Python programs.  It exactly mimics the behavior of the
Python interpreter when it prints a stack trace.  This is useful when
you want to print stack traces under program control, e.g. in a
``wrapper'' around the interpreter.

The module uses traceback objects --- this is the object type
that is stored in the variables \code{sys.exc_traceback} and
\code{sys.last_traceback}.
\obindex{traceback}

The module defines the following functions:

\begin{funcdesc}{print_tb}{traceback\optional{, limit}}
Print up to \var{limit} stack trace entries from \var{traceback}.  If
\var{limit} is omitted or \code{None}, all entries are printed.
\end{funcdesc}

\begin{funcdesc}{extract_tb}{traceback\optional{, limit}}
Return a list of up to \var{limit} ``pre-processed'' stack trace
entries extracted from \var{traceback}.  It is useful for alternate
formatting of stack traces.  If \var{limit} is omitted or \code{None},
all entries are extracted.  A ``pre-processed'' stack trace entry is a
quadruple (\var{filename}, \var{line number}, \var{function name},
\var{line text}) representing the information that is usually printed
for a stack trace.  The \var{line text} is a string with leading and
trailing whitespace stripped; if the source is not available it is
\code{None}.
\end{funcdesc}

\begin{funcdesc}{print_exception}{type, value, traceback\optional{, limit}}
Print exception information and up to \var{limit} stack trace entries
from \var{traceback}.  This differs from \function{print_tb()} in the
following ways: (1) if \var{traceback} is not \code{None}, it prints a
header \samp{Traceback (innermost last):}; (2) it prints the
exception \var{type} and \var{value} after the stack trace; (3) if
\var{type} is \exception{SyntaxError} and \var{value} has the appropriate
format, it prints the line where the syntax error occurred with a
caret indicating the approximate position of the error.
\end{funcdesc}

\begin{funcdesc}{print_exc}{\optional{limit}}
This is a shorthand for `\code{print_exception(sys.exc_type,}
\code{sys.exc_value,} \code{sys.exc_traceback,} \var{limit}\code{)}'.
\end{funcdesc}

\begin{funcdesc}{print_last}{\optional{limit}}
This is a shorthand for `\code{print_exception(sys.last_type,}
\code{sys.last_value,} \code{sys.last_traceback,} \var{limit}\code{)}'.
\end{funcdesc}
