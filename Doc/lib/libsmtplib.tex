% Documentation by ESR
\section{Standard Module \module{smtp}}
\stmodindex{smtp}
\label{module-smtp}

The \code{smtp} module defines an SMTP session object that can be used
to send mail to any Internet machine with an SMTP or ESMTP listener daemon.
For details of SMTP and ESMTP operation, consult RFC 821 (Simple Mail
Transfer Protocol) and RFC1869 (SMTP Service Extensions).

\begin{classdesc}{SMTP}{\optional{host, port}}
A \class{SMTP} instance encapsulates an SMTP connection.  It has
methods that support a full repertoire of SMTP and ESMTP
operations. If the optional host and port parameters are given, the
SMTP connect method is called with those parameters during
initialization.

For normal use, you should only require the initialization/connect,
\var{sendmail}, and \var{quit} methods  An example is included below.
\end{classdesc}

\subsection{SMTP Objects}
\label{SMTP-objects}

A \class{SMTP} instance has the following methods:

\begin{methoddesc}{set_debuglevel}{level}
Set the debug output level.  A non-false value results in debug
messages for connection and for all messages sent to and received from
the server.
\end{methoddesc}

\begin{methoddesc}{connect}{\optional{host='localhost',port=0}}
Connect to a host on a given port.

If the hostname ends with a colon (`:') followed by a number,
that suffix will be stripped off and the number interpreted as
the port number to use.

Note:  This method is automatically invoked by __init__,
if a host is specified during instantiation.
\end{methoddesc}

\begin{methoddesc}{docmd}{cmd, \optional{, argstring}}
Send a command to the server.  The optional argument
string is simply concatenated to the command.

Get back a 2-tuple composed of a numeric response code and the actual
response line (multiline responses are joined into one long line.)

In normal operation it should not be necessary to call this method
explicitly.  It is used to implement other methods and may be useful
for testing private extensions.
\end{methoddesc}

\begin{methoddesc}{helo}{\optional{hostname}}
Identify yourself to the SMTP server using HELO.  The hostname
argument defaults to the FQDN of the local host.

In normal operation it should not be necessary to call this method
explicitly.  It will be implicitly called by the \var{sendmail} method
when necessary.
\end{methoddesc}

\begin{methoddesc}{ehlo}{\optional{hostname}}
Identify yourself to an ESMTP server using HELO.  The hostname
argument defaults to the FQDN of the local host.  Examine the 
response for ESMTP option and store them for use by the
\var{has_option} method.

Unless you wish to use the \var{has_option} method before sending
mail, it should not be necessary to call this method explicitly.  It
will be implicitly called by the \var{sendmail} method when necessary.
\end{methoddesc}

\begin{methoddesc}{has_option}{name}
Return 1 if name is in the set of ESMTP options returned by the
server, 0 otherwise.  Case is ignored.
\end{methoddesc}

\begin{methoddesc}{verify}{address}
Check the validity of an address on this server using SMTP VRFY.
Returns a tuple consisting of code 250 and a full RFC822 address
(including human name) if the user address is valid. Otherwise returns
an SMTP error code of 400 or greater and an error string.

Note: many sites disable SMTP VRFY in order to foil spammers.
\end{methoddesc}

\begin{methoddesc}{sendmail}{from_addr, to_addrs, msg\optional{, options=[]}}
Send mail.  The required arguments are an RFC822 from-address string,
a list of RFC822 to-address strings, and a message string.  The caller
may pass a list of ESMTP options to be used in MAIL FROM commands.

If there has been no previous EHLO or HELO command this session, this
method tries ESMTP EHLO first. If the server does ESMTP, message size
and each of the specified options will be passed to it (if the option
is in the feature set the server advertises).  If EHLO fails, HELO
will be tried and ESMTP options suppressed.

This method will return normally if the mail is accepted for at least 
one recipient. Otherwise it will throw an exception (either
SMTPSenderRefused, SMTPRecipientsRefused, or SMTPDataError)
That is, if this method does not throw an exception, then someone 
should get your mail.  If this method does not throw an exception,
it returns a dictionary, with one entry for each recipient that was 
refused. 
\end{methoddesc}

\begin{methoddesc}{quit}{}
Terminate the SMTP session and close the connection.
\end{methoddesc}

Low-level methods corresponding to the standard SMTP/ESMTP commands
HELP, RSET, NOOP, MAIL, RCPT, and DATA are also supported.  Normally
these do not need to be called directly, so they are not documented
here. For details, consult the module code.

Example:

\begin{verbatim}
    import sys, rfc822

    def prompt(prompt):
        sys.stdout.write(prompt + ": ")
        return string.strip(sys.stdin.readline())

    fromaddr = prompt("From")
    toaddrs  = string.splitfields(prompt("To"), ',')
    print "Enter message, end with ^D:"
    msg = ''
    while 1:
        line = sys.stdin.readline()
        if not line:
            break
        msg = msg + line
    print "Message length is " + `len(msg)`

    server = SMTP('localhost')
    server.set_debuglevel(1)
    server.sendmail(fromaddr, toaddrs, msg)
    server.quit()
\end{verbatim}

