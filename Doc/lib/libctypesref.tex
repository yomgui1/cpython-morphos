\subsection{ctypes reference\label{ctypes-reference}}

ctypes defines a lot of C compatible datatypes, and also allows to
define your own types.  Among other things, a ctypes type instance
holds a memory block that contains C compatible data.

\begin{funcdesc}{addressof}{obj}
Returns the address of the memory buffer as integer.  \var{obj} must
be an instance of a ctypes type.
\end{funcdesc}

\begin{funcdesc}{alignment}{obj_or_type}
Returns the alignment requirements of a ctypes type.
\var{obj_or_type} must be a ctypes type or an instance.
\end{funcdesc}

\begin{excclassdesc}{ArgumentError}{}
\end{excclassdesc}

\begin{classdesc}{BigEndianStructure}{}
\end{classdesc}

\begin{funcdesc}{byref}{obj}
Returns a light-weight pointer to \var{obj}, which must be an instance
of a ctypes type.  The returned object can only be used as a foreign
function call parameter.  It behaves similar to \code{pointer(obj)},
but the construction is a lot faster.
\end{funcdesc}

\begin{classdesc}{c_byte}{\optional{value}}
Represents a C \code{signed char} datatype, and interprets the value
as small integer.  The constructor accepts an optional integer
initializer; no overflow checking is done.
\end{classdesc}

\begin{classdesc}{c_char}{\optional{value}}
Represents a C \code{char} datatype, and interprets the value as a
single character.  The constructor accepts an optional string
initializer, the length of the string must be exactly one character.
\end{classdesc}

\begin{classdesc}{c_char_p}{\optional{value}}
\end{classdesc}

\begin{classdesc}{c_double}{\optional{value}}
Represents a C \code{double} datatype.  The constructor accepts an
optional float initializer.
\end{classdesc}

\begin{classdesc}{c_float}{\optional{value}}
Represents a C \code{double} datatype.  The constructor accepts an
optional float initializer.
\end{classdesc}

\begin{classdesc}{c_int}{\optional{value}}
Represents a C \code{signed int} datatype.  The constructor accepts an
optional integer initializer; no overflow checking is done.  On
platforms where \code{sizeof(int) == sizeof(long)} \var{c_int} is an
alias to \var{c_long}.
\end{classdesc}

\begin{classdesc}{c_int16}{\optional{value}}
Represents a C 16-bit \code{signed int} datatype.  Usually an alias
for \var{c_short}.
\end{classdesc}

\begin{classdesc}{c_int32}{\optional{value}}
Represents a C 32-bit \code{signed int} datatype.  Usually an alias
for \code{c_int}.
\end{classdesc}

\begin{classdesc}{c_int64}{\optional{value}}
Represents a C 64-bit \code{signed int} datatype.  Usually an alias
for \code{c_longlong}.
\end{classdesc}

\begin{classdesc}{c_int8}{\optional{value}}
Represents a C 8-bit \code{signed int} datatype.  Usually an alias for \code{c_byte}.
\end{classdesc}

\begin{classdesc}{c_long}{\optional{value}}
Represents a C \code{signed long} datatype.  The constructor accepts
an optional integer initializer; no overflow checking is done.
\end{classdesc}

\begin{classdesc}{c_longlong}{\optional{value}}
Represents a C \code{signed long long} datatype.  The constructor
accepts an optional integer initializer; no overflow checking is done.
\end{classdesc}

\begin{classdesc}{c_short}{\optional{value}}
Represents a C \code{signed short} datatype.  The constructor accepts
an optional integer initializer; no overflow checking is done.
\end{classdesc}

\begin{classdesc}{c_size_t}{\optional{value}}
Represents a C \code{size_t} datatype.
\end{classdesc}

\begin{classdesc}{c_ubyte}{\optional{value}}
\end{classdesc}

\begin{classdesc}{c_uint}{\optional{value}}
\end{classdesc}

\begin{classdesc}{c_uint16}{\optional{value}}
\end{classdesc}

\begin{classdesc}{c_uint32}{\optional{value}}
\end{classdesc}

\begin{classdesc}{c_uint64}{\optional{value}}
\end{classdesc}

\begin{classdesc}{c_uint8}{\optional{value}}
\end{classdesc}

\begin{classdesc}{c_ulong}{\optional{value}}
\end{classdesc}

\begin{classdesc}{c_ulonglong}{\optional{value}}
\end{classdesc}

\begin{classdesc}{c_ushort}{\optional{value}}
\end{classdesc}

\begin{classdesc}{c_void_p}{\optional{value}}
\end{classdesc}

\begin{classdesc}{c_wchar}{\optional{value}}
\end{classdesc}

\begin{classdesc}{c_wchar_p}{\optional{value}}
\end{classdesc}

\begin{funcdesc}{cast}{obj, type}
\end{funcdesc}

\begin{classdesc}{CDLL}{name, mode=RTLD_LOCAL, handle=None}
\end{classdesc}

\begin{datadesc}{cdll}
\end{datadesc}

\begin{funcdesc}{CFUNCTYPE}{restype, *argtypes}
\end{funcdesc}

\begin{funcdesc}{create_string_buffer}{init\optional{, size}}
\end{funcdesc}

\begin{funcdesc}{create_unicode_buffer}{init\optional{, size}}
\end{funcdesc}

\begin{funcdesc}{DllCanUnloadNow}{}
\end{funcdesc}

\begin{funcdesc}{DllGetClassObject}{}
\end{funcdesc}

\begin{funcdesc}{FormatError}{}
\end{funcdesc}

\begin{funcdesc}{GetLastError}{}
\end{funcdesc}

\begin{classdesc}{HRESULT}{}
\end{classdesc}

\begin{classdesc}{LibraryLoader}{dlltype}
\end{classdesc}

\begin{classdesc}{LittleEndianStructure}{}
\end{classdesc}

\begin{funcdesc}{memmove}{dst, src, count}
\end{funcdesc}

\begin{funcdesc}{memset}{dst, c, count}
\end{funcdesc}

\begin{classdesc}{OleDLL}{name, mode=RTLD_LOCAL, handle=None}
\end{classdesc}

\begin{datadesc}{oledll}
\end{datadesc}

\begin{funcdesc}{POINTER}{}
\end{funcdesc}

\begin{funcdesc}{pointer}{}
\end{funcdesc}

\begin{classdesc}{py_object}{}
\end{classdesc}

\begin{classdesc}{PyDLL}{name, mode=RTLD_LOCAL, handle=None}
\end{classdesc}

\begin{datadesc}{pydll}{}
\end{datadesc}

\begin{funcdesc}{PYFUNCTYPE}{restype, *argtypes}
\end{funcdesc}

\begin{funcdesc}{pythonapi}{}
\end{funcdesc}

\begin{funcdesc}{resize}{}
\end{funcdesc}

\begin{datadesc}{RTLD_GLOBAL}
\end{datadesc}

\begin{datadesc}{RTLD_LOCAL}
\end{datadesc}

\begin{funcdesc}{set_conversion_mode}{}
\end{funcdesc}

\begin{funcdesc}{sizeof}{}
\end{funcdesc}

\begin{funcdesc}{string_at}{address}
\end{funcdesc}

\begin{classdesc}{Structure}{}
\end{classdesc}

\begin{classdesc}{Union}{}
\end{classdesc}

\begin{classdesc}{WinDLL}{name, mode=RTLD_LOCAL, handle=None}
\end{classdesc}

\begin{datadesc}{windll}
\end{datadesc}

\begin{funcdesc}{WinError}{}
\end{funcdesc}

\begin{funcdesc}{WINFUNCTYPE}{restype, *argtypes}
\end{funcdesc}

\begin{funcdesc}{wstring_at}{address}
\end{funcdesc}

