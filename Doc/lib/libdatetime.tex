% XXX what order should the types be discussed in?

\section{\module{datetime} ---
         Basic date and time types}

\declaremodule{builtin}{datetime}
\modulesynopsis{Basic date and time types.}
\moduleauthor{Tim Peters}{tim@zope.com}
\sectionauthor{Tim Peters}{tim@zope.com}
\sectionauthor{A.M. Kuchling}{amk@amk.ca}

\versionadded{2.3}


The \module{datetime} module supplies classes for manipulating dates
and times in both simple and complex ways.  While date and time
arithmetic is supported, the focus of the implementation is on
efficient member extraction for output formatting and manipulation.

There are two kinds of date and time objects: ``naive'' and ``aware''.
This distinction refers to whether the object has any notion of time
zone, daylight saving time, or other kind of algorithmic or political
time adjustment.  Whether a naive \class{datetime} object represents
Coordinated Universal Time (UTC), local time, or time in some other
timezone is purely up to the program, just like it's up to the program
whether a particular number represents meters, miles, or mass.  Naive
\class{datetime} objects are easy to understand and to work with, at
the cost of ignoring some aspects of reality.

For applications requiring more, \class{datetime} and \class{time}
objects have an optional time zone information member,
\member{tzinfo}, that can contain an instance of a subclass of
the abstract \class{tzinfo} class.  These \class{tzinfo} objects
capture information about the offset from UTC time, the time zone
name, and whether Daylight Saving Time is in effect.  Note that no
concrete \class{tzinfo} classes are supplied by the \module{datetime}
module.  Supporting timezones at whatever level of detail is required
is up to the application.  The rules for time adjustment across the
world are more political than rational, and there is no standard
suitable for every application.

The \module{datetime} module exports the following constants:

\begin{datadesc}{MINYEAR}
  The smallest year number allowed in a \class{date} or
  \class{datetime} object.  \constant{MINYEAR}
  is \code{1}.
\end{datadesc}

\begin{datadesc}{MAXYEAR}
  The largest year number allowed in a \class{date} or \class{datetime}
  object.  \constant{MAXYEAR} is \code{9999}.
\end{datadesc}

\begin{seealso}
  \seemodule{calendar}{General calendar related functions.}
  \seemodule{time}{Time access and conversions.}
\end{seealso}

\subsection{Available Types}

\begin{classdesc*}{date}
  An idealized naive date, assuming the current Gregorian calendar
  always was, and always will be, in effect.
  Attributes: \member{year}, \member{month}, and \member{day}.
\end{classdesc*}

\begin{classdesc*}{time}
  An idealized time, independent of any particular day, assuming
  that every day has exactly 24*60*60 seconds (there is no notion
  of "leap seconds" here).
  Attributes: \member{hour}, \member{minute}, \member{second},
              \member{microsecond}, and \member{tzinfo}.
\end{classdesc*}

\begin{classdesc*}{datetime}
  A combination of a date and a time.
  Attributes: \member{year}, \member{month}, \member{day},
              \member{hour}, \member{minute}, \member{second},
              \member{microsecond}, and \member{tzinfo}.
\end{classdesc*}

\begin{classdesc*}{timedelta}
  A duration expressing the difference between two \class{date},
  \class{time}, or \class{datetime} instances to microsecond
  resolution.
\end{classdesc*}

\begin{classdesc*}{tzinfo}
  An abstract base class for time zone information objects.  These
  are used by the  \class{datetime} and \class{time} classes to
  provide a customizable notion of time adjustment (for example, to
  account for time zone and/or daylight saving time).
\end{classdesc*}

Objects of these types are immutable.

Objects of the \class{date} type are always naive.

An object \var{d} of type \class{time} or \class{datetime} may be
naive or aware.  \var{d} is aware if \code{\var{d}.tzinfo} is not
\code{None} and \code{\var{d}.tzinfo.utcoffset(\var{d})} does not return
\code{None}.  If \code{\var{d}.tzinfo} is \code{None}, or if
\code{\var{d}.tzinfo} is not \code{None} but
\code{\var{d}.tzinfo.utcoffset(\var{d})} returns \code{None}, \var{d}
is naive.

The distinction between naive and aware doesn't apply to
\code{timedelta} objects.

Subclass relationships:

\begin{verbatim}
object
    timedelta
    tzinfo
    time
    date
        datetime
\end{verbatim}

\subsection{\class{timedelta} Objects \label{datetime-timedelta}}

A \class{timedelta} object represents a duration, the difference
between two dates or times.

\begin{classdesc}{timedelta}{days=0, seconds=0, microseconds=0,
                             milliseconds=0, minutes=0, hours=0, weeks=0}

  All arguments are optional.  Arguments may be ints, longs, or floats,
  and may be positive or negative.

  Only \var{days}, \var{seconds} and \var{microseconds} are stored
  internally.  Arguments are converted to those units:

\begin{itemize}
  \item A millisecond is converted to 1000 microseconds.
  \item A minute is converted to 60 seconds.
  \item An hour is converted to 3600 seconds.
  \item A week is converted to 7 days.
\end{itemize}

  and days, seconds and microseconds are then normalized so that the
  representation is unique, with

\begin{itemize}
  \item \code{0 <= \var{microseconds} < 1000000}
  \item \code{0 <= \var{seconds} < 3600*24} (the number of seconds in one day)
  \item \code{-999999999 <= \var{days} <= 999999999}
\end{itemize}

  If any argument is a float and there are fractional microseconds,
  the fractional microseconds left over from all arguments are combined
  and their sum is rounded to the nearest microsecond.  If no
  argument is a float, the conversion and normalization processes
  are exact (no information is lost).

  If the normalized value of days lies outside the indicated range,
  \exception{OverflowError} is raised.

  Note that normalization of negative values may be surprising at first.
  For example,

\begin{verbatim}
>>> d = timedelta(microseconds=-1)
>>> (d.days, d.seconds, d.microseconds)
(-1, 86399, 999999)
\end{verbatim}
\end{classdesc}

Class attributes are:

\begin{memberdesc}{min}
  The most negative \class{timedelta} object,
  \code{timedelta(-999999999)}.
\end{memberdesc}

\begin{memberdesc}{max}
  The most positive \class{timedelta} object,
  \code{timedelta(days=999999999, hours=23, minutes=59, seconds=59,
                  microseconds=999999)}.
\end{memberdesc}

\begin{memberdesc}{resolution}
  The smallest possible difference between non-equal
  \class{timedelta} objects, \code{timedelta(microseconds=1)}.
\end{memberdesc}

Note that, because of normalization, \code{timedelta.max} \textgreater
\code{-timedelta.min}.  \code{-timedelta.max} is not representable as
a \class{timedelta} object.

Instance attributes (read-only):

\begin{tableii}{c|l}{code}{Attribute}{Value}
  \lineii{days}{Between -999999999 and 999999999 inclusive}
  \lineii{seconds}{Between 0 and 86399 inclusive}
  \lineii{microseconds}{Between 0 and 999999 inclusive}
\end{tableii}

Supported operations:

% XXX this table is too wide!
\begin{tableii}{c|l}{code}{Operation}{Result}
  \lineii{\var{t1} = \var{t2} + \var{t3}}
          {Sum of \var{t2} and \var{t3}.
           Afterwards \var{t1}-\var{t2} == \var{t3} and \var{t1}-\var{t3}
           == \var{t2} are true.
          (1)}
  \lineii{\var{t1} = \var{t2} - \var{t3}}
          {Difference of \var{t2} and \var{t3}.
           Afterwards \var{t1} == \var{t2} - \var{t3} and \var{t2} == \var{t1} + \var{t3} are
           true.
          (1)}
  \lineii{\var{t1} = \var{t2} * \var{i} or \var{t1} = \var{i} * \var{t2}}
          {Delta multiplied by an integer or long.
           Afterwards \var{t1} // i == \var{t2} is true,
           provided \code{i != 0}.}
  \lineii{}{In general, \var{t1} * i == \var{t1} * (i-1) + \var{t1} is true.
          (1)}
  \lineii{\var{t1} = \var{t2} // \var{i}}
          {The floor is computed and the remainder (if any) is thrown away.
          (3)}
  \lineii{+\var{t1}}
          {Returns a \class{timedelta} object with the same value.
          (2)}
  \lineii{-\var{t1}}
          {equivalent to \class{timedelta}(-\var{t1.days}, -\var{t1.seconds},
           -\var{t1.microseconds}), and to \var{t1}* -1.
          (1)(4)}
  \lineii{abs(\var{t})}
          {equivalent to +\var{t} when \code{t.days >= 0}, and to
           -\var{t} when \code{t.days < 0}.
          (2)}
\end{tableii}
\noindent
Notes:

\begin{description}
\item[(1)]
  This is exact, but may overflow.

\item[(2)]
  This is exact, and cannot overflow.

\item[(3)]
  Division by 0 raises \exception{ZeroDivisionError}.

\item[(4)]
  -\var{timedelta.max} is not representable as a \class{timedelta} object.
\end{description}

In addition to the operations listed above \class{timedelta} objects
support certain additions and subtractions with \class{date} and
\class{datetime} objects (see below).

Comparisons of \class{timedelta} objects are supported with the
\class{timedelta} object representing the smaller duration considered
to be the smaller timedelta.
In order to stop mixed-type comparisons from falling back to the
default comparison by object address, when a \class{timedelta} object is
compared to an object of a different type, \exception{TypeError} is
raised unless the comparison is \code{==} or \code{!=}.  The latter
cases return \constant{False} or \constant{True}, respectively.

\class{timedelta} objects are hashable (usable as dictionary keys),
support efficient pickling, and in Boolean contexts, a \class{timedelta}
object is considered to be true if and only if it isn't equal to
\code{timedelta(0)}.


\subsection{\class{date} Objects \label{datetime-date}}

A \class{date} object represents a date (year, month and day) in an idealized
calendar, the current Gregorian calendar indefinitely extended in both
directions.  January 1 of year 1 is called day number 1, January 2 of year
1 is called day number 2, and so on.  This matches the definition of the
"proleptic Gregorian" calendar in Dershowitz and Reingold's book
\citetitle{Calendrical Calculations}, where it's the base calendar for all
computations.  See the book for algorithms for converting between
proleptic Gregorian ordinals and many other calendar systems.

\begin{classdesc}{date}{year, month, day}
  All arguments are required.  Arguments may be ints or longs, in the
  following ranges:

  \begin{itemize}
    \item \code{MINYEAR <= \var{year} <= MAXYEAR}
    \item \code{1 <= \var{month} <= 12}
    \item \code{1 <= \var{day} <= number of days in the given month and year}
  \end{itemize}

  If an argument outside those ranges is given, \exception{ValueError}
  is raised.
\end{classdesc}

Other constructors, all class methods:

\begin{methoddesc}{today}{}
  Return the current local date.  This is equivalent to
  \code{date.fromtimestamp(time.time())}.
\end{methoddesc}

\begin{methoddesc}{fromtimestamp}{timestamp}
  Return the local date corresponding to the POSIX timestamp, such
  as is returned by \function{time.time()}.  This may raise
  \exception{ValueError}, if the timestamp is out of the range of
  values supported by the platform C \cfunction{localtime()}
  function.  It's common for this to be restricted to years from 1970
  through 2038.  Note that on non-POSIX systems that include leap
  seconds in their notion of a timestamp, leap seconds are ignored by
  \method{fromtimestamp()}.
\end{methoddesc}

\begin{methoddesc}{fromordinal}{ordinal}
  Return the date corresponding to the proleptic Gregorian ordinal,
  where January 1 of year 1 has ordinal 1.  \exception{ValueError} is
  raised unless \code{1 <= \var{ordinal} <= date.max.toordinal()}.
  For any date \var{d}, \code{date.fromordinal(\var{d}.toordinal()) ==
  \var{d}}.
\end{methoddesc}

Class attributes:

\begin{memberdesc}{min}
  The earliest representable date, \code{date(MINYEAR, 1, 1)}.
\end{memberdesc}

\begin{memberdesc}{max}
  The latest representable date, \code{date(MAXYEAR, 12, 31)}.
\end{memberdesc}

\begin{memberdesc}{resolution}
  The smallest possible difference between non-equal date
  objects, \code{timedelta(days=1)}.
\end{memberdesc}

Instance attributes (read-only):

\begin{memberdesc}{year}
  Between \constant{MINYEAR} and \constant{MAXYEAR} inclusive.
\end{memberdesc}

\begin{memberdesc}{month}
  Between 1 and 12 inclusive.
\end{memberdesc}

\begin{memberdesc}{day}
  Between 1 and the number of days in the given month of the given
  year.
\end{memberdesc}

Supported operations:

% XXX rewrite to be a table
\begin{tableii}{c|l}{code}{Operation}{Result}
  \lineii{\var{date2} = \var{date1} + \var{timedelta}}
    {\var{date2} is \code{\var{timedelta}.days} days removed from
    \var{date1}.  (1)}


  \lineii{\var{date2} = \var{date1} - \var{timedelta}}
   {Computes \var{date2} such that \code{\var{date2} + \var{timedelta}
   == \var{date1}}. (2)}

  \lineii{\var{timedelta} = \var{date1} - \var{date2}}
   {(3)}

  \lineii{\var{date1}<\var{date2}}
   {\var{date1} is considered less than \var{date2} when \var{date1}
   precedes \var{date2} in time. (4)}

\end{tableii}

Notes:
\begin{description}

\item[(1)]
 \var{date2} is moved forward in time if \code{\var{timedelta}.days
    > 0}, or backward if \code{\var{timedelta}.days < 0}.  Afterward
    \code{\var{date2} - \var{date1} == \var{timedelta}.days}.
    \code{\var{timedelta}.seconds} and
    \code{\var{timedelta}.microseconds} are ignored.
    \exception{OverflowError} is raised if \code{\var{date2}.year}
    would be smaller than \constant{MINYEAR} or larger than
    \constant{MAXYEAR}.

\item[(2)]
 This isn't quite equivalent to date1 +
   (-timedelta), because -timedelta in isolation can overflow in cases
   where date1 - timedelta does not.  \code{\var{timedelta}.seconds}
   and \code{\var{timedelta}.microseconds} are ignored.

\item[(3)]
This is exact, and cannot overflow.  timedelta.seconds and
    timedelta.microseconds are 0, and date2 + timedelta == date1
    after.

\item[(4)]
In other words, \code{date1 < date2}
   if and only if \code{\var{date1}.toordinal() <
   \var{date2}.toordinal()}. 
In order to stop comparison from falling back to the default
scheme of comparing object addresses, date comparison
normally raises \exception{TypeError} if the other comparand
isn't also a \class{date} object.  However, \code{NotImplemented}
is returned instead if the other comparand has a
\method{timetuple} attribute.  This hook gives other kinds of
date objects a chance at implementing mixed-type comparison.
If not, when a \class{date} object is
compared to an object of a different type, \exception{TypeError} is
raised unless the comparison is \code{==} or \code{!=}.  The latter
cases return \constant{False} or \constant{True}, respectively.

\end{description}


Dates can be used as dictionary keys. In Boolean contexts, all
\class{date} objects are considered to be true.

Instance methods:

\begin{methoddesc}{replace}{year, month, day}
  Return a date with the same value, except for those members given
  new values by whichever keyword arguments are specified.  For
  example, if \code{d == date(2002, 12, 31)}, then
  \code{d.replace(day=26) == date(2000, 12, 26)}.
\end{methoddesc}

\begin{methoddesc}{timetuple}{}
  Return a \class{time.struct_time} such as returned by
  \function{time.localtime()}.  The hours, minutes and seconds are
  0, and the DST flag is -1.
  \code{\var{d}.timetuple()} is equivalent to
      \code{time.struct_time((\var{d}.year, \var{d}.month, \var{d}.day,
             0, 0, 0, 
             \var{d}.weekday(), 
             \var{d}.toordinal() - date(\var{d}.year, 1, 1).toordinal() + 1,
            -1))}
\end{methoddesc}

\begin{methoddesc}{toordinal}{}
  Return the proleptic Gregorian ordinal of the date, where January 1
  of year 1 has ordinal 1.  For any \class{date} object \var{d},
  \code{date.fromordinal(\var{d}.toordinal()) == \var{d}}.
\end{methoddesc}

\begin{methoddesc}{weekday}{}
  Return the day of the week as an integer, where Monday is 0 and
  Sunday is 6.  For example, \code{date(2002, 12, 4).weekday() == 2}, a
  Wednesday.
  See also \method{isoweekday()}.
\end{methoddesc}

\begin{methoddesc}{isoweekday}{}
  Return the day of the week as an integer, where Monday is 1 and
  Sunday is 7.  For example, \code{date(2002, 12, 4).isoweekday() == 3}, a
  Wednesday.
  See also \method{weekday()}, \method{isocalendar()}.
\end{methoddesc}

\begin{methoddesc}{isocalendar}{}
  Return a 3-tuple, (ISO year, ISO week number, ISO weekday).

  The ISO calendar is a widely used variant of the Gregorian calendar.
  See \url{http://www.phys.uu.nl/~vgent/calendar/isocalendar.htm}
  for a good explanation.

  The ISO year consists of 52 or 53 full weeks, and where a week starts
  on a Monday and ends on a Sunday.  The first week of an ISO year is
  the first (Gregorian) calendar week of a year containing a Thursday.
  This is called week number 1, and the ISO year of that Thursday is
  the same as its Gregorian year.

  For example, 2004 begins on a Thursday, so the first week of ISO
  year 2004 begins on Monday, 29 Dec 2003 and ends on Sunday, 4 Jan
  2004, so that
  \code{date(2003, 12, 29).isocalendar() == (2004, 1, 1)}
  and 
  \code{date(2004, 1, 4).isocalendar() == (2004, 1, 7)}.
\end{methoddesc}

\begin{methoddesc}{isoformat}{}
  Return a string representing the date in ISO 8601 format,
  'YYYY-MM-DD'.  For example,
  \code{date(2002, 12, 4).isoformat() == '2002-12-04'}.
\end{methoddesc}

\begin{methoddesc}{__str__}{}
  For a date \var{d}, \code{str(\var{d})} is equivalent to
  \code{\var{d}.isoformat()}.
\end{methoddesc}

\begin{methoddesc}{ctime}{}
  Return a string representing the date, for example
  date(2002, 12, 4).ctime() == 'Wed Dec  4 00:00:00 2002'.
  \code{\var{d}.ctime()} is equivalent to
  \code{time.ctime(time.mktime(\var{d}.timetuple()))}
  on platforms where the native C \cfunction{ctime()} function
  (which \function{time.ctime()} invokes, but which
  \method{date.ctime()} does not invoke) conforms to the C standard.
\end{methoddesc}

\begin{methoddesc}{strftime}{format}
  Return a string representing the date, controlled by an explicit
  format string.  Format codes referring to hours, minutes or seconds
  will see 0 values.
  See the section on \method{strftime()} behavior.
\end{methoddesc}


\subsection{\class{datetime} Objects \label{datetime-datetime}}

A \class{datetime} object is a single object containing all the
information from a \class{date} object and a \class{time} object.  Like a
\class{date} object, \class{datetime} assumes the current Gregorian
calendar extended in both directions; like a time object,
\class{datetime} assumes there are exactly 3600*24 seconds in every
day.

Constructor:

\begin{classdesc}{datetime}{year, month, day,
                            hour=0, minute=0, second=0, microsecond=0,
                            tzinfo=None}
  The year, month and day arguments are required.  \var{tzinfo} may
  be \code{None}, or an instance of a \class{tzinfo} subclass.  The
  remaining arguments may be ints or longs, in the following ranges:

  \begin{itemize}
    \item \code{MINYEAR <= \var{year} <= MAXYEAR}
    \item \code{1 <= \var{month} <= 12}
    \item \code{1 <= \var{day} <= number of days in the given month and year}
    \item \code{0 <= \var{hour} < 24}
    \item \code{0 <= \var{minute} < 60}
    \item \code{0 <= \var{second} < 60}
    \item \code{0 <= \var{microsecond} < 1000000}
  \end{itemize}

  If an argument outside those ranges is given,
  \exception{ValueError} is raised.
\end{classdesc}

Other constructors, all class methods:

\begin{methoddesc}{today}{}
  Return the current local datetime, with \member{tzinfo} \code{None}.
  This is equivalent to
  \code{datetime.fromtimestamp(time.time())}.
  See also \method{now()}, \method{fromtimestamp()}.
\end{methoddesc}

\begin{methoddesc}{now(tz=None)}{}
  Return the current local date and time.  If optional argument
  \var{tz} is \code{None} or not specified, this is like
  \method{today()}, but, if possible, supplies more precision than can
  be gotten from going through a \function{time.time()} timestamp (for
  example, this may be possible on platforms supplying the C
  \cfunction{gettimeofday()} function).

  Else \var{tz} must be an instance of a class \class{tzinfo} subclass,
  and the current date and time are converted to \var{tz}'s time
  zone.  In this case the result is equivalent to
  \code{\var{tz}.fromutc(datetime.utcnow().replace(tzinfo=\var{tz}))}.
  See also \method{today()}, \method{utcnow()}.
\end{methoddesc}

\begin{methoddesc}{utcnow}{}
  Return the current UTC date and time, with \member{tzinfo} \code{None}.
  This is like \method{now()}, but returns the current UTC date and time,
  as a naive \class{datetime} object.
  See also \method{now()}.
\end{methoddesc}

\begin{methoddesc}{fromtimestamp}{timestamp, tz=None}
  Return the local date and time corresponding to the \POSIX{}
  timestamp, such as is returned by \function{time.time()}.
  If optional argument \var{tz} is \code{None} or not specified, the
  timestamp is converted to the platform's local date and time, and
  the returned \class{datetime} object is naive.

  Else \var{tz} must be an instance of a class \class{tzinfo} subclass,
  and the timestamp is converted to \var{tz}'s time zone.  In this case
  the result is equivalent to
  \code{\var{tz}.fromutc(datetime.utcfromtimestamp(\var{timestamp}).replace(tzinfo=\var{tz}))}.

  \method{fromtimestamp()} may raise \exception{ValueError}, if the
  timestamp is out of the range of values supported by the platform C
  \cfunction{localtime()} or \cfunction{gmtime()} functions.  It's common
  for this to be restricted to years in 1970 through 2038.
  Note that on non-POSIX systems that include leap seconds in their
  notion of a timestamp, leap seconds are ignored by
  \method{fromtimestamp()}, and then it's possible to have two timestamps
  differing by a second that yield identical \class{datetime} objects.
  See also \method{utcfromtimestamp()}.
\end{methoddesc}

\begin{methoddesc}{utcfromtimestamp}{timestamp}
  Return the UTC \class{datetime} corresponding to the \POSIX{}
  timestamp, with \member{tzinfo} \code{None}.
  This may raise \exception{ValueError}, if the
  timestamp is out of the range of values supported by the platform
  C \cfunction{gmtime()} function.  It's common for this to be
  restricted to years in 1970 through 2038.
  See also \method{fromtimestamp()}.
\end{methoddesc}

\begin{methoddesc}{fromordinal}{ordinal}
  Return the \class{datetime} corresponding to the proleptic
  Gregorian ordinal, where January 1 of year 1 has ordinal 1.
  \exception{ValueError} is raised unless \code{1 <= ordinal <=
  datetime.max.toordinal()}.  The hour, minute, second and
  microsecond of the result are all 0,
  and \member{tzinfo} is \code{None}.
\end{methoddesc}

\begin{methoddesc}{combine}{date, time}
  Return a new \class{datetime} object whose date members are
  equal to the given \class{date} object's, and whose time
  and \member{tzinfo} members are equal to the given \class{time} object's.
  For any \class{datetime} object \var{d}, \code{\var{d} ==
  datetime.combine(\var{d}.date(), \var{d}.timetz())}.  If date is a
  \class{datetime} object, its time and \member{tzinfo} members are
  ignored.
  \end{methoddesc}

Class attributes:

\begin{memberdesc}{min}
  The earliest representable \class{datetime},
  \code{datetime(MINYEAR, 1, 1, tzinfo=None)}.
\end{memberdesc}

\begin{memberdesc}{max}
  The latest representable \class{datetime},
  \code{datetime(MAXYEAR, 12, 31, 23, 59, 59, 999999, tzinfo=None)}.
\end{memberdesc}

\begin{memberdesc}{resolution}
  The smallest possible difference between non-equal \class{datetime}
  objects, \code{timedelta(microseconds=1)}.
\end{memberdesc}

Instance attributes (read-only):

\begin{memberdesc}{year}
  Between \constant{MINYEAR} and \constant{MAXYEAR} inclusive.
\end{memberdesc}

\begin{memberdesc}{month}
  Between 1 and 12 inclusive.
\end{memberdesc}

\begin{memberdesc}{day}
  Between 1 and the number of days in the given month of the given
  year.
\end{memberdesc}

\begin{memberdesc}{hour}
  In \code{range(24)}.
\end{memberdesc}

\begin{memberdesc}{minute}
  In \code{range(60)}.
\end{memberdesc}

\begin{memberdesc}{second}
  In \code{range(60)}.
\end{memberdesc}

\begin{memberdesc}{microsecond}
  In \code{range(1000000)}.
\end{memberdesc}

\begin{memberdesc}{tzinfo}
  The object passed as the \var{tzinfo} argument to the
  \class{datetime} constructor, or \code{None} if none was passed.
\end{memberdesc}

Supported operations:

\begin{tableii}{c|l}{code}{Operation}{Result}
  \lineii{\var{datetime2} = \var{datetime1} + \var{timedelta}}{(1)}

  \lineii{\var{datetime2} = \var{datetime1} - \var{timedelta}}{(2)}

  \lineii{\var{timedelta} = \var{datetime1} - \var{datetime2}}{(3)}

  \lineii{\var{datetime1} < \var{datetime2}}
   {Compares \class{datetime} to \class{datetime}. 
    (4)}

\end{tableii}

\begin{description}

\item[(1)]

    datetime2 is a duration of timedelta removed from datetime1, moving
    forward in time if \code{\var{timedelta}.days} > 0, or backward if
    \code{\var{timedelta}.days} < 0.  The result has the same \member{tzinfo} member
    as the input datetime, and datetime2 - datetime1 == timedelta after.
    \exception{OverflowError} is raised if datetime2.year would be
    smaller than \constant{MINYEAR} or larger than \constant{MAXYEAR}.
    Note that no time zone adjustments are done even if the input is an
    aware object.

\item[(2)]
    Computes the datetime2 such that datetime2 + timedelta == datetime1.
    As for addition, the result has the same \member{tzinfo} member
    as the input datetime, and no time zone adjustments are done even
    if the input is aware.
    This isn't quite equivalent to datetime1 + (-timedelta), because
    -timedelta in isolation can overflow in cases where
    datetime1 - timedelta does not.

\item[(3)]
    Subtraction of a \class{datetime} from a
    \class{datetime} is defined only if both
    operands are naive, or if both are aware.  If one is aware and the
    other is naive, \exception{TypeError} is raised.

    If both are naive, or both are aware and have the same \member{tzinfo}
    member, the \member{tzinfo} members are ignored, and the result is
    a \class{timedelta} object \var{t} such that
    \code{\var{datetime2} + \var{t} == \var{datetime1}}.  No time zone
    adjustments are done in this case.

    If both are aware and have different \member{tzinfo} members,
    \code{a-b} acts as if \var{a} and \var{b} were first converted to
    naive UTC datetimes first.  The result is
    \code{(\var{a}.replace(tzinfo=None) - \var{a}.utcoffset()) -
          (\var{b}.replace(tzinfo=None) - \var{b}.utcoffset())}
    except that the implementation never overflows.

\item[(4)]

\var{datetime1} is considered less than \var{datetime2}
when \var{datetime1} precedes \var{datetime2} in time.  

If one comparand is naive and
the other is aware, \exception{TypeError} is raised.  If both
    comparands are aware, and have the same \member{tzinfo} member,
    the common \member{tzinfo} member is ignored and the base datetimes
    are compared.  If both comparands are aware and have different
    \member{tzinfo} members, the comparands are first adjusted by
    subtracting their UTC offsets (obtained from \code{self.utcoffset()}).
    \note{In order to stop comparison from falling back to the default
          scheme of comparing object addresses, datetime comparison
          normally raises \exception{TypeError} if the other comparand
          isn't also a \class{datetime} object.  However,
          \code{NotImplemented} is returned instead if the other comparand
          has a \method{timetuple} attribute.  This hook gives other
          kinds of date objects a chance at implementing mixed-type
          comparison.  If not, when a \class{datetime} object is
          compared to an object of a different type, \exception{TypeError}
          is raised unless the comparison is \code{==} or \code{!=}.  The
          latter cases return \constant{False} or \constant{True},
          respectively.}

\end{description}

\class{datetime} objects can be used as dictionary keys. In Boolean
contexts, all \class{datetime} objects are considered to be true.


Instance methods:

\begin{methoddesc}{date}{}
  Return \class{date} object with same year, month and day.
\end{methoddesc}

\begin{methoddesc}{time}{}
  Return \class{time} object with same hour, minute, second and microsecond.
  \member{tzinfo} is \code{None}.  See also method \method{timetz()}.
\end{methoddesc}

\begin{methoddesc}{timetz}{}
  Return \class{time} object with same hour, minute, second, microsecond,
  and tzinfo members.  See also method \method{time()}.
\end{methoddesc}

\begin{methoddesc}{replace}{year=, month=, day=, hour=, minute=, second=,
                            microsecond=, tzinfo=}
  Return a datetime with the same members, except for those members given
  new values by whichever keyword arguments are specified.  Note that
  \code{tzinfo=None} can be specified to create a naive datetime from
  an aware datetime with no conversion of date and time members.
\end{methoddesc}

\begin{methoddesc}{astimezone}{tz}
  Return a \class{datetime} object with new \member{tzinfo} member
  \var{tz}, adjusting the date and time members so the result is the
  same UTC time as \var{self}, but in \var{tz}'s local time.

  \var{tz} must be an instance of a \class{tzinfo} subclass, and its
  \method{utcoffset()} and \method{dst()} methods must not return
  \code{None}.  \var{self} must be aware (\code{\var{self}.tzinfo} must
  not be \code{None}, and \code{\var{self}.utcoffset()} must not return
  \code{None}).

  If \code{\var{self}.tzinfo} is \var{tz},
  \code{\var{self}.astimezone(\var{tz})} is equal to \var{self}:  no
  adjustment of date or time members is performed.
  Else the result is local time in time zone \var{tz}, representing the
  same UTC time as \var{self}:  after \code{\var{astz} =
  \var{dt}.astimezone(\var{tz})},
  \code{\var{astz} - \var{astz}.utcoffset()} will usually have the same
  date and time members as \code{\var{dt} - \var{dt}.utcoffset()}.
  The discussion of class \class{tzinfo} explains the cases at Daylight
  Saving Time transition boundaries where this cannot be achieved (an issue
  only if \var{tz} models both standard and daylight time).

  If you merely want to attach a time zone object \var{tz} to a
  datetime \var{dt} without adjustment of date and time members,
  use \code{\var{dt}.replace(tzinfo=\var{tz})}.  If
  you merely want to remove the time zone object from an aware datetime
  \var{dt} without conversion of date and time members, use
  \code{\var{dt}.replace(tzinfo=None)}.

  Note that the default \method{tzinfo.fromutc()} method can be overridden
  in a \class{tzinfo} subclass to effect the result returned by
  \method{astimezone()}.  Ignoring error cases, \method{astimezone()}
  acts like:

  \begin{verbatim}
  def astimezone(self, tz):
      if self.tzinfo is tz:
          return self
      # Convert self to UTC, and attach the new time zone object.
      utc = (self - self.utcoffset()).replace(tzinfo=tz)
      # Convert from UTC to tz's local time.
      return tz.fromutc(utc)
  \end{verbatim}
\end{methoddesc}

\begin{methoddesc}{utcoffset}{}
  If \member{tzinfo} is \code{None}, returns \code{None}, else
  returns \code{\var{self}.tzinfo.utcoffset(\var{self})}, and
  raises an exception if the latter doesn't return \code{None}, or
  a \class{timedelta} object representing a whole number of minutes
  with magnitude less than one day.
\end{methoddesc}

\begin{methoddesc}{dst}{}
  If \member{tzinfo} is \code{None}, returns \code{None}, else
  returns \code{\var{self}.tzinfo.dst(\var{self})}, and
  raises an exception if the latter doesn't return \code{None}, or
  a \class{timedelta} object representing a whole number of minutes
  with magnitude less than one day.
\end{methoddesc}

\begin{methoddesc}{tzname}{}
  If \member{tzinfo} is \code{None}, returns \code{None}, else
  returns \code{\var{self}.tzinfo.tzname(\var{self})},
  raises an exception if the latter doesn't return \code{None} or
  a string object,
\end{methoddesc}

\begin{methoddesc}{timetuple}{}
  Return a \class{time.struct_time} such as returned by
  \function{time.localtime()}.
  \code{\var{d}.timetuple()} is equivalent to
  \code{time.struct_time((\var{d}.year, \var{d}.month, \var{d}.day,
         \var{d}.hour, \var{d}.minute, \var{d}.second,
         \var{d}.weekday(),
         \var{d}.toordinal() - date(\var{d}.year, 1, 1).toordinal() + 1,
         dst))}
  The \member{tm_isdst} flag of the result is set according to
  the \method{dst()} method:  if \member{tzinfo} is \code{None} or
  \method{dst()} returns \code{None},
  \member{tm_isdst} is set to  \code{-1}; else if \method{dst()} returns
  a non-zero value, \member{tm_isdst} is set to \code{1};
  else \code{tm_isdst} is set to \code{0}.
\end{methoddesc}

\begin{methoddesc}{utctimetuple}{}
  If \class{datetime} instance \var{d} is naive, this is the same as
  \code{\var{d}.timetuple()} except that \member{tm_isdst} is forced to 0
  regardless of what \code{d.dst()} returns.  DST is never in effect
  for a UTC time.

  If \var{d} is aware, \var{d} is normalized to UTC time, by subtracting
  \code{\var{d}.utcoffset()}, and a \class{time.struct_time} for the
  normalized time is returned.  \member{tm_isdst} is forced to 0.
  Note that the result's \member{tm_year} member may be
  \constant{MINYEAR}-1 or \constant{MAXYEAR}+1, if \var{d}.year was
  \code{MINYEAR} or \code{MAXYEAR} and UTC adjustment spills over a
  year boundary.
\end{methoddesc}

\begin{methoddesc}{toordinal}{}
  Return the proleptic Gregorian ordinal of the date.  The same as
  \code{self.date().toordinal()}.
\end{methoddesc}

\begin{methoddesc}{weekday}{}
  Return the day of the week as an integer, where Monday is 0 and
  Sunday is 6.  The same as \code{self.date().weekday()}.
  See also \method{isoweekday()}.
\end{methoddesc}

\begin{methoddesc}{isoweekday}{}
  Return the day of the week as an integer, where Monday is 1 and
  Sunday is 7.  The same as \code{self.date().isoweekday()}.
  See also \method{weekday()}, \method{isocalendar()}.
\end{methoddesc}

\begin{methoddesc}{isocalendar}{}
  Return a 3-tuple, (ISO year, ISO week number, ISO weekday).  The
  same as \code{self.date().isocalendar()}.
\end{methoddesc}

\begin{methoddesc}{isoformat}{sep='T'}
  Return a string representing the date and time in ISO 8601 format,
      YYYY-MM-DDTHH:MM:SS.mmmmmm
  or, if \member{microsecond} is 0,
      YYYY-MM-DDTHH:MM:SS

  If \method{utcoffset()} does not return \code{None}, a 6-character
  string is appended, giving the UTC offset in (signed) hours and
  minutes:
      YYYY-MM-DDTHH:MM:SS.mmmmmm+HH:MM
  or, if \member{microsecond} is 0
      YYYY-MM-DDTHH:MM:SS+HH:MM

  The optional argument \var{sep} (default \code{'T'}) is a
  one-character separator, placed between the date and time portions
  of the result.  For example,

\begin{verbatim}
>>> from datetime import tzinfo, timedelta, datetime
>>> class TZ(tzinfo):
...     def utcoffset(self, dt): return timedelta(minutes=-399)
...
>>> datetime(2002, 12, 25, tzinfo=TZ()).isoformat(' ')
'2002-12-25 00:00:00-06:39'
\end{verbatim}
\end{methoddesc}

\begin{methoddesc}{__str__}{}
  For a \class{datetime} instance \var{d}, \code{str(\var{d})} is
  equivalent to \code{\var{d}.isoformat(' ')}.
\end{methoddesc}

\begin{methoddesc}{ctime}{}
  Return a string representing the date and time, for example
  \code{datetime(2002, 12, 4, 20, 30, 40).ctime() ==
   'Wed Dec  4 20:30:40 2002'}.
  \code{d.ctime()} is equivalent to
  \code{time.ctime(time.mktime(d.timetuple()))} on platforms where
  the native C \cfunction{ctime()} function (which
  \function{time.ctime()} invokes, but which
  \method{datetime.ctime()} does not invoke) conforms to the C
  standard.
\end{methoddesc}

\begin{methoddesc}{strftime}{format}
  Return a string representing the date and time, controlled by an
  explicit format string.  See the section on \method{strftime()}
  behavior.
\end{methoddesc}


\subsection{\class{time} Objects \label{datetime-time}}

A time object represents a (local) time of day, independent of any
particular day, and subject to adjustment via a \class{tzinfo} object.

\begin{classdesc}{time}{hour=0, minute=0, second=0, microsecond=0,
                        tzinfo=None}
  All arguments are optional.  \var{tzinfo} may be \code{None}, or
  an instance of a \class{tzinfo} subclass.  The remaining arguments
  may be ints or longs, in the following ranges:

  \begin{itemize}
    \item \code{0 <= \var{hour} < 24}
    \item \code{0 <= \var{minute} < 60}
    \item \code{0 <= \var{second} < 60}
    \item \code{0 <= \var{microsecond} < 1000000}.
  \end{itemize}

  If an argument outside those ranges is given,
  \exception{ValueError} is raised.
\end{classdesc}

Class attributes:

\begin{memberdesc}{min}
  The earliest representable \class{time}, \code{time(0, 0, 0, 0)}.
\end{memberdesc}

\begin{memberdesc}{max}
  The latest representable \class{time}, \code{time(23, 59, 59, 999999)}.
\end{memberdesc}

\begin{memberdesc}{resolution}
  The smallest possible difference between non-equal \class{time}
  objects, \code{timedelta(microseconds=1)}, although note that
  arithmetic on \class{time} objects is not supported.
\end{memberdesc}

Instance attributes (read-only):

\begin{memberdesc}{hour}
  In \code{range(24)}.
\end{memberdesc}

\begin{memberdesc}{minute}
  In \code{range(60)}.
\end{memberdesc}

\begin{memberdesc}{second}
  In \code{range(60)}.
\end{memberdesc}

\begin{memberdesc}{microsecond}
  In \code{range(1000000)}.
\end{memberdesc}

\begin{memberdesc}{tzinfo}
  The object passed as the tzinfo argument to the \class{time}
  constructor, or \code{None} if none was passed.
\end{memberdesc}

Supported operations:

\begin{itemize}
  \item
    comparison of \class{time} to \class{time},
    where \var{a} is considered less than \var{b} when \var{a} precedes
    \var{b} in time.  If one comparand is naive and the other is aware,
    \exception{TypeError} is raised.  If both comparands are aware, and
    have the same \member{tzinfo} member, the common \member{tzinfo}
    member is ignored and the base times are compared.  If both
    comparands are aware and have different \member{tzinfo} members,
    the comparands are first adjusted by subtracting their UTC offsets
    (obtained from \code{self.utcoffset()}).
    In order to stop mixed-type comparisons from falling back to the
    default comparison by object address, when a \class{time} object is
    compared to an object of a different type, \exception{TypeError} is
    raised unless the comparison is \code{==} or \code{!=}.  The latter
    cases return \constant{False} or \constant{True}, respectively.

  \item
    hash, use as dict key

  \item
    efficient pickling

  \item
    in Boolean contexts, a \class{time} object is considered to be
    true if and only if, after converting it to minutes and
    subtracting \method{utcoffset()} (or \code{0} if that's
    \code{None}), the result is non-zero.
\end{itemize}

Instance methods:

\begin{methoddesc}{replace}(hour=, minute=, second=, microsecond=, tzinfo=)
  Return a \class{time} with the same value, except for those members given
  new values by whichever keyword arguments are specified.  Note that
  \code{tzinfo=None} can be specified to create a naive \class{time} from
  an aware \class{time}, without conversion of the time members.
\end{methoddesc}

\begin{methoddesc}{isoformat}{}
  Return a string representing the time in ISO 8601 format,
      HH:MM:SS.mmmmmm
  or, if self.microsecond is 0,
      HH:MM:SS
  If \method{utcoffset()} does not return \code{None}, a 6-character
  string is appended, giving the UTC offset in (signed) hours and
  minutes:
      HH:MM:SS.mmmmmm+HH:MM
  or, if self.microsecond is 0,
      HH:MM:SS+HH:MM
\end{methoddesc}

\begin{methoddesc}{__str__}{}
  For a time \var{t}, \code{str(\var{t})} is equivalent to
  \code{\var{t}.isoformat()}.
\end{methoddesc}

\begin{methoddesc}{strftime}{format}
  Return a string representing the time, controlled by an explicit
  format string.  See the section on \method{strftime()} behavior.
\end{methoddesc}

\begin{methoddesc}{utcoffset}{}
  If \member{tzinfo} is \code{None}, returns \code{None}, else
  returns \code{\var{self}.tzinfo.utcoffset(None)}, and
  raises an exception if the latter doesn't return \code{None} or
  a \class{timedelta} object representing a whole number of minutes
  with magnitude less than one day.
\end{methoddesc}

\begin{methoddesc}{dst}{}
  If \member{tzinfo} is \code{None}, returns \code{None}, else
  returns \code{\var{self}.tzinfo.dst(None)}, and
  raises an exception if the latter doesn't return \code{None}, or
  a \class{timedelta} object representing a whole number of minutes
  with magnitude less than one day.
\end{methoddesc}

\begin{methoddesc}{tzname}{}
  If \member{tzinfo} is \code{None}, returns \code{None}, else
  returns \code{\var{self}.tzinfo.tzname(None)}, or
  raises an exception if the latter doesn't return \code{None} or
  a string object.
\end{methoddesc}


\subsection{\class{tzinfo} Objects \label{datetime-tzinfo}}

\class{tzinfo} is an abstract base clase, meaning that this class
should not be instantiated directly.  You need to derive a concrete
subclass, and (at least) supply implementations of the standard
\class{tzinfo} methods needed by the \class{datetime} methods you
use.  The \module{datetime} module does not supply any concrete
subclasses of \class{tzinfo}.

An instance of (a concrete subclass of) \class{tzinfo} can be passed
to the constructors for \class{datetime} and \class{time} objects.
The latter objects view their members as being in local time, and the
\class{tzinfo} object supports methods revealing offset of local time
from UTC, the name of the time zone, and DST offset, all relative to a
date or time object passed to them.

Special requirement for pickling:  A \class{tzinfo} subclass must have an
\method{__init__} method that can be called with no arguments, else it
can be pickled but possibly not unpickled again.  This is a technical
requirement that may be relaxed in the future.

A concrete subclass of \class{tzinfo} may need to implement the
following methods.  Exactly which methods are needed depends on the
uses made of aware \module{datetime} objects.  If in doubt, simply
implement all of them.

\begin{methoddesc}{utcoffset}{self, dt}
  Return offset of local time from UTC, in minutes east of UTC.  If
  local time is west of UTC, this should be negative.  Note that this
  is intended to be the total offset from UTC; for example, if a
  \class{tzinfo} object represents both time zone and DST adjustments,
  \method{utcoffset()} should return their sum.  If the UTC offset
  isn't known, return \code{None}.  Else the value returned must be
  a \class{timedelta} object specifying a whole number of minutes in the
  range -1439 to 1439 inclusive (1440 = 24*60; the magnitude of the offset
  must be less than one day).  Most implementations of
  \method{utcoffset()} will probably look like one of these two:

\begin{verbatim}
    return CONSTANT                 # fixed-offset class
    return CONSTANT + self.dst(dt)  # daylight-aware class
\end{verbatim}

    If \method{utcoffset()} does not return \code{None},
    \method{dst()} should not return \code{None} either.

    The default implementation of \method{utcoffset()} raises
    \exception{NotImplementedError}.
\end{methoddesc}

\begin{methoddesc}{dst}{self, dt}
  Return the daylight saving time (DST) adjustment, in minutes east of
  UTC, or \code{None} if DST information isn't known.  Return
  \code{timedelta(0)} if DST is not in effect.
  If DST is in effect, return the offset as a
  \class{timedelta} object (see \method{utcoffset()} for details).
  Note that DST offset, if applicable, has
  already been added to the UTC offset returned by
  \method{utcoffset()}, so there's no need to consult \method{dst()}
  unless you're interested in obtaining DST info separately.  For
  example, \method{datetime.timetuple()} calls its \member{tzinfo}
  member's \method{dst()} method to determine how the
  \member{tm_isdst} flag should be set, and
  \method{tzinfo.fromutc()} calls \method{dst()} to account for
  DST changes when crossing time zones.

  An instance \var{tz} of a \class{tzinfo} subclass that models both
  standard and daylight times must be consistent in this sense:

      \code{\var{tz}.utcoffset(\var{dt}) - \var{tz}.dst(\var{dt})}

  must return the same result for every \class{datetime} \var{dt}
  with \code{\var{dt}.tzinfo==\var{tz}}  For sane \class{tzinfo}
  subclasses, this expression yields the time zone's "standard offset",
  which should not depend on the date or the time, but only on geographic
  location.  The implementation of \method{datetime.astimezone()} relies
  on this, but cannot detect violations; it's the programmer's
  responsibility to ensure it.  If a \class{tzinfo} subclass cannot
  guarantee this, it may be able to override the default implementation
  of \method{tzinfo.fromutc()} to work correctly with \method{astimezone()}
  regardless.

  Most implementations of \method{dst()} will probably look like one
  of these two:

\begin{verbatim}
    return timedelta(0)   # a fixed-offset class:  doesn't account for DST

  or

    # Code to set dston and dstoff to the time zone's DST transition
    # times based on the input dt.year, and expressed in standard local
    # time.  Then

    if dston <= dt.replace(tzinfo=None) < dstoff:
        return timedelta(hours=1)
    else:
        return timedelta(0)
\end{verbatim}

  The default implementation of \method{dst()} raises
  \exception{NotImplementedError}.
\end{methoddesc}

\begin{methoddesc}{tzname}{self, dt}
  Return the time zone name corresponding to the \class{datetime}
  object \var{dt}, as a string.
  Nothing about string names is defined by the
  \module{datetime} module, and there's no requirement that it mean
  anything in particular.  For example, "GMT", "UTC", "-500", "-5:00",
  "EDT", "US/Eastern", "America/New York" are all valid replies.  Return
  \code{None} if a string name isn't known.  Note that this is a method
  rather than a fixed string primarily because some \class{tzinfo}
  subclasses will wish to return different names depending on the specific
  value of \var{dt} passed, especially if the \class{tzinfo} class is
  accounting for daylight time.

  The default implementation of \method{tzname()} raises
  \exception{NotImplementedError}.
\end{methoddesc}

These methods are called by a \class{datetime} or \class{time} object,
in response to their methods of the same names.  A \class{datetime}
object passes itself as the argument, and a \class{time} object passes
\code{None} as the argument.  A \class{tzinfo} subclass's methods should
therefore be prepared to accept a \var{dt} argument of \code{None}, or of
class \class{datetime}.

When \code{None} is passed, it's up to the class designer to decide the
best response.  For example, returning \code{None} is appropriate if the
class wishes to say that time objects don't participate in the
\class{tzinfo} protocols.  It may be more useful for \code{utcoffset(None)}
to return the standard UTC offset, as there is no other convention for
discovering the standard offset.

When a \class{datetime} object is passed in response to a
\class{datetime} method, \code{dt.tzinfo} is the same object as
\var{self}.  \class{tzinfo} methods can rely on this, unless
user code calls \class{tzinfo} methods directly.  The intent is that
the \class{tzinfo} methods interpret \var{dt} as being in local time,
and not need worry about objects in other timezones.

There is one more \class{tzinfo} method that a subclass may wish to
override:

\begin{methoddesc}{fromutc}{self, dt}
  This is called from the default \class{datetime.astimezone()}
  implementation.  When called from that, \code{\var{dt}.tzinfo} is
  \var{self}, and \var{dt}'s date and time members are to be viewed as
  expressing a UTC time.  The purpose of \method{fromutc()} is to
  adjust the date and time members, returning an equivalent datetime in
  \var{self}'s local time.

  Most \class{tzinfo} subclasses should be able to inherit the default
  \method{fromutc()} implementation without problems.  It's strong enough
  to handle fixed-offset time zones, and time zones accounting for both
  standard and daylight time, and the latter even if the DST transition
  times differ in different years.  An example of a time zone the default
  \method{fromutc()} implementation may not handle correctly in all cases
  is one where the standard offset (from UTC) depends on the specific date
  and time passed, which can happen for political reasons.
  The default implementations of \method{astimezone()} and
  \method{fromutc()} may not produce the result you want if the result is
  one of the hours straddling the moment the standard offset changes.

  Skipping code for error cases, the default \method{fromutc()}
  implementation acts like:

  \begin{verbatim}
  def fromutc(self, dt):
      # raise ValueError error if dt.tzinfo is not self
      dtoff = dt.utcoffset()
      dtdst = dt.dst()
      # raise ValueError if dtoff is None or dtdst is None
      delta = dtoff - dtdst  # this is self's standard offset
      if delta:
          dt += delta   # convert to standard local time
          dtdst = dt.dst()
          # raise ValueError if dtdst is None
      if dtdst:
          return dt + dtdst
      else:
          return dt
  \end{verbatim}
\end{methoddesc}

Example \class{tzinfo} classes:

\verbatiminput{tzinfo-examples.py}

Note that there are unavoidable subtleties twice per year in a
\class{tzinfo}
subclass accounting for both standard and daylight time, at the DST
transition points.  For concreteness, consider US Eastern (UTC -0500),
where EDT begins the minute after 1:59 (EST) on the first Sunday in
April, and ends the minute after 1:59 (EDT) on the last Sunday in October:

\begin{verbatim}
    UTC   3:MM  4:MM  5:MM  6:MM  7:MM  8:MM
    EST  22:MM 23:MM  0:MM  1:MM  2:MM  3:MM
    EDT  23:MM  0:MM  1:MM  2:MM  3:MM  4:MM

  start  22:MM 23:MM  0:MM  1:MM  3:MM  4:MM

    end  23:MM  0:MM  1:MM  1:MM  2:MM  3:MM
\end{verbatim}

When DST starts (the "start" line), the local wall clock leaps from 1:59
to 3:00.  A wall time of the form 2:MM doesn't really make sense on that
day, so \code{astimezone(Eastern)} won't deliver a result with
\code{hour==2} on the
day DST begins.  In order for \method{astimezone()} to make this
guarantee, the \method{rzinfo.dst()} method must consider times
in the "missing hour" (2:MM for Eastern) to be in daylight time.

When DST ends (the "end" line), there's a potentially worse problem:
there's an hour that can't be spelled unambiguously in local wall time:
the last hour of daylight time.  In Eastern, that's times of
the form 5:MM UTC on the day daylight time ends.  The local wall clock
leaps from 1:59 (daylight time) back to 1:00 (standard time) again.
Local times of the form 1:MM are ambiguous.  \method{astimezone()} mimics
the local clock's behavior by mapping two adjacent UTC hours into the
same local hour then.  In the Eastern example, UTC times of the form
5:MM and 6:MM both map to 1:MM when converted to Eastern.  In order for
\method{astimezone()} to make this guarantee, the \method{tzinfo.dst()}
method must consider times in the "repeated hour" to be in
standard time.  This is easily arranged, as in the example, by expressing
DST switch times in the time zone's standard local time.

Applications that can't bear such ambiguities should avoid using hybrid
\class{tzinfo} subclasses; there are no ambiguities when using UTC, or
any other fixed-offset \class{tzinfo} subclass (such as a class
representing only EST (fixed offset -5 hours), or only EDT (fixed offset
-4 hours)).


\subsection{\method{strftime()} Behavior}

\class{date}, \class{datetime}, and \class{time}
objects all support a \code{strftime(\var{format})}
method, to create a string representing the time under the control of
an explicit format string.  Broadly speaking,
\code{d.strftime(fmt)}
acts like the \refmodule{time} module's
\code{time.strftime(fmt, d.timetuple())}
although not all objects support a \method{timetuple()} method.

For \class{time} objects, the format codes for
year, month, and day should not be used, as time objects have no such
values.  If they're used anyway, \code{1900} is substituted for the
year, and \code{0} for the month and day.

For \class{date} objects, the format codes for hours, minutes, and
seconds should not be used, as \class{date} objects have no such
values.  If they're used anyway, \code{0} is substituted for them.

For a naive object, the \code{\%z} and \code{\%Z} format codes are
replaced by empty strings.

For an aware object:

\begin{itemize}
  \item[\code{\%z}]
    \method{utcoffset()} is transformed into a 5-character string of
    the form +HHMM or -HHMM, where HH is a 2-digit string giving the
    number of UTC offset hours, and MM is a 2-digit string giving the
    number of UTC offset minutes.  For example, if
    \method{utcoffset()} returns \code{timedelta(hours=-3, minutes=-30)},
    \code{\%z} is replaced with the string \code{'-0330'}.

  \item[\code{\%Z}]
    If \method{tzname()} returns \code{None}, \code{\%Z} is replaced
    by an empty string.  Otherwise \code{\%Z} is replaced by the returned
    value, which must be a string.
\end{itemize}

The full set of format codes supported varies across platforms,
because Python calls the platform C library's \function{strftime()}
function, and platform variations are common.  The documentation for
Python's \refmodule{time} module lists the format codes that the C
standard (1989 version) requires, and those work on all platforms
with a standard C implementation.  Note that the 1999 version of the
C standard added additional format codes.

The exact range of years for which \method{strftime()} works also
varies across platforms.  Regardless of platform, years before 1900
cannot be used.

