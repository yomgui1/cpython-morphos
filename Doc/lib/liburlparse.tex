\section{\module{urlparse} ---
         Parse URLs into components}
\declaremodule{standard}{urlparse}

\modulesynopsis{Parse URLs into components.}

\index{WWW}
\index{World Wide Web}
\index{URL}
\indexii{URL}{parsing}
\indexii{relative}{URL}


This module defines a standard interface to break Uniform Resource
Locator (URL) strings up in components (addressing scheme, network
location, path etc.), to combine the components back into a URL
string, and to convert a ``relative URL'' to an absolute URL given a
``base URL.''

The module has been designed to match the Internet RFC on Relative
Uniform Resource Locators (and discovered a bug in an earlier
draft!). It supports the following URL schemes:
\code{file}, \code{ftp}, \code{gopher}, \code{hdl}, \code{http}, 
\code{https}, \code{imap}, \code{mailto}, \code{mms}, \code{news}, 
\code{nntp}, \code{prospero}, \code{rsync}, \code{rtsp}, \code{rtspu}, 
\code{sftp}, \code{shttp}, \code{sip}, \code{sips}, \code{snews}, \code{svn}, 
\code{svn+ssh}, \code{telnet}, \code{wais}.

\versionadded[Support for the \code{sftp} and \code{sips} schemes]{2.5}

The \module{urlparse} module defines the following functions:

\begin{funcdesc}{urlparse}{urlstring\optional{,
                           default_scheme\optional{, allow_fragments}}}
Parse a URL into six components, returning a 6-tuple.  This
corresponds to the general structure of a URL:
\code{\var{scheme}://\var{netloc}/\var{path};\var{parameters}?\var{query}\#\var{fragment}}.
Each tuple item is a string, possibly empty.
The components are not broken up in smaller parts (for example, the network
location is a single string), and \% escapes are not expanded.
The delimiters as shown above are not part of the result,
except for a leading slash in the \var{path} component, which is
retained if present.  For example:

\begin{verbatim}
>>> from urlparse import urlparse
>>> o = urlparse('http://www.cwi.nl:80/%7Eguido/Python.html')
>>> o
('http', 'www.cwi.nl:80', '/%7Eguido/Python.html', '', '', '')
>>> o.scheme
'http'
>>> o.port
80
>>> o.geturl()
'http://www.cwi.nl:80/%7Eguido/Python.html'
\end{verbatim}

If the \var{default_scheme} argument is specified, it gives the
default addressing scheme, to be used only if the URL does not
specify one.  The default value for this argument is the empty string.

If the \var{allow_fragments} argument is false, fragment identifiers
are not allowed, even if the URL's addressing scheme normally does
support them.  The default value for this argument is \constant{True}.

The return value is actually an instance of a subclass of
\pytype{tuple}.  This class has the following additional read-only
convenience attributes:

\begin{tableiv}{l|c|l|c}{member}{Attribute}{Index}{Value}{Value if not present}
  \lineiv{scheme}  {0} {URL scheme specifier}             {empty string}
  \lineiv{netloc}  {1} {Network location part}            {empty string}
  \lineiv{path}    {2} {Hierarchical path}                {empty string}
  \lineiv{params}  {3} {Parameters for last path element} {empty string}
  \lineiv{query}   {4} {Query component}                  {empty string}
  \lineiv{fragment}{5} {Fragment identifier}              {empty string}
  \lineiv{username}{ } {User name}                        {\constant{None}}
  \lineiv{password}{ } {Password}                         {\constant{None}}
  \lineiv{hostname}{ } {Host name (lower case)}           {\constant{None}}
  \lineiv{port}    { } {Port number as integer, if present} {\constant{None}}
\end{tableiv}

See section~\ref{urlparse-result-object}, ``Results of
\function{urlparse()} and \function{urlsplit()},'' for more
information on the result object.

\versionchanged[Added attributes to return value]{2.5}
\end{funcdesc}

\begin{funcdesc}{urlunparse}{parts}
Construct a URL from a tuple as returned by \code{urlparse()}.
The \var{parts} argument can be any six-item iterable.
This may result in a slightly different, but equivalent URL, if the
URL that was parsed originally had unnecessary delimiters (for example,
a ? with an empty query; the RFC states that these are equivalent).
\end{funcdesc}

\begin{funcdesc}{urlsplit}{urlstring\optional{,
                           default_scheme\optional{, allow_fragments}}}
This is similar to \function{urlparse()}, but does not split the
params from the URL.  This should generally be used instead of
\function{urlparse()} if the more recent URL syntax allowing
parameters to be applied to each segment of the \var{path} portion of
the URL (see \rfc{2396}) is wanted.  A separate function is needed to
separate the path segments and parameters.  This function returns a
5-tuple: (addressing scheme, network location, path, query, fragment
identifier).

The return value is actually an instance of a subclass of
\pytype{tuple}.  This class has the following additional read-only
convenience attributes:

\begin{tableiv}{l|c|l|c}{member}{Attribute}{Index}{Value}{Value if not present}
  \lineiv{scheme}   {0} {URL scheme specifier}   {empty string}
  \lineiv{netloc}   {1} {Network location part}  {empty string}
  \lineiv{path}     {2} {Hierarchical path}      {empty string}
  \lineiv{query}    {3} {Query component}        {empty string}
  \lineiv{fragment} {4} {Fragment identifier}    {empty string}
  \lineiv{username} { } {User name}              {\constant{None}}
  \lineiv{password} { } {Password}               {\constant{None}}
  \lineiv{hostname} { } {Host name (lower case)} {\constant{None}}
  \lineiv{port}     { } {Port number as integer, if present} {\constant{None}}
\end{tableiv}

See section~\ref{urlparse-result-object}, ``Results of
\function{urlparse()} and \function{urlsplit()},'' for more
information on the result object.

\versionadded{2.2}
\versionchanged[Added attributes to return value]{2.5}
\end{funcdesc}

\begin{funcdesc}{urlunsplit}{parts}
Combine the elements of a tuple as returned by \function{urlsplit()}
into a complete URL as a string.
The \var{parts} argument can be any five-item iterable.
This may result in a slightly different, but equivalent URL, if the
URL that was parsed originally had unnecessary delimiters (for example,
a ? with an empty query; the RFC states that these are equivalent).
\versionadded{2.2}
\end{funcdesc}

\begin{funcdesc}{urljoin}{base, url\optional{, allow_fragments}}
Construct a full (``absolute'') URL by combining a ``base URL''
(\var{base}) with another URL (\var{url}).  Informally, this
uses components of the base URL, in particular the addressing scheme,
the network location and (part of) the path, to provide missing
components in the relative URL.  For example:

\begin{verbatim}
>>> from urlparse import urljoin
>>> urljoin('http://www.cwi.nl/%7Eguido/Python.html', 'FAQ.html')
'http://www.cwi.nl/%7Eguido/FAQ.html'
\end{verbatim}

The \var{allow_fragments} argument has the same meaning and default as
for \function{urlparse()}.

\note{If \var{url} is an absolute URL (that is, starting with \code{//}
      or \code{scheme://}, the \var{url}'s host name and/or scheme
      will be present in the result.  For example:}

\begin{verbatim}
>>> urljoin('http://www.cwi.nl/%7Eguido/Python.html',
...         '//www.python.org/%7Eguido')
'http://www.python.org/%7Eguido'
\end{verbatim}
      
If you do not want that behavior, preprocess
the \var{url} with \function{urlsplit()} and \function{urlunsplit()},
removing possible \em{scheme} and \em{netloc} parts.
\end{funcdesc}

\begin{funcdesc}{urldefrag}{url}
If \var{url} contains a fragment identifier, returns a modified
version of \var{url} with no fragment identifier, and the fragment
identifier as a separate string.  If there is no fragment identifier
in \var{url}, returns \var{url} unmodified and an empty string.
\end{funcdesc}


\begin{seealso}
  \seerfc{1738}{Uniform Resource Locators (URL)}{
        This specifies the formal syntax and semantics of absolute
        URLs.}
  \seerfc{1808}{Relative Uniform Resource Locators}{
        This Request For Comments includes the rules for joining an
        absolute and a relative URL, including a fair number of
        ``Abnormal Examples'' which govern the treatment of border
        cases.}
  \seerfc{2396}{Uniform Resource Identifiers (URI): Generic Syntax}{
        Document describing the generic syntactic requirements for
        both Uniform Resource Names (URNs) and Uniform Resource
        Locators (URLs).}
\end{seealso}


\subsection{Results of \function{urlparse()} and \function{urlsplit()}
            \label{urlparse-result-object}}

The result objects from the \function{urlparse()} and
\function{urlsplit()} functions are subclasses of the \pytype{tuple}
type.  These subclasses add the attributes described in those
functions, as well as provide an additional method:

\begin{methoddesc}[ParseResult]{geturl}{}
  Return the re-combined version of the original URL as a string.
  This may differ from the original URL in that the scheme will always
  be normalized to lower case and empty components may be dropped.
  Specifically, empty parameters, queries, and fragment identifiers
  will be removed.

  The result of this method is a fixpoint if passed back through the
  original parsing function:

\begin{verbatim}
>>> import urlparse
>>> url = 'HTTP://www.Python.org/doc/#'

>>> r1 = urlparse.urlsplit(url)
>>> r1.geturl()
'http://www.Python.org/doc/'

>>> r2 = urlparse.urlsplit(r1.geturl())
>>> r2.geturl()
'http://www.Python.org/doc/'
\end{verbatim}

\versionadded{2.5}
\end{methoddesc}

The following classes provide the implementations of the parse results::

\begin{classdesc*}{BaseResult}
  Base class for the concrete result classes.  This provides most of
  the attribute definitions.  It does not provide a \method{geturl()}
  method.  It is derived from \class{tuple}, but does not override the
  \method{__init__()} or \method{__new__()} methods.
\end{classdesc*}


\begin{classdesc}{ParseResult}{scheme, netloc, path, params, query, fragment}
  Concrete class for \function{urlparse()} results.  The
  \method{__new__()} method is overridden to support checking that the
  right number of arguments are passed.
\end{classdesc}


\begin{classdesc}{SplitResult}{scheme, netloc, path, query, fragment}
  Concrete class for \function{urlsplit()} results.  The
  \method{__new__()} method is overridden to support checking that the
  right number of arguments are passed.
\end{classdesc}
