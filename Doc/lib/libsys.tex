\section{\module{sys} ---
         System-specific parameters and functions}

\declaremodule{builtin}{sys}
\modulesynopsis{Access system-specific parameters and functions.}

This module provides access to some variables used or maintained by the
interpreter and to functions that interact strongly with the interpreter.
It is always available.


\begin{datadesc}{argv}
  The list of command line arguments passed to a Python script.
  \code{argv[0]} is the script name (it is operating system dependent
  whether this is a full pathname or not).  If the command was
  executed using the \programopt{-c} command line option to the
  interpreter, \code{argv[0]} is set to the string \code{'-c'}.  If no
  script name was passed to the Python interpreter, \code{argv} has
  zero length.
\end{datadesc}

\begin{datadesc}{byteorder}
  An indicator of the native byte order.  This will have the value
  \code{'big'} on big-endian (most-signigicant byte first) platforms,
  and \code{'little'} on little-endian (least-significant byte first)
  platforms.
  \versionadded{2.0}
\end{datadesc}

\begin{datadesc}{builtin_module_names}
  A tuple of strings giving the names of all modules that are compiled
  into this Python interpreter.  (This information is not available in
  any other way --- \code{modules.keys()} only lists the imported
  modules.)
\end{datadesc}

\begin{datadesc}{copyright}
  A string containing the copyright pertaining to the Python
  interpreter.
\end{datadesc}

\begin{datadesc}{dllhandle}
  Integer specifying the handle of the Python DLL.
  Availability: Windows.
\end{datadesc}

\begin{funcdesc}{displayhook}{\var{value}}
  If \var{value} is not \code{None}, this function prints it to
  \code{sys.stdout}, and saves it in \code{__builtin__._}.

  \code{sys.displayhook} is called on the result of evaluating an
  expression entered in an interactive Python session.  The display of
  these values can be customized by assigning another one-argument
  function to \code{sys.displayhook}.
\end{funcdesc}

\begin{funcdesc}{excepthook}{\var{type}, \var{value}, \var{traceback}}
  This function prints out a given traceback and exception to
  \code{sys.stderr}.

  When an exception is raised and uncaught, the interpreter calls
  \code{sys.excepthook} with three arguments, the exception class,
  exception instance, and a traceback object.  In an interactive
  session this happens just before control is returned to the prompt;
  in a Python program this happens just before the program exits.  The
  handling of such top-level exceptions can be customized by assigning
  another three-argument function to \code{sys.excepthook}.
\end{funcdesc}

\begin{datadesc}{__displayhook__}
\dataline{__excepthook__}
  These objects contain the original values of \code{displayhook} and
  \code{excepthook} at the start of the program.  They are saved so
  that \code{displayhook} and \code{excepthook} can be restored in
  case they happen to get replaced with broken objects.
\end{datadesc}

\begin{funcdesc}{exc_info}{}
  This function returns a tuple of three values that give information
  about the exception that is currently being handled.  The
  information returned is specific both to the current thread and to
  the current stack frame.  If the current stack frame is not handling
  an exception, the information is taken from the calling stack frame,
  or its caller, and so on until a stack frame is found that is
  handling an exception.  Here, ``handling an exception'' is defined
  as ``executing or having executed an except clause.''  For any stack
  frame, only information about the most recently handled exception is
  accessible.

  If no exception is being handled anywhere on the stack, a tuple
  containing three \code{None} values is returned.  Otherwise, the
  values returned are \code{(\var{type}, \var{value},
  \var{traceback})}.  Their meaning is: \var{type} gets the exception
  type of the exception being handled (a string or class object);
  \var{value} gets the exception parameter (its \dfn{associated value}
  or the second argument to \keyword{raise}, which is always a class
  instance if the exception type is a class object); \var{traceback}
  gets a traceback object (see the Reference Manual) which
  encapsulates the call stack at the point where the exception
  originally occurred.  \obindex{traceback}

  \strong{Warning:} assigning the \var{traceback} return value to a
  local variable in a function that is handling an exception will
  cause a circular reference.  This will prevent anything referenced
  by a local variable in the same function or by the traceback from
  being garbage collected.  Since most functions don't need access to
  the traceback, the best solution is to use something like
  \code{type, value = sys.exc_info()[:2]} to extract only the
  exception type and value.  If you do need the traceback, make sure
  to delete it after use (best done with a \keyword{try}
  ... \keyword{finally} statement) or to call \function{exc_info()} in
  a function that does not itself handle an exception.
\end{funcdesc}

\begin{datadesc}{exc_type}
\dataline{exc_value}
\dataline{exc_traceback}
\deprecated {1.5}
            {Use \function{exc_info()} instead.}
  Since they are global variables, they are not specific to the
  current thread, so their use is not safe in a multi-threaded
  program.  When no exception is being handled, \code{exc_type} is set
  to \code{None} and the other two are undefined.
\end{datadesc}

\begin{datadesc}{exec_prefix}
  A string giving the site-specific directory prefix where the
  platform-dependent Python files are installed; by default, this is
  also \code{'/usr/local'}.  This can be set at build time with the
  \longprogramopt{exec-prefix} argument to the \program{configure}
  script.  Specifically, all configuration files (e.g. the
  \file{pyconfig.h} header file) are installed in the directory
  \code{exec_prefix + '/lib/python\var{version}/config'}, and shared
  library modules are installed in \code{exec_prefix +
  '/lib/python\var{version}/lib-dynload'}, where \var{version} is
  equal to \code{version[:3]}.
\end{datadesc}

\begin{datadesc}{executable}
  A string giving the name of the executable binary for the Python
  interpreter, on systems where this makes sense.
\end{datadesc}

\begin{funcdesc}{exit}{\optional{arg}}
  Exit from Python.  This is implemented by raising the
  \exception{SystemExit} exception, so cleanup actions specified by
  finally clauses of \keyword{try} statements are honored, and it is
  possible to intercept the exit attempt at an outer level.  The
  optional argument \var{arg} can be an integer giving the exit status
  (defaulting to zero), or another type of object.  If it is an
  integer, zero is considered ``successful termination'' and any
  nonzero value is considered ``abnormal termination'' by shells and
  the like.  Most systems require it to be in the range 0-127, and
  produce undefined results otherwise.  Some systems have a convention
  for assigning specific meanings to specific exit codes, but these
  are generally underdeveloped; Unix programs generally use 2 for
  command line syntax errors and 1 for all other kind of errors.  If
  another type of object is passed, \code{None} is equivalent to
  passing zero, and any other object is printed to \code{sys.stderr}
  and results in an exit code of 1.  In particular,
  \code{sys.exit("some error message")} is a quick way to exit a
  program when an error occurs.
\end{funcdesc}

\begin{datadesc}{exitfunc}
  This value is not actually defined by the module, but can be set by
  the user (or by a program) to specify a clean-up action at program
  exit.  When set, it should be a parameterless function.  This
  function will be called when the interpreter exits.  Only one
  function may be installed in this way; to allow multiple functions
  which will be called at termination, use the \refmodule{atexit}
  module.  Note: the exit function is not called when the program is
  killed by a signal, when a Python fatal internal error is detected,
  or when \code{os._exit()} is called.
\end{datadesc}

\begin{funcdesc}{getdefaultencoding}{}
  Return the name of the current default string encoding used by the
  Unicode implementation.
  \versionadded{2.0}
\end{funcdesc}

\begin{funcdesc}{getdlopenflags}{}
  Return the current value of the flags that are used for
  \cfunction{dlopen()} calls. The flag constants are defined in the
  \refmodule{dl} and \module{DLFCN} modules.
  Availability: \UNIX.
  \versionadded{2.2}
\end{funcdesc}

\begin{funcdesc}{getrefcount}{object}
  Return the reference count of the \var{object}.  The count returned
  is generally one higher than you might expect, because it includes
  the (temporary) reference as an argument to
  \function{getrefcount()}.
\end{funcdesc}

\begin{funcdesc}{getrecursionlimit}{}
  Return the current value of the recursion limit, the maximum depth
  of the Python interpreter stack.  This limit prevents infinite
  recursion from causing an overflow of the C stack and crashing
  Python.  It can be set by \function{setrecursionlimit()}.
\end{funcdesc}

\begin{funcdesc}{_getframe}{\optional{depth}}
  Return a frame object from the call stack.  If optional integer
  \var{depth} is given, return the frame object that many calls below
  the top of the stack.  If that is deeper than the call stack,
  \exception{ValueError} is raised.  The default for \var{depth} is
  zero, returning the frame at the top of the call stack.

  This function should be used for internal and specialized purposes
  only.
\end{funcdesc}

\begin{datadesc}{hexversion}
  The version number encoded as a single integer.  This is guaranteed
  to increase with each version, including proper support for
  non-production releases.  For example, to test that the Python
  interpreter is at least version 1.5.2, use:

\begin{verbatim}
if sys.hexversion >= 0x010502F0:
    # use some advanced feature
    ...
else:
    # use an alternative implementation or warn the user
    ...
\end{verbatim}

  This is called \samp{hexversion} since it only really looks
  meaningful when viewed as the result of passing it to the built-in
  \function{hex()} function.  The \code{version_info} value may be
  used for a more human-friendly encoding of the same information.
  \versionadded{1.5.2}
\end{datadesc}

\begin{datadesc}{last_type}
\dataline{last_value}
\dataline{last_traceback}
  These three variables are not always defined; they are set when an
  exception is not handled and the interpreter prints an error message
  and a stack traceback.  Their intended use is to allow an
  interactive user to import a debugger module and engage in
  post-mortem debugging without having to re-execute the command that
  caused the error.  (Typical use is \samp{import pdb; pdb.pm()} to
  enter the post-mortem debugger; see chapter \ref{debugger}, ``The
  Python Debugger,'' for more information.)

  The meaning of the variables is the same as that of the return
  values from \function{exc_info()} above.  (Since there is only one
  interactive thread, thread-safety is not a concern for these
  variables, unlike for \code{exc_type} etc.)
\end{datadesc}

\begin{datadesc}{maxint}
  The largest positive integer supported by Python's regular integer
  type.  This is at least 2**31-1.  The largest negative integer is
  \code{-maxint-1} -- the asymmetry results from the use of 2's
  complement binary arithmetic.
\end{datadesc}

\begin{datadesc}{modules}
  This is a dictionary that maps module names to modules which have
  already been loaded.  This can be manipulated to force reloading of
  modules and other tricks.  Note that removing a module from this
  dictionary is \emph{not} the same as calling
  \function{reload()}\bifuncindex{reload} on the corresponding module
  object.
\end{datadesc}

\begin{datadesc}{path}
\indexiii{module}{search}{path}
  A list of strings that specifies the search path for modules.
  Initialized from the environment variable \envvar{PYTHONPATH}, or an
  installation-dependent default.

  The first item of this list, \code{path[0]}, is the directory
  containing the script that was used to invoke the Python
  interpreter.  If the script directory is not available (e.g.  if the
  interpreter is invoked interactively or if the script is read from
  standard input), \code{path[0]} is the empty string, which directs
  Python to search modules in the current directory first.  Notice
  that the script directory is inserted \emph{before} the entries
  inserted as a result of \envvar{PYTHONPATH}.
\end{datadesc}

\begin{datadesc}{platform}
  This string contains a platform identifier, e.g. \code{'sunos5'} or
  \code{'linux1'}.  This can be used to append platform-specific
  components to \code{path}, for instance.
\end{datadesc}

\begin{datadesc}{prefix}
  A string giving the site-specific directory prefix where the
  platform independent Python files are installed; by default, this is
  the string \code{'/usr/local'}.  This can be set at build time with
  the \longprogramopt{prefix} argument to the \program{configure}
  script.  The main collection of Python library modules is installed
  in the directory \code{prefix + '/lib/python\var{version}'} while
  the platform independent header files (all except \file{pyconfig.h})
  are stored in \code{prefix + '/include/python\var{version}'}, where
  \var{version} is equal to \code{version[:3]}.
\end{datadesc}

\begin{datadesc}{ps1}
\dataline{ps2}
\index{interpreter prompts}
\index{prompts, interpreter}
  Strings specifying the primary and secondary prompt of the
  interpreter.  These are only defined if the interpreter is in
  interactive mode.  Their initial values in this case are
  \code{'>\code{>}> '} and \code{'... '}.  If a non-string object is
  assigned to either variable, its \function{str()} is re-evaluated
  each time the interpreter prepares to read a new interactive
  command; this can be used to implement a dynamic prompt.
\end{datadesc}

\begin{funcdesc}{setcheckinterval}{interval}
  Set the interpreter's ``check interval''.  This integer value
  determines how often the interpreter checks for periodic things such
  as thread switches and signal handlers.  The default is \code{10},
  meaning the check is performed every 10 Python virtual instructions.
  Setting it to a larger value may increase performance for programs
  using threads.  Setting it to a value \code{<=} 0 checks every
  virtual instruction, maximizing responsiveness as well as overhead.
\end{funcdesc}

\begin{funcdesc}{setdefaultencoding}{name}
  Set the current default string encoding used by the Unicode
  implementation.  If \var{name} does not match any available
  encoding, \exception{LookupError} is raised.  This function is only
  intended to be used by the \refmodule{site} module implementation
  and, where needed, by \module{sitecustomize}.  Once used by the
  \refmodule{site} module, it is removed from the \module{sys}
  module's namespace.
%  Note that \refmodule{site} is not imported if
%  the \programopt{-S} option is passed to the interpreter, in which
%  case this function will remain available.
  \versionadded{2.0}
\end{funcdesc}

\begin{funcdesc}{setdlopenflags}{n}
  Set the flags used by the interpreter for \cfunction{dlopen()}
  calls, such as when the interpreter loads extension modules.  Among
  other things, this will enable a lazy resolving of symbols when
  importing a module, if called as \code{sys.setdlopenflags(0)}.  To
  share symbols across extension modules, call as
  \code{sys.setdlopenflags(dl.RTLD_NOW | dl.RTLD_GLOBAL)}.  Symbolic
  names for the flag modules can be either found in the \refmodule{dl}
  module, or in the \module{DLFCN} module. If \module{DLFCN} is not
  available, it can be generated from \file{/usr/include/dlfcn.h}
  using the \program{h2py} script.
  Availability: \UNIX.
  \versionadded{2.2}
\end{funcdesc}

\begin{funcdesc}{setprofile}{profilefunc}
  Set the system's profile function,\index{profile function} which
  allows you to implement a Python source code profiler in
  Python.\index{profiler}  See chapter \ref{profile} for more
  information on the Python profiler.  The system's profile function
  is called similarly to the system's trace function (see
  \function{settrace()}), but it isn't called for each executed line
  of code (only on call and return and when an exception occurs).
  Also, its return value is not used, so it can simply return
  \code{None}.
\end{funcdesc}

\begin{funcdesc}{setrecursionlimit}{limit}
  Set the maximum depth of the Python interpreter stack to
  \var{limit}.  This limit prevents infinite recursion from causing an
  overflow of the C stack and crashing Python.

  The highest possible limit is platform-dependent.  A user may need
  to set the limit higher when she has a program that requires deep
  recursion and a platform that supports a higher limit.  This should
  be done with care, because a too-high limit can lead to a crash.
\end{funcdesc}

\begin{funcdesc}{settrace}{tracefunc}
  Set the system's trace function,\index{trace function} which allows
  you to implement a Python source code debugger in Python.  See
  section \ref{debugger-hooks}, ``How It Works,'' in the chapter on
  the Python debugger.\index{debugger}
\end{funcdesc}

\begin{datadesc}{stdin}
\dataline{stdout}
\dataline{stderr}
  File objects corresponding to the interpreter's standard input,
  output and error streams.  \code{stdin} is used for all interpreter
  input except for scripts but including calls to
  \function{input()}\bifuncindex{input} and
  \function{raw_input()}\bifuncindex{raw_input}.  \code{stdout} is
  used for the output of \keyword{print} and expression statements and
  for the prompts of \function{input()} and \function{raw_input()}.
  The interpreter's own prompts and (almost all of) its error messages
  go to \code{stderr}.  \code{stdout} and \code{stderr} needn't be
  built-in file objects: any object is acceptable as long as it has a
  \method{write()} method that takes a string argument.  (Changing
  these objects doesn't affect the standard I/O streams of processes
  executed by \function{os.popen()}, \function{os.system()} or the
  \function{exec*()} family of functions in the \refmodule{os}
  module.)
\end{datadesc}

\begin{datadesc}{__stdin__}
\dataline{__stdout__}
\dataline{__stderr__}
  These objects contain the original values of \code{stdin},
  \code{stderr} and \code{stdout} at the start of the program.  They
  are used during finalization, and could be useful to restore the
  actual files to known working file objects in case they have been
  overwritten with a broken object.
\end{datadesc}

\begin{datadesc}{tracebacklimit}
  When this variable is set to an integer value, it determines the
  maximum number of levels of traceback information printed when an
  unhandled exception occurs.  The default is \code{1000}.  When set
  to \code{0} or less, all traceback information is suppressed and
  only the exception type and value are printed.
\end{datadesc}

\begin{datadesc}{version}
  A string containing the version number of the Python interpreter
  plus additional information on the build number and compiler used.
  It has a value of the form \code{'\var{version}
  (\#\var{build_number}, \var{build_date}, \var{build_time})
  [\var{compiler}]'}.  The first three characters are used to identify
  the version in the installation directories (where appropriate on
  each platform).  An example:

\begin{verbatim}
>>> import sys
>>> sys.version
'1.5.2 (#0 Apr 13 1999, 10:51:12) [MSC 32 bit (Intel)]'
\end{verbatim}
\end{datadesc}

\begin{datadesc}{version_info}
  A tuple containing the five components of the version number:
  \var{major}, \var{minor}, \var{micro}, \var{releaselevel}, and
  \var{serial}.  All values except \var{releaselevel} are integers;
  the release level is \code{'alpha'}, \code{'beta'},
  \code{'candidate'}, or \code{'final'}.  The \code{version_info}
  value corresponding to the Python version 2.0 is \code{(2, 0, 0,
  'final', 0)}.
  \versionadded{2.0}
\end{datadesc}

\begin{datadesc}{winver}
  The version number used to form registry keys on Windows platforms.
  This is stored as string resource 1000 in the Python DLL.  The value
  is normally the first three characters of \constant{version}.  It is
  provided in the \module{sys} module for informational purposes;
  modifying this value has no effect on the registry keys used by
  Python.
  Availability: Windows.
\end{datadesc}
