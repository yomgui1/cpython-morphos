\section{\module{chunk} ---
         Helper for reading IFF chunks}

\declaremodule{standard}{chunk}
\sectionauthor{Moshe Zadka}{mzadka@geocities.com}
\modulesynopsis{Helper class for reading from IFF-based file formats.}

The \module{chunk} module defines a class for interfacing to ``IFF''
chunk-based files, like TIFF or AIFF.  This is used as a helper module
for the \refmodule{aifc} and \refmodule{wave} modules.

The \module{chunk} module defines the following class:

\begin{classdesc}{Chunk}{file\optional{, align}}
The chunk from \var{file} starting at \var{file}'s current
position. The \var{align} argument, which defaults to true, determines 
whether to align chunk data on 2-byte boundaries.

\exception{EOFError} is raised if \var{file} does not contain enough
data to read the IFF header.
\end{classdesc}

The IFF header format is described in this table:

\begin{tableiii}{c|c|l}{textrm}{Offset}{Length}{Contents}
  \lineiii{0}{4}{Chunk ID}
  \lineiii{4}{4}{Size of chunk in big-endian byte order, including the 
                 header}
\end{tableiii}


\subsection{Chunk Objects \label{iff-chunk-objects}}

Chunk objects have the following methods:

\begin{methoddesc}{getname}{}
Return the ID of the chunk.
\end{methoddesc}

\begin{methoddesc}{close}{}
Close the chunk, forwarding the file pointer to the end of the chunk.
\end{methoddesc}

\begin{methoddesc}{isatty}{}
Returns false unless the chunk has been closed, in which case
\exception{ValueError} is raised.
\end{methoddesc}

\begin{methoddesc}{seek}{offset\optional{, whence}}
Seek to a position within the chunk. If file pointer is not seekable,
or \var{offset} would point outside the chunk, an error is raised.
\var{whence} is interpreted the same as for the \method{seek()} method
on file objects; see section \ref{bltin-file-objects} for more
information.
\end{methoddesc}

\begin{methoddesc}{tell}{}
Return the current position within this chunk.
\end{methoddesc}

\begin{methoddesc}{read}{\optional{n}}
Read at most \var{n} bytes from the chunk. If \var{n} is omitted
or negative, read until the end of the chunk.
\end{methoddesc}

\begin{methoddesc}{skip}{}
Skip to the end of the chunk.  All further calls to \method{read()}
for the chunk will return \code{''}.  If you are not interested in the
contents of the chunk, this method should be called so that the file
points to the start of the next chunk.
\end{methoddesc}
