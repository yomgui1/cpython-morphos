\section{Built-in Module \sectcode{dbm}}
\bimodindex{dbm}

Dbm provides python programs with an interface to the unix \code{ndbm}
database library.  Dbm objects are of the mapping type, so they can be
handled just like objects of the built-in \dfn{dictionary} type,
except that keys and values are always strings, and printing a dbm
object doesn't print the keys and values.

The module defines the following constant and functions:

\renewcommand{\indexsubitem}{(in module dbm)}
\begin{excdesc}{error}
Raised on dbm-specific errors, such as I/O errors. \code{KeyError} is
raised for general mapping errors like specifying an incorrect key.
\end{excdesc}

\begin{funcdesc}{open}{filename\, rwmode\, filemode}
Open a dbm database and return a mapping object.  \var{filename} is
the name of the database file (without the \file{.dir} or \file{.pag}
extensions), \var{rwmode} is \code{'r'}, \code{'w'} or \code{'rw'} as for
\code{open}, and \var{filemode} is the \UNIX{} mode of the file, used only
when the database has to be created.
\end{funcdesc}
