\section{\module{sets} ---
         Unordered collections of unique elements}

\declaremodule{standard}{sets}
\modulesynopsis{Implementation of sets of unique elements.}
\moduleauthor{Greg V. Wilson}{gvwilson@nevex.com}
\moduleauthor{Alex Martelli}{aleax@aleax.it}
\moduleauthor{Guido van Rossum}{guido@python.org}
\sectionauthor{Raymond D. Hettinger}{python@rcn.com}

\versionadded{2.3}

The \module{sets} module provides classes for constructing and manipulating
unordered collections of unique elements.  Common uses include membership
testing, removing duplicates from a sequence, and computing standard math
operations on sets such as intersection, union, difference, and symmetric
difference.

Like other collections, sets support \code{\var{x} in \var{set}},
\code{len(\var{set})}, and \code{for \var{x} in \var{set}}.  Being an
unordered collection, sets do not record element position or order of
insertion.  Accordingly, sets do not support indexing, slicing, or
other sequence-like behavior.

Most set applications use the \class{Set} class which provides every set
method except for \method{__hash__()}. For advanced applications requiring
a hash method, the \class{ImmutableSet} class adds a \method{__hash__()}
method but omits methods which alter the contents of the set. Both
\class{Set} and \class{ImmutableSet} derive from \class{BaseSet}, an
abstract class useful for determining whether something is a set:
\code{isinstance(\var{obj}, BaseSet)}.

The set classes are implemented using dictionaries.  As a result, sets
cannot contain mutable elements such as lists or dictionaries.
However, they can contain immutable collections such as tuples or
instances of \class{ImmutableSet}.  For convenience in implementing
sets of sets, inner sets are automatically converted to immutable
form, for example, \code{Set([Set(['dog'])])} is transformed to
\code{Set([ImmutableSet(['dog'])])}.

\begin{classdesc}{Set}{\optional{iterable}}
Constructs a new empty \class{Set} object.  If the optional \var{iterable}
parameter is supplied, updates the set with elements obtained from iteration.
All of the elements in \var{iterable} should be immutable or be transformable
to an immutable using the protocol described in
section~\ref{immutable-transforms}.
\end{classdesc}

\begin{classdesc}{ImmutableSet}{\optional{iterable}}
Constructs a new empty \class{ImmutableSet} object.  If the optional
\var{iterable} parameter is supplied, updates the set with elements obtained
from iteration.  All of the elements in \var{iterable} should be immutable or
be transformable to an immutable using the protocol described in
section~\ref{immutable-transforms}.

Because \class{ImmutableSet} objects provide a \method{__hash__()} method,
they can be used as set elements or as dictionary keys.  \class{ImmutableSet}
objects do not have methods for adding or removing elements, so all of the
elements must be known when the constructor is called.
\end{classdesc}


\subsection{Set Objects}

Instances of \class{Set} and \class{ImmutableSet} both provide
the following operations:

\begin{tableii}{c|l}{code}{Operation}{Result}
  \lineii{len(\var{s})}{cardinality of set \var{s}}

  \hline
  \lineii{\var{x} in \var{s}}
         {test \var{x} for membership in \var{s}}
  \lineii{\var{x} not in \var{s}}
         {test \var{x} for non-membership in \var{s}}
  \lineii{\var{s}.issubset(\var{t})}
         {test whether every element in \var{s} is in \var{t}}
  \lineii{\var{s}.issuperset(\var{t})}
         {test whether every element in \var{t} is in \var{s}}

  \hline
  \lineii{\var{s} | \var{t}}
         {new set with elements from both \var{s} and \var{t}}
  \lineii{\var{s}.union(\var{t})}
         {new set with elements from both \var{s} and \var{t}}
  \lineii{\var{s} \&\ \var{t}}
         {new set with elements common to \var{s} and \var{t}}
  \lineii{\var{s}.intersection(\var{t})}
         {new set with elements common to \var{s} and \var{t}}
  \lineii{\var{s} - \var{t}}
         {new set with elements in \var{s} but not in \var{t}}
  \lineii{\var{s}.difference(\var{t})}
         {new set with elements in \var{s} but not in \var{t}}
  \lineii{\var{s} \textasciicircum\ \var{t}}
         {new set with elements in either \var{s} or \var{t} but not both}
  \lineii{\var{s}.symmetric_difference(\var{t})}
         {new set with elements in either \var{s} or \var{t} but not both}
  \lineii{\var{s}.copy()}
         {new set with a shallow copy of \var{s}}
\end{tableii}

In addition to the above operations, both \class{Set} and \class{ImmutableSet}
support set to set comparison operators based on the contents of their
internal dictionaries.  Two sets are equal if and only if every element of
each set is contained in the other.

The following table lists operations available in \class{ImmutableSet}
but not found in \class{Set}:

\begin{tableii}{c|l|c}{code}{Operation}{Result}
  \lineii{hash(\var{s})}{returns a hash value for \var{s}}
\end{tableii}

The following table lists operations available in \class{Set}
but not found in \class{ImmutableSet}:

\begin{tableii}{c|l}{code}{Operation}{Result}
  \lineii{\var{s} |= \var{t}}
         {return set \var{s} with elements added from \var{t}}
  \lineii{\var{s}.union_update(\var{t})}
         {return set \var{s} with elements added from \var{t}}
  \lineii{\var{s} \&= \var{t}}
         {return set \var{s} keeping only elements also found in \var{t}}
  \lineii{\var{s}.intersection_update(\var{t})}
         {return set \var{s} keeping only elements also found in \var{t}}
  \lineii{\var{s} -= \var{t}}
         {return set \var{s} after removing elements found in \var{t}}
  \lineii{\var{s}.difference_update(\var{t})}
         {return set \var{s} after removing elements found in \var{t}}
  \lineii{\var{s} \textasciicircum= \var{t}}
         {return set \var{s} with elements from \var{s} or \var{t}
          but not both}
  \lineii{\var{s}.symmetric_difference_update(\var{t})}
         {return set \var{s} with elements from \var{s} or \var{t}
          but not both}

  \hline
  \lineii{\var{s}.add(\var{x})}
         {Add element \var{x} to set \var{s}}
  \lineii{\var{s}.remove(\var{x})}
         {Remove element \var{x} from set \var{s}}
  \lineii{\var{s}.discard(\var{x})}
         {Removes element \var{x} from set \var{s}. Like \var{s}.remove(\var{x})
          but does not raise KeyError if \var{x} is not in \var{s}}
  \lineii{\var{s}.pop()}
         {Remove and return an element from \var{s}; no guarantee is
          made about which element is removed}
  \lineii{\var{s}.update(\var{t})}
         {Add elements from \var{t} to set \var{s}}
  \lineii{\var{s}.clear()}
         {Remove all elements from set \var{s}}
\end{tableii}


\subsection{Example}

\begin{verbatim}
>>> from sets import Set
>>> engineers = Set(['John', 'Jane', 'Jack', 'Janice'])
>>> programmers = Set(['Jack', 'Sam', 'Susan', 'Janice'])
>>> management = Set(['Jane', 'Jack', 'Susan', 'Zack'])
>>> employees = engineers | programmers | management           # union
>>> engineering_management = engineers & programmers           # intersection
>>> fulltime_management = management - engineers - programmers # difference
>>> engineers.add('Marvin')                                    # add element
>>> print engineers
Set(['Jane', 'Marvin', 'Janice', 'John', 'Jack'])
>>> employees.issuperset(engineers)           # superset test
False
>>> employees.update(engineers)               # update from another set
>>> employees.issuperset(engineers)
True
>>> for group in [engineers, programmers, management, employees]:
        group.discard('Susan')                # unconditionally remove element
        print group

Set(['Jane', 'Marvin', 'Janice', 'John', 'Jack'])
Set(['Janice', 'Jack', 'Sam'])
Set(['Jane', 'Zack', 'Jack'])
Set(['Jack', 'Sam', 'Jane', 'Marvin', 'Janice', 'John', 'Zack'])
\end{verbatim}


\subsection{Protocol for automatic conversion to immutable
            \label{immutable-transforms}}

Sets can only contain immutable elements.  For convenience, mutable
\class{Set} objects are automatically copied to an \class{ImmutableSet}
before being added as a set element.

The mechanism is to always add a hashable element, or if it is not
hashable, the element is checked to see if it has an
\method{_as_immutable()} method which returns an immutable equivalent.

Since \class{Set} objects have a \method{_as_immutable()} method
returning an instance of \class{ImmutableSet}, it is possible to
construct sets of sets.

A similar mechanism is needed by the \method{__contains__()} and
\method{remove()} methods which need to hash an element to check
for membership in a set.  Those methods check an element for hashability
and, if not, check for a \method{_as_temporarily_immutable()} method
which returns the element wrapped by a class that provides temporary
methods for \method{__hash__()}, \method{__eq__()}, and \method{__ne__()}.

The alternate mechanism spares the need to build a separate copy of
the original mutable object.

\class{Set} objects implement the \method{_as_temporarily_immutable()}
method which returns the \class{Set} object wrapped by a new class
\class{_TemporarilyImmutableSet}.

The two mechanisms for adding hashability are normally invisible to the
user; however, a conflict can arise in a multi-threaded environment
where one thread is updating a set while another has temporarily wrapped it
in \class{_TemporarilyImmutableSet}.  In other words, sets of mutable sets
are not thread-safe.
