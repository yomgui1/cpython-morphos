\section{\module{curses.ascii} ---
         Constants and set-membership functions for ASCII characters.}

\declaremodule{standard}{curses.ascii}
\modulesynopsis{Constants and set-membership functions for ASCII characters.}
\moduleauthor{Eric S. Raymond}{esr@thyrsus.com}
\sectionauthor{Eric S. Raymond}{esr@thyrsus.com}

\versionadded{1.6}

The \module{curses.ascii} module supplies name constants for ASCII characters
and functions to test membership in various ASCII character classes.  
The constants supplied are names for control characters as follows:

NUL, SOH, STX, ETX, EOT, ENQ, ACK, BEL, BS, TAB, HT, LF, NL, VT, FF, CR,
SO, SI, DLE, DC1, DC2, DC3, DC4, NAK, SYN, ETB, CAN, EM, SUB, ESC, FS, 
GS, RS, US, SP, DEL.

NL and LF are synonyms; so are HT and TAB.  The module also supplies
the following functions, patterned on those in the standard C library:

\begin{funcdesc}{isalnum}{c}
Checks for an ASCII alphanumeric character; it is equivalent to
isalpha(c) or isdigit(c))
\end{funcdesc}

\begin{funcdesc}{isalpha}{c}
Checks for an ASCII alphabetic character; it is equivalent to
isupper(c) or islower(c))
\end{funcdesc}

\begin{funcdesc}{isascii}{c}
Checks for a character value that fits in the 7-bit ASCII set.
\end{funcdesc}

\begin{funcdesc}{isblank}{c}
Checks for an ASCII alphanumeric character; it is equivalent to
isalpha(c) or isdigit(c))
\end{funcdesc}

\begin{funcdesc}{iscntrl}{c}
Checks for an ASCII control character (range 0x00 to 0x1f).
\end{funcdesc}

\begin{funcdesc}{isdigit}{c}
Checks for an ASCII decimal digit, 0 through 9.
\end{funcdesc}

\begin{funcdesc}{isgraph}{c}
Checks for ASCII any printable character except space.
\end{funcdesc}

\begin{funcdesc}{islower}{c}
Checks for an ASCII lower-case character.
\end{funcdesc}

\begin{funcdesc}{isprint}{c}
Checks for any ASCII printable character including space.
\end{funcdesc}

\begin{funcdesc}{ispunct}{c}
Checks for any printable ASCII character which is not a space or an
alphanumeric character.
\end{funcdesc}

\begin{funcdesc}{isspace}{c}
Checks for ASCII white-space characters; space, tab, line feed,
carriage return, form feed, horizontal tab, vertical tab.
\end{funcdesc}

\begin{funcdesc}{isupper}{c}
Checks for an ASCII uppercase letter.
\end{funcdesc}

\begin{funcdesc}{isxdigit}{c}
Checks for an ASCII hexadecimal digit, i.e. one of 0123456789abcdefABCDEF.
\end{funcdesc}

\begin{funcdesc}{isctrl}{c}
Checks for an ASCII control character, bit values 0 to 31.
\end{funcdesc}

\begin{funcdesc}{ismeta}{c}
Checks for a (non-ASCII) character, bit values 0x80 and above.
\end{funcdesc}

These functions accept either integers or strings; when the argument
is a string, it is first converted using the built-in function ord().

Note that all these functions check ordinal bit values derived from the 
first character of the string you pass in; they do not actually know
anything about the host machine's character encoding.  For functions 
that know about the character encoding (and handle
internationalization properly) see the string module.

The following two functions take either a single-character string or
integer byte value; they return a value of the same type.

\begin{funcdesc}{ascii}{c}
Return the ASCII value corresponding to the low 7 bits of c.
\end{funcdesc}

\begin{funcdesc}{ctrl}{c}
Return the control character corresponding to the given character
(the character bit value is logical-anded with 0x1f).
\end{funcdesc}

\begin{funcdesc}{alt}{c}
Return the 8-bit character corresponding to the given ASCII character
(the character bit value is logical-ored with 0x80).
\end{funcdesc}

The following function takes either a single-character string or
integer byte value; it returns a string.

\begin{funcdesc}{unctrl}{c}
Return a string representation of the ASCII character c.  If c is
printable, this string is the character itself.  If the character
is a control character (0x00-0x1f) the string consists of a caret
(^) followed by the corresponding uppercase letter.  If the character
is an ASCII delete (0x7f) the string is "^?".  If the character has
its meta bit (0x80) set, the meta bit is stripped, the preceding rules
applied, and "!" prepended to the result.
\end{funcdesc}

Finally, the module supplies a 33-element string array 
called controlnames that contains the ASCII mnemonics for the
thirty-two ASCII control characters from 0 (NUL) to 0x1f (US),
in order, plus the mnemonic "SP" for space.

 
