\section{\module{xml.dom} ---
         The Document Object Model API}

\declaremodule{standard}{xml.dom}
\modulesynopsis{Document Object Model API for Python.}
\sectionauthor{Paul Prescod}{paul@prescod.net}
\sectionauthor{Martin v. L\"owis}{loewis@informatik.hu-berlin.de}

\versionadded{2.0}

The Document Object Model, or ``DOM,'' is a cross-language API from
the World Wide Web Consortium (W3C) for accessing and modifying XML
documents.  A DOM implementation presents an XML document as a tree
structure, or allows client code to build such a structure from
scratch.  It then gives access to the structure through a set of
objects which provided well-known interfaces.

The DOM is extremely useful for random-access applications.  SAX only
allows you a view of one bit of the document at a time.  If you are
looking at one SAX element, you have no access to another.  If you are
looking at a text node, you have no access to a containing element.
When you write a SAX application, you need to keep track of your
program's position in the document somewhere in your own code.  SAX
does not do it for you.  Also, if you need to look ahead in the XML
document, you are just out of luck.

Some applications are simply impossible in an event driven model with
no access to a tree.  Of course you could build some sort of tree
yourself in SAX events, but the DOM allows you to avoid writing that
code.  The DOM is a standard tree representation for XML data.

%What if your needs are somewhere between SAX and the DOM?  Perhaps
%you cannot afford to load the entire tree in memory but you find the
%SAX model somewhat cumbersome and low-level.  There is also a module
%called xml.dom.pulldom that allows you to build trees of only the
%parts of a document that you need structured access to.  It also has
%features that allow you to find your way around the DOM.
% See http://www.prescod.net/python/pulldom

The Document Object Model is being defined by the W3C in stages, or
``levels'' in their terminology.  The Python mapping of the API is
substantially based on the DOM Level~2 recommendation.  The mapping of
the Level~3 specification, currently only available in draft form, is
being developed by the \ulink{Python XML Special Interest
Group}{http://www.python.org/sigs/xml-sig/} as part of the
\ulink{PyXML package}{http://pyxml.sourceforge.net/}.  Refer to the
documentation bundled with that package for information on the current
state of DOM Level~3 support.

DOM applications typically start by parsing some XML into a DOM.  How
this is accomplished is not covered at all by DOM Level~1, and Level~2
provides only limited improvements: There is a
\class{DOMImplementation} object class which provides access to
\class{Document} creation methods, but no way to access an XML
reader/parser/Document builder in an implementation-independent way.
There is also no well-defined way to access these methods without an
existing \class{Document} object.  In Python, each DOM implementation
will provide a function \function{getDOMImplementation()}. DOM Level~3
adds a Load/Store specification, which defines an interface to the
reader, but this is not yet available in the Python standard library.

Once you have a DOM document object, you can access the parts of your
XML document through its properties and methods.  These properties are
defined in the DOM specification; this portion of the reference manual
describes the interpretation of the specification in Python.

The specification provided by the W3C defines the DOM API for Java,
ECMAScript, and OMG IDL.  The Python mapping defined here is based in
large part on the IDL version of the specification, but strict
compliance is not required (though implementations are free to support
the strict mapping from IDL).  See section \ref{dom-conformance},
``Conformance,'' for a detailed discussion of mapping requirements.


\begin{seealso}
  \seetitle[http://www.w3.org/TR/DOM-Level-2-Core/]{Document Object
            Model (DOM) Level~2 Specification}
           {The W3C recommendation upon which the Python DOM API is
            based.}
  \seetitle[http://www.w3.org/TR/REC-DOM-Level-1/]{Document Object
            Model (DOM) Level~1 Specification}
           {The W3C recommendation for the
            DOM supported by \module{xml.dom.minidom}.}
  \seetitle[http://pyxml.sourceforge.net]{PyXML}{Users that require a
            full-featured implementation of DOM should use the PyXML
            package.}
  \seetitle[http://cgi.omg.org/cgi-bin/doc?orbos/99-08-02.pdf]{CORBA
            Scripting with Python}
           {This specifies the mapping from OMG IDL to Python.}
\end{seealso}

\subsection{Module Contents}

The \module{xml.dom} contains the following functions:

\begin{funcdesc}{registerDOMImplementation}{name, factory}
Register the \var{factory} function with the name \var{name}.  The
factory function should return an object which implements the
\class{DOMImplementation} interface.  The factory function can return
the same object every time, or a new one for each call, as appropriate
for the specific implementation (e.g. if that implementation supports
some customization).
\end{funcdesc}

\begin{funcdesc}{getDOMImplementation}{\optional{name\optional{, features}}}
Return a suitable DOM implementation. The \var{name} is either
well-known, the module name of a DOM implementation, or
\code{None}. If it is not \code{None}, imports the corresponding
module and returns a \class{DOMImplementation} object if the import
succeeds.  If no name is given, and if the environment variable
\envvar{PYTHON_DOM} is set, this variable is used to find the
implementation.

If name is not given, this examines the available implementations to
find one with the required feature set.  If no implementation can be
found, raise an \exception{ImportError}.  The features list must be a
sequence of \code{(\var{feature}, \var{version})} pairs which are
passed to the \method{hasFeature()} method on available
\class{DOMImplementation} objects.
\end{funcdesc}


Some convenience constants are also provided:

\begin{datadesc}{EMPTY_NAMESPACE}
  The value used to indicate that no namespace is associated with a
  node in the DOM.  This is typically found as the
  \member{namespaceURI} of a node, or used as the \var{namespaceURI}
  parameter to a namespaces-specific method.
  \versionadded{2.2}
\end{datadesc}

\begin{datadesc}{XML_NAMESPACE}
  The namespace URI associated with the reserved prefix \code{xml}, as
  defined by
  \citetitle[http://www.w3.org/TR/REC-xml-names/]{Namespaces in XML}
  (section~4).
  \versionadded{2.2}
\end{datadesc}

\begin{datadesc}{XMLNS_NAMESPACE}
  The namespace URI for namespace declarations, as defined by
  \citetitle[http://www.w3.org/TR/DOM-Level-2-Core/core.html]{Document
  Object Model (DOM) Level~2 Core Specification} (section~1.1.8).
  \versionadded{2.2}
\end{datadesc}

\begin{datadesc}{XHTML_NAMESPACE}
  The URI of the XHTML namespace as defined by
  \citetitle[http://www.w3.org/TR/xhtml1/]{XHTML 1.0: The Extensible
  HyperText Markup Language} (section~3.1.1).
  \versionadded{2.2}
\end{datadesc}


% Should the Node documentation go here?

In addition, \module{xml.dom} contains a base \class{Node} class and
the DOM exception classes.  The \class{Node} class provided by this
module does not implement any of the methods or attributes defined by
the DOM specification; concrete DOM implementations must provide
those.  The \class{Node} class provided as part of this module does
provide the constants used for the \member{nodeType} attribute on
concrete \class{Node} objects; they are located within the class
rather than at the module level to conform with the DOM
specifications.


\subsection{Objects in the DOM \label{dom-objects}}

The definitive documentation for the DOM is the DOM specification from
the W3C.

Note that DOM attributes may also be manipulated as nodes instead of
as simple strings.  It is fairly rare that you must do this, however,
so this usage is not yet documented.


\begin{tableiii}{l|l|l}{class}{Interface}{Section}{Purpose}
  \lineiii{DOMImplementation}{\ref{dom-implementation-objects}}
          {Interface to the underlying implementation.}
  \lineiii{Node}{\ref{dom-node-objects}}
          {Base interface for most objects in a document.}
  \lineiii{NodeList}{\ref{dom-nodelist-objects}}
          {Interface for a sequence of nodes.}
  \lineiii{DocumentType}{\ref{dom-documenttype-objects}}
          {Information about the declarations needed to process a document.}
  \lineiii{Document}{\ref{dom-document-objects}}
          {Object which represents an entire document.}
  \lineiii{Element}{\ref{dom-element-objects}}
          {Element nodes in the document hierarchy.}
  \lineiii{Attr}{\ref{dom-attr-objects}}
          {Attribute value nodes on element nodes.}
  \lineiii{Comment}{\ref{dom-comment-objects}}
          {Representation of comments in the source document.}
  \lineiii{Text}{\ref{dom-text-objects}}
          {Nodes containing textual content from the document.}
  \lineiii{ProcessingInstruction}{\ref{dom-pi-objects}}
          {Processing instruction representation.}
\end{tableiii}

An additional section describes the exceptions defined for working
with the DOM in Python.


\subsubsection{DOMImplementation Objects
               \label{dom-implementation-objects}}

The \class{DOMImplementation} interface provides a way for
applications to determine the availability of particular features in
the DOM they are using.  DOM Level~2 added the ability to create new
\class{Document} and \class{DocumentType} objects using the
\class{DOMImplementation} as well.

\begin{methoddesc}[DOMImplementation]{hasFeature}{feature, version}
\end{methoddesc}


\subsubsection{Node Objects \label{dom-node-objects}}

All of the components of an XML document are subclasses of
\class{Node}.

\begin{memberdesc}[Node]{nodeType}
An integer representing the node type.  Symbolic constants for the
types are on the \class{Node} object:
\constant{ELEMENT_NODE}, \constant{ATTRIBUTE_NODE},
\constant{TEXT_NODE}, \constant{CDATA_SECTION_NODE},
\constant{ENTITY_NODE}, \constant{PROCESSING_INSTRUCTION_NODE},
\constant{COMMENT_NODE}, \constant{DOCUMENT_NODE},
\constant{DOCUMENT_TYPE_NODE}, \constant{NOTATION_NODE}.
This is a read-only attribute.
\end{memberdesc}

\begin{memberdesc}[Node]{parentNode}
The parent of the current node, or \code{None} for the document node.
The value is always a \class{Node} object or \code{None}.  For
\class{Element} nodes, this will be the parent element, except for the
root element, in which case it will be the \class{Document} object.
For \class{Attr} nodes, this is always \code{None}.
This is a read-only attribute.
\end{memberdesc}

\begin{memberdesc}[Node]{attributes}
A \class{NamedNodeMap} of attribute objects.  Only elements have
actual values for this; others provide \code{None} for this attribute.
This is a read-only attribute.
\end{memberdesc}

\begin{memberdesc}[Node]{previousSibling}
The node that immediately precedes this one with the same parent.  For
instance the element with an end-tag that comes just before the
\var{self} element's start-tag.  Of course, XML documents are made
up of more than just elements so the previous sibling could be text, a
comment, or something else.  If this node is the first child of the
parent, this attribute will be \code{None}.
This is a read-only attribute.
\end{memberdesc}

\begin{memberdesc}[Node]{nextSibling}
The node that immediately follows this one with the same parent.  See
also \member{previousSibling}.  If this is the last child of the
parent, this attribute will be \code{None}.
This is a read-only attribute.
\end{memberdesc}

\begin{memberdesc}[Node]{childNodes}
A list of nodes contained within this node.
This is a read-only attribute.
\end{memberdesc}

\begin{memberdesc}[Node]{firstChild}
The first child of the node, if there are any, or \code{None}.
This is a read-only attribute.
\end{memberdesc}

\begin{memberdesc}[Node]{lastChild}
The last child of the node, if there are any, or \code{None}.
This is a read-only attribute.
\end{memberdesc}

\begin{memberdesc}[Node]{localName}
The part of the \member{tagName} following the colon if there is one,
else the entire \member{tagName}.  The value is a string.
\end{memberdesc}

\begin{memberdesc}[Node]{prefix}
The part of the \member{tagName} preceding the colon if there is one,
else the empty string.  The value is a string, or \code{None}
\end{memberdesc}

\begin{memberdesc}[Node]{namespaceURI}
The namespace associated with the element name.  This will be a
string or \code{None}.  This is a read-only attribute.
\end{memberdesc}

\begin{memberdesc}[Node]{nodeName}
This has a different meaning for each node type; see the DOM
specification for details.  You can always get the information you
would get here from another property such as the \member{tagName}
property for elements or the \member{name} property for attributes.
For all node types, the value of this attribute will be either a
string or \code{None}.  This is a read-only attribute.
\end{memberdesc}

\begin{memberdesc}[Node]{nodeValue}
This has a different meaning for each node type; see the DOM
specification for details.  The situation is similar to that with
\member{nodeName}.  The value is a string or \code{None}.
\end{memberdesc}

\begin{methoddesc}[Node]{hasAttributes}{}
Returns true if the node has any attributes.
\end{methoddesc}

\begin{methoddesc}[Node]{hasChildNodes}{}
Returns true if the node has any child nodes.
\end{methoddesc}

\begin{methoddesc}[Node]{isSameNode}{other}
Returns true if \var{other} refers to the same node as this node.
This is especially useful for DOM implementations which use any sort
of proxy architecture (because more than one object can refer to the
same node).

\begin{notice}
  This is based on a proposed DOM Level~3 API which is still in the
  ``working draft'' stage, but this particular interface appears
  uncontroversial.  Changes from the W3C will not necessarily affect
  this method in the Python DOM interface (though any new W3C API for
  this would also be supported).
\end{notice}
\end{methoddesc}

\begin{methoddesc}[Node]{appendChild}{newChild}
Add a new child node to this node at the end of the list of children,
returning \var{newChild}.
\end{methoddesc}

\begin{methoddesc}[Node]{insertBefore}{newChild, refChild}
Insert a new child node before an existing child.  It must be the case
that \var{refChild} is a child of this node; if not,
\exception{ValueError} is raised.  \var{newChild} is returned.
\end{methoddesc}

\begin{methoddesc}[Node]{removeChild}{oldChild}
Remove a child node.  \var{oldChild} must be a child of this node; if
not, \exception{ValueError} is raised.  \var{oldChild} is returned on
success.  If \var{oldChild} will not be used further, its
\method{unlink()} method should be called.
\end{methoddesc}

\begin{methoddesc}[Node]{replaceChild}{newChild, oldChild}
Replace an existing node with a new node. It must be the case that 
\var{oldChild} is a child of this node; if not,
\exception{ValueError} is raised.
\end{methoddesc}

\begin{methoddesc}[Node]{normalize}{}
Join adjacent text nodes so that all stretches of text are stored as
single \class{Text} instances.  This simplifies processing text from a
DOM tree for many applications.
\versionadded{2.1}
\end{methoddesc}

\begin{methoddesc}[Node]{cloneNode}{deep}
Clone this node.  Setting \var{deep} means to clone all child nodes as
well.  This returns the clone.
\end{methoddesc}


\subsubsection{NodeList Objects \label{dom-nodelist-objects}}

A \class{NodeList} represents a sequence of nodes.  These objects are
used in two ways in the DOM Core recommendation:  the
\class{Element} objects provides one as it's list of child nodes, and
the \method{getElementsByTagName()} and
\method{getElementsByTagNameNS()} methods of \class{Node} return
objects with this interface to represent query results.

The DOM Level~2 recommendation defines one method and one attribute
for these objects:

\begin{methoddesc}[NodeList]{item}{i}
  Return the \var{i}'th item from the sequence, if there is one, or
  \code{None}.  The index \var{i} is not allowed to be less then zero
  or greater than or equal to the length of the sequence.
\end{methoddesc}

\begin{memberdesc}[NodeList]{length}
  The number of nodes in the sequence.
\end{memberdesc}

In addition, the Python DOM interface requires that some additional
support is provided to allow \class{NodeList} objects to be used as
Python sequences.  All \class{NodeList} implementations must include
support for \method{__len__()} and \method{__getitem__()}; this allows
iteration over the \class{NodeList} in \keyword{for} statements and
proper support for the \function{len()} built-in function.

If a DOM implementation supports modification of the document, the
\class{NodeList} implementation must also support the
\method{__setitem__()} and \method{__delitem__()} methods.


\subsubsection{DocumentType Objects \label{dom-documenttype-objects}}

Information about the notations and entities declared by a document
(including the external subset if the parser uses it and can provide
the information) is available from a \class{DocumentType} object.  The
\class{DocumentType} for a document is available from the
\class{Document} object's \member{doctype} attribute; if there is no
\code{DOCTYPE} declaration for the document, the document's
\member{doctype} attribute will be set to \code{None} instead of an
instance of this interface.

\class{DocumentType} is a specialization of \class{Node}, and adds the
following attributes:

\begin{memberdesc}[DocumentType]{publicId}
  The public identifier for the external subset of the document type
  definition.  This will be a string or \code{None}.
\end{memberdesc}

\begin{memberdesc}[DocumentType]{systemId}
  The system identifier for the external subset of the document type
  definition.  This will be a URI as a string, or \code{None}.
\end{memberdesc}

\begin{memberdesc}[DocumentType]{internalSubset}
  A string giving the complete internal subset from the document.
  This does not include the brackets which enclose the subset.  If the
  document has no internal subset, this should be \code{None}.
\end{memberdesc}

\begin{memberdesc}[DocumentType]{name}
  The name of the root element as given in the \code{DOCTYPE}
  declaration, if present.
\end{memberdesc}

\begin{memberdesc}[DocumentType]{entities}
  This is a \class{NamedNodeMap} giving the definitions of external
  entities.  For entity names defined more than once, only the first
  definition is provided (others are ignored as required by the XML
  recommendation).  This may be \code{None} if the information is not
  provided by the parser, or if no entities are defined.
\end{memberdesc}

\begin{memberdesc}[DocumentType]{notations}
  This is a \class{NamedNodeMap} giving the definitions of notations.
  For notation names defined more than once, only the first definition
  is provided (others are ignored as required by the XML
  recommendation).  This may be \code{None} if the information is not
  provided by the parser, or if no notations are defined.
\end{memberdesc}


\subsubsection{Document Objects \label{dom-document-objects}}

A \class{Document} represents an entire XML document, including its
constituent elements, attributes, processing instructions, comments
etc.  Remeber that it inherits properties from \class{Node}.

\begin{memberdesc}[Document]{documentElement}
The one and only root element of the document.
\end{memberdesc}

\begin{methoddesc}[Document]{createElement}{tagName}
Create and return a new element node.  The element is not inserted
into the document when it is created.  You need to explicitly insert
it with one of the other methods such as \method{insertBefore()} or
\method{appendChild()}.
\end{methoddesc}

\begin{methoddesc}[Document]{createElementNS}{namespaceURI, tagName}
Create and return a new element with a namespace.  The
\var{tagName} may have a prefix.  The element is not inserted into the
document when it is created.  You need to explicitly insert it with
one of the other methods such as \method{insertBefore()} or
\method{appendChild()}.
\end{methoddesc}

\begin{methoddesc}[Document]{createTextNode}{data}
Create and return a text node containing the data passed as a
parameter.  As with the other creation methods, this one does not
insert the node into the tree.
\end{methoddesc}

\begin{methoddesc}[Document]{createComment}{data}
Create and return a comment node containing the data passed as a
parameter.  As with the other creation methods, this one does not
insert the node into the tree.
\end{methoddesc}

\begin{methoddesc}[Document]{createProcessingInstruction}{target, data}
Create and return a processing instruction node containing the
\var{target} and \var{data} passed as parameters.  As with the other
creation methods, this one does not insert the node into the tree.
\end{methoddesc}

\begin{methoddesc}[Document]{createAttribute}{name}
Create and return an attribute node.  This method does not associate
the attribute node with any particular element.  You must use
\method{setAttributeNode()} on the appropriate \class{Element} object
to use the newly created attribute instance.
\end{methoddesc}

\begin{methoddesc}[Document]{createAttributeNS}{namespaceURI, qualifiedName}
Create and return an attribute node with a namespace.  The
\var{tagName} may have a prefix.  This method does not associate the
attribute node with any particular element.  You must use
\method{setAttributeNode()} on the appropriate \class{Element} object
to use the newly created attribute instance.
\end{methoddesc}

\begin{methoddesc}[Document]{getElementsByTagName}{tagName}
Search for all descendants (direct children, children's children,
etc.) with a particular element type name.
\end{methoddesc}

\begin{methoddesc}[Document]{getElementsByTagNameNS}{namespaceURI, localName}
Search for all descendants (direct children, children's children,
etc.) with a particular namespace URI and localname.  The localname is
the part of the namespace after the prefix.
\end{methoddesc}


\subsubsection{Element Objects \label{dom-element-objects}}

\class{Element} is a subclass of \class{Node}, so inherits all the
attributes of that class.

\begin{memberdesc}[Element]{tagName}
The element type name.  In a namespace-using document it may have
colons in it.  The value is a string.
\end{memberdesc}

\begin{methoddesc}[Element]{getElementsByTagName}{tagName}
Same as equivalent method in the \class{Document} class.
\end{methoddesc}

\begin{methoddesc}[Element]{getElementsByTagNameNS}{tagName}
Same as equivalent method in the \class{Document} class.
\end{methoddesc}

\begin{methoddesc}[Element]{getAttribute}{attname}
Return an attribute value as a string.
\end{methoddesc}

\begin{methoddesc}[Element]{getAttributeNode}{attrname}
Return the \class{Attr} node for the attribute named by
\var{attrname}.
\end{methoddesc}

\begin{methoddesc}[Element]{getAttributeNS}{namespaceURI, localName}
Return an attribute value as a string, given a \var{namespaceURI} and
\var{localName}.
\end{methoddesc}

\begin{methoddesc}[Element]{getAttributeNodeNS}{namespaceURI, localName}
Return an attribute value as a node, given a \var{namespaceURI} and
\var{localName}.
\end{methoddesc}

\begin{methoddesc}[Element]{removeAttribute}{attname}
Remove an attribute by name.  No exception is raised if there is no
matching attribute.
\end{methoddesc}

\begin{methoddesc}[Element]{removeAttributeNode}{oldAttr}
Remove and return \var{oldAttr} from the attribute list, if present.
If \var{oldAttr} is not present, \exception{NotFoundErr} is raised.
\end{methoddesc}

\begin{methoddesc}[Element]{removeAttributeNS}{namespaceURI, localName}
Remove an attribute by name.  Note that it uses a localName, not a
qname.  No exception is raised if there is no matching attribute.
\end{methoddesc}

\begin{methoddesc}[Element]{setAttribute}{attname, value}
Set an attribute value from a string.
\end{methoddesc}

\begin{methoddesc}[Element]{setAttributeNode}{newAttr}
Add a new attibute node to the element, replacing an existing
attribute if necessary if the \member{name} attribute matches.  If a
replacement occurs, the old attribute node will be returned.  If
\var{newAttr} is already in use, \exception{InuseAttributeErr} will be
raised.
\end{methoddesc}

\begin{methoddesc}[Element]{setAttributeNodeNS}{newAttr}
Add a new attibute node to the element, replacing an existing
attribute if necessary if the \member{namespaceURI} and
\member{localName} attributes match.  If a replacement occurs, the old
attribute node will be returned.  If \var{newAttr} is already in use,
\exception{InuseAttributeErr} will be raised.
\end{methoddesc}

\begin{methoddesc}[Element]{setAttributeNS}{namespaceURI, qname, value}
Set an attribute value from a string, given a \var{namespaceURI} and a
\var{qname}.  Note that a qname is the whole attribute name.  This is
different than above.
\end{methoddesc}


\subsubsection{Attr Objects \label{dom-attr-objects}}

\class{Attr} inherits from \class{Node}, so inherits all its
attributes.

\begin{memberdesc}[Attr]{name}
The attribute name.  In a namespace-using document it may have colons
in it.
\end{memberdesc}

\begin{memberdesc}[Attr]{localName}
The part of the name following the colon if there is one, else the
entire name.  This is a read-only attribute.
\end{memberdesc}

\begin{memberdesc}[Attr]{prefix}
The part of the name preceding the colon if there is one, else the
empty string.
\end{memberdesc}


\subsubsection{NamedNodeMap Objects \label{dom-attributelist-objects}}

\class{NamedNodeMap} does \emph{not} inherit from \class{Node}.

\begin{memberdesc}[NamedNodeMap]{length}
The length of the attribute list.
\end{memberdesc}

\begin{methoddesc}[NamedNodeMap]{item}{index}
Return an attribute with a particular index.  The order you get the
attributes in is arbitrary but will be consistent for the life of a
DOM.  Each item is an attribute node.  Get its value with the
\member{value} attribbute.
\end{methoddesc}

There are also experimental methods that give this class more mapping
behavior.  You can use them or you can use the standardized
\method{getAttribute*()} family of methods on the \class{Element}
objects.


\subsubsection{Comment Objects \label{dom-comment-objects}}

\class{Comment} represents a comment in the XML document.  It is a
subclass of \class{Node}, but cannot have child nodes.

\begin{memberdesc}[Comment]{data}
The content of the comment as a string.  The attribute contains all
characters between the leading \code{<!-}\code{-} and trailing
\code{-}\code{->}, but does not include them.
\end{memberdesc}


\subsubsection{Text and CDATASection Objects \label{dom-text-objects}}

The \class{Text} interface represents text in the XML document.  If
the parser and DOM implementation support the DOM's XML extension,
portions of the text enclosed in CDATA marked sections are stored in
\class{CDATASection} objects.  These two interfaces are identical, but
provide different values for the \member{nodeType} attribute.

These interfaces extend the \class{Node} interface.  They cannot have
child nodes.

\begin{memberdesc}[Text]{data}
The content of the text node as a string.
\end{memberdesc}

\begin{notice}
  The use of a \class{CDATASection} node does not indicate that the
  node represents a complete CDATA marked section, only that the
  content of the node was part of a CDATA section.  A single CDATA
  section may be represented by more than one node in the document
  tree.  There is no way to determine whether two adjacent
  \class{CDATASection} nodes represent different CDATA marked
  sections.
\end{notice}


\subsubsection{ProcessingInstruction Objects \label{dom-pi-objects}}

Represents a processing instruction in the XML document; this inherits
from the \class{Node} interface and cannot have child nodes.

\begin{memberdesc}[ProcessingInstruction]{target}
The content of the processing instruction up to the first whitespace
character.  This is a read-only attribute.
\end{memberdesc}

\begin{memberdesc}[ProcessingInstruction]{data}
The content of the processing instruction following the first
whitespace character.
\end{memberdesc}


\subsubsection{Exceptions \label{dom-exceptions}}

\versionadded{2.1}

The DOM Level~2 recommendation defines a single exception,
\exception{DOMException}, and a number of constants that allow
applications to determine what sort of error occurred.
\exception{DOMException} instances carry a \member{code} attribute
that provides the appropriate value for the specific exception.

The Python DOM interface provides the constants, but also expands the
set of exceptions so that a specific exception exists for each of the
exception codes defined by the DOM.  The implementations must raise
the appropriate specific exception, each of which carries the
appropriate value for the \member{code} attribute.

\begin{excdesc}{DOMException}
  Base exception class used for all specific DOM exceptions.  This
  exception class cannot be directly instantiated.
\end{excdesc}

\begin{excdesc}{DomstringSizeErr}
  Raised when a specified range of text does not fit into a string.
  This is not known to be used in the Python DOM implementations, but
  may be received from DOM implementations not written in Python.
\end{excdesc}

\begin{excdesc}{HierarchyRequestErr}
  Raised when an attempt is made to insert a node where the node type
  is not allowed.
\end{excdesc}

\begin{excdesc}{IndexSizeErr}
  Raised when an index or size parameter to a method is negative or
  exceeds the allowed values.
\end{excdesc}

\begin{excdesc}{InuseAttributeErr}
  Raised when an attempt is made to insert an \class{Attr} node that
  is already present elsewhere in the document.
\end{excdesc}

\begin{excdesc}{InvalidAccessErr}
  Raised if a parameter or an operation is not supported on the
  underlying object.
\end{excdesc}

\begin{excdesc}{InvalidCharacterErr}
  This exception is raised when a string parameter contains a
  character that is not permitted in the context it's being used in by
  the XML 1.0 recommendation.  For example, attempting to create an
  \class{Element} node with a space in the element type name will
  cause this error to be raised.
\end{excdesc}

\begin{excdesc}{InvalidModificationErr}
  Raised when an attempt is made to modify the type of a node.
\end{excdesc}

\begin{excdesc}{InvalidStateErr}
  Raised when an attempt is made to use an object that is not or is no
  longer usable.
\end{excdesc}

\begin{excdesc}{NamespaceErr}
  If an attempt is made to change any object in a way that is not
  permitted with regard to the
  \citetitle[http://www.w3.org/TR/REC-xml-names/]{Namespaces in XML}
  recommendation, this exception is raised.
\end{excdesc}

\begin{excdesc}{NotFoundErr}
  Exception when a node does not exist in the referenced context.  For
  example, \method{NamedNodeMap.removeNamedItem()} will raise this if
  the node passed in does not exist in the map.
\end{excdesc}

\begin{excdesc}{NotSupportedErr}
  Raised when the implementation does not support the requested type
  of object or operation.
\end{excdesc}

\begin{excdesc}{NoDataAllowedErr}
  This is raised if data is specified for a node which does not
  support data.
  % XXX  a better explanation is needed!
\end{excdesc}

\begin{excdesc}{NoModificationAllowedErr}
  Raised on attempts to modify an object where modifications are not
  allowed (such as for read-only nodes).
\end{excdesc}

\begin{excdesc}{SyntaxErr}
  Raised when an invalid or illegal string is specified.
  % XXX  how is this different from InvalidCharacterErr ???
\end{excdesc}

\begin{excdesc}{WrongDocumentErr}
  Raised when a node is inserted in a different document than it
  currently belongs to, and the implementation does not support
  migrating the node from one document to the other.
\end{excdesc}

The exception codes defined in the DOM recommendation map to the
exceptions described above according to this table:

\begin{tableii}{l|l}{constant}{Constant}{Exception}
  \lineii{DOMSTRING_SIZE_ERR}{\exception{DomstringSizeErr}}
  \lineii{HIERARCHY_REQUEST_ERR}{\exception{HierarchyRequestErr}}
  \lineii{INDEX_SIZE_ERR}{\exception{IndexSizeErr}}
  \lineii{INUSE_ATTRIBUTE_ERR}{\exception{InuseAttributeErr}}
  \lineii{INVALID_ACCESS_ERR}{\exception{InvalidAccessErr}}
  \lineii{INVALID_CHARACTER_ERR}{\exception{InvalidCharacterErr}}
  \lineii{INVALID_MODIFICATION_ERR}{\exception{InvalidModificationErr}}
  \lineii{INVALID_STATE_ERR}{\exception{InvalidStateErr}}
  \lineii{NAMESPACE_ERR}{\exception{NamespaceErr}}
  \lineii{NOT_FOUND_ERR}{\exception{NotFoundErr}}
  \lineii{NOT_SUPPORTED_ERR}{\exception{NotSupportedErr}}
  \lineii{NO_DATA_ALLOWED_ERR}{\exception{NoDataAllowedErr}}
  \lineii{NO_MODIFICATION_ALLOWED_ERR}{\exception{NoModificationAllowedErr}}
  \lineii{SYNTAX_ERR}{\exception{SyntaxErr}}
  \lineii{WRONG_DOCUMENT_ERR}{\exception{WrongDocumentErr}}
\end{tableii}


\subsection{Conformance \label{dom-conformance}}

This section describes the conformance requirements and relationships
between the Python DOM API, the W3C DOM recommendations, and the OMG
IDL mapping for Python.


\subsubsection{Type Mapping \label{dom-type-mapping}}

The primitive IDL types used in the DOM specification are mapped to
Python types according to the following table.

\begin{tableii}{l|l}{code}{IDL Type}{Python Type}
  \lineii{boolean}{\code{IntegerType} (with a value of \code{0} or \code{1})}
  \lineii{int}{\code{IntegerType}}
  \lineii{long int}{\code{IntegerType}}
  \lineii{unsigned int}{\code{IntegerType}}
\end{tableii}

Additionally, the \class{DOMString} defined in the recommendation is
mapped to a Python string or Unicode string.  Applications should
be able to handle Unicode whenever a string is returned from the DOM.

The IDL \keyword{null} value is mapped to \code{None}, which may be
accepted or provided by the implementation whenever \keyword{null} is
allowed by the API.


\subsubsection{Accessor Methods \label{dom-accessor-methods}}

The mapping from OMG IDL to Python defines accessor functions for IDL
\keyword{attribute} declarations in much the way the Java mapping
does.  Mapping the IDL declarations

\begin{verbatim}
readonly attribute string someValue;
         attribute string anotherValue;
\end{verbatim}

yields three accessor functions:  a ``get'' method for
\member{someValue} (\method{_get_someValue()}), and ``get'' and
``set'' methods for
\member{anotherValue} (\method{_get_anotherValue()} and
\method{_set_anotherValue()}).  The mapping, in particular, does not
require that the IDL attributes are accessible as normal Python
attributes:  \code{\var{object}.someValue} is \emph{not} required to
work, and may raise an \exception{AttributeError}.

The Python DOM API, however, \emph{does} require that normal attribute
access work.  This means that the typical surrogates generated by
Python IDL compilers are not likely to work, and wrapper objects may
be needed on the client if the DOM objects are accessed via CORBA.
While this does require some additional consideration for CORBA DOM
clients, the implementers with experience using DOM over CORBA from
Python do not consider this a problem.  Attributes that are declared
\keyword{readonly} may not restrict write access in all DOM
implementations.

Additionally, the accessor functions are not required.  If provided,
they should take the form defined by the Python IDL mapping, but
these methods are considered unnecessary since the attributes are
accessible directly from Python.  ``Set'' accessors should never be
provided for \keyword{readonly} attributes.
