\chapter{Undocumented Modules}
\label{undoc}

Here's a quick listing of modules that are currently undocumented, but
that should be documented.  Feel free to contribute documentation for
them!  (The idea and original contents for this chapter were taken
from a posting by Fredrik Lundh; I have revised some modules' status.)


\section{Frameworks}

Frameworks tend to be harder to document, but are well worth the
effort spent.

\begin{description}
\item[\module{Tkinter}]
--- Interface to Tcl/Tk for graphical user interfaces; Fredrik Lundh
is working on this one!  See
\citetitle[http://www.pythonware.com/library.htm]{An Introduction to
Tkinter} at \url{http://www.pythonware.com/library.htm} for on-line
reference material.

\item[\module{Tkdnd}]
--- Drag-and-drop support for \module{Tkinter}.

\item[\module{turtle}]
--- Turtle graphics in a Tk window.

\item[\module{test}]
--- Regression testing framework.  This is used for the Python
regression test, but is useful for other Python libraries as well.
This is a package rather than a module.
\end{description}


\section{Miscellaneous useful utilities}

Some of these are very old and/or not very robust; marked with ``hmm.''

\begin{description}
\item[\module{dircmp}]
--- Class to build directory diff tools on (may become a demo or tool).
\deprecated{1.6}{The \refmodule{filecmp} module will replace
\module{dircmp}.}

\item[\module{bdb}]
--- A generic Python debugger base class (used by pdb)

\item[\module{ihooks}]
--- Import hook support (for \refmodule{rexec}; may become obsolete)

\item[\module{tzparse}]
--- Parse a timezone specification (unfinished; may disappear in the
future)
\end{description}


\section{Platform specific modules}

These modules are used to implement the \refmodule{os.path} module,
and are not documented beyond this mention.  There's little need to
document these.

\begin{description}
\item[\module{dospath}]
--- implementation of \module{os.path} on MS-DOS

\item[\module{ntpath}]
--- implementation on \module{os.path} on 32-bit Windows

\item[\module{posixpath}]
--- implementation on \module{os.path} on \POSIX{}
\end{description}


\section{Multimedia}

\begin{description}
\item[\module{audiodev}]
--- Platform-independent API for playing audio data

\item[\module{sunaudio}]
--- Interpret Sun audio headers (may become obsolete or a tool/demo)

\item[\module{toaiff}]
--- Convert "arbitrary" sound files to AIFF files; should probably
become a tool or demo.  Requires the external program \program{sox}.
\end{description}


\section{Obsolete \label{obsolete-modules}}

These modules are not normally available for import; additional work
must be done to make them available.

Those which are written in Python will be installed into the directory 
\file{lib-old/} installed as part of the standard library.  To use
these, the directory must be added to \code{sys.path}, possibly using
\envvar{PYTHONPATH}.

Obsolete extension modules written in C are not built by default.
Under \UNIX, these must be enabled by uncommenting the appropriate
lines in \file{Modules/Setup} in the build tree and either rebuilding
Python if the modules are statically linked, or building and
installing the shared object if using dynamically-loaded extensions.

% XXX need Windows instructions!

\begin{description}
\item[\module{addpack}]
--- alternate approach to packages

\item[\module{cmp}]
--- File comparison function.  Use the newer \refmodule{filecmp} instead.

\item[\module{cmpcache}]
--- Caching version of the obsolete \module{cmp} module.  Use the
newer \refmodule{filecmp} instead.

\item[\module{codehack}]
--- Extract function name or line number from a function
code object (these are now accessible as attributes:
\member{co.co_name}, \member{func.func_name},
\member{co.co_firstlineno}).

\item[\module{dircmp}]
--- class to build directory diff tools on (may become a demo or tool)

\item[\module{dump}]
--- Print python code that reconstructs a variable

\item[\module{fmt}]
--- text formatting abstractions (too slow)

\item[\module{lockfile}]
--- wrapper around FCNTL file locking (use
\function{fcntl.lockf()}/\function{flock()} intead; see \refmodule{fcntl})

\item[\module{newdir}]
--- New \function{dir()} function (the standard \function{dir()} is
now just as good)

\item[\module{Para}]
--- helper for fmt.py

\item[\module{poly}]
--- Polynomials

\item[\module{tb}]
--- Print tracebacks, with a dump of local variables (use
\function{pdb.pm()} or \refmodule{traceback} instead)

\item[\module{timing}]
--- Measure time intervals to high resolution (use
\function{time.clock()} instead).  (This is an extension module.)

\item[\module{util}]
--- Useful functions that don't fit elsewhere.

\item[\module{wdb}]
--- A primitive windowing debugger based on STDWIN.

\item[\module{whatsound}]
--- Recognize sound files; use \refmodule{sndhdr} instead.

\item[\module{zmod}]
--- Compute properties of mathematical "fields"
\end{description}


The following modules are obsolete, but are likely re-surface as tools
or scripts.

\begin{description}
\item[\module{find}]
--- find files matching pattern in directory tree

\item[\module{grep}]
--- grep

\item[\module{packmail}]
--- create a self-unpacking \UNIX{} shell archive
\end{description}


The following modules were documented in previous versions of this
manual, but are now considered obsolete.  The source for the
documentation is still available as part of the documentation source
archive.

\begin{description}
\item[\module{ni}]
--- Import modules in ``packages.''  Basic package support is now
built in.

\item[\module{rand}]
--- Old interface to the random number generator.

\item[\module{soundex}]
--- Algorithm for collapsing names which sound similar to a shared
key.  (This is an extension module.)
\end{description}


\section{Extension modules}

\begin{description}
\item[\module{stdwin}]
--- Interface to STDWIN (an old, unsupported
platform-independent GUI package).  Obsolete; use \module{Tkinter} for
a platform-independent GUI instead.
\end{description}

The following are SGI specific, and may be out of touch with the
current version of reality.

\begin{description}
\item[\module{cl}]
--- Interface to the SGI compression library.

\item[\module{sv}]
--- Interface to the ``simple video'' board on SGI Indigo
(obsolete hardware).
\end{description}
