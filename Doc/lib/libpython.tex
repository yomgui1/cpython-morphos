\chapter{Python Services}

The modules described in this chapter provide a wide range of services
related to the Python interpreter and its interaction with its
environment.  Here's an overview:

\begin{description}

\item[sys]
--- Access system specific parameters and functions.

\item[types]
--- Names for all built-in types.

\item[UserDict]
\item[UserList]
--- Class wrappers for dictionary and list objects.

\item[operator]
--- All python's standard operators as built-in functions.

\item[traceback]
--- Print or retrieve a stack traceback.

\item[pickle]
--- Convert Python objects to streams of bytes and back.

\item[copy_reg]
--- Register \code{pickle} support functions.

\item[shelve]
--- Python object persistency.

\item[copy]
--- Shallow and deep copy operations.

\item[marshal]
--- Convert Python objects to streams of bytes and back (with
different constraints).

\item[imp]
--- Access the implementation of the \code{import} statement.

\item[ni]
--- New import (obsolete).

\item[parser]
--- Retrieve and submit parse trees from and to the runtime support
environment.

\item[symbol]
--- Constants representing internal nodes of the parse tree.

\item[token]
--- Constants representing terminal nodes of the parse tree.

\item[keyword]
--- Test whether a string is a keyword in the Python language.

\item[code]
--- Code object services.

\item[pprint]
--- Data pretty printer.

\item[dis]
--- Disassembler.

\item[site]
--- A standard way to reference site-specific modules.

\item[user]
--- A standard way to reference user-specific modules.

\item[__builtin__]
--- The set of built-in functions.

\item[__main__]
--- The environment where the top-level script is run.

\end{description}
