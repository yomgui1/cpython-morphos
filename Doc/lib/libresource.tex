\section{Built-in Module \sectcode{resource}}

\bimodindex{resource}
This module provides basic mechanisms for measuring and controlling
system resources utilized by a program.

Symbolic constants are used to specify particular system resources and
to request usage information about either the current process or its
children. 

Resources usage can be limited using the \code{setrlimit} function
described below. Each resource is controlled by a pair of limits: a
soft limit and a hard limit. The soft limit is the current limit, and
may be lowered or raised by a process over time. The soft limit can
never exceed the hard limit. The hard limit can be lowered to any
value greater than the soft limit, but not raised. (Only process with
the effective UID of the super-user can raise a hard limit).

The specific resources that can be limited are system dependent. They
are described in the \code{getrlimit} man page. Typical resources
include:

\begin{description}

\item[RLIMIT_CORE]
The maximum size (in bytes) of a core file that the current process
can create.

\item[RLIMIT_CPU]
The maximum amount of CPU time (in seconds) that a process can use. If
this limit is exceeded, a \code{SIGXCPU} signal is sent to the
process. (See the \code{signal} module documentation for information
about how to catch this signal and do something useful, e.g. flush
open files to disk.)

\end{description}

\begin{datadesc}{RLIMIT_*}
  These symbols define resources whose consumption can be controlled
  using the \code{setrlimit} and \code{getrlimit} functions defined
  below. The values of these symbols are exactly the constants used
  by C programs.

  The \UNIX{} man page for \file{getrlimit} lists the available
  resources. Note that not all systems use the same symbol or same
  value to denote the same resource.
\end{datadesc}

\begin{datadesc}{RUSAGE_*}
  These symbols are passed to the \code{getrusage} function to specify
  whether usage information is being request for the current process,
  \code{RUSAGE_SELF} or its child processes \code{RUSAGE_CHILDREN}. On
  some system, \code{RUSAGE_BOTH} requests information for both.
\end{datadesc}

\begin{datadesc}{error}
  The functions described below may raise this error if the underlying
  system call failures unexpectedly.
\end{datadesc}

The resource module defines the following functions:

\begin{funcdesc}{getrusage}{who}
  This function returns a large tuple that describes the resources
  consumed by either the current process or its children, as specified
  by the \var{who} parameter. The elements of the return value each
  describe how a particular system resource has been used, e.g. amount
  of time spent running is user mode or number of times the process was
  swapped out of main memory. Some values are dependent on the clock
  tick internal, e.g. the amount of memory the process is using.

  The first two elements of the return value are floating point values
  representing the amount of time spent executing in user mode and the
  amount of time spent executing in system mode, respectively. The
  remaining values are integers. Consult the \code{getrusage} man page
  for detailed information about these values. A brief summary is
  presented here:

\begin{tabular}{rl}
	\emph{offset} &	\emph{resource} \\
	0  &	time in user mode (float) \\
	1  &	time in system mode (float) \\
	2  &	maximum resident set size \\
	3  &	shared memory size \\
	4  &	unshared memory size \\
	5  &	unshared stack size \\
	6  &	page faults not requiring I/O \\
	7  &	page faults requiring I/O \\
	8  &	number of swap outs \\
	9  &	block input operations \\
	10 &	block output operations \\
	11 &	messages sent \\
	12 &	messages received \\
	13 &	signals received \\
	14 &	voluntary context switches \\
	15 &	involuntary context switches \\
\end{tabular}

  This function will raise a ValueError if an invalid \var{who}
  parameter is specified. It may also raise a \code{resource.error}
  exception in unusual circumstances.
\end{funcdesc}

\begin{funcdesc}{getpagesize}{}
  Returns the number of bytes in a system page. (This need not be the
  same as the hardware page size.) This function is useful for
  determining the number of bytes of memory a process is using. The
  third element of the tuple returned by \code{getrusage} describes
  memory usage in pages; multiplying by page size produces number of
  bytes. 
\end{funcdesc}

\begin{funcdesc}{getrlimit}{resource}
  Returns a tuple \code{(\var{soft}, \var{hard})} with the current
  soft and hard limits of \var{resource}. Raises ValueError if
  an invalid resource is specified, or \code{resource.error} if the
  underyling system call fails unexpectedly.
\end{funcdesc}

\begin{funcdesc}{setrlimit}{resource\, limits}
  Sets new limits of consumption of \var{resource}. The \var{limits}
  argument must be a tuple \code{(\var{soft}, \var{hard})} of two
  integers describing the new limits. A value of -1 can be used to
  specify the maximum possible upper limit.

  Raises ValueError if an invalid resource is specified, if the new
  soft limit exceeds the hard limit, or if a process tries to raise its
  hard limit (unless the process has an effective UID of
  super-user). Can also raise a \code{resource.error} if the
  underyling system call fails.
\end{funcdesc}
