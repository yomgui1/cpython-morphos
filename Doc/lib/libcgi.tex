\section{\module{cgi} ---
         Common Gateway Interface support.}
\declaremodule{standard}{cgi}

\modulesynopsis{Common Gateway Interface support, used to interpret
forms in server-side scripts.}

\indexii{WWW}{server}
\indexii{CGI}{protocol}
\indexii{HTTP}{protocol}
\indexii{MIME}{headers}
\index{URL}


Support module for Common Gateway Interface (CGI) scripts.%
\index{Common Gateway Interface}

This module defines a number of utilities for use by CGI scripts
written in Python.

\subsection{Introduction}
\nodename{cgi-intro}

A CGI script is invoked by an HTTP server, usually to process user
input submitted through an HTML \code{<FORM>} or \code{<ISINDEX>} element.

Most often, CGI scripts live in the server's special \file{cgi-bin}
directory.  The HTTP server places all sorts of information about the
request (such as the client's hostname, the requested URL, the query
string, and lots of other goodies) in the script's shell environment,
executes the script, and sends the script's output back to the client.

The script's input is connected to the client too, and sometimes the
form data is read this way; at other times the form data is passed via
the ``query string'' part of the URL.  This module is intended
to take care of the different cases and provide a simpler interface to
the Python script.  It also provides a number of utilities that help
in debugging scripts, and the latest addition is support for file
uploads from a form (if your browser supports it --- Grail 0.3 and
Netscape 2.0 do).

The output of a CGI script should consist of two sections, separated
by a blank line.  The first section contains a number of headers,
telling the client what kind of data is following.  Python code to
generate a minimal header section looks like this:

\begin{verbatim}
print "Content-Type: text/html"     # HTML is following
print                               # blank line, end of headers
\end{verbatim}

The second section is usually HTML, which allows the client software
to display nicely formatted text with header, in-line images, etc.
Here's Python code that prints a simple piece of HTML:

\begin{verbatim}
print "<TITLE>CGI script output</TITLE>"
print "<H1>This is my first CGI script</H1>"
print "Hello, world!"
\end{verbatim}

\subsection{Using the cgi module}
\nodename{Using the cgi module}

Begin by writing \samp{import cgi}.  Do not use \samp{from cgi import
*} --- the module defines all sorts of names for its own use or for
backward compatibility that you don't want in your namespace.

When you write a new script, consider adding the line:

\begin{verbatim}
import cgitb; cgitb.enable()
\end{verbatim}

This activates a special exception handler that will display detailed
reports in the Web browser if any errors occur.  If you'd rather not
show the guts of your program to users of your script, you can have
the reports saved to files instead, with a line like this:

\begin{verbatim}
import cgitb; cgitb.enable(display=0, logdir="/tmp")
\end{verbatim}

It's very helpful to use this feature during script development.
The reports produced by \refmodule{cgitb} provide information that
can save you a lot of time in tracking down bugs.  You can always
remove the \code{cgitb} line later when you have tested your script
and are confident that it works correctly.

To get at submitted form data,
it's best to use the \class{FieldStorage} class.  The other classes
defined in this module are provided mostly for backward compatibility.
Instantiate it exactly once, without arguments.  This reads the form
contents from standard input or the environment (depending on the
value of various environment variables set according to the CGI
standard).  Since it may consume standard input, it should be
instantiated only once.

The \class{FieldStorage} instance can be indexed like a Python
dictionary, and also supports the standard dictionary methods
\method{has_key()} and \method{keys()}.  The built-in \function{len()}
is also supported.  Form fields containing empty strings are ignored
and do not appear in the dictionary; to keep such values, provide
a true value for the optional \var{keep_blank_values} keyword
parameter when creating the \class{FieldStorage} instance.

For instance, the following code (which assumes that the 
\mailheader{Content-Type} header and blank line have already been
printed) checks that the fields \code{name} and \code{addr} are both
set to a non-empty string:

\begin{verbatim}
form = cgi.FieldStorage()
if not (form.has_key("name") and form.has_key("addr")):
    print "<H1>Error</H1>"
    print "Please fill in the name and addr fields."
    return
print "<p>name:", form["name"].value
print "<p>addr:", form["addr"].value
...further form processing here...
\end{verbatim}

Here the fields, accessed through \samp{form[\var{key}]}, are
themselves instances of \class{FieldStorage} (or
\class{MiniFieldStorage}, depending on the form encoding).
The \member{value} attribute of the instance yields the string value
of the field.  The \method{getvalue()} method returns this string value
directly; it also accepts an optional second argument as a default to
return if the requested key is not present.

If the submitted form data contains more than one field with the same
name, the object retrieved by \samp{form[\var{key}]} is not a
\class{FieldStorage} or \class{MiniFieldStorage}
instance but a list of such instances.  Similarly, in this situation,
\samp{form.getvalue(\var{key})} would return a list of strings.
If you expect this possibility
(when your HTML form contains multiple fields with the same name), use
the \function{getlist()} function, which always returns a list of values (so that you
do not need to special-case the single item case).  For example, this
code concatenates any number of username fields, separated by
commas:

\begin{verbatim}
value = form.getlist("username")
usernames = ",".join(value)
\end{verbatim}

If a field represents an uploaded file, accessing the value via the
\member{value} attribute or the \function{getvalue()} method reads the
entire file in memory as a string.  This may not be what you want.
You can test for an uploaded file by testing either the \member{filename}
attribute or the \member{file} attribute.  You can then read the data at
leisure from the \member{file} attribute:

\begin{verbatim}
fileitem = form["userfile"]
if fileitem.file:
    # It's an uploaded file; count lines
    linecount = 0
    while 1:
        line = fileitem.file.readline()
        if not line: break
        linecount = linecount + 1
\end{verbatim}

The file upload draft standard entertains the possibility of uploading
multiple files from one field (using a recursive
\mimetype{multipart/*} encoding).  When this occurs, the item will be
a dictionary-like \class{FieldStorage} item.  This can be determined
by testing its \member{type} attribute, which should be
\mimetype{multipart/form-data} (or perhaps another MIME type matching
\mimetype{multipart/*}).  In this case, it can be iterated over
recursively just like the top-level form object.

When a form is submitted in the ``old'' format (as the query string or
as a single data part of type
\mimetype{application/x-www-form-urlencoded}), the items will actually
be instances of the class \class{MiniFieldStorage}.  In this case, the
\member{list}, \member{file}, and \member{filename} attributes are
always \code{None}.


\subsection{Higher Level Interface}

\versionadded{2.2}  % XXX: Is this true ? 

The previous section explains how to read CGI form data using the
\class{FieldStorage} class.  This section describes a higher level
interface which was added to this class to allow one to do it in a
more readable and intuitive way.  The interface doesn't make the
techniques described in previous sections obsolete --- they are still
useful to process file uploads efficiently, for example.

The interface consists of two simple methods. Using the methods
you can process form data in a generic way, without the need to worry
whether only one or more values were posted under one name.

In the previous section, you learned to write following code anytime
you expected a user to post more than one value under one name:

\begin{verbatim}
item = form.getvalue("item")
if isinstance(item, list):
    # The user is requesting more than one item.
else:
    # The user is requesting only one item.
\end{verbatim}

This situation is common for example when a form contains a group of
multiple checkboxes with the same name:

\begin{verbatim}
<input type="checkbox" name="item" value="1" />
<input type="checkbox" name="item" value="2" />
\end{verbatim}

In most situations, however, there's only one form control with a
particular name in a form and then you expect and need only one value
associated with this name.  So you write a script containing for
example this code:

\begin{verbatim}
user = form.getvalue("user").upper()
\end{verbatim}

The problem with the code is that you should never expect that a
client will provide valid input to your scripts.  For example, if a
curious user appends another \samp{user=foo} pair to the query string,
then the script would crash, because in this situation the
\code{getvalue("user")} method call returns a list instead of a
string.  Calling the \method{toupper()} method on a list is not valid
(since lists do not have a method of this name) and results in an
\exception{AttributeError} exception.

Therefore, the appropriate way to read form data values was to always
use the code which checks whether the obtained value is a single value
or a list of values.  That's annoying and leads to less readable
scripts.

A more convenient approach is to use the methods \method{getfirst()}
and \method{getlist()} provided by this higher level interface.

\begin{methoddesc}[FieldStorage]{getfirst}{name\optional{, default}}
  Thin method always returns only one value associated with form field
  \var{name}.  The method returns only the first value in case that
  more values were posted under such name.  Please note that the order
  in which the values are received may vary from browser to browser
  and should not be counted on.\footnote{Note that some recent
      versions of the HTML specification do state what order the
      field values should be supplied in, but knowing whether a
      request was received from a conforming browser, or even from a
      browser at all, is tedious and error-prone.}  If no such form
  field or value exists then the method returns the value specified by
  the optional parameter \var{default}.  This parameter defaults to
  \code{None} if not specified.
\end{methoddesc}

\begin{methoddesc}[FieldStorage]{getlist}{name}
  This method always returns a list of values associated with form
  field \var{name}.  The method returns an empty list if no such form
  field or value exists for \var{name}.  It returns a list consisting
  of one item if only one such value exists.
\end{methoddesc}

Using these methods you can write nice compact code:

\begin{verbatim}
import cgi
form = cgi.FieldStorage()
user = form.getfirst("user", "").upper()    # This way it's safe.
for item in form.getlist("item"):
    do_something(item)
\end{verbatim}


\subsection{Old classes}

These classes, present in earlier versions of the \module{cgi} module,
are still supported for backward compatibility.  New applications
should use the \class{FieldStorage} class.

\class{SvFormContentDict} stores single value form content as
dictionary; it assumes each field name occurs in the form only once.

\class{FormContentDict} stores multiple value form content as a
dictionary (the form items are lists of values).  Useful if your form
contains multiple fields with the same name.

Other classes (\class{FormContent}, \class{InterpFormContentDict}) are
present for backwards compatibility with really old applications only.
If you still use these and would be inconvenienced when they
disappeared from a next version of this module, drop me a note.


\subsection{Functions}
\nodename{Functions in cgi module}

These are useful if you want more control, or if you want to employ
some of the algorithms implemented in this module in other
circumstances.

\begin{funcdesc}{parse}{fp\optional{, keep_blank_values\optional{,
                        strict_parsing}}}
  Parse a query in the environment or from a file (the file defaults
  to \code{sys.stdin}).  The \var{keep_blank_values} and
  \var{strict_parsing} parameters are passed to \function{parse_qs()}
  unchanged.
\end{funcdesc}

\begin{funcdesc}{parse_qs}{qs\optional{, keep_blank_values\optional{,
                           strict_parsing}}}
Parse a query string given as a string argument (data of type 
\mimetype{application/x-www-form-urlencoded}).  Data are
returned as a dictionary.  The dictionary keys are the unique query
variable names and the values are lists of values for each name.

The optional argument \var{keep_blank_values} is
a flag indicating whether blank values in
URL encoded queries should be treated as blank strings.  
A true value indicates that blanks should be retained as 
blank strings.  The default false value indicates that
blank values are to be ignored and treated as if they were
not included.

The optional argument \var{strict_parsing} is a flag indicating what
to do with parsing errors.  If false (the default), errors
are silently ignored.  If true, errors raise a ValueError
exception.

Use the \function{\refmodule{urllib}.urlencode()} function to convert
such dictionaries into query strings.

\end{funcdesc}

\begin{funcdesc}{parse_qsl}{qs\optional{, keep_blank_values\optional{,
                            strict_parsing}}}
Parse a query string given as a string argument (data of type 
\mimetype{application/x-www-form-urlencoded}).  Data are
returned as a list of name, value pairs.

The optional argument \var{keep_blank_values} is
a flag indicating whether blank values in
URL encoded queries should be treated as blank strings.  
A true value indicates that blanks should be retained as 
blank strings.  The default false value indicates that
blank values are to be ignored and treated as if they were
not included.

The optional argument \var{strict_parsing} is a flag indicating what
to do with parsing errors.  If false (the default), errors
are silently ignored.  If true, errors raise a ValueError
exception.

Use the \function{\refmodule{urllib}.urlencode()} function to convert
such lists of pairs into query strings.
\end{funcdesc}

\begin{funcdesc}{parse_multipart}{fp, pdict}
Parse input of type \mimetype{multipart/form-data} (for 
file uploads).  Arguments are \var{fp} for the input file and
\var{pdict} for a dictionary containing other parameters in
the \mailheader{Content-Type} header.

Returns a dictionary just like \function{parse_qs()} keys are the
field names, each value is a list of values for that field.  This is
easy to use but not much good if you are expecting megabytes to be
uploaded --- in that case, use the \class{FieldStorage} class instead
which is much more flexible.

Note that this does not parse nested multipart parts --- use
\class{FieldStorage} for that.
\end{funcdesc}

\begin{funcdesc}{parse_header}{string}
Parse a MIME header (such as \mailheader{Content-Type}) into a main
value and a dictionary of parameters.
\end{funcdesc}

\begin{funcdesc}{test}{}
Robust test CGI script, usable as main program.
Writes minimal HTTP headers and formats all information provided to
the script in HTML form.
\end{funcdesc}

\begin{funcdesc}{print_environ}{}
Format the shell environment in HTML.
\end{funcdesc}

\begin{funcdesc}{print_form}{form}
Format a form in HTML.
\end{funcdesc}

\begin{funcdesc}{print_directory}{}
Format the current directory in HTML.
\end{funcdesc}

\begin{funcdesc}{print_environ_usage}{}
Print a list of useful (used by CGI) environment variables in
HTML.
\end{funcdesc}

\begin{funcdesc}{escape}{s\optional{, quote}}
Convert the characters
\character{\&}, \character{<} and \character{>} in string \var{s} to
HTML-safe sequences.  Use this if you need to display text that might
contain such characters in HTML.  If the optional flag \var{quote} is
true, the double-quote character (\character{"}) is also translated;
this helps for inclusion in an HTML attribute value, as in \code{<A
HREF="...">}.  If the value to be quoted might include single- or
double-quote characters, or both, consider using the
\function{quoteattr()} function in the \refmodule{xml.sax.saxutils}
module instead.
\end{funcdesc}


\subsection{Caring about security \label{cgi-security}}

\indexii{CGI}{security}

There's one important rule: if you invoke an external program (via the
\function{os.system()} or \function{os.popen()} functions. or others
with similar functionality), make very sure you don't pass arbitrary
strings received from the client to the shell.  This is a well-known
security hole whereby clever hackers anywhere on the Web can exploit a
gullible CGI script to invoke arbitrary shell commands.  Even parts of
the URL or field names cannot be trusted, since the request doesn't
have to come from your form!

To be on the safe side, if you must pass a string gotten from a form
to a shell command, you should make sure the string contains only
alphanumeric characters, dashes, underscores, and periods.


\subsection{Installing your CGI script on a \UNIX\ system}

Read the documentation for your HTTP server and check with your local
system administrator to find the directory where CGI scripts should be
installed; usually this is in a directory \file{cgi-bin} in the server tree.

Make sure that your script is readable and executable by ``others''; the
\UNIX{} file mode should be \code{0755} octal (use \samp{chmod 0755
\var{filename}}).  Make sure that the first line of the script contains
\code{\#!} starting in column 1 followed by the pathname of the Python
interpreter, for instance:

\begin{verbatim}
#!/usr/local/bin/python
\end{verbatim}

Make sure the Python interpreter exists and is executable by ``others''.

Make sure that any files your script needs to read or write are
readable or writable, respectively, by ``others'' --- their mode
should be \code{0644} for readable and \code{0666} for writable.  This
is because, for security reasons, the HTTP server executes your script
as user ``nobody'', without any special privileges.  It can only read
(write, execute) files that everybody can read (write, execute).  The
current directory at execution time is also different (it is usually
the server's cgi-bin directory) and the set of environment variables
is also different from what you get when you log in.  In particular, don't
count on the shell's search path for executables (\envvar{PATH}) or
the Python module search path (\envvar{PYTHONPATH}) to be set to
anything interesting.

If you need to load modules from a directory which is not on Python's
default module search path, you can change the path in your script,
before importing other modules.  For example:

\begin{verbatim}
import sys
sys.path.insert(0, "/usr/home/joe/lib/python")
sys.path.insert(0, "/usr/local/lib/python")
\end{verbatim}

(This way, the directory inserted last will be searched first!)

Instructions for non-\UNIX{} systems will vary; check your HTTP server's
documentation (it will usually have a section on CGI scripts).


\subsection{Testing your CGI script}

Unfortunately, a CGI script will generally not run when you try it
from the command line, and a script that works perfectly from the
command line may fail mysteriously when run from the server.  There's
one reason why you should still test your script from the command
line: if it contains a syntax error, the Python interpreter won't
execute it at all, and the HTTP server will most likely send a cryptic
error to the client.

Assuming your script has no syntax errors, yet it does not work, you
have no choice but to read the next section.


\subsection{Debugging CGI scripts} \indexii{CGI}{debugging}

First of all, check for trivial installation errors --- reading the
section above on installing your CGI script carefully can save you a
lot of time.  If you wonder whether you have understood the
installation procedure correctly, try installing a copy of this module
file (\file{cgi.py}) as a CGI script.  When invoked as a script, the file
will dump its environment and the contents of the form in HTML form.
Give it the right mode etc, and send it a request.  If it's installed
in the standard \file{cgi-bin} directory, it should be possible to send it a
request by entering a URL into your browser of the form:

\begin{verbatim}
http://yourhostname/cgi-bin/cgi.py?name=Joe+Blow&addr=At+Home
\end{verbatim}

If this gives an error of type 404, the server cannot find the script
-- perhaps you need to install it in a different directory.  If it
gives another error, there's an installation problem that
you should fix before trying to go any further.  If you get a nicely
formatted listing of the environment and form content (in this
example, the fields should be listed as ``addr'' with value ``At Home''
and ``name'' with value ``Joe Blow''), the \file{cgi.py} script has been
installed correctly.  If you follow the same procedure for your own
script, you should now be able to debug it.

The next step could be to call the \module{cgi} module's
\function{test()} function from your script: replace its main code
with the single statement

\begin{verbatim}
cgi.test()
\end{verbatim}

This should produce the same results as those gotten from installing
the \file{cgi.py} file itself.

When an ordinary Python script raises an unhandled exception (for
whatever reason: of a typo in a module name, a file that can't be
opened, etc.), the Python interpreter prints a nice traceback and
exits.  While the Python interpreter will still do this when your CGI
script raises an exception, most likely the traceback will end up in
one of the HTTP server's log files, or be discarded altogether.

Fortunately, once you have managed to get your script to execute
\emph{some} code, you can easily send tracebacks to the Web browser
using the \refmodule{cgitb} module.  If you haven't done so already,
just add the line:

\begin{verbatim}
import cgitb; cgitb.enable()
\end{verbatim}

to the top of your script.  Then try running it again; when a
problem occurs, you should see a detailed report that will
likely make apparent the cause of the crash.

If you suspect that there may be a problem in importing the
\refmodule{cgitb} module, you can use an even more robust approach
(which only uses built-in modules):

\begin{verbatim}
import sys
sys.stderr = sys.stdout
print "Content-Type: text/plain"
print
...your code here...
\end{verbatim}

This relies on the Python interpreter to print the traceback.  The
content type of the output is set to plain text, which disables all
HTML processing.  If your script works, the raw HTML will be displayed
by your client.  If it raises an exception, most likely after the
first two lines have been printed, a traceback will be displayed.
Because no HTML interpretation is going on, the traceback will be
readable.


\subsection{Common problems and solutions}

\begin{itemize}
\item Most HTTP servers buffer the output from CGI scripts until the
script is completed.  This means that it is not possible to display a
progress report on the client's display while the script is running.

\item Check the installation instructions above.

\item Check the HTTP server's log files.  (\samp{tail -f logfile} in a
separate window may be useful!)

\item Always check a script for syntax errors first, by doing something
like \samp{python script.py}.

\item If your script does not have any syntax errors, try adding
\samp{import cgitb; cgitb.enable()} to the top of the script.

\item When invoking external programs, make sure they can be found.
Usually, this means using absolute path names --- \envvar{PATH} is
usually not set to a very useful value in a CGI script.

\item When reading or writing external files, make sure they can be read
or written by the userid under which your CGI script will be running:
this is typically the userid under which the web server is running, or some
explicitly specified userid for a web server's \samp{suexec} feature.

\item Don't try to give a CGI script a set-uid mode.  This doesn't work on
most systems, and is a security liability as well.
\end{itemize}

