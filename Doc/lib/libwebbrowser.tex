\section{\module{webbrowser} ---
         Convenient Web-browser controller}

\declaremodule{standard}{webbrowser}
\modulesynopsis{Easy-to-use controller for Web browsers.}
\moduleauthor{Fred L. Drake, Jr.}{fdrake@acm.org}
\sectionauthor{Fred L. Drake, Jr.}{fdrake@acm.org}

The \module{webbrowser} module provides a very high-level interface to
allow displaying Web-based documents to users.  The controller objects
are easy to use and are platform independent.

Under \UNIX, graphical browsers are preferred under X11, but text-mode
browsers will be used if graphical browsers are not available or an X11
display isn't available.  If text-mode browsers are used, the calling
process will block until the user exits the browser.

Under \UNIX, if the environment variable \envvar{BROWSER} exists, it
is interpreted to override the platform default browser, as a
colon-separated list of browsers to try in order.  When the value of
a list part contains the string \code{\%s}, then it is interpreted as
a literal browser command line to be used with the argument URL
substituted for the \code{\%s}; if the part does not contain,
\code{\%s}, it is simply interpreted as the name of the browser to
launch.

For non-\UNIX{} platforms, or when X11 browsers are available on
\UNIX, the controlling process will not wait for the user to finish
with the browser, but allow the browser to maintain its own window on
the display.

The following exception is defined:

\begin{excdesc}{Error}
  Exception raised when a browser control error occurs.
\end{excdesc}

The following functions are defined:

\begin{funcdesc}{open}{url\optional{, new}}
  Display \var{url} using the default browser.  If \var{new} is true,
  a new browser window is opened if possible.
\end{funcdesc}

\begin{funcdesc}{open_new}{url}
  Open \var{url} in a new window of the default browser, if possible,
  otherwise, open \var{url} in the only browser window.  (This entry
  point is deprecated and may be removed in 2.1.)
\end{funcdesc}

\begin{funcdesc}{get}{\optional{name}}
  Return a controller object for the browser type \var{name}.  If
  \var{name} is empty, return a controller for a default browser
  appropriate to the caller's environment.
\end{funcdesc}

\begin{funcdesc}{register}{name, constructor\optional{, instance}}
  Register the browser type \var{name}.  Once a browser type is
  registered, the \function{get()} function can return a controller
  for that browser type.  If \var{instance} is not provided, or is
  \code{None}, \var{constructor} will be called without parameters to
  create an instance when needed.  If \var{instance} is provided,
  \var{constructor} will never be called, and may be \code{None}.

  This entry point is only useful if you plan to either set the
  \envvar{BROWSER} variable or call \function{get} with a nonempty
  argument matching the name of a handler you declare.  
\end{funcdesc}

Several browser types are defined.  This table gives the type names
that may be passed to the \function{get()} function and the names of
the implementation classes, all defined in this module.

\begin{tableiii}{l|l|c}{code}{Type Name}{Class Name}{Notes}
  \lineiii{'mozilla'}{\class{Mozilla}}{}
  \lineiii{'netscape'}{\class{Netscape}}{}
  \lineiii{'mosaic'}{\class{Mosaic}}{}
  \lineiii{'kfm'}{\class{Konquerer}}{(1)}
  \lineiii{'grail'}{\class{Grail}}{}
  \lineiii{'links'}{\class{links}}{}
  \lineiii{'lynx'}{\class{Lynx}}{}
  \lineiii{'w3m'}{\class{w3m}}{}
  \lineiii{'windows-default'}{\class{WindowsDefault}}{(2)}
  \lineiii{'internet-config'}{\class{InternetConfig}}{(3)}
\end{tableiii}

\noindent
Notes:

\begin{description}
\item[(1)]
``Konquerer'' is the file manager for the KDE desktop environment for
UNIX, and only makes sense to use if KDE is running.  Some way of
reliably detecting KDE would be nice; the \envvar{KDEDIR} variable is
not sufficient.

\item[(2)]
Only on Windows platforms; requires the common
extension modules \module{win32api} and \module{win32con}.

\item[(3)]
Only on MacOS platforms; requires the standard MacPython \module{ic}
module, described in the \citetitle[../mac/module-ic.html]{Macintosh
Library Modules} manual.
\end{description}


\subsection{Browser Controller Objects \label{browser-controllers}}

Browser controllers provide two methods which parallel two of the
module-level convenience functions:

\begin{funcdesc}{open}{url\optional{, new}}
  Display \var{url} using the browser handled by this controller.  If
  \var{new} is true, a new browser window is opened if possible.
\end{funcdesc}

\begin{funcdesc}{open_new}{url}
  Open \var{url} in a new window of the browser handled by this
  controller, if possible, otherwise, open \var{url} in the only
  browser window.  (This method is deprecated and may be removed in
  2.1.)
\end{funcdesc}
