\section{Standard Module \sectcode{getopt}}

\stmodindex{getopt}
This module helps scripts to parse the command line arguments in
\code{sys.argv}.
It uses the same conventions as the \UNIX{}
\code{getopt()}
function.
It defines the function
\code{getopt.getopt(args, options)}
and the exception
\code{getopt.error}.

The first argument to
\code{getopt()}
is the argument list passed to the script with its first element
chopped off (i.e.,
\code{sys.argv[1:]}).
The second argument is the string of option letters that the
script wants to recognize, with options that require an argument
followed by a colon (i.e., the same format that \UNIX{}
\code{getopt()}
uses).
The return value consists of two elements: the first is a list of
option-and-value pairs; the second is the list of program arguments
left after the option list was stripped (this is a trailing slice of the
first argument).
Each option-and-value pair returned has the option as its first element,
prefixed with a hyphen (e.g.,
\code{'-x'}),
and the option argument as its second element, or an empty string if the
option has no argument.
The options occur in the list in the same order in which they were
found, thus allowing multiple occurrences.
Example:

\bcode\begin{verbatim}
>>> import getopt, string
>>> args = string.split('-a -b -cfoo -d bar a1 a2')
>>> args
['-a', '-b', '-cfoo', '-d', 'bar', 'a1', 'a2']
>>> optlist, args = getopt.getopt(args, 'abc:d:')
>>> optlist
[('-a', ''), ('-b', ''), ('-c', 'foo'), ('-d', 'bar')]
>>> args
['a1', 'a2']
>>> 
\end{verbatim}\ecode

The exception
\code{getopt.error = 'getopt error'}
is raised when an unrecognized option is found in the argument list or
when an option requiring an argument is given none.
The argument to the exception is a string indicating the cause of the
error.
