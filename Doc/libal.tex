\section{Built-in Module \sectcode{al}}
\bimodindex{al}

This module provides access to the audio facilities of the SGI Indy
and Indigo workstations.  See section 3A of the IRIX man pages for
details.  You'll need to read those man pages to understand what these
functions do!  Some of the functions are not available in IRIX
releases before 4.0.5.  Again, see the manual to check whether a
specific function is available on your platform.

All functions and methods defined in this module are equivalent to
the C functions with \samp{AL} prefixed to their name.

Symbolic constants from the C header file \file{<audio.h>} are defined
in the standard module \code{AL}, see below.

\strong{Warning:} the current version of the audio library may dump core
when bad argument values are passed rather than returning an error
status.  Unfortunately, since the precise circumstances under which
this may happen are undocumented and hard to check, the Python
interface can provide no protection against this kind of problems.
(One example is specifying an excessive queue size --- there is no
documented upper limit.)

The module defines the following functions:

\renewcommand{\indexsubitem}{(in module al)}

\begin{funcdesc}{openport}{name\, direction\optional{\, config}}
The name and direction arguments are strings.  The optional config
argument is a configuration object as returned by
\code{al.newconfig()}.  The return value is an \dfn{port object};
methods of port objects are described below.
\end{funcdesc}

\begin{funcdesc}{newconfig}{}
The return value is a new \dfn{configuration object}; methods of
configuration objects are described below.
\end{funcdesc}

\begin{funcdesc}{queryparams}{device}
The device argument is an integer.  The return value is a list of
integers containing the data returned by ALqueryparams().
\end{funcdesc}

\begin{funcdesc}{getparams}{device\, list}
The device argument is an integer.  The list argument is a list such
as returned by \code{queryparams}; it is modified in place (!).
\end{funcdesc}

\begin{funcdesc}{setparams}{device\, list}
The device argument is an integer.  The list argument is a list such
as returned by \code{al.queryparams}.
\end{funcdesc}

\subsection{Configuration Objects}

Configuration objects (returned by \code{al.newconfig()} have the
following methods:

\renewcommand{\indexsubitem}{(audio configuration object method)}

\begin{funcdesc}{getqueuesize}{}
Return the queue size.
\end{funcdesc}

\begin{funcdesc}{setqueuesize}{size}
Set the queue size.
\end{funcdesc}

\begin{funcdesc}{getwidth}{}
Get the sample width.
\end{funcdesc}

\begin{funcdesc}{setwidth}{width}
Set the sample width.
\end{funcdesc}

\begin{funcdesc}{getchannels}{}
Get the channel count.
\end{funcdesc}

\begin{funcdesc}{setchannels}{nchannels}
Set the channel count.
\end{funcdesc}

\begin{funcdesc}{getsampfmt}{}
Get the sample format.
\end{funcdesc}

\begin{funcdesc}{setsampfmt}{sampfmt}
Set the sample format.
\end{funcdesc}

\begin{funcdesc}{getfloatmax}{}
Get the maximum value for floating sample formats.
\end{funcdesc}

\begin{funcdesc}{setfloatmax}{floatmax}
Set the maximum value for floating sample formats.
\end{funcdesc}

\subsection{Port Objects}

Port objects (returned by \code{al.openport()} have the following
methods:

\renewcommand{\indexsubitem}{(audio port object method)}

\begin{funcdesc}{closeport}{}
Close the port.
\end{funcdesc}

\begin{funcdesc}{getfd}{}
Return the file descriptor as an int.
\end{funcdesc}

\begin{funcdesc}{getfilled}{}
Return the number of filled samples.
\end{funcdesc}

\begin{funcdesc}{getfillable}{}
Return the number of fillable samples.
\end{funcdesc}

\begin{funcdesc}{readsamps}{nsamples}
Read a number of samples from the queue, blocking if necessary.
Return the data as a string containing the raw data, (e.g., 2 bytes per
sample in big-endian byte order (high byte, low byte) if you have set
the sample width to 2 bytes).
\end{funcdesc}

\begin{funcdesc}{writesamps}{samples}
Write samples into the queue, blocking if necessary.  The samples are
encoded as described for the \code{readsamps} return value.
\end{funcdesc}

\begin{funcdesc}{getfillpoint}{}
Return the `fill point'.
\end{funcdesc}

\begin{funcdesc}{setfillpoint}{fillpoint}
Set the `fill point'.
\end{funcdesc}

\begin{funcdesc}{getconfig}{}
Return a configuration object containing the current configuration of
the port.
\end{funcdesc}

\begin{funcdesc}{setconfig}{config}
Set the configuration from the argument, a configuration object.
\end{funcdesc}

\begin{funcdesc}{getstatus}{list}
Get status information on last error.
\end{funcdesc}

\section{Standard Module \sectcode{AL}}
\nodename{AL (uppercase)}
\stmodindex{AL}

This module defines symbolic constants needed to use the built-in
module \code{al} (see above); they are equivalent to those defined in
the C header file \file{<audio.h>} except that the name prefix
\samp{AL_} is omitted.  Read the module source for a complete list of
the defined names.  Suggested use:

\bcode\begin{verbatim}
import al
from AL import *
\end{verbatim}\ecode
