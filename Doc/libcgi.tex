\section{Standard Module \module{cgi}}
\label{module-cgi}
\stmodindex{cgi}
\indexii{WWW}{server}
\indexii{CGI}{protocol}
\indexii{HTTP}{protocol}
\indexii{MIME}{headers}
\index{URL}


Support module for CGI (Common Gateway Interface) scripts.%
\index{Common Gateway Interface}

This module defines a number of utilities for use by CGI scripts
written in Python.

\subsection{Introduction}
\nodename{cgi-intro}

A CGI script is invoked by an HTTP server, usually to process user
input submitted through an HTML \code{<FORM>} or \code{<ISINPUT>} element.

Most often, CGI scripts live in the server's special \file{cgi-bin}
directory.  The HTTP server places all sorts of information about the
request (such as the client's hostname, the requested URL, the query
string, and lots of other goodies) in the script's shell environment,
executes the script, and sends the script's output back to the client.

The script's input is connected to the client too, and sometimes the
form data is read this way; at other times the form data is passed via
the ``query string'' part of the URL.  This module is intended
to take care of the different cases and provide a simpler interface to
the Python script.  It also provides a number of utilities that help
in debugging scripts, and the latest addition is support for file
uploads from a form (if your browser supports it --- Grail 0.3 and
Netscape 2.0 do).

The output of a CGI script should consist of two sections, separated
by a blank line.  The first section contains a number of headers,
telling the client what kind of data is following.  Python code to
generate a minimal header section looks like this:

\begin{verbatim}
print "Content-type: text/html"     # HTML is following
print                               # blank line, end of headers
\end{verbatim}

The second section is usually HTML, which allows the client software
to display nicely formatted text with header, in-line images, etc.
Here's Python code that prints a simple piece of HTML:

\begin{verbatim}
print "<TITLE>CGI script output</TITLE>"
print "<H1>This is my first CGI script</H1>"
print "Hello, world!"
\end{verbatim}

(It may not be fully legal HTML according to the letter of the
standard, but any browser will understand it.)

\subsection{Using the cgi module}
\nodename{Using the cgi module}

Begin by writing \samp{import cgi}.  Do not use \samp{from cgi import
*} --- the module defines all sorts of names for its own use or for
backward compatibility that you don't want in your namespace.

It's best to use the \class{FieldStorage} class.  The other classes
defined in this module are provided mostly for backward compatibility.
Instantiate it exactly once, without arguments.  This reads the form
contents from standard input or the environment (depending on the
value of various environment variables set according to the CGI
standard).  Since it may consume standard input, it should be
instantiated only once.

The \class{FieldStorage} instance can be accessed as if it were a Python 
dictionary.  For instance, the following code (which assumes that the 
\code{content-type} header and blank line have already been printed)
checks that the fields \code{name} and \code{addr} are both set to a
non-empty string:

\begin{verbatim}
form = cgi.FieldStorage()
form_ok = 0
if form.has_key("name") and form.has_key("addr"):
    if form["name"].value != "" and form["addr"].value != "":
        form_ok = 1
if not form_ok:
    print "<H1>Error</H1>"
    print "Please fill in the name and addr fields."
    return
...further form processing here...
\end{verbatim}

Here the fields, accessed through \samp{form[\var{key}]}, are
themselves instances of \class{FieldStorage} (or
\class{MiniFieldStorage}, depending on the form encoding).

If the submitted form data contains more than one field with the same
name, the object retrieved by \samp{form[\var{key}]} is not a
\class{FieldStorage} or \class{MiniFieldStorage}
instance but a list of such instances.  If you expect this possibility
(i.e., when your HTML form comtains multiple fields with the same
name), use the \function{type()} function to determine whether you
have a single instance or a list of instances.  For example, here's
code that concatenates any number of username fields, separated by
commas:

\begin{verbatim}
username = form["username"]
if type(username) is type([]):
    # Multiple username fields specified
    usernames = ""
    for item in username:
        if usernames:
            # Next item -- insert comma
            usernames = usernames + "," + item.value
        else:
            # First item -- don't insert comma
            usernames = item.value
else:
    # Single username field specified
    usernames = username.value
\end{verbatim}

If a field represents an uploaded file, the value attribute reads the
entire file in memory as a string.  This may not be what you want.
You can test for an uploaded file by testing either the filename
attribute or the file attribute.  You can then read the data at
leasure from the file attribute:

\begin{verbatim}
fileitem = form["userfile"]
if fileitem.file:
    # It's an uploaded file; count lines
    linecount = 0
    while 1:
        line = fileitem.file.readline()
        if not line: break
        linecount = linecount + 1
\end{verbatim}

The file upload draft standard entertains the possibility of uploading
multiple files from one field (using a recursive
\mimetype{multipart/*} encoding).  When this occurs, the item will be
a dictionary-like \class{FieldStorage} item.  This can be determined
by testing its \member{type} attribute, which should be
\mimetype{multipart/form-data} (or perhaps another MIME type matching
\mimetype{multipart/*}).  It this case, it can be iterated over
recursively just like the top-level form object.

When a form is submitted in the ``old'' format (as the query string or
as a single data part of type
\mimetype{application/x-www-form-urlencoded}), the items will actually
be instances of the class \class{MiniFieldStorage}.  In this case, the
list, file and filename attributes are always \code{None}.


\subsection{Old classes}

These classes, present in earlier versions of the \module{cgi} module,
are still supported for backward compatibility.  New applications
should use the \class{FieldStorage} class.

\class{SvFormContentDict} stores single value form content as
dictionary; it assumes each field name occurs in the form only once.

\class{FormContentDict} stores multiple value form content as a
dictionary (the form items are lists of values).  Useful if your form
contains multiple fields with the same name.

Other classes (\class{FormContent}, \class{InterpFormContentDict}) are
present for backwards compatibility with really old applications only.
If you still use these and would be inconvenienced when they
disappeared from a next version of this module, drop me a note.


\subsection{Functions}
\nodename{Functions in cgi module}

These are useful if you want more control, or if you want to employ
some of the algorithms implemented in this module in other
circumstances.

\begin{funcdesc}{parse}{fp}
Parse a query in the environment or from a file (default
\code{sys.stdin}).
\end{funcdesc}

\begin{funcdesc}{parse_qs}{qs}
Parse a query string given as a string argument (data of type 
\mimetype{application/x-www-form-urlencoded}).
\end{funcdesc}

\begin{funcdesc}{parse_multipart}{fp, pdict}
Parse input of type \mimetype{multipart/form-data} (for 
file uploads).  Arguments are \var{fp} for the input file and
\var{pdict} for the dictionary containing other parameters of
\code{content-type} header

Returns a dictionary just like \function{parse_qs()} keys are the
field names, each value is a list of values for that field.  This is
easy to use but not much good if you are expecting megabytes to be
uploaded --- in that case, use the \class{FieldStorage} class instead
which is much more flexible.  Note that \code{content-type} is the
raw, unparsed contents of the \code{content-type} header.

Note that this does not parse nested multipart parts --- use
\class{FieldStorage} for that.
\end{funcdesc}

\begin{funcdesc}{parse_header}{string}
Parse a header like \code{content-type} into a main
content-type and a dictionary of parameters.
\end{funcdesc}

\begin{funcdesc}{test}{}
Robust test CGI script, usable as main program.
Writes minimal HTTP headers and formats all information provided to
the script in HTML form.
\end{funcdesc}

\begin{funcdesc}{print_environ}{}
Format the shell environment in HTML.
\end{funcdesc}

\begin{funcdesc}{print_form}{form}
Format a form in HTML.
\end{funcdesc}

\begin{funcdesc}{print_directory}{}
Format the current directory in HTML.
\end{funcdesc}

\begin{funcdesc}{print_environ_usage}{}
Print a list of useful (used by CGI) environment variables in
HTML.
\end{funcdesc}

\begin{funcdesc}{escape}{s\optional{, quote}}
Convert the characters
\character{\&}, \character{<} and \character{>} in string \var{s} to
HTML-safe sequences.  Use this if you need to display text that might
contain such characters in HTML.  If the optional flag \var{quote} is
true, the double quote character (\character{"}) is also translated;
this helps for inclusion in an HTML attribute value, e.g. in \code{<A
HREF="...">}.
\end{funcdesc}


\subsection{Caring about security}

There's one important rule: if you invoke an external program (e.g.
via the \function{os.system()} or \function{os.popen()} functions),
make very sure you don't pass arbitrary strings received from the
client to the shell.  This is a well-known security hole whereby
clever hackers anywhere on the web can exploit a gullible CGI script
to invoke arbitrary shell commands.  Even parts of the URL or field
names cannot be trusted, since the request doesn't have to come from
your form!

To be on the safe side, if you must pass a string gotten from a form
to a shell command, you should make sure the string contains only
alphanumeric characters, dashes, underscores, and periods.


\subsection{Installing your CGI script on a Unix system}

Read the documentation for your HTTP server and check with your local
system administrator to find the directory where CGI scripts should be
installed; usually this is in a directory \file{cgi-bin} in the server tree.

Make sure that your script is readable and executable by ``others''; the
\UNIX{} file mode should be \code{0755} octal (use \samp{chmod 0755
filename}).  Make sure that the first line of the script contains
\code{\#!} starting in column 1 followed by the pathname of the Python
interpreter, for instance:

\begin{verbatim}
#!/usr/local/bin/python
\end{verbatim}

Make sure the Python interpreter exists and is executable by ``others''.

Make sure that any files your script needs to read or write are
readable or writable, respectively, by ``others'' --- their mode
should be \code{0644} for readable and \code{0666} for writable.  This
is because, for security reasons, the HTTP server executes your script
as user ``nobody'', without any special privileges.  It can only read
(write, execute) files that everybody can read (write, execute).  The
current directory at execution time is also different (it is usually
the server's cgi-bin directory) and the set of environment variables
is also different from what you get at login.  In particular, don't
count on the shell's search path for executables (\envvar{PATH}) or
the Python module search path (\envvar{PYTHONPATH}) to be set to
anything interesting.

If you need to load modules from a directory which is not on Python's
default module search path, you can change the path in your script,
before importing other modules, e.g.:

\begin{verbatim}
import sys
sys.path.insert(0, "/usr/home/joe/lib/python")
sys.path.insert(0, "/usr/local/lib/python")
\end{verbatim}

(This way, the directory inserted last will be searched first!)

Instructions for non-\UNIX{} systems will vary; check your HTTP server's
documentation (it will usually have a section on CGI scripts).


\subsection{Testing your CGI script}

Unfortunately, a CGI script will generally not run when you try it
from the command line, and a script that works perfectly from the
command line may fail mysteriously when run from the server.  There's
one reason why you should still test your script from the command
line: if it contains a syntax error, the Python interpreter won't
execute it at all, and the HTTP server will most likely send a cryptic
error to the client.

Assuming your script has no syntax errors, yet it does not work, you
have no choice but to read the next section.


\subsection{Debugging CGI scripts}

First of all, check for trivial installation errors --- reading the
section above on installing your CGI script carefully can save you a
lot of time.  If you wonder whether you have understood the
installation procedure correctly, try installing a copy of this module
file (\file{cgi.py}) as a CGI script.  When invoked as a script, the file
will dump its environment and the contents of the form in HTML form.
Give it the right mode etc, and send it a request.  If it's installed
in the standard \file{cgi-bin} directory, it should be possible to send it a
request by entering a URL into your browser of the form:

\begin{verbatim}
http://yourhostname/cgi-bin/cgi.py?name=Joe+Blow&addr=At+Home
\end{verbatim}

If this gives an error of type 404, the server cannot find the script
-- perhaps you need to install it in a different directory.  If it
gives another error (e.g.  500), there's an installation problem that
you should fix before trying to go any further.  If you get a nicely
formatted listing of the environment and form content (in this
example, the fields should be listed as ``addr'' with value ``At Home''
and ``name'' with value ``Joe Blow''), the \file{cgi.py} script has been
installed correctly.  If you follow the same procedure for your own
script, you should now be able to debug it.

The next step could be to call the \module{cgi} module's
\function{test()} function from your script: replace its main code
with the single statement

\begin{verbatim}
cgi.test()
\end{verbatim}

This should produce the same results as those gotten from installing
the \file{cgi.py} file itself.

When an ordinary Python script raises an unhandled exception
(e.g. because of a typo in a module name, a file that can't be opened,
etc.), the Python interpreter prints a nice traceback and exits.
While the Python interpreter will still do this when your CGI script
raises an exception, most likely the traceback will end up in one of
the HTTP server's log file, or be discarded altogether.

Fortunately, once you have managed to get your script to execute
\emph{some} code, it is easy to catch exceptions and cause a traceback
to be printed.  The \function{test()} function below in this module is
an example.  Here are the rules:

\begin{enumerate}
\item Import the traceback module before entering the \keyword{try}
   ... \keyword{except} statement

\item Assign \code{sys.stderr} to be \code{sys.stdout}

\item Make sure you finish printing the headers and the blank line
   early

\item Wrap all remaining code in a \keyword{try} ... \keyword{except}
   statement

\item In the except clause, call \function{traceback.print_exc()}
\end{enumerate}

For example:

\begin{verbatim}
import sys
import traceback
print "Content-type: text/html"
print
sys.stderr = sys.stdout
try:
    ...your code here...
except:
    print "\n\n<PRE>"
    traceback.print_exc()
\end{verbatim}

Notes: The assignment to \code{sys.stderr} is needed because the
traceback prints to \code{sys.stderr}.
The \code{print "{\e}n{\e}n<PRE>"} statement is necessary to
disable the word wrapping in HTML.

If you suspect that there may be a problem in importing the traceback
module, you can use an even more robust approach (which only uses
built-in modules):

\begin{verbatim}
import sys
sys.stderr = sys.stdout
print "Content-type: text/plain"
print
...your code here...
\end{verbatim}

This relies on the Python interpreter to print the traceback.  The
content type of the output is set to plain text, which disables all
HTML processing.  If your script works, the raw HTML will be displayed
by your client.  If it raises an exception, most likely after the
first two lines have been printed, a traceback will be displayed.
Because no HTML interpretation is going on, the traceback will
readable.


\subsection{Common problems and solutions}

\begin{itemize}
\item Most HTTP servers buffer the output from CGI scripts until the
script is completed.  This means that it is not possible to display a
progress report on the client's display while the script is running.

\item Check the installation instructions above.

\item Check the HTTP server's log files.  (\samp{tail -f logfile} in a
separate window may be useful!)

\item Always check a script for syntax errors first, by doing something
like \samp{python script.py}.

\item When using any of the debugging techniques, don't forget to add
\samp{import sys} to the top of the script.

\item When invoking external programs, make sure they can be found.
Usually, this means using absolute path names --- \envvar{PATH} is
usually not set to a very useful value in a CGI script.

\item When reading or writing external files, make sure they can be read
or written by every user on the system.

\item Don't try to give a CGI script a set-uid mode.  This doesn't work on
most systems, and is a security liability as well.
\end{itemize}

