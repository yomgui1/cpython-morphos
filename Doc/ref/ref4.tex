\chapter{Execution model \label{execmodel}}
\index{execution model}


\section{Naming and binding \label{naming}}
\indexii{code}{block}
\index{namespace}
\index{scope}

\dfn{Names}\index{name} refer to objects.  Names are introduced by
name binding operations.  Each occurrence of a name in the program
text refers to the \dfn{binding}\indexii{binding}{name} of that name
established in the innermost function block containing the use.

A \dfn{block}\index{block} is a piece of Python program text that is
executed as a unit.  The following are blocks: a module, a function
body, and a class definition.  Each command typed interactively is a
block.  A script file (a file given as standard input to the
interpreter or specified on the interpreter command line the first
argument) is a code block.  A script command (a command specified on
the interpreter command line with the `\strong{-c}' option) is a code
block.  The file read by the built-in function \function{execfile()}
is a code block.  The string argument passed to the built-in function
\function{eval()} and to the \keyword{exec} statement is a code block.
The expression read and evaluated by the built-in function
\function{input()} is a code block.

A \dfn{scope}\index{scope} defines the visibility of a name within a
block.  If a local variable is defined in a block, it's scope includes
that block.  If the definition occurs in a function block, the scope
extends to any blocks contained within the defining one, unless a
contained block introduces a different binding for the name.  The
scope of names defined in a class block is limited to the class block;
it does not extend to the code blocks of methods.

When a name is used in a code block, it is resolved using the nearest
enclosing scope.  The set of all such scopes visible to a code block
is called the block's \dfn{environment}\index{environment}.  

If a name is bound in a block, it is a local variable of that block.
If a name is bound at the module level, it is a global variable.  (The
variables of the module code block are local and global.)  If a
variable is used in a code block but not defined there, it is a
\dfn{free variable}\indexii{free}{variable}.

When a name is not found at all, a
\exception{NameError}\withsubitem{(built-in
exception)}{\ttindex{NameError}} exception is raised.  If the name
refers to a local variable that has not been bound, a
\exception{UnboundLocalError}\ttindex{UnboundLocalError} exception is
raised.  \exception{UnboundLocalError} is a subclass of
\exception{NameError}.

The following constructs bind names: formal parameters to functions,
\keyword{import} statements, class and function definitions (these
bind the class or function name in the defining block), and targets
that are identifiers if occurring in an assignment, \keyword{for} loop
header, or in the second position of an \keyword{except} clause
header.  The \keyword{import} statement of the form ``\samp{from
\ldots import *}''\stindex{from} binds all names defined in the
imported module, except those beginning with an underscore.  This form
may only be used at the module level.

A target occurring in a \keyword{del} statement is also considered bound
for this purpose (though the actual semantics are to unbind the
name).  It is illegal to unbind a name that is referenced by an
enclosing scope; the compiler will report a \exception{SyntaxError}.

Each assignment or import statement occurs within a block defined by a
class or function definition or at the module level (the top-level
code block).

If a name binding operation occurs anywhere within a code block, all
uses of the name within the block are treated as references to the
current block.  This can lead to errors when a name is used within a
block before it is bound.

The previous rule is a subtle.  Python lacks declarations and allows
name binding operations to occur anywhere within a code block.  The
local variables of a code block can be determined by scanning the
entire text of the block for name binding operations.

If the global statement occurs within a block, all uses of the name
specified in the statement refer to the binding of that name in the
top-level namespace.  Names are resolved in the top-level namespace by
searching the global namespace, i.e. the namespace of the module
containing the code block, and the builtin namespace, the namespace of
the module \module{__builtin__}.  The global namespace is searched
first.  If the name is not found there, the builtin namespace is
searched.  The global statement must precede all uses of the name.

The built-in namespace associated with the execution of a code block
is actually found by looking up the name \code{__builtins__} in its
global namespace; this should be a dictionary or a module (in the
latter case the module's dictionary is used).  Normally, the
\code{__builtins__} namespace is the dictionary of the built-in module
\module{__builtin__} (note: no `s').  If it isn't, restricted
execution\indexii{restricted}{execution} mode is in effect.

The namespace for a module is automatically created the first time a
module is imported.  The main module for a script is always called
\module{__main__}\refbimodindex{__main__}.

The global statement has the same scope as a name binding operation
in the same block.  If the nearest enclosing scope for a free variable
contains a global statement, the free variable is treated as a global.

A class definition is an executable statement that may use and define
names.  These references follow the normal rules for name resolution.
The namespace of the class definition becomes the attribute dictionary
of the class.  Names defined at the class scope are not visible in
methods. 

\subsection{Interaction with dynamic features \label{dynamic-features}}

There are several cases where Python statements are illegal when
used in conjunction with nested scopes that contain free
variables.

If a variable is referenced in an enclosing scope, it is illegal
to delete the name.  An error will be reported at compile time.

If the wild card form of import --- \samp{import *} --- is used in a
function and the function contains or is a nested block with free
variables, the compiler will raise a SyntaxError.

If \keyword{exec} is used in a function and the function contains or
is a nested block with free variables, the compiler will raise a
\exception{SyntaxError} unless the exec explicitly specifies the local
namespace for the \keyword{exec}.  (In other words, \samp{exec obj}
would be illegal, but \samp{exec obj in ns} would be legal.)

The \function{eval()}, \function{execfile()}, and \function{input()}
functions and the \keyword{exec} statement do not have access to the
full environment for resolving names.  Names may be resolved in the
local and global namespaces of the caller.  Free variables are not
resolved in the nearest enclosing namespace, but in the global
namespace.\footnote{This limitation occurs because the code that is
    executed by these operations is not available at the time the
    module is compiled.}
The \keyword{exec} statement and the \function{eval()} and
\function{execfile()} functions have optional arguments to override
the global and local namespace.  If only one namespace is specified,
it is used for both.

\section{Exceptions \label{exceptions}}
\index{exception}

Exceptions are a means of breaking out of the normal flow of control
of a code block in order to handle errors or other exceptional
conditions.  An exception is
\emph{raised}\index{raise an exception} at the point where the error
is detected; it may be \emph{handled}\index{handle an exception} by
the surrounding code block or by any code block that directly or
indirectly invoked the code block where the error occurred.
\index{exception handler}
\index{errors}
\index{error handling}

The Python interpreter raises an exception when it detects a run-time
error (such as division by zero).  A Python program can also
explicitly raise an exception with the \keyword{raise} statement.
Exception handlers are specified with the \keyword{try} ... \keyword{except}
statement.  The \keyword{try} ... \keyword{finally} statement
specifies cleanup code which does not handle the exception, but is
executed whether an exception occurred or not in the preceding code.

Python uses the ``termination'' \index{termination model}model of
error handling: an exception handler can find out what happened and
continue execution at an outer level, but it cannot repair the cause
of the error and retry the failing operation (except by re-entering
the offending piece of code from the top).

When an exception is not handled at all, the interpreter terminates
execution of the program, or returns to its interactive main loop.  In
either case, it prints a stack backtrace, except when the exception is 
\exception{SystemExit}\withsubitem{(built-in
exception)}{\ttindex{SystemExit}}.

Exceptions are identified by string objects or class instances.
Selection of a matching except clause is based on object identity
(i.e., two different string objects with the same value represent
different exceptions!)  For string exceptions, the \keyword{except}
clause must reference the same string object.  For class exceptions,
the \keyword{except} clause must reference the same class or a base
class of it.

When an exception is raised, an object (maybe \code{None}) is passed
as the exception's ``parameter'' or ``value''; this object does not
affect the selection of an exception handler, but is passed to the
selected exception handler as additional information.  For class
exceptions, this object must be an instance of the exception class
being raised.

See also the description of the \keyword{try} statement in section
\ref{try} and \keyword{raise} statement in section \ref{raise}.
