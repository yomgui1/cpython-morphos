\section{Standard Module \sectcode{regsub}}
\label{module-regsub}

\stmodindex{regsub}
This module defines a number of functions useful for working with
regular expressions (see built-in module \code{regex}).

Warning: these functions are not thread-safe.

\strong{Obsolescence note:}
This module is obsolete as of Python version 1.5; it is still being
maintained because much existing code still uses it.  All new code in
need of regular expressions should use the new \module{re} module, which
supports the more powerful and regular Perl-style regular expressions.
Existing code should be converted.  The standard library module
\module{reconvert} helps in converting \code{regex} style regular
expressions to \module{re} style regular expressions.  (For more
conversion help, see the URL
\url{http://starship.skyport.net/crew/amk/howto/regex-to-re.html}.)

\setindexsubitem{(in module regsub)}

\begin{funcdesc}{sub}{pat, repl, str}
Replace the first occurrence of pattern \var{pat} in string
\var{str} by replacement \var{repl}.  If the pattern isn't found,
the string is returned unchanged.  The pattern may be a string or an
already compiled pattern.  The replacement may contain references
\samp{\e \var{digit}} to subpatterns and escaped backslashes.
\end{funcdesc}

\begin{funcdesc}{gsub}{pat, repl, str}
Replace all (non-overlapping) occurrences of pattern \var{pat} in
string \var{str} by replacement \var{repl}.  The same rules as for
\code{sub()} apply.  Empty matches for the pattern are replaced only
when not adjacent to a previous match, so e.g.
\code{gsub('', '-', 'abc')} returns \code{'-a-b-c-'}.
\end{funcdesc}

\begin{funcdesc}{split}{str, pat\optional{, maxsplit}}
Split the string \var{str} in fields separated by delimiters matching
the pattern \var{pat}, and return a list containing the fields.  Only
non-empty matches for the pattern are considered, so e.g.
\code{split('a:b', ':*')} returns \code{['a', 'b']} and
\code{split('abc', '')} returns \code{['abc']}.  The \var{maxsplit}
defaults to 0. If it is nonzero, only \var{maxsplit} number of splits
occur, and the remainder of the string is returned as the final
element of the list.
\end{funcdesc}

\begin{funcdesc}{splitx}{str, pat\optional{, maxsplit}}
Split the string \var{str} in fields separated by delimiters matching
the pattern \var{pat}, and return a list containing the fields as well
as the separators.  For example, \code{splitx('a:::b', ':*')} returns
\code{['a', ':::', 'b']}.  Otherwise, this function behaves the same
as \code{split}.
\end{funcdesc}

\begin{funcdesc}{capwords}{s\optional{, pat}}
Capitalize words separated by optional pattern \var{pat}.  The default
pattern uses any characters except letters, digits and underscores as
word delimiters.  Capitalization is done by changing the first
character of each word to upper case.
\end{funcdesc}

\begin{funcdesc}{clear_cache}{}
The regsub module maintains a cache of compiled regular expressions,
keyed on the regular expression string and the syntax of the regex
module at the time the expression was compiled.  This function clears
that cache.
\end{funcdesc}
