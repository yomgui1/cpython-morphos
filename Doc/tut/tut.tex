% Format this file with latex.
 
\documentstyle[myformat]{report}

\title{\bf
	Python Tutorial
}
	
\author{
	Guido van Rossum \\
	Dept. CST, CWI, Kruislaan 413 \\
	1098 SJ Amsterdam, The Netherlands \\
	E-mail: {\tt guido@cwi.nl}
}

\begin{document}

\pagenumbering{roman}

\maketitle

\begin{abstract}

\noindent
Python is a simple, yet powerful programming language that bridges the
gap between C and shell programming, and is thus ideally suited for
``throw-away programming''
and rapid prototyping.  Its syntax is put
together from constructs borrowed from a variety of other languages;
most prominent are influences from ABC, C, Modula-3 and Icon.

The Python interpreter is easily extended with new functions and data
types implemented in C.  Python is also suitable as an extension
language for highly customizable C applications such as editors or
window managers.

Python is available for various operating systems, amongst which
several flavors of {\UNIX}, Amoeba, the Apple Macintosh O.S.,
and MS-DOS.

This tutorial introduces the reader informally to the basic concepts
and features of the Python language and system.  It helps to have a
Python interpreter handy for hands-on experience, but as the examples
are self-contained, the tutorial can be read off-line as well.

For a description of standard objects and modules, see the {\em
Library Reference} document.  The {\em Language Reference} document
(when it is ever written)
will give a more formal definition of the language.

\end{abstract}

\pagebreak

\tableofcontents

\pagebreak

\pagenumbering{arabic}

\chapter{Whetting Your Appetite}

If you ever wrote a large shell script, you probably know this
feeling: you'd love to add yet another feature, but it's already so
slow, and so big, and so complicated; or the feature involves a system
call or other funcion that is only accessible from C \ldots  Usually
the problem at hand isn't serious enough to warrant rewriting the
script in C; perhaps because the problem requires variable-length
strings or other data types (like sorted lists of file names) that are
easy in the shell but lots of work to implement in C; or perhaps just
because you're not sufficiently familiar with C.

In such cases, Python may be just the language for you.  Python is
simple to use, but it is a real programming language, offering much
more structure and support for large programs than the shell has.  On
the other hand, it also offers much more error checking than C, and,
being a {\em very-high-level language}, it has high-level data types
built in, such as flexible arrays and dictionaries that would cost you
days to implement efficiently in C.  Because of its more general data
types Python is applicable to a much larger problem domain than {\em
Awk} or even {\em Perl}, yet most simple things are at least as easy
in Python as in those languages.

Python allows you to split up your program in modules that can be
reused in other Python programs.  It comes with a large collection of
standard modules that you can use as the basis of your programs ---
or as examples to start learning to program in Python.  There are also
built-in modules that provide things like file I/O, system calls, and
even a generic interface to window systems (STDWIN).

Python is an interpreted language, which saves you considerable time
during program development because no compilation and linking is
necessary.  The interpreter can be used interactively, which makes it
easy to experiment with features of the language, to write throw-away
programs, or to test functions during bottom-up program development.
It is also a handy desk calculator.

Python allows writing very compact and readable programs.  Programs
written in Python are typically much shorter than equivalent C
programs, for several reasons:
\begin{itemize}
\item
the high-level data types allow you to express complex operations in a
single statement;
\item
statement grouping is done by indentation instead of begin/end
brackets;
\item
no variable or argument declarations are necessary.
\end{itemize}

Python is {\em extensible}: if you know how to program in C it is easy
to add a new built-in
function or
module to the interpreter, either to
perform critical operations at maximum speed, or to link Python
programs to libraries that may only be available in binary form (such
as a vendor-specific graphics library).  Once you are really hooked,
you can link the Python interpreter into an application written in C
and use it as an extension or command language.

\section{Where From Here}

Now that you are all excited about Python, you'll want to examine it
in some more detail.  Since the best introduction to a language is
using it, you are invited here to do so.

In the next chapter, the mechanics of using the interpreter are
explained.  This is rather mundane information, but essential for
trying out the examples shown later.

The rest of the tutorial introduces various features of the Python
language and system though examples, beginning with simple
expressions, statements and data types, through functions and modules,
and finally touching upon advanced concepts like exceptions.

When you're through with the turtorial (or just getting bored), you
should read the Library Reference, which gives complete (though terse)
reference material about built-in and standard types, functions and
modules that can save you a lot of time when writing Python programs.

\chapter{Using the Python Interpreter}

The Python interpreter is usually installed as {\tt /usr/local/python}
on those machines where it is available; putting {\tt /usr/local} in
your {\UNIX} shell's search path makes it possible to start it by
typing the command
\bcode\begin{verbatim}
python
\end{verbatim}\ecode
to the shell.  Since the choice of the directory where the interpreter
lives is an installation option, other places are possible; check with
your local Python guru or system administrator.

The interpreter operates somewhat like the {\UNIX} shell: when called
with standard input connected to a tty device, it reads and executes
commands interactively; when called with a file name argument or with
a file as standard input, it reads and executes a {\em script} from
that file.

Note that there is a difference between ``{\tt python file}'' and
``{\tt python $<$file}''.  In the latter case, input requests from the
program, such as calls to {\tt input()} and {\tt raw\_input()}, are
satisfied from {\em file}.  Since this file has already been read
until the end by the parser before the program starts executing, the
program will encounter EOF immediately.  In the former case (which is
usually what you want) they are satisfied from whatever file or device
is connected to standard input of the Python interpreter.

A third possibility is ``{\tt python -c command [arg] ...}'', which
executes the statement(s) in {\tt command}, analogous to the shell's
{\tt -c} option.  Usually {\tt command} will contain spaces or other
characters that are special to the shell, so it is best to quote it.

When available, the script name and additional arguments thereafter
are passed to the script in the variable {\tt sys.argv}, which is a
list of strings.
When {\tt -c command} is used, {\tt sys.argv} is set to {\tt '-c'}.

When commands are read from a tty, the interpreter is said to be in
{\em interactive\ mode}.  In this mode it prompts for the next command
with the {\em primary\ prompt}, usually three greater-than signs ({\tt
>>>}); for continuation lines it prompts with the {\em secondary\
prompt}, by default three dots ({\tt ...}).  Typing an EOF (Control-D)
at the primary prompt causes the interpreter to exit with a zero exit
status.

When an error occurs in interactive mode, the interpreter prints a
message and a stack trace and returns to the primary prompt; with
input from a file, it exits with a nonzero exit status after printing
the stack trace.  (Exceptions handled by an {\tt except} clause in a
{\tt try} statement are not errors in this context.)  Some errors are
unconditionally fatal and cause an exit with a nonzero exit; this
applies to internal inconsistencies and some cases of running out of
memory.  All error messages are written to the standard error stream;
normal output from the executed commands is written to standard
output.

Typing an interrupt (normally Control-C or DEL) to the primary or
secondary prompt cancels the input and returns to the primary prompt.
Typing an interrupt while a command is being executed raises the {\tt
KeyboardInterrupt} exception, which may be handled by a {\tt try}
statement.

When a module named
{\tt foo}
is imported, the interpreter searches for a file named
{\tt foo.py}
in a list of directories specified by the environment variable
{\tt PYTHONPATH}.
It has the same syntax as the {\UNIX} shell variable
{\tt PATH},
i.e., a list of colon-separated directory names.
When
{\tt PYTHONPATH}
is not set, an installation-dependent default path is used, usually
{\tt .:/usr/local/lib/python}.
(Modules are really searched in the list of directories given by the
variable {\tt sys.path} which is initialized from {\tt PYTHONPATH} or
from the installation-dependent default.  See the section on Standard
Modules later.)

As an important speed-up of the start-up time for short programs, if a
file called {\tt foo.pyc} exists in the directory where {\tt foo.py}
is found, this is assumed to contain an already-``compiled'' version
of the module {\tt foo}.  The last modification time of {\tt foo.py}
is recorded in {\tt foo.pyc}, and the file is ignored if these don't
match.  Whenever {\tt foo.py} is successfully compiled, an attempt is
made to write the compiled version to {\tt foo.pyc}.

On BSD'ish {\UNIX} systems, Python scripts can be made directly
executable, like shell scripts, by putting the line
\bcode\begin{verbatim}
#! /usr/local/python
\end{verbatim}\ecode
(assuming that's the name of the interpreter) at the beginning of the
script and giving the file an executable mode.  (The {\tt \#!} must be
the first two characters of the file.)

\section{Interactive Input Editing and History Substitution}

Some versions of the Python interpreter support editing of the current
input line and history substitution, similar to facilities found in
the Korn shell and the GNU Bash shell.  This is implemented using the
{\em GNU\ Readline} library, which supports Emacs-style and vi-style
editing.  This library has its own documentation which I won't
duplicate here; however, the basics are easily explained.

Perhaps the quickest check to see whether command line editing is
supported is typing Control-P to the first Python prompt you get.  If
it beeps, you have command line editing.  If nothing appears to
happen, or if \verb/^P/ is echoed, you can skip the rest of this
section.

If supported, input line editing is active whenever the interpreter
prints a primary or secondary prompt.  The current line can be edited
using the conventional Emacs control characters.  The most important
of these are: C-A (Control-A) moves the cursor to the beginning of the
line, C-E to the end, C-B moves it one position to the left, C-F to
the right.  Backspace erases the character to the left of the cursor,
C-D the character to its right.  C-K kills (erases) the rest of the
line to the right of the cursor, C-Y yanks back the last killed
string.  C-underscore undoes the last change you made; it can be
repeated for cumulative effect.

History substitution works as follows.  All non-empty input lines
issued are saved in a history buffer, and when a new prompt is given
you are positioned on a new line at the bottom of this buffer.  C-P
moves one line up (back) in the history buffer, C-N moves one down.
Any line in the history buffer can be edited; an asterisk appears in
front of the prompt to mark a line as modified.  Pressing the Return
key passes the current line to the interpreter.  C-R starts an
incremental reverse search; C-S starts a forward search.

The key bindings and some other parameters of the Readline library can
be customized by placing commands in an initialization file called
{\tt \$HOME/.inputrc}.  Key bindings have the form
\bcode\begin{verbatim}
key-name: function-name
\end{verbatim}\ecode
and options can be set with
\bcode\begin{verbatim}
set option-name value
\end{verbatim}\ecode
Example:
\bcode\begin{verbatim}
# I prefer vi-style editing:
set editing-mode vi
# Edit using a single line:
set horizontal-scroll-mode On
# Rebind some keys:
Meta-h: backward-kill-word
Control-u: universal-argument
\end{verbatim}\ecode
Note that the default binding for TAB in Python is to insert a TAB
instead of Readline's default filename completion function.  If you
insist, you can override this by putting
\bcode\begin{verbatim}
TAB: complete
\end{verbatim}\ecode
in your {\tt \$HOME/.inputrc}.  (Of course, this makes it hard to type
indented continuation lines.)

This facility is an enormous step forward compared to previous
versions of the interpreter; however, some wishes are left: It would
be nice if the proper indentation were suggested on continuation lines
(the parser knows if an indent token is required next).  The
completion mechanism might use the interpreter's symbol table.  A
function to check (or even suggest) matching parentheses, quotes etc.
would also be useful.

\chapter{An Informal Introduction to Python}

In the following examples, input and output are distinguished by the
presence or absence of prompts ({\tt >>>} and {\tt ...}): to repeat the
example, you must type everything after the prompt, when the prompt
appears;
lines that do not begin with a prompt are output from the interpreter.
Note that a secondary prompt on a line by itself in an example means
you must type a blank line; this is used to end a multi-line command.

\section{Using Python as a Calculator}

Let's try some simple Python commands.  Start the interpreter and wait
for the primary prompt, {\tt >>>}.

The interpreter acts as a simple calculator: you can type an
expression at it and it will write the value.  Expression syntax is
straightforward: the operators {\tt +}, {\tt -}, {\tt *} and {\tt /}
work just as in most other languages (e.g., Pascal or C); parentheses
can be used for grouping.  For example:
\bcode\begin{verbatim}
>>> # This is a comment
>>> 2+2
4
>>> 
>>> (50-5+5*6+25)/4
25
>>> # Division truncates towards zero:
>>> 7/3
2
>>> 
\end{verbatim}\ecode
As in C, the equal sign ({\tt =}) is used to assign a value to a
variable.  The value of an assignment is not written:
\bcode\begin{verbatim}
>>> width = 20
>>> height = 5*9
>>> width * height
900
>>> 
\end{verbatim}\ecode
A value can be assigned to several variables simultaneously:
\bcode\begin{verbatim}
>>> # Zero x, y and z
>>> x = y = z = 0
>>>
\end{verbatim}\ecode
There is full support for floating point; operators with mixed type
operands convert the integer operand to floating point:
\bcode\begin{verbatim}
>>> 4 * 2.5 / 3.3
3.0303030303
>>> 
\end{verbatim}\ecode
Besides numbers, Python can also manipulate strings, enclosed in
single quotes:
\bcode\begin{verbatim}
>>> 'foo bar'
'foo bar'
>>> 'doesn\'t'
'doesn\'t'
>>> 
\end{verbatim}\ecode
Strings are written
the same way as they are typed for input:
inside quotes and with quotes and other funny characters escaped by
backslashes, to show the precise value.  (There is also a way to write
strings without quotes and escapes.)

Strings can be concatenated (glued together) with the {\tt +}
operator, and repeated with {\tt *}:
\bcode\begin{verbatim}
>>> word = 'Help' + 'A'
>>> word
'HelpA'
>>> '<' + word*5 + '>'
'<HelpAHelpAHelpAHelpAHelpA>'
>>> 
\end{verbatim}\ecode
Strings can be subscripted; as in C, the first character of a string
has subscript 0.

There is no separate character type; a character is simply a string of
size one.  As in Icon, substrings can be specified with the {\em
slice} notation: two subscripts (indices) separated by a colon.
\bcode\begin{verbatim}
>>> word[4]
'A'
>>> word[0:2]
'He'
>>> word[2:4]
'lp'
>>> # Slice indices have useful defaults:
>>> word[:2]    # Take first two characters
'He'
>>> word[2:]    # Drop first two characters
'lpA'
>>> # A useful invariant: s[:i] + s[i:] = s
>>> word[:3] + word[3:]
'HelpA'
>>> 
\end{verbatim}\ecode
Degenerate cases are handled gracefully: an index that is too large is
replaced by the string size, an upper bound smaller than the lower
bound returns an empty string.
\bcode\begin{verbatim}
>>> word[1:100]
'elpA'
>>> word[10:]
''
>>> word[2:1]
''
>>> 
\end{verbatim}\ecode
Slice indices (but not simple subscripts) may be negative numbers, to
start counting from the right.  For example:
\bcode\begin{verbatim}
>>> word[-2:]    # Take last two characters
'pA'
>>> word[:-2]    # Drop last two characters
'Hel'
>>> # But -0 does not count from the right!
>>> word[-0:]    # (since -0 equals 0)
'HelpA'
>>> 
\end{verbatim}\ecode
The best way to remember how slices work is to think of the indices as
pointing {\em between} characters, with the left edge of the first
character numbered 0.  Then the right edge of the last character of a
string of {\tt n} characters has index {\tt n}, for example:
\bcode\begin{verbatim}
 +---+---+---+---+---+ 
 | H | e | l | p | A |
 +---+---+---+---+---+ 
 0   1   2   3   4   5 
-5  -4  -3  -2  -1
\end{verbatim}\ecode
The first row of numbers gives the position of the indices 0...5 in
the string; the second row gives the corresponding negative indices.
For nonnegative indices, the length of a slice is the difference of
the indices, if both are within bounds, e.g., the length of {\tt
word[1:3]} is 3--1 = 2.

The built-in function {\tt len()} computes the length of a string:
\bcode\begin{verbatim}
>>> s = 'supercalifragilisticexpialidocious'
>>> len(s)
34
>>> 
\end{verbatim}\ecode
Python knows a number of {\em compound} data types, used to group
together other values.  The most versatile is the {\em list}, which
can be written as a list of comma-separated values between square
brackets:
\bcode\begin{verbatim}
>>> a = ['foo', 'bar', 100, 1234]
>>> a
['foo', 'bar', 100, 1234]
>>> 
\end{verbatim}\ecode
As for strings, list subscripts start at 0:
\bcode\begin{verbatim}
>>> a[0]
'foo'
>>> a[3]
1234
>>> 
\end{verbatim}\ecode
Lists can be sliced, concatenated and so on, like strings:
\bcode\begin{verbatim}
>>> a[1:3]
['bar', 100]
>>> a[:2] + ['bletch', 2*2]
['foo', 'bar', 'bletch', 4]
>>> 3*a[:3] + ['Boe!']
['foo', 'bar', 100, 'foo', 'bar', 100, 'foo', 'bar', 100, 'Boe!']
>>> 
\end{verbatim}\ecode
Unlike strings, which are {\em immutable}, it is possible to change
individual elements of a list:
\bcode\begin{verbatim}
>>> a
['foo', 'bar', 100, 1234]
>>> a[2] = a[2] + 23
>>> a
['foo', 'bar', 123, 1234]
>>>
\end{verbatim}\ecode
Assignment to slices is also possible, and this may even change the size
of the list:
\bcode\begin{verbatim}
>>> # Replace some items:
>>> a[0:2] = [1, 12]
>>> a
[1, 12, 123, 1234]
>>> # Remove some:
>>> a[0:2] = []
>>> a
[123, 1234]
>>> # Insert some:
>>> a[1:1] = ['bletch', 'xyzzy']
>>> a
[123, 'bletch', 'xyzzy', 1234]
>>> 
\end{verbatim}\ecode
The built-in function {\tt len()} also applies to lists:
\bcode\begin{verbatim}
>>> len(a)
4
>>> 
\end{verbatim}\ecode
It is possible to nest lists (create lists containing other lists),
for example:
\bcode\begin{verbatim}
>>> p = [1, [2, 3], 4]
>>> len(p)
3
>>> p[1]
[2, 3]
>>> p[1][0]
2
>>> p[1].append('xtra')
>>> p
[1, [2, 3, 'xtra'], 4]
>>>
\end{verbatim}\ecode

\section{First Steps Towards Programming}

Of course, we can use Python for more complicated tasks than adding
two and two together.  For instance, we can write an initial
subsequence of the {\em Fibonacci} series as follows:
\bcode\begin{verbatim}
>>> # Fibonacci series:
>>> # the sum of two elements defines the next
>>> a, b = 0, 1
>>> while b < 10:
...       print b
...       a, b = b, a+b
... 
1
1
2
3
5
8
>>> 
\end{verbatim}\ecode
This example introduces several new features.

\begin{itemize}

\item
The first line contains a {\em multiple assignment}: the variables
{\tt a} and {\tt b} simultaneously get the new values 0 and 1.  On the
last line this is used again, demonstrating that the expressions on
the right-hand side are all evaluated first before any of the
assignments take place.

\item
The {\tt while} loop executes as long as the condition (here: {\tt b <
100}) remains true.  In Python, as in C, any non-zero integer value is
true; zero is false.  The condition may also be a string or list value,
in fact any sequence; anything with a non-zero length is true, empty
sequences are false.  The test used in the example is a simple
comparison.  The standard comparison operators are written as {\tt <},
{\tt >}, {\tt =}, {\tt <=}, {\tt >=} and {\tt <>}.%
\footnote{
	The ambiguity of using {\tt =}
	for both assignment and equality is resolved by disallowing
	unparenthesized conditions on the right hand side of assignments.
	Parenthesized assignment is also disallowed; instead it is
	interpreted as an equality test.
}

\item
The {\em body} of the loop is {\em indented}: indentation is Python's
way of grouping statements.  Python does not (yet!) provide an
intelligent input line editing facility, so you have to type a tab or
space(s) for each indented line.  In practice you will prepare more
complicated input for Python with a text editor; most text editors have
an auto-indent facility.  When a compound statement is entered
interactively, it must be followed by a blank line to indicate
completion (since the parser cannot guess when you have typed the last
line).

\item
The {\tt print} statement writes the value of the expression(s) it is
given.  It differs from just writing the expression you want to write
(as we did earlier in the calculator examples) in the way it handles
multiple expressions and strings.  Strings are written without quotes,
and a space is inserted between items, so you can format things nicely,
like this:
\bcode\begin{verbatim}
>>> i = 256*256
>>> print 'The value of i is', i
The value of i is 65536
>>> 
\end{verbatim}\ecode
A trailing comma avoids the newline after the output:
\bcode\begin{verbatim}
>>> a, b = 0, 1
>>> while b < 1000:
...     print b,
...     a, b = b, a+b
... 
1 1 2 3 5 8 13 21 34 55 89 144 233 377 610 987
>>> 
\end{verbatim}\ecode
Note that the interpreter inserts a newline before it prints the next
prompt if the last line was not completed.

\end{itemize}

\chapter{More Control Flow Tools}

Besides the {\tt while} statement just introduced, Python knows the
usual control flow statements known from other languages, with some
twists.

\section{If Statements}

Perhaps the most well-known statement type is the {\tt if} statement.
For example:
\bcode\begin{verbatim}
>>> if x < 0:
...      x = 0
...      print 'Negative changed to zero'
... elif x = 0:
...      print 'Zero'
... elif x = 1:
...      print 'Single'
... else:
...      print 'More'
... 
\end{verbatim}\ecode
There can be zero or more {\tt elif} parts, and the {\tt else} part is
optional.  The keyword `{\tt elif}' is short for `{\tt else if}', and is
useful to avoid excessive indentation.  An {\tt if...elif...elif...}
sequence is a substitute for the {\em switch} or {\em case} statements
found in other languages.

\section{For Statements}

The {\tt for} statement in Python differs a bit from what you may be
used to in C or Pascal.  Rather than always iterating over an
arithmetic progression of numbers (as in Pascal), or leaving the user
completely free in the iteration test and step (as C), Python's {\tt
for} statement iterates over the items of any sequence (e.g., a list
or a string), in the order that they appear in the sequence.  For
example (no pun intended):
\bcode\begin{verbatim}
>>> # Measure some strings:
>>> a = ['cat', 'window', 'defenestrate']
>>> for x in a:
...     print x, len(x)
... 
cat 3
window 6
defenestrate 12
>>> 
\end{verbatim}\ecode
It is not safe to modify the sequence being iterated over in the loop
(this can only happen for mutable sequence types, i.e., lists).  If
you need to modify the list you are iterating over, e.g., duplicate
selected items, you must iterate over a copy.  The slice notation
makes this particularly convenient:
\bcode\begin{verbatim}
>>> for x in a[:]: # make a slice copy of the entire list
...    if len(x) > 6: a.insert(0, x)
... 
>>> a
['defenestrate', 'cat', 'window', 'defenestrate']
>>> 
\end{verbatim}\ecode

\section{The {\tt range()} Function}

If you do need to iterate over a sequence of numbers, the built-in
function {\tt range()} comes in handy.  It generates lists containing
arithmetic progressions, e.g.:
\bcode\begin{verbatim}
>>> range(10)
[0, 1, 2, 3, 4, 5, 6, 7, 8, 9]
>>> 
\end{verbatim}\ecode
The given end point is never part of the generated list; {\tt range(10)}
generates a list of 10 values, exactly the legal indices for items of a
sequence of length 10.  It is possible to let the range start at another
number, or to specify a different increment (even negative):
\bcode\begin{verbatim}
>>> range(5, 10)
[5, 6, 7, 8, 9]
>>> range(0, 10, 3)
[0, 3, 6, 9]
>>> range(-10, -100, -30)
[-10, -40, -70]
>>> 
\end{verbatim}\ecode
To iterate over the indices of a sequence, combine {\tt range()} and
{\tt len()} as follows:
\bcode\begin{verbatim}
>>> a = ['Mary', 'had', 'a', 'little', 'lamb']
>>> for i in range(len(a)):
...     print i, a[i]
... 
0 Mary
1 had
2 a
3 little
4 lamb
>>> 
\end{verbatim}\ecode

\section{Break and Continue Statements, and Else Clauses on Loops}

The {\tt break} statement, like in C, breaks out of the smallest
enclosing {\tt for} or {\tt while} loop.

The {\tt continue} statement, also borrowed from C, continues with the
next iteration of the loop.

Loop statements may have an {\tt else} clause; it is executed when the
loop terminates through exhaustion of the list (with {\tt for}) or when
the condition becomes false (with {\tt while}), but not when the loop is
terminated by a {\tt break} statement.  This is exemplified by the
following loop, which searches for a list item of value 0:
\bcode\begin{verbatim}
>>> for n in range(2, 10):
...     for x in range(2, n):
...         if n % x = 0:
...            print n, 'equals', x, '*', n/x
...            break
...     else:
...          print n, 'is a prime number'
... 
2 is a prime number
3 is a prime number
4 equals 2 * 2
5 is a prime number
6 equals 2 * 3
7 is a prime number
8 equals 2 * 4
9 equals 3 * 3
>>> 
\end{verbatim}\ecode

\section{Pass Statements}

The {\tt pass} statement does nothing.
It can be used when a statement is required syntactically but the
program requires no action.
For example:
\bcode\begin{verbatim}
>>> while 1:
...       pass # Busy-wait for keyboard interrupt
... 
\end{verbatim}\ecode

\section{Defining Functions}

We can create a function that writes the Fibonacci series to an
arbitrary boundary:
\bcode\begin{verbatim}
>>> def fib(n):    # write Fibonacci series up to n
...     a, b = 0, 1
...     while b <= n:
...           print b,
...           a, b = b, a+b
... 
>>> # Now call the function we just defined:
>>> fib(2000)
1 1 2 3 5 8 13 21 34 55 89 144 233 377 610 987 1597
>>> 
\end{verbatim}\ecode
The keyword {\tt def} introduces a function {\em definition}.  It must
be followed by the function name and the parenthesized list of formal
parameters.  The statements that form the body of the function starts at
the next line, indented by a tab stop.

The {\em execution} of a function introduces a new symbol table used
for the local variables of the function.  More precisely, all variable
assignments in a function store the value in the local symbol table;
whereas
 variable references first look in the local symbol table, then
in the global symbol table, and then in the table of built-in names.
Thus,
global variables cannot be directly assigned to from within a
function, although they may be referenced.

The actual parameters (arguments) to a function call are introduced in
the local symbol table of the called function when it is called; thus,
arguments are passed using {\em call\ by\ value}.%
\footnote{
	 Actually, {\em call  by  object reference} would be a better
	 description, since if a mutable object is passed, the caller
	 will see any changes the callee makes to it (e.g., items
	 inserted into a list).
}
When a function calls another function, a new local symbol table is
created for that call.

A function definition introduces the function name in the
current
symbol table.  The value
of the function name
has a type that is recognized by the interpreter as a user-defined
function.  This value can be assigned to another name which can then
also be used as a function.  This serves as a general renaming
mechanism:
\bcode\begin{verbatim}
>>> fib
<function object at 10042ed0>
>>> f = fib
>>> f(100)
1 1 2 3 5 8 13 21 34 55 89
>>> 
\end{verbatim}\ecode
You might object that {\tt fib} is not a function but a procedure.  In
Python, as in C, procedures are just functions that don't return a
value.  In fact, technically speaking, procedures do return a value,
albeit a rather boring one.  This value is called {\tt None} (it's a
built-in name).  Writing the value {\tt None} is normally suppressed by
the interpreter if it would be the only value written.  You can see it
if you really want to:
\bcode\begin{verbatim}
>>> print fib(0)
None
>>> 
\end{verbatim}\ecode
It is simple to write a function that returns a list of the numbers of
the Fibonacci series, instead of printing it:
\bcode\begin{verbatim}
>>> def fib2(n): # return Fibonacci series up to n
...     result = []
...     a, b = 0, 1
...     while b <= n:
...           result.append(b)    # see below
...           a, b = b, a+b
...     return result
... 
>>> f100 = fib2(100)    # call it
>>> f100                # write the result
[1, 1, 2, 3, 5, 8, 13, 21, 34, 55, 89]
>>> 
\end{verbatim}\ecode
This example, as usual, demonstrates some new Python features:

\begin{itemize}

\item
The {\tt return} statement returns with a value from a function.  {\tt
return} without an expression argument is used to return from the middle
of a procedure (falling off the end also returns from a proceduce), in
which case the {\tt None} value is returned.

\item
The statement {\tt result.append(b)} calls a {\em method} of the list
object {\tt result}.  A method is a function that `belongs' to an
object and is named {\tt obj.methodname}, where {\tt obj} is some
object (this may be an expression), and {\tt methodname} is the name
of a method that is defined by the object's type.  Different types
define different methods.  Methods of different types may have the
same name without causing ambiguity.  (It is possible to define your
own object types and methods, using {\em classes}.  This is an
advanced feature that is not discussed in this tutorial.)
The method {\tt append} shown in the example, is defined for
list objects; it adds a new element at the end of the list.  In this
example
it is equivalent to {\tt result = result + [b]}, but more efficient.

\end{itemize}

\chapter{Odds and Ends}

This chapter describes some things you've learned about already in
more detail, and adds some new things as well.

\section{More on Lists}

The list data type has some more methods.  Here are all of the methods
of lists objects:

\begin{description}

\item[{\tt insert(i, x)}]
Insert an item at a given position.  The first argument is the index of
the element before which to insert, so {\tt a.insert(0, x)} inserts at
the front of the list, and {\tt a.insert(len(a), x)} is equivalent to
{\tt a.append(x)}.

\item[{\tt append(x)}]
Equivalent to {\tt a.insert(len(a), x)}.

\item[{\tt index(x)}]
Return the index in the list of the first item whose value is {\tt x}.
It is an error if there is no such item.

\item[{\tt remove(x)}]
Remove the first item from the list whose value is {\tt x}.
It is an error if there is no such item.

\item[{\tt sort()}]
Sort the items of the list, in place.

\item[{\tt reverse()}]
Reverse the elements of the list, in place.

\end{description}

An example that uses all list methods:
\bcode\begin{verbatim}
>>> a = [66.6, 333, 333, 1, 1234.5]
>>> a.insert(2, -1)
>>> a.append(333)
>>> a
[66.6, 333, -1, 333, 1, 1234.5, 333]
>>> a.index(333)
1
>>> a.remove(333)
>>> a
[66.6, -1, 333, 1, 1234.5, 333]
>>> a.reverse()
>>> a
[333, 1234.5, 1, 333, -1, 66.6]
>>> a.sort()
>>> a
[-1, 1, 66.6, 333, 333, 1234.5]
>>>
\end{verbatim}\ecode

\section{The {\tt del} statement}

There is a way to remove an item from a list given its index instead
of its value: the {\tt del} statement.  This can also be used to
remove slices from a list (which we did earlier by assignment of an
empty list to the slice).  For example:
\bcode\begin{verbatim}
>>> a
[-1, 1, 66.6, 333, 333, 1234.5]
>>> del a[0]
>>> a
[1, 66.6, 333, 333, 1234.5]
>>> del a[2:4]
>>> a
[1, 66.6, 1234.5]
>>>
\end{verbatim}\ecode

{\tt del} can also be used to delete entire variables:
\bcode\begin{verbatim}
>>> del a
>>>
\end{verbatim}\ecode
Referencing the name {\tt a} hereafter is an error (at least until
another value is assigned to it).  We'll find other uses for {\tt del}
later.

\section{Tuples and Sequences}

We saw that lists and strings have many common properties, e.g.,
subscripting and slicing operations.  They are two examples of {\em
sequence} data types.  As Python is an evolving language, other
sequence data types may be added.  There is also another standard
sequence data type: the {\em tuple}.

A tuple consists of a number of values separated by commas, for
instance:
\bcode\begin{verbatim}
>>> t = 12345, 54321, 'hello!'
>>> t[0]
12345
>>> t
(12345, 54321, 'hello!')
>>> # Tuples may be nested:
>>> u = t, (1, 2, 3, 4, 5)
>>> u
((12345, 54321, 'hello!'), (1, 2, 3, 4, 5))
>>>
\end{verbatim}\ecode
As you see, on output tuples are alway enclosed in parentheses, so
that nested tuples are interpreted correctly; they may be input with
or without surrounding parentheses, although often parentheses are
necessary anyway (if the tuple is part of a larger expression).

Tuples have many uses, e.g., (x, y) coordinate pairs, employee records
from a database, etc.  Tuples, like strings, are immutable: it is not
possible to assign to the individual items of a tuple (you can
simulate much of the same effect with slicing and concatenation,
though).

A special problem is the construction of tuples containing 0 or 1
items: the syntax has some extra quirks to accomodate these.  Empty
tuples are constructed by an empty pair of parentheses; a tuple with
one item is constructed by following a value with a comma
(it is not sufficient to enclose a single value in parentheses).
Ugly, but effective.  For example:
\bcode\begin{verbatim}
>>> empty = ()
>>> singleton = 'hello',    # <-- note trailing comma
>>> len(empty)
0
>>> len(singleton)
1
>>> singleton
('hello',)
>>>
\end{verbatim}\ecode

The statement {\tt t = 12345, 54321, 'hello!'} is an example of {\em
tuple packing}: the values {\tt 12345}, {\tt 54321} and {\tt 'hello!'}
are packed together in a tuple.  The reverse operation is also
possible, e.g.:
\bcode\begin{verbatim}
>>> x, y, z = t
>>>
\end{verbatim}\ecode
This is called, appropriately enough, {\em tuple unpacking}.  Tuple
unpacking requires that the list of variables on the left has the same
number of elements as the length of the tuple.  Note that multiple
assignment is really just a combination of tuple packing and tuple
unpacking!

Occasionally, the corresponding operation on lists is useful: {\em list
unpacking}.  This is supported by enclosing the list of variables in
square brackets:
\bcode\begin{verbatim}
>>> a = ['foo', 'bar', 100, 1234]
>>> [a1, a2, a3, a4] = a
>>>
\end{verbatim}\ecode

\section{Dictionaries}

Another useful data type built into Python is the {\em dictionary}.
Dictionaries are sometimes found in other languages as ``associative
memories'' or ``associative arrays''.  Unlike sequences, which are
indexed by a range of numbers, dictionaries are indexed by {\em keys},
which are strings.  It is best to think of a dictionary as an unordered set of
{\em key:value} pairs, with the requirement that the keys are unique
(within one dictionary).
A pair of braces creates an empty dictionary: \verb/{}/.
Placing a comma-separated list of key:value pairs within the
braces adds initial key:value pairs to the dictionary; this is also the
way dictionaries are written on output.

The main operations on a dictionary are storing a value with some key
and extracting the value given the key.  It is also possible to delete
a key:value pair
with {\tt del}.
If you store using a key that is already in use, the old value
associated with that key is forgotten.  It is an error to extract a
value using a non-existant key.

The {\tt keys()} method of a dictionary object returns a list of all the
keys used in the dictionary, in random order (if you want it sorted,
just apply the {\tt sort()} method to the list of keys).  To check
whether a single key is in the dictionary, use the \verb/has_key()/
method of the dictionary.

Here is a small example using a dictionary:

\bcode\begin{verbatim}
>>> tel = {'jack': 4098, 'sape': 4139}
>>> tel['guido'] = 4127
>>> tel
{'sape': 4139; 'guido': 4127; 'jack': 4098}
>>> tel['jack']
4098
>>> del tel['sape']
>>> tel['irv'] = 4127
>>> tel
{'guido': 4127; 'irv': 4127; 'jack': 4098}
>>> tel.keys()
['guido', 'irv', 'jack']
>>> tel.has_key('guido')
1
>>> 
\end{verbatim}\ecode

\section{More on Conditions}

The conditions used in {\tt while} and {\tt if} statements above can
contain other operators besides comparisons.

The comparison operators {\tt in} and {\tt not in} check whether a value
occurs (does not occur) in a sequence.  The operators {\tt is} and {\tt
is not} compare whether two objects are really the same object; this
only matters for mutable objects like lists.  All comparison operators
have the same priority, which is lower than that of all numerical
operators.

Comparisons can be chained: e.g., {\tt a < b = c} tests whether {\tt a}
is less than {\tt b} and moreover {\tt b} equals {\tt c}.

Comparisons may be combined by the Boolean operators {\tt and} and {\tt
or}, and the outcome of a comparison (or of any other Boolean
expression) may be negated with {\tt not}.  These all have lower
priorities than comparison operators again; between them, {\tt not} has
the highest priority, and {\tt or} the lowest, so that
{\tt A and not B or C} is equivalent to {\tt (A and (not B)) or C}.  Of
course, parentheses can be used to express the desired composition.

The Boolean operators {\tt and} and {\tt or} are so-called {\em
shortcut} operators: their arguments are evaluated from left to right,
and evaluation stops as soon as the outcome is determined.  E.g., if
{\tt A} and {\tt C} are true but {\tt B} is false, {\tt A and B and C}
does not evaluate the expression C.  In general, the return value of a
shortcut operator, when used as a general value and not as a Boolean, is
the last evaluated argument.

It is possible to assign the result of a comparison or other Boolean
expression to a variable, but you must enclose the entire Boolean
expression in parentheses.  This is necessary because otherwise an
assignment like \verb/a = b = c/ would be ambiguous: does it assign the
value of {\tt c} to {\tt a} and {\tt b}, or does it compare {\tt b} to
{\tt c} and assign the outcome (0 or 1) to {\tt a}?  As it is, the first
meaning is what you get, and to get the latter you have to write
\verb/a = (b = c)/.  (In Python, unlike C, assignment cannot occur
inside expressions.)

\section{Comparing Sequences and Other Types}

Sequence objects may be compared to other objects with the same
sequence type.  The comparison uses {\em lexicographical} ordering:
first the first two items are compared, and if they differ this
determines the outcome of the comparison; if they are equal, the next
two items are compared, and so on, until either sequence is exhausted.
If two items to be compared are themselves sequences of the same type,
the lexiographical comparison is carried out recursively.  If all
items of two sequences compare equal, the sequences are considered
equal.  If one sequence is an initial subsequence of the other, the
shorted sequence is the smaller one.  Lexicographical ordering for
strings uses the ASCII ordering for individual characters.  Some
examples of comparisons between sequences with the same types:
\bcode\begin{verbatim}
(1, 2, 3)              < (1, 2, 4)
[1, 2, 3]              < [1, 2, 4]
'ABC' < 'C' < 'Pascal' < 'Python'
(1, 2, 3, 4)           < (1, 2, 4)
(1, 2)                 < (1, 2, -1)
(1, 2, 3)              = (1.0, 2.0, 3.0)
(1, 2, ('aa', 'ab'))   < (1, 2, ('abc', 'a'), 4)
\end{verbatim}\ecode

Note that comparing objects of different types is legal.  The outcome
is deterministic but arbitrary: the types are ordered by their name.
Thus, a list is always smaller than a string, a string is always
smaller than a tuple, etc.  Mixed numeric types are compared according
to their numeric value, so 0 equals 0.0, etc.%
\footnote{
	The rules for comparing objects of different types should
	not be relied upon; they may change in a future version of
	the language.
}

\chapter{Modules}

If you quit from the Python interpreter and enter it again, the
definitions you have made (functions and variables) are lost.
Therefore, if you want to write a somewhat longer program, you are
better off using a text editor to prepare the input for the interpreter
and run it with that file as input instead.  This is known as creating a
{\em script}.  As your program gets longer, you may want to split it
into several files for easier maintenance.  You may also want to use a
handy function that you've written in several programs without copying
its definition into each program.

To support this, Python has a way to put definitions in a file and use
them in a script or in an interactive instance of the interpreter.
Such a file is called a {\em module}; definitions from a module can be
{\em imported} into other modules or into the {\em main} module (the
collection of variables that you have access to in a script
executed at the top level
and in calculator mode).

A module is a file containing Python definitions and statements.  The
file name is the module name with the suffix {\tt .py} appended.  For
instance, use your favorite text editor to create a file called {\tt
fibo.py} in the current directory with the following contents:
\bcode\begin{verbatim}
# Fibonacci numbers module

def fib(n):    # write Fibonacci series up to n
    a, b = 0, 1
    while b <= n:
          print b,
          a, b = b, a+b

def fib2(n): # return Fibonacci series up to n
    result = []
    a, b = 0, 1
    while b <= n:
          result.append(b)
          a, b = b, a+b
    return result
\end{verbatim}\ecode
Now enter the Python interpreter and import this module with the
following command:
\bcode\begin{verbatim}
>>> import fibo
>>> 
\end{verbatim}\ecode
This does not enter the names of the functions defined in
{\tt fibo}
directly in the current symbol table; it only enters the module name
{\tt fibo}
there.
Using the module name you can access the functions:
\bcode\begin{verbatim}
>>> fibo.fib(1000)
1 1 2 3 5 8 13 21 34 55 89 144 233 377 610 987
>>> fibo.fib2(100)
[1, 1, 2, 3, 5, 8, 13, 21, 34, 55, 89]
>>> 
\end{verbatim}\ecode
If you intend to use a function often you can assign it to a local name:
\bcode\begin{verbatim}
>>> fib = fibo.fib
>>> fib(500)
1 1 2 3 5 8 13 21 34 55 89 144 233 377
>>> 
\end{verbatim}\ecode

\section{More on Modules}

A module can contain executable statements as well as function
definitions.
These statements are intended to initialize the module.
They are executed only the
{\em first}
time the module is imported somewhere.%
\footnote{
	In fact function definitions are also `statements' that are
	`executed'; the execution enters the function name in the
	module's global symbol table.
}

Each module has its own private symbol table, which is used as the
global symbol table by all functions defined in the module.
Thus, the author of a module can use global variables in the module
without worrying about accidental clashes with a user's global
variables.
On the other hand, if you know what you are doing you can touch a
module's global variables with the same notation used to refer to its
functions,
{\tt modname.itemname}.

Modules can import other modules.
It is customary but not required to place all
{\tt import}
statements at the beginning of a module (or script, for that matter).
The imported module names are placed in the importing module's global
symbol table.

There is a variant of the
{\tt import}
statement that imports names from a module directly into the importing
module's symbol table.
For example:
\bcode\begin{verbatim}
>>> from fibo import fib, fib2
>>> fib(500)
1 1 2 3 5 8 13 21 34 55 89 144 233 377
>>> 
\end{verbatim}\ecode
This does not introduce the module name from which the imports are taken
in the local symbol table (so in the example, {\tt fibo} is not
defined).

There is even a variant to import all names that a module defines:
\bcode\begin{verbatim}
>>> from fibo import *
>>> fib(500)
1 1 2 3 5 8 13 21 34 55 89 144 233 377
>>> 
\end{verbatim}\ecode
This imports all names except those beginning with an underscore
({\tt \_}).

\section{Standard Modules}

Python comes with a library of standard modules, described in a separate
document (Python Library Reference).  Some modules are built into the
interpreter; these provide access to operations that are not part of the
core of the language but are nevertheless built in, either for
efficiency or to provide access to operating system primitives such as
system calls.  The set of such modules is a configuration option; e.g.,
the {\tt amoeba} module is only provided on systems that somehow support
Amoeba primitives.  One particular module deserves some attention: {\tt
sys}, which is built into every Python interpreter.  The variables {\tt
sys.ps1} and {\tt sys.ps2} define the strings used as primary and
secondary prompts:
\bcode\begin{verbatim}
>>> import sys
>>> sys.ps1
'>>> '
>>> sys.ps2
'... '
>>> sys.ps1 = 'C> '
C> print 'Yuck!'
Yuck!
C> 
\end{verbatim}\ecode
These two variables are only defined if the interpreter is in
interactive mode.

The variable
{\tt sys.path}
is a list of strings that determine the interpreter's search path for
modules.
It is initialized to a default path taken from the environment variable
{\tt PYTHONPATH},
or from a built-in default if
{\tt PYTHONPATH}
is not set.
You can modify it using standard list operations, e.g.:
\bcode\begin{verbatim}
>>> import sys
>>> sys.path.append('/ufs/guido/lib/python')
>>> 
\end{verbatim}\ecode

\section{The {\tt dir()} function}

The built-in function {\tt dir} is used to find out which names a module
defines.  It returns a sorted list of strings:
\bcode\begin{verbatim}
>>> import fibo, sys
>>> dir(fibo)
['fib', 'fib2']
>>> dir(sys)
['argv', 'exit', 'modules', 'path', 'ps1', 'ps2', 'stderr', 'stdin', 'stdout']
>>>
\end{verbatim}\ecode
Without arguments, {\tt dir()} lists the names you have defined currently:
\bcode\begin{verbatim}
>>> a = [1, 2, 3, 4, 5]
>>> import fibo, sys
>>> fib = fibo.fib
>>> dir()
['a', 'fib', 'fibo', 'sys']
>>>
\end{verbatim}\ecode
Note that it lists all types of names: variables, modules, functions, etc.

{\tt dir()} does not list the names of built-in functions and variables.
If you want a list of those, they are defined in the standard module
{\tt builtin}:
\bcode\begin{verbatim}
>>> import builtin
>>> dir(builtin)
['EOFError', 'KeyboardInterrupt', 'MemoryError', 'NameError', 'None', 'Runti
meError', 'SystemError', 'TypeError', 'abs', 'chr', 'dir', 'divmod', 'eval',
 'exec', 'float', 'input', 'int', 'len', 'long', 'max', 'min', 'open', 'ord'
, 'pow', 'range', 'raw_input', 'reload', 'type']
>>>
\end{verbatim}\ecode

\chapter{Output Formatting}

So far we've encountered two ways of writing values: {\em expression
statements} and the {\tt print} statement.  (A third way is using the
{\tt write} method of file objects; the standard output file can be
referenced as {\tt sys.stdout}.  See the Library Reference for more
information on this.)

Often you'll want more control over the formatting of your output than
simply printing space-separated values.  The key to nice formatting in
Python is to do all the string handling yourself; using string slicing
and concatenation operations you can create any lay-out you can imagine.
The standard module {\tt string} contains some useful operations for
padding strings to a given column width; these will be discussed shortly.

One question remains, of course: how do you convert values to strings?
Luckily, Python has a way to convert any value to a string: just write
the value between reverse quotes (\verb/``/).  Some examples:
\bcode\begin{verbatim}
>>> x = 10 * 3.14
>>> y = 200*200
>>> s = 'The value of x is ' + `x` + ', and y is ' + `y` + '...'
>>> print s
The value of x is 31.4, and y is 40000...
>>> # Reverse quotes work on other types besides numbers:
>>> p = [x, y]
>>> ps = `p`
>>> ps
'[31.4, 40000]'
>>> # Converting a string adds string quotes and backslashes:
>>> hello = 'hello, world\n'
>>> hellos = `hello`
>>> print hellos
'hello, world\012'
>>> # The argument of reverse quotes may be a tuple:
>>> `x, y, ('foo', 'bar')`
'(31.4, 40000, (\'foo\', \'bar\'))'
>>>
\end{verbatim}\ecode

Here is how you write a table of squares and cubes:
\bcode\begin{verbatim}
>>> import string
>>> for x in range(1, 11):
...     print string.rjust(`x`, 2), string.rjust(`x*x`, 3),
...     # Note trailing comma on previous line
...     print string.rjust(`x*x*x`, 4)
...
 1   1    1
 2   4    8
 3   9   27
 4  16   64
 5  25  125
 6  36  216
 7  49  343
 8  64  512
 9  81  729
10 100 1000
>>>
\end{verbatim}\ecode
(Note that one space between each column was added by the way {\tt print}
works: it always adds spaces between its arguments.)

This example demonstrates the function {\tt string.rjust()}, which
right-justifies a string in a field of a given width by padding it with
spaces on the left.  There are similar functions {\tt string.ljust()}
and {\tt string.center()}.  These functions do not write anything, they
just return a new string.  If the input string is too long, they don't
truncate it, but return it unchanged; this will mess up your column
lay-out but that's usually better than the alternative, which would be
lying about a value.  (If you really want truncation you can always add
a slice operation, as in {\tt string.ljust(x,~n)[0:n]}.)

There is another function, {\tt string.zfill}, which pads a numeric
string on the left with zeros.  It understands about plus and minus
signs:%
\footnote{
	Better facilities for formatting floating point numbers are
	lacking at this moment.
}
\bcode\begin{verbatim}
>>> string.zfill('12', 5)
'00012'
>>> string.zfill('-3.14', 7)
'-003.14'
>>> string.zfill('3.14159265359', 5)
'3.14159265359'
>>>
\end{verbatim}\ecode

\chapter{Errors and Exceptions}

Until now error messages haven't been more than mentioned, but if you
have tried out the examples you have probably seen some.  There are
(at least) two distinguishable kinds of errors: {\em syntax\ errors}
and {\em exceptions}.

\section{Syntax Errors}

Syntax errors, also known as parsing errors, are perhaps the most common
kind of complaint you get while you are still learning Python:
\bcode\begin{verbatim}
>>> while 1 print 'Hello world'
Parsing error: file <stdin>, line 1:
while 1 print 'Hello world'
             ^
Unhandled exception: run-time error: syntax error
>>> 
\end{verbatim}\ecode
The parser repeats the offending line and displays a little `arrow'
pointing at the earliest point in the line where the error was detected.
The error is caused by (or at least detected at) the token
{\em preceding}
the arrow: in the example, the error is detected at the keyword
{\tt print}, since a colon ({\tt :}) is missing before it.
File name and line number are printed so you know where to look in case
the input came from a script.

\section{Exceptions}

Even if a statement or expression is syntactically correct, it may
cause an error when an attempt is made to execute it.
Errors detected during execution are called {\em exceptions} and are
not unconditionally fatal: you will soon learn how to handle them in
Python programs.  Most exceptions are not handled by programs,
however, and result in error messages as shown here:
\bcode\small\begin{verbatim}
>>> 10 * (1/0)
Unhandled exception: run-time error: integer division by zero
Stack backtrace (innermost last):
  File "<stdin>", line 1
>>> 4 + foo*3
Unhandled exception: undefined name: foo
Stack backtrace (innermost last):
  File "<stdin>", line 1
>>> '2' + 2
Unhandled exception: type error: illegal argument type for built-in operation
Stack backtrace (innermost last):
  File "<stdin>", line 1
>>> 
\end{verbatim}\ecode

The first line of the error message indicates what happened.
Exceptions come in different types, and the type is printed as part of
the message: the types in the example are
{\tt run-time error},
{\tt undefined name}
and
{\tt type error}.
The rest of the line is a detail whose interpretation depends on the
exception type.

The rest of the error message shows the context where the
exception happened.
In general it contains a stack backtrace listing source lines; however,
it will not display lines read from standard input.

Here is a summary of the most common exceptions:
\begin{itemize}
\item
{\em Run-time\ errors}
are generally caused by wrong data used by the program; this can be the
programmer's fault or caused by bad input.
The detail states the cause of the error in more detail.
\item
{\em Undefined\ name}
errors are more serious: these are usually caused by misspelled
identifiers.%
\footnote{
	The parser does not check whether names used in a program are at
	all defined elsewhere in the program; such checks are
	postponed until run-time.  The same holds for type checking.
}
The detail is the offending identifier.
\item
{\em Type\ errors} are also pretty serious: this is another case of
using wrong data (or better, using data the wrong way), but here the
error can be gleaned from the object type(s) alone.  The detail shows
in what context the error was detected.
\end{itemize}

\section{Handling Exceptions}

It is possible to write programs that handle selected exceptions.
Look at the following example, which prints a table of inverses of
some floating point numbers:
\bcode\begin{verbatim}
>>> numbers = [0.3333, 2.5, 0, 10]
>>> for x in numbers:
...     print x,
...     try:
...         print 1.0 / x
...     except RuntimeError:
...         print '*** has no inverse ***'
... 
0.3333 3.00030003
2.5 0.4
0 *** has no inverse ***
10 0.1
>>> 
\end{verbatim}\ecode
The {\tt try} statement works as follows.
\begin{itemize}
\item
First, the
{\em try\ clause}
(the statement(s) between the {\tt try} and {\tt except} keywords) is
executed.
\item
If no exception occurs, the
{\em except\ clause}
is skipped and execution of the {\tt try} statement is finished.
\item
If an exception occurs during execution of the try clause,
the rest of the clause is skipped.  Then if
its type matches the exception named after the {\tt except} keyword,
the rest of the try clause is skipped, the except clause is executed,
and then execution continues after the {\tt try} statement.
\item
If an exception occurs which does not match the exception named in the
except clause, it is passed on to outer try statements; if no handler is
found, it is an
{\em unhandled\ exception}
and execution stops with a message as shown above.
\end{itemize}
A {\tt try} statement may have more than one except clause, to specify
handlers for different exceptions.
At most one handler will be executed.
Handlers only handle exceptions that occur in the corresponding try
clause, not in other handlers of the same {\tt try} statement.
An except clause may name multiple exceptions as a parenthesized list,
e.g.:
\bcode\begin{verbatim}
... except (RuntimeError, TypeError, NameError):
...     pass
\end{verbatim}\ecode
The last except clause may omit the exception name(s), to serve as a
wildcard.
Use this with extreme caution!

When an exception occurs, it may have an associated value, also known as
the exceptions's
{\em argument}.
The presence and type of the argument depend on the exception type.
For exception types which have an argument, the except clause may
specify a variable after the exception name (or list) to receive the
argument's value, as follows:
\bcode\begin{verbatim}
>>> try:
...     foo()
... except NameError, x:
...     print 'name', x, 'undefined'
... 
name foo undefined
>>> 
\end{verbatim}\ecode
If an exception has an argument, it is printed as the third part
(`detail') of the message for unhandled exceptions.

Standard exception names are built-in identifiers (not reserved
keywords).
These are in fact string objects whose
{\em object\ identity}
(not their value!) identifies the exceptions.
The string is printed as the second part of the message for unhandled
exceptions.
Their names and values are:
\bcode\begin{verbatim}
EOFError              'end-of-file read'
KeyboardInterrupt     'keyboard interrupt'
MemoryError           'out of memory'           *
NameError             'undefined name'          *
RuntimeError          'run-time error'          *
SystemError           'system error'            *
TypeError             'type error'              *
\end{verbatim}\ecode
The meanings should be clear enough.
Those exceptions with a {\tt *} in the third column have an argument.

Exception handlers don't just handle exceptions if they occur
immediately in the try clause, but also if they occur inside functions
that are called (even indirectly) in the try clause.
For example:
\bcode\begin{verbatim}
>>> def this_fails():
...     x = 1/0
... 
>>> try:
...     this_fails()
... except RuntimeError, detail:
...     print 'Handling run-time error:', detail
... 
Handling run-time error: integer division by zero
>>> 
\end{verbatim}\ecode

\section{Raising Exceptions}

The {\tt raise} statement allows the programmer to force a specified
exception to occur.
For example:
\bcode\begin{verbatim}
>>> raise NameError, 'Hi There!'
Unhandled exception: undefined name: Hi There!
Stack backtrace (innermost last):
  File "<stdin>", line 1
>>> 
\end{verbatim}\ecode
The first argument to {\tt raise} names the exception to be raised.
The optional second argument specifies the exception's argument.

\section{User-defined Exceptions}

Programs may name their own exceptions by assigning a string to a
variable.
For example:
\bcode\begin{verbatim}
>>> my_exc = 'nobody likes me!'
>>> try:
...     raise my_exc, 2*2
... except my_exc, val:
...     print 'My exception occurred, value:', val
... 
My exception occured, value: 4
>>> raise my_exc, 1
Unhandled exception: nobody likes me!: 1
Stack backtrace (innermost last):
  File "<stdin>", line 7
>>> 
\end{verbatim}\ecode
Many standard modules use this to report errors that may occur in
functions they define.

\section{Defining Clean-up Actions}

The {\tt try} statement has another optional clause which is intended to
define clean-up actions that must be executed under all circumstances.
For example:
\bcode\begin{verbatim}
>>> try:
...     raise KeyboardInterrupt
... finally:
...     print 'Goodbye, world!'
... 
Goodbye, world!
Unhandled exception: keyboard interrupt
Stack backtrace (innermost last):
  File "<stdin>", line 2
>>> 
\end{verbatim}\ecode
The
{\em finally\ clause}
must follow the except clauses(s), if any.
It is executed whether or not an exception occurred,
or whether or not an exception is handled.
If the exception is handled, the finally clause is executed after the
handler (and even if another exception occurred in the handler).
It is also executed when the {\tt try} statement is left via a
{\tt break} or {\tt return} statement.

\end{document}
