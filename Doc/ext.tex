\documentclass[twoside,openright]{report}
\usepackage{myformat}

% XXX PM Modulator

\title{Extending and Embedding the Python Interpreter}

\author{Guido van Rossum\\
	Fred L. Drake, Jr., editor}
\authoraddress{
	PythonLabs\\
	E-mail: \email{python-docs@python.org}
}

\date{June 15, 2001}		% XXX update before release!
\release{2.0.1c1}		% software release, not documentation
\setshortversion{2.0}		% major.minor only for software


% Tell \index to actually write the .idx file
\makeindex

\begin{document}

\maketitle

Copyright 1991, 1992, 1993, 1994 by Stichting Mathematisch Centrum,
Amsterdam, The Netherlands.

\begin{center}
All Rights Reserved
\end{center}

Permission to use, copy, modify, and distribute this software and its
documentation for any purpose and without fee is hereby granted,
provided that the above copyright notice appear in all copies and that
both that copyright notice and this permission notice appear in
supporting documentation, and that the names of Stichting Mathematisch
Centrum or CWI not be used in advertising or publicity pertaining to
distribution of the software without specific, written prior permission.

STICHTING MATHEMATISCH CENTRUM DISCLAIMS ALL WARRANTIES WITH REGARD TO
THIS SOFTWARE, INCLUDING ALL IMPLIED WARRANTIES OF MERCHANTABILITY AND
FITNESS, IN NO EVENT SHALL STICHTING MATHEMATISCH CENTRUM BE LIABLE
FOR ANY SPECIAL, INDIRECT OR CONSEQUENTIAL DAMAGES OR ANY DAMAGES
WHATSOEVER RESULTING FROM LOSS OF USE, DATA OR PROFITS, WHETHER IN AN
ACTION OF CONTRACT, NEGLIGENCE OR OTHER TORTIOUS ACTION, ARISING OUT
OF OR IN CONNECTION WITH THE USE OR PERFORMANCE OF THIS SOFTWARE.


\begin{abstract}

\noindent
Python is an interpreted, object-oriented programming language.  This
document describes how to write modules in \C{} or \Cpp{} to extend the
Python interpreter with new modules.  Those modules can define new
functions but also new object types and their methods.  The document
also describes how to embed the Python interpreter in another
application, for use as an extension language.  Finally, it shows how
to compile and link extension modules so that they can be loaded
dynamically (at run time) into the interpreter, if the underlying
operating system supports this feature.

This document assumes basic knowledge about Python.  For an informal
introduction to the language, see the Python Tutorial.  The \emph{Python
Reference Manual} gives a more formal definition of the language.  The
\emph{Python Library Reference} documents the existing object types,
functions and modules (both built-in and written in Python) that give
the language its wide application range.

For a detailed description of the whole Python/\C{} API, see the separate
\emph{Python/\C{} API Reference Manual}.  \strong{Note:} While that
manual is still in a state of flux, it is safe to say that it is much
more up to date than the manual you're reading currently (which has
been in need for an upgrade for some time now).


\end{abstract}

\tableofcontents


\chapter{Extending Python with \C{} or \Cpp{} code}


%\section{Introduction}
\label{intro}

It is quite easy to add new built-in modules to Python, if you know
how to program in \C{}.  Such \dfn{extension modules} can do two things
that can't be done directly in Python: they can implement new built-in
object types, and they can call \C{} library functions and system calls.

To support extensions, the Python API (Application Programmers
Interface) defines a set of functions, macros and variables that
provide access to most aspects of the Python run-time system.  The
Python API is incorporated in a \C{} source file by including the header
\code{"Python.h"}.

The compilation of an extension module depends on its intended use as
well as on your system setup; details are given in a later section.


\section{A Simple Example}
\label{simpleExample}

Let's create an extension module called \samp{spam} (the favorite food
of Monty Python fans...) and let's say we want to create a Python
interface to the \C{} library function \cfunction{system()}.\footnote{An
interface for this function already exists in the standard module
\module{os} --- it was chosen as a simple and straightfoward example.}
This function takes a null-terminated character string as argument and
returns an integer.  We want this function to be callable from Python
as follows:

\begin{verbatim}
>>> import spam
>>> status = spam.system("ls -l")
\end{verbatim}

Begin by creating a file \samp{spammodule.c}.  (In general, if a
module is called \samp{spam}, the \C{} file containing its implementation
is called \file{spammodule.c}; if the module name is very long, like
\samp{spammify}, the module name can be just \file{spammify.c}.)

The first line of our file can be:

\begin{verbatim}
#include "Python.h"
\end{verbatim}

which pulls in the Python API (you can add a comment describing the
purpose of the module and a copyright notice if you like).

All user-visible symbols defined by \code{"Python.h"} have a prefix of
\samp{Py} or \samp{PY}, except those defined in standard header files.
For convenience, and since they are used extensively by the Python
interpreter, \code{"Python.h"} includes a few standard header files:
\code{<stdio.h>}, \code{<string.h>}, \code{<errno.h>}, and
\code{<stdlib.h>}.  If the latter header file does not exist on your
system, it declares the functions \cfunction{malloc()},
\cfunction{free()} and \cfunction{realloc()} directly.

The next thing we add to our module file is the \C{} function that will
be called when the Python expression \samp{spam.system(\var{string})}
is evaluated (we'll see shortly how it ends up being called):

\begin{verbatim}
static PyObject *
spam_system(self, args)
    PyObject *self;
    PyObject *args;
{
    char *command;
    int sts;
    if (!PyArg_ParseTuple(args, "s", &command))
        return NULL;
    sts = system(command);
    return Py_BuildValue("i", sts);
}
\end{verbatim}

There is a straightforward translation from the argument list in
Python (e.g.\ the single expression \code{"ls -l"}) to the arguments
passed to the \C{} function.  The \C{} function always has two arguments,
conventionally named \var{self} and \var{args}.

The \var{self} argument is only used when the \C{} function implements a
builtin method.  This will be discussed later. In the example,
\var{self} will always be a \NULL{} pointer, since we are defining
a function, not a method.  (This is done so that the interpreter
doesn't have to understand two different types of \C{} functions.)

The \var{args} argument will be a pointer to a Python tuple object
containing the arguments.  Each item of the tuple corresponds to an
argument in the call's argument list.  The arguments are Python
objects --- in order to do anything with them in our \C{} function we have
to convert them to \C{} values.  The function \cfunction{PyArg_ParseTuple()}
in the Python API checks the argument types and converts them to \C{}
values.  It uses a template string to determine the required types of
the arguments as well as the types of the \C{} variables into which to
store the converted values.  More about this later.

\cfunction{PyArg_ParseTuple()} returns true (nonzero) if all arguments have
the right type and its components have been stored in the variables
whose addresses are passed.  It returns false (zero) if an invalid
argument list was passed.  In the latter case it also raises an
appropriate exception by so the calling function can return
\NULL{} immediately (as we saw in the example).


\section{Intermezzo: Errors and Exceptions}
\label{errors}

An important convention throughout the Python interpreter is the
following: when a function fails, it should set an exception condition
and return an error value (usually a \NULL{} pointer).  Exceptions
are stored in a static global variable inside the interpreter; if this
variable is \NULL{} no exception has occurred.  A second global
variable stores the ``associated value'' of the exception (the second
argument to \keyword{raise}).  A third variable contains the stack
traceback in case the error originated in Python code.  These three
variables are the \C{} equivalents of the Python variables
\code{sys.exc_type}, \code{sys.exc_value} and \code{sys.exc_traceback}
(see the section on module \module{sys} in the \emph{Python Library
Reference}).  It is important to know about them to understand how
errors are passed around.

The Python API defines a number of functions to set various types of
exceptions.

The most common one is \cfunction{PyErr_SetString()}.  Its arguments
are an exception object and a \C{} string.  The exception object is
usually a predefined object like \cdata{PyExc_ZeroDivisionError}.  The
\C{} string indicates the cause of the error and is converted to a
Python string object and stored as the ``associated value'' of the
exception.

Another useful function is \cfunction{PyErr_SetFromErrno()}, which only
takes an exception argument and constructs the associated value by
inspection of the (\UNIX{}) global variable \cdata{errno}.  The most
general function is \cfunction{PyErr_SetObject()}, which takes two object
arguments, the exception and its associated value.  You don't need to
\cfunction{Py_INCREF()} the objects passed to any of these functions.

You can test non-destructively whether an exception has been set with
\cfunction{PyErr_Occurred()}.  This returns the current exception object,
or \NULL{} if no exception has occurred.  You normally don't need
to call \cfunction{PyErr_Occurred()} to see whether an error occurred in a
function call, since you should be able to tell from the return value.

When a function \var{f} that calls another function \var{g} detects
that the latter fails, \var{f} should itself return an error value
(e.g. \NULL{} or \code{-1}).  It should \emph{not} call one of the
\cfunction{PyErr_*()} functions --- one has already been called by \var{g}.
\var{f}'s caller is then supposed to also return an error indication
to \emph{its} caller, again \emph{without} calling \cfunction{PyErr_*()},
and so on --- the most detailed cause of the error was already
reported by the function that first detected it.  Once the error
reaches the Python interpreter's main loop, this aborts the currently
executing Python code and tries to find an exception handler specified
by the Python programmer.

(There are situations where a module can actually give a more detailed
error message by calling another \cfunction{PyErr_*()} function, and in
such cases it is fine to do so.  As a general rule, however, this is
not necessary, and can cause information about the cause of the error
to be lost: most operations can fail for a variety of reasons.)

To ignore an exception set by a function call that failed, the exception
condition must be cleared explicitly by calling \cfunction{PyErr_Clear()}. 
The only time \C{} code should call \cfunction{PyErr_Clear()} is if it doesn't
want to pass the error on to the interpreter but wants to handle it
completely by itself (e.g. by trying something else or pretending
nothing happened).

Note that a failing \cfunction{malloc()} call must be turned into an
exception --- the direct caller of \cfunction{malloc()} (or
\cfunction{realloc()}) must call \cfunction{PyErr_NoMemory()} and
return a failure indicator itself.  All the object-creating functions
(\cfunction{PyInt_FromLong()} etc.) already do this, so only if you
call \cfunction{malloc()} directly this note is of importance.

Also note that, with the important exception of
\cfunction{PyArg_ParseTuple()} and friends, functions that return an
integer status usually return a positive value or zero for success and
\code{-1} for failure, like \UNIX{} system calls.

Finally, be careful to clean up garbage (by making
\cfunction{Py_XDECREF()} or \cfunction{Py_DECREF()} calls for objects
you have already created) when you return an error indicator!

The choice of which exception to raise is entirely yours.  There are
predeclared \C{} objects corresponding to all built-in Python exceptions,
e.g. \cdata{PyExc_ZeroDevisionError} which you can use directly.  Of
course, you should choose exceptions wisely --- don't use
\cdata{PyExc_TypeError} to mean that a file couldn't be opened (that
should probably be \cdata{PyExc_IOError}).  If something's wrong with
the argument list, the \cfunction{PyArg_ParseTuple()} function usually
raises \cdata{PyExc_TypeError}.  If you have an argument whose value
which must be in a particular range or must satisfy other conditions,
\cdata{PyExc_ValueError} is appropriate.

You can also define a new exception that is unique to your module.
For this, you usually declare a static object variable at the
beginning of your file, e.g.

\begin{verbatim}
static PyObject *SpamError;
\end{verbatim}

and initialize it in your module's initialization function
(\cfunction{initspam()}) with an exception object, e.g. (leaving out
the error checking for now):

\begin{verbatim}
void
initspam()
{
    PyObject *m, *d;
    m = Py_InitModule("spam", SpamMethods);
    d = PyModule_GetDict(m);
    SpamError = PyErr_NewException("spam.error", NULL, NULL);
    PyDict_SetItemString(d, "error", SpamError);
}
\end{verbatim}

Note that the Python name for the exception object is
\exception{spam.error}.  The \cfunction{PyErr_NewException()} function
may create either a string or class, depending on whether the
\samp{-X} flag was passed to the interpreter.  If \samp{-X} was used,
\cdata{SpamError} will be a string object, otherwise it will be a
class object with the base class being \exception{Exception},
described in the \emph{Python Library Reference} under ``Built-in
Exceptions.''


\section{Back to the Example}
\label{backToExample}

Going back to our example function, you should now be able to
understand this statement:

\begin{verbatim}
    if (!PyArg_ParseTuple(args, "s", &command))
        return NULL;
\end{verbatim}

It returns \NULL{} (the error indicator for functions returning
object pointers) if an error is detected in the argument list, relying
on the exception set by \cfunction{PyArg_ParseTuple()}.  Otherwise the
string value of the argument has been copied to the local variable
\cdata{command}.  This is a pointer assignment and you are not supposed
to modify the string to which it points (so in Standard \C{}, the variable
\cdata{command} should properly be declared as \samp{const char
*command}).

The next statement is a call to the \UNIX{} function
\cfunction{system()}, passing it the string we just got from
\cfunction{PyArg_ParseTuple()}:

\begin{verbatim}
    sts = system(command);
\end{verbatim}

Our \function{spam.system()} function must return the value of
\cdata{sts} as a Python object.  This is done using the function
\cfunction{Py_BuildValue()}, which is something like the inverse of
\cfunction{PyArg_ParseTuple()}: it takes a format string and an
arbitrary number of \C{} values, and returns a new Python object.
More info on \cfunction{Py_BuildValue()} is given later.

\begin{verbatim}
    return Py_BuildValue("i", sts);
\end{verbatim}

In this case, it will return an integer object.  (Yes, even integers
are objects on the heap in Python!)

If you have a \C{} function that returns no useful argument (a function
returning \ctype{void}), the corresponding Python function must return
\code{None}.   You need this idiom to do so:

\begin{verbatim}
    Py_INCREF(Py_None);
    return Py_None;
\end{verbatim}

\cdata{Py_None} is the \C{} name for the special Python object
\code{None}.  It is a genuine Python object (not a \NULL{}
pointer, which means ``error'' in most contexts, as we have seen).


\section{The Module's Method Table and Initialization Function}
\label{methodTable}

I promised to show how \cfunction{spam_system()} is called from Python
programs.  First, we need to list its name and address in a ``method
table'':

\begin{verbatim}
static PyMethodDef SpamMethods[] = {
    ...
    {"system",  spam_system, METH_VARARGS},
    ...
    {NULL,      NULL}        /* Sentinel */
};
\end{verbatim}

Note the third entry (\samp{METH_VARARGS}).  This is a flag telling
the interpreter the calling convention to be used for the \C{}
function.  It should normally always be \samp{METH_VARARGS} or
\samp{METH_VARARGS | METH_KEYWORDS}; a value of \samp{0} means that an
obsolete variant of \cfunction{PyArg_ParseTuple()} is used.

When using only \samp{METH_VARARGS}, the function should expect
the Python-level parameters to be passed in as a tuple acceptable for
parsing via \cfunction{PyArg_ParseTuple()}; more information on this
function is provided below.

The \constant{METH_KEYWORDS} bit may be set in the third field if keyword
arguments should be passed to the function.  In this case, the \C{}
function should accept a third \samp{PyObject *} parameter which will
be a dictionary of keywords.  Use \cfunction{PyArg_ParseTupleAndKeywords()}
to parse the arguemts to such a function.

The method table must be passed to the interpreter in the module's
initialization function (which should be the only non-\code{static}
item defined in the module file):

\begin{verbatim}
void
initspam()
{
    (void) Py_InitModule("spam", SpamMethods);
}
\end{verbatim}

When the Python program imports module \module{spam} for the first
time, \cfunction{initspam()} is called.  It calls
\cfunction{Py_InitModule()}, which creates a ``module object'' (which
is inserted in the dictionary \code{sys.modules} under the key
\code{"spam"}), and inserts built-in function objects into the newly
created module based upon the table (an array of \ctype{PyMethodDef}
structures) that was passed as its second argument.
\cfunction{Py_InitModule()} returns a pointer to the module object
that it creates (which is unused here).  It aborts with a fatal error
if the module could not be initialized satisfactorily, so the caller
doesn't need to check for errors.


\section{Compilation and Linkage}
\label{compilation}

There are two more things to do before you can use your new extension:
compiling and linking it with the Python system.  If you use dynamic
loading, the details depend on the style of dynamic loading your
system uses; see the chapter on Dynamic Loading for more info about
this.

If you can't use dynamic loading, or if you want to make your module a
permanent part of the Python interpreter, you will have to change the
configuration setup and rebuild the interpreter.  Luckily, this is
very simple: just place your file (\file{spammodule.c} for example) in
the \file{Modules} directory, add a line to the file
\file{Modules/Setup} describing your file:

\begin{verbatim}
spam spammodule.o
\end{verbatim}

and rebuild the interpreter by running \program{make} in the toplevel
directory.  You can also run \program{make} in the \file{Modules}
subdirectory, but then you must first rebuilt the \file{Makefile}
there by running `\program{make} Makefile'.  (This is necessary each
time you change the \file{Setup} file.)

If your module requires additional libraries to link with, these can
be listed on the line in the \file{Setup} file as well, for instance:

\begin{verbatim}
spam spammodule.o -lX11
\end{verbatim}

\section{Calling Python Functions From \C{}}
\label{callingPython}

So far we have concentrated on making \C{} functions callable from
Python.  The reverse is also useful: calling Python functions from \C{}.
This is especially the case for libraries that support so-called
``callback'' functions.  If a \C{} interface makes use of callbacks, the
equivalent Python often needs to provide a callback mechanism to the
Python programmer; the implementation will require calling the Python
callback functions from a \C{} callback.  Other uses are also imaginable.

Fortunately, the Python interpreter is easily called recursively, and
there is a standard interface to call a Python function.  (I won't
dwell on how to call the Python parser with a particular string as
input --- if you're interested, have a look at the implementation of
the \samp{-c} command line option in \file{Python/pythonmain.c}.)

Calling a Python function is easy.  First, the Python program must
somehow pass you the Python function object.  You should provide a
function (or some other interface) to do this.  When this function is
called, save a pointer to the Python function object (be careful to
\cfunction{Py_INCREF()} it!) in a global variable --- or whereever you
see fit. For example, the following function might be part of a module
definition:

\begin{verbatim}
static PyObject *my_callback = NULL;

static PyObject *
my_set_callback(dummy, arg)
    PyObject *dummy, *arg;
{
    Py_XDECREF(my_callback); /* Dispose of previous callback */
    Py_XINCREF(arg); /* Add a reference to new callback */
    my_callback = arg; /* Remember new callback */
    /* Boilerplate to return "None" */
    Py_INCREF(Py_None);
    return Py_None;
}
\end{verbatim}

The macros \cfunction{Py_XINCREF()} and \cfunction{Py_XDECREF()}
increment/decrement the reference count of an object and are safe in
the presence of \NULL{} pointers.  More info on them in the section on
Reference Counts below.

Later, when it is time to call the function, you call the \C{} function
\cfunction{PyEval_CallObject()}.  This function has two arguments, both
pointers to arbitrary Python objects: the Python function, and the
argument list.  The argument list must always be a tuple object, whose
length is the number of arguments.  To call the Python function with
no arguments, pass an empty tuple; to call it with one argument, pass
a singleton tuple.  \cfunction{Py_BuildValue()} returns a tuple when its
format string consists of zero or more format codes between
parentheses.  For example:

\begin{verbatim}
    int arg;
    PyObject *arglist;
    PyObject *result;
    ...
    arg = 123;
    ...
    /* Time to call the callback */
    arglist = Py_BuildValue("(i)", arg);
    result = PyEval_CallObject(my_callback, arglist);
    Py_DECREF(arglist);
\end{verbatim}

\cfunction{PyEval_CallObject()} returns a Python object pointer: this is
the return value of the Python function.  \cfunction{PyEval_CallObject()} is
``reference-count-neutral'' with respect to its arguments.  In the
example a new tuple was created to serve as the argument list, which
is \cfunction{Py_DECREF()}-ed immediately after the call.

The return value of \cfunction{PyEval_CallObject()} is ``new'': either it
is a brand new object, or it is an existing object whose reference
count has been incremented.  So, unless you want to save it in a
global variable, you should somehow \cfunction{Py_DECREF()} the result,
even (especially!) if you are not interested in its value.

Before you do this, however, it is important to check that the return
value isn't \NULL{}.  If it is, the Python function terminated by
raising an exception.  If the \C{} code that called
\cfunction{PyEval_CallObject()} is called from Python, it should now
return an error indication to its Python caller, so the interpreter
can print a stack trace, or the calling Python code can handle the
exception.  If this is not possible or desirable, the exception should
be cleared by calling \cfunction{PyErr_Clear()}.  For example:

\begin{verbatim}
    if (result == NULL)
        return NULL; /* Pass error back */
    ...use result...
    Py_DECREF(result); 
\end{verbatim}

Depending on the desired interface to the Python callback function,
you may also have to provide an argument list to
\cfunction{PyEval_CallObject()}.  In some cases the argument list is
also provided by the Python program, through the same interface that
specified the callback function.  It can then be saved and used in the
same manner as the function object.  In other cases, you may have to
construct a new tuple to pass as the argument list.  The simplest way
to do this is to call \cfunction{Py_BuildValue()}.  For example, if
you want to pass an integral event code, you might use the following
code:

\begin{verbatim}
    PyObject *arglist;
    ...
    arglist = Py_BuildValue("(l)", eventcode);
    result = PyEval_CallObject(my_callback, arglist);
    Py_DECREF(arglist);
    if (result == NULL)
        return NULL; /* Pass error back */
    /* Here maybe use the result */
    Py_DECREF(result);
\end{verbatim}

Note the placement of \samp{Py_DECREF(arglist)} immediately after the
call, before the error check!  Also note that strictly spoken this
code is not complete: \cfunction{Py_BuildValue()} may run out of
memory, and this should be checked.


\section{Format Strings for \sectcode{PyArg_ParseTuple()}}
\label{parseTuple}

The \cfunction{PyArg_ParseTuple()} function is declared as follows:

\begin{verbatim}
int PyArg_ParseTuple(PyObject *arg, char *format, ...);
\end{verbatim}

The \var{arg} argument must be a tuple object containing an argument
list passed from Python to a \C{} function.  The \var{format} argument
must be a format string, whose syntax is explained below.  The
remaining arguments must be addresses of variables whose type is
determined by the format string.  For the conversion to succeed, the
\var{arg} object must match the format and the format must be
exhausted.

Note that while \cfunction{PyArg_ParseTuple()} checks that the Python
arguments have the required types, it cannot check the validity of the
addresses of \C{} variables passed to the call: if you make mistakes
there, your code will probably crash or at least overwrite random bits
in memory.  So be careful!

A format string consists of zero or more ``format units''.  A format
unit describes one Python object; it is usually a single character or
a parenthesized sequence of format units.  With a few exceptions, a
format unit that is not a parenthesized sequence normally corresponds
to a single address argument to \cfunction{PyArg_ParseTuple()}.  In the
following description, the quoted form is the format unit; the entry
in (round) parentheses is the Python object type that matches the
format unit; and the entry in [square] brackets is the type of the \C{}
variable(s) whose address should be passed.  (Use the \samp{\&}
operator to pass a variable's address.)

\begin{description}

\item[\samp{s} (string) [char *{]}]
Convert a Python string to a \C{} pointer to a character string.  You
must not provide storage for the string itself; a pointer to an
existing string is stored into the character pointer variable whose
address you pass.  The \C{} string is null-terminated.  The Python string
must not contain embedded null bytes; if it does, a \exception{TypeError}
exception is raised.

\item[\samp{s\#} (string) {[char *, int]}]
This variant on \samp{s} stores into two \C{} variables, the first one
a pointer to a character string, the second one its length.  In this
case the Python string may contain embedded null bytes.

\item[\samp{z} (string or \code{None}) {[char *]}]
Like \samp{s}, but the Python object may also be \code{None}, in which
case the \C{} pointer is set to \NULL{}.

\item[\samp{z\#} (string or \code{None}) {[char *, int]}]
This is to \samp{s\#} as \samp{z} is to \samp{s}.

\item[\samp{b} (integer) {[char]}]
Convert a Python integer to a tiny int, stored in a \C{} \ctype{char}.

\item[\samp{h} (integer) {[short int]}]
Convert a Python integer to a \C{} \ctype{short int}.

\item[\samp{i} (integer) {[int]}]
Convert a Python integer to a plain \C{} \ctype{int}.

\item[\samp{l} (integer) {[long int]}]
Convert a Python integer to a \C{} \ctype{long int}.

\item[\samp{c} (string of length 1) {[char]}]
Convert a Python character, represented as a string of length 1, to a
\C{} \ctype{char}.

\item[\samp{f} (float) {[float]}]
Convert a Python floating point number to a \C{} \ctype{float}.

\item[\samp{d} (float) {[double]}]
Convert a Python floating point number to a \C{} \ctype{double}.

\item[\samp{D} (complex) {[Py_complex]}]
Convert a Python complex number to a \C{} \ctype{Py_complex} structure.

\item[\samp{O} (object) {[PyObject *]}]
Store a Python object (without any conversion) in a \C{} object pointer.
The \C{} program thus receives the actual object that was passed.  The
object's reference count is not increased.  The pointer stored is not
\NULL{}.

\item[\samp{O!} (object) {[\var{typeobject}, PyObject *{]}}]
Store a Python object in a \C{} object pointer.  This is similar to
\samp{O}, but takes two \C{} arguments: the first is the address of a
Python type object, the second is the address of the \C{} variable (of
type \ctype{PyObject *}) into which the object pointer is stored.
If the Python object does not have the required type, a
\exception{TypeError} exception is raised.

\item[\samp{O\&} (object) {[\var{converter}, \var{anything}{]}}]
Convert a Python object to a \C{} variable through a \var{converter}
function.  This takes two arguments: the first is a function, the
second is the address of a \C{} variable (of arbitrary type), converted
to \ctype{void *}.  The \var{converter} function in turn is called as
follows:

\code{\var{status} = \var{converter}(\var{object}, \var{address});}

where \var{object} is the Python object to be converted and
\var{address} is the \ctype{void *} argument that was passed to
\cfunction{PyArg_ConvertTuple()}.  The returned \var{status} should be
\code{1} for a successful conversion and \code{0} if the conversion
has failed.  When the conversion fails, the \var{converter} function
should raise an exception.

\item[\samp{S} (string) {[PyStringObject *]}]
Like \samp{O} but requires that the Python object is a string object.
Raises a \exception{TypeError} exception if the object is not a string
object.  The \C{} variable may also be declared as \ctype{PyObject *}.

\item[\samp{(\var{items})} (tuple) {[\var{matching-items}{]}}]
The object must be a Python tuple whose length is the number of format
units in \var{items}.  The \C{} arguments must correspond to the
individual format units in \var{items}.  Format units for tuples may
be nested.

\end{description}

It is possible to pass Python long integers where integers are
requested; however no proper range checking is done --- the most
significant bits are silently truncated when the receiving field is
too small to receive the value (actually, the semantics are inherited
from downcasts in \C{} --- your milage may vary).

A few other characters have a meaning in a format string.  These may
not occur inside nested parentheses.  They are:

\begin{description}

\item[\samp{|}]
Indicates that the remaining arguments in the Python argument list are
optional.  The \C{} variables corresponding to optional arguments should
be initialized to their default value --- when an optional argument is
not specified, \cfunction{PyArg_ParseTuple()} does not touch the contents
of the corresponding \C{} variable(s).

\item[\samp{:}]
The list of format units ends here; the string after the colon is used
as the function name in error messages (the ``associated value'' of
the exceptions that \cfunction{PyArg_ParseTuple()} raises).

\item[\samp{;}]
The list of format units ends here; the string after the colon is used
as the error message \emph{instead} of the default error message.
Clearly, \samp{:} and \samp{;} mutually exclude each other.

\end{description}

Some example calls:

\begin{verbatim}
    int ok;
    int i, j;
    long k, l;
    char *s;
    int size;

    ok = PyArg_ParseTuple(args, ""); /* No arguments */
        /* Python call: f() */

    ok = PyArg_ParseTuple(args, "s", &s); /* A string */
        /* Possible Python call: f('whoops!') */

    ok = PyArg_ParseTuple(args, "lls", &k, &l, &s); /* Two longs and a string */
        /* Possible Python call: f(1, 2, 'three') */

    ok = PyArg_ParseTuple(args, "(ii)s#", &i, &j, &s, &size);
        /* A pair of ints and a string, whose size is also returned */
        /* Possible Python call: f((1, 2), 'three') */

    {
        char *file;
        char *mode = "r";
        int bufsize = 0;
        ok = PyArg_ParseTuple(args, "s|si", &file, &mode, &bufsize);
        /* A string, and optionally another string and an integer */
        /* Possible Python calls:
           f('spam')
           f('spam', 'w')
           f('spam', 'wb', 100000) */
    }

    {
        int left, top, right, bottom, h, v;
        ok = PyArg_ParseTuple(args, "((ii)(ii))(ii)",
                 &left, &top, &right, &bottom, &h, &v);
                 /* A rectangle and a point */
                 /* Possible Python call:
                    f(((0, 0), (400, 300)), (10, 10)) */
    }

    {
        Py_complex c;
        ok = PyArg_ParseTuple(args, "D:myfunction", &c);
        /* a complex, also providing a function name for errors */
        /* Possible Python call: myfunction(1+2j) */
    }
\end{verbatim}


\section{Keyword Parsing with \sectcode{PyArg_ParseTupleAndKeywords()}}
\label{parseTupleAndKeywords}

The \cfunction{PyArg_ParseTupleAndKeywords()} function is declared as
follows:

\begin{verbatim}
int PyArg_ParseTupleAndKeywords(PyObject *arg, PyObject *kwdict,
                                char *format, char **kwlist, ...);
\end{verbatim}

The \var{arg} and \var{format} parameters are identical to those of the
\cfunction{PyArg_ParseTuple()} function.  The \var{kwdict} parameter
is the dictionary of keywords received as the third parameter from the 
Python runtime.  The \var{kwlist} parameter is a \NULL{}-terminated
list of strings which identify the parameters; the names are matched
with the type information from \var{format} from left to right.

\strong{Note:}  Nested tuples cannot be parsed when using keyword
arguments!  Keyword parameters passed in which are not present in the
\var{kwlist} will cause a \exception{TypeError} to be raised.

Here is an example module which uses keywords, based on an example by
Geoff Philbrick (\email{philbrick@hks.com}):

\begin{verbatim}
#include <stdio.h>
#include "Python.h"

static PyObject *
keywdarg_parrot(self, args, keywds)
    PyObject *self;
    PyObject *args;
    PyObject *keywds;
{  
    int voltage;
    char *state = "a stiff";
    char *action = "voom";
    char *type = "Norwegian Blue";

    static char *kwlist[] = {"voltage", "state", "action", "type", NULL};

    if (!PyArg_ParseTupleAndKeywords(args, keywds, "i|sss", kwlist, 
                                     &voltage, &state, &action, &type))
        return NULL; 
  
    printf("-- This parrot wouldn't %s if you put %i Volts through it.\n", 
           action, voltage);
    printf("-- Lovely plumage, the %s -- It's %s!\n", type, state);

    Py_INCREF(Py_None);

    return Py_None;
}

static PyMethodDef keywdarg_methods[] = {
    {"parrot", (PyCFunction)keywdarg_parrot, METH_VARARGS|METH_KEYWORDS},
    {NULL,  NULL}   /* sentinel */
};

void
initkeywdarg()
{
  /* Create the module and add the functions */
  Py_InitModule("keywdarg", keywdarg_methods);  
  
}
\end{verbatim}


\section{The \sectcode{Py_BuildValue()} Function}
\label{buildValue}

This function is the counterpart to \cfunction{PyArg_ParseTuple()}.  It is
declared as follows:

\begin{verbatim}
PyObject *Py_BuildValue(char *format, ...);
\end{verbatim}

It recognizes a set of format units similar to the ones recognized by
\cfunction{PyArg_ParseTuple()}, but the arguments (which are input to the
function, not output) must not be pointers, just values.  It returns a
new Python object, suitable for returning from a \C{} function called
from Python.

One difference with \cfunction{PyArg_ParseTuple()}: while the latter
requires its first argument to be a tuple (since Python argument lists
are always represented as tuples internally),
\cfunction{Py_BuildValue()} does not always build a tuple.  It builds
a tuple only if its format string contains two or more format units.
If the format string is empty, it returns \code{None}; if it contains
exactly one format unit, it returns whatever object is described by
that format unit.  To force it to return a tuple of size 0 or one,
parenthesize the format string.

In the following description, the quoted form is the format unit; the
entry in (round) parentheses is the Python object type that the format
unit will return; and the entry in [square] brackets is the type of
the \C{} value(s) to be passed.

The characters space, tab, colon and comma are ignored in format
strings (but not within format units such as \samp{s\#}).  This can be
used to make long format strings a tad more readable.

\begin{description}

\item[\samp{s} (string) {[char *]}]
Convert a null-terminated \C{} string to a Python object.  If the \C{}
string pointer is \NULL{}, \code{None} is returned.

\item[\samp{s\#} (string) {[char *, int]}]
Convert a \C{} string and its length to a Python object.  If the \C{} string
pointer is \NULL{}, the length is ignored and \code{None} is
returned.

\item[\samp{z} (string or \code{None}) {[char *]}]
Same as \samp{s}.

\item[\samp{z\#} (string or \code{None}) {[char *, int]}]
Same as \samp{s\#}.

\item[\samp{i} (integer) {[int]}]
Convert a plain \C{} \ctype{int} to a Python integer object.

\item[\samp{b} (integer) {[char]}]
Same as \samp{i}.

\item[\samp{h} (integer) {[short int]}]
Same as \samp{i}.

\item[\samp{l} (integer) {[long int]}]
Convert a \C{} \ctype{long int} to a Python integer object.

\item[\samp{c} (string of length 1) {[char]}]
Convert a \C{} \ctype{int} representing a character to a Python string of
length 1.

\item[\samp{d} (float) {[double]}]
Convert a \C{} \ctype{double} to a Python floating point number.

\item[\samp{f} (float) {[float]}]
Same as \samp{d}.

\item[\samp{O} (object) {[PyObject *]}]
Pass a Python object untouched (except for its reference count, which
is incremented by one).  If the object passed in is a \NULL{}
pointer, it is assumed that this was caused because the call producing
the argument found an error and set an exception.  Therefore,
\cfunction{Py_BuildValue()} will return \NULL{} but won't raise an
exception.  If no exception has been raised yet,
\cdata{PyExc_SystemError} is set.

\item[\samp{S} (object) {[PyObject *]}]
Same as \samp{O}.

\item[\samp{O\&} (object) {[\var{converter}, \var{anything}]}]
Convert \var{anything} to a Python object through a \var{converter}
function.  The function is called with \var{anything} (which should be
compatible with \ctype{void *}) as its argument and should return a
``new'' Python object, or \NULL{} if an error occurred.

\item[\samp{(\var{items})} (tuple) {[\var{matching-items}]}]
Convert a sequence of \C{} values to a Python tuple with the same number
of items.

\item[\samp{[\var{items}]} (list) {[\var{matching-items}]}]
Convert a sequence of \C{} values to a Python list with the same number
of items.

\item[\samp{\{\var{items}\}} (dictionary) {[\var{matching-items}]}]
Convert a sequence of \C{} values to a Python dictionary.  Each pair of
consecutive \C{} values adds one item to the dictionary, serving as key
and value, respectively.

\end{description}

If there is an error in the format string, the
\cdata{PyExc_SystemError} exception is raised and \NULL{} returned.

Examples (to the left the call, to the right the resulting Python value):

\begin{verbatim}
    Py_BuildValue("")                        None
    Py_BuildValue("i", 123)                  123
    Py_BuildValue("iii", 123, 456, 789)      (123, 456, 789)
    Py_BuildValue("s", "hello")              'hello'
    Py_BuildValue("ss", "hello", "world")    ('hello', 'world')
    Py_BuildValue("s#", "hello", 4)          'hell'
    Py_BuildValue("()")                      ()
    Py_BuildValue("(i)", 123)                (123,)
    Py_BuildValue("(ii)", 123, 456)          (123, 456)
    Py_BuildValue("(i,i)", 123, 456)         (123, 456)
    Py_BuildValue("[i,i]", 123, 456)         [123, 456]
    Py_BuildValue("{s:i,s:i}",
                  "abc", 123, "def", 456)    {'abc': 123, 'def': 456}
    Py_BuildValue("((ii)(ii)) (ii)",
                  1, 2, 3, 4, 5, 6)          (((1, 2), (3, 4)), (5, 6))
\end{verbatim}

\section{Reference Counts}
\label{refcounts}

%\subsection{Introduction}

In languages like \C{} or \Cpp{}, the programmer is responsible for
dynamic allocation and deallocation of memory on the heap.  In \C{},
this is done using the functions \cfunction{malloc()} and
\cfunction{free()}.  In \Cpp{}, the operators \keyword{new} and
\keyword{delete} are used with essentially the same meaning; they are
actually implemented using \cfunction{malloc()} and
\cfunction{free()}, so we'll restrict the following discussion to the
latter.

Every block of memory allocated with \cfunction{malloc()} should
eventually be returned to the pool of available memory by exactly one
call to \cfunction{free()}.  It is important to call
\cfunction{free()} at the right time.  If a block's address is
forgotten but \cfunction{free()} is not called for it, the memory it
occupies cannot be reused until the program terminates.  This is
called a \dfn{memory leak}.  On the other hand, if a program calls
\cfunction{free()} for a block and then continues to use the block, it
creates a conflict with re-use of the block through another
\cfunction{malloc()} call.  This is called \dfn{using freed memory}.
It has the same bad consequences as referencing uninitialized data ---
core dumps, wrong results, mysterious crashes.

Common causes of memory leaks are unusual paths through the code.  For
instance, a function may allocate a block of memory, do some
calculation, and then free the block again.  Now a change in the
requirements for the function may add a test to the calculation that
detects an error condition and can return prematurely from the
function.  It's easy to forget to free the allocated memory block when
taking this premature exit, especially when it is added later to the
code.  Such leaks, once introduced, often go undetected for a long
time: the error exit is taken only in a small fraction of all calls,
and most modern machines have plenty of virtual memory, so the leak
only becomes apparent in a long-running process that uses the leaking
function frequently.  Therefore, it's important to prevent leaks from
happening by having a coding convention or strategy that minimizes
this kind of errors.

Since Python makes heavy use of \cfunction{malloc()} and
\cfunction{free()}, it needs a strategy to avoid memory leaks as well
as the use of freed memory.  The chosen method is called
\dfn{reference counting}.  The principle is simple: every object
contains a counter, which is incremented when a reference to the
object is stored somewhere, and which is decremented when a reference
to it is deleted.  When the counter reaches zero, the last reference
to the object has been deleted and the object is freed.

An alternative strategy is called \dfn{automatic garbage collection}.
(Sometimes, reference counting is also referred to as a garbage
collection strategy, hence my use of ``automatic'' to distinguish the
two.)  The big advantage of automatic garbage collection is that the
user doesn't need to call \cfunction{free()} explicitly.  (Another claimed
advantage is an improvement in speed or memory usage --- this is no
hard fact however.)  The disadvantage is that for \C{}, there is no
truly portable automatic garbage collector, while reference counting
can be implemented portably (as long as the functions \cfunction{malloc()}
and \cfunction{free()} are available --- which the \C{} Standard guarantees).
Maybe some day a sufficiently portable automatic garbage collector
will be available for \C{}.  Until then, we'll have to live with
reference counts.

\subsection{Reference Counting in Python}
\label{refcountsInPython}

There are two macros, \code{Py_INCREF(x)} and \code{Py_DECREF(x)},
which handle the incrementing and decrementing of the reference count.
\cfunction{Py_DECREF()} also frees the object when the count reaches zero.
For flexibility, it doesn't call \cfunction{free()} directly --- rather, it
makes a call through a function pointer in the object's \dfn{type
object}.  For this purpose (and others), every object also contains a
pointer to its type object.

The big question now remains: when to use \code{Py_INCREF(x)} and
\code{Py_DECREF(x)}?  Let's first introduce some terms.  Nobody
``owns'' an object; however, you can \dfn{own a reference} to an
object.  An object's reference count is now defined as the number of
owned references to it.  The owner of a reference is responsible for
calling \cfunction{Py_DECREF()} when the reference is no longer
needed.  Ownership of a reference can be transferred.  There are three
ways to dispose of an owned reference: pass it on, store it, or call
\cfunction{Py_DECREF()}.  Forgetting to dispose of an owned reference
creates a memory leak.

It is also possible to \dfn{borrow}\footnote{The metaphor of
``borrowing'' a reference is not completely correct: the owner still
has a copy of the reference.} a reference to an object.  The borrower
of a reference should not call \cfunction{Py_DECREF()}.  The borrower must
not hold on to the object longer than the owner from which it was
borrowed.  Using a borrowed reference after the owner has disposed of
it risks using freed memory and should be avoided
completely.\footnote{Checking that the reference count is at least 1
\strong{does not work} --- the reference count itself could be in
freed memory and may thus be reused for another object!}

The advantage of borrowing over owning a reference is that you don't
need to take care of disposing of the reference on all possible paths
through the code --- in other words, with a borrowed reference you
don't run the risk of leaking when a premature exit is taken.  The
disadvantage of borrowing over leaking is that there are some subtle
situations where in seemingly correct code a borrowed reference can be
used after the owner from which it was borrowed has in fact disposed
of it.

A borrowed reference can be changed into an owned reference by calling
\cfunction{Py_INCREF()}.  This does not affect the status of the owner from
which the reference was borrowed --- it creates a new owned reference,
and gives full owner responsibilities (i.e., the new owner must
dispose of the reference properly, as well as the previous owner).

\subsection{Ownership Rules}
\label{ownershipRules}

Whenever an object reference is passed into or out of a function, it
is part of the function's interface specification whether ownership is
transferred with the reference or not.

Most functions that return a reference to an object pass on ownership
with the reference.  In particular, all functions whose function it is
to create a new object, e.g.\ \cfunction{PyInt_FromLong()} and
\cfunction{Py_BuildValue()}, pass ownership to the receiver.  Even if in
fact, in some cases, you don't receive a reference to a brand new
object, you still receive ownership of the reference.  For instance,
\cfunction{PyInt_FromLong()} maintains a cache of popular values and can
return a reference to a cached item.

Many functions that extract objects from other objects also transfer
ownership with the reference, for instance
\cfunction{PyObject_GetAttrString()}.  The picture is less clear, here,
however, since a few common routines are exceptions:
\cfunction{PyTuple_GetItem()}, \cfunction{PyList_GetItem()},
\cfunction{PyDict_GetItem()}, and \cfunction{PyDict_GetItemString()}
all return references that you borrow from the tuple, list or
dictionary.

The function \cfunction{PyImport_AddModule()} also returns a borrowed
reference, even though it may actually create the object it returns:
this is possible because an owned reference to the object is stored in
\code{sys.modules}.

When you pass an object reference into another function, in general,
the function borrows the reference from you --- if it needs to store
it, it will use \cfunction{Py_INCREF()} to become an independent
owner.  There are exactly two important exceptions to this rule:
\cfunction{PyTuple_SetItem()} and \cfunction{PyList_SetItem()}.  These
functions take over ownership of the item passed to them --- even if
they fail!  (Note that \cfunction{PyDict_SetItem()} and friends don't
take over ownership --- they are ``normal''.)

When a \C{} function is called from Python, it borrows references to its
arguments from the caller.  The caller owns a reference to the object,
so the borrowed reference's lifetime is guaranteed until the function
returns.  Only when such a borrowed reference must be stored or passed
on, it must be turned into an owned reference by calling
\cfunction{Py_INCREF()}.

The object reference returned from a \C{} function that is called from
Python must be an owned reference --- ownership is tranferred from the
function to its caller.

\subsection{Thin Ice}
\label{thinIce}

There are a few situations where seemingly harmless use of a borrowed
reference can lead to problems.  These all have to do with implicit
invocations of the interpreter, which can cause the owner of a
reference to dispose of it.

The first and most important case to know about is using
\cfunction{Py_DECREF()} on an unrelated object while borrowing a
reference to a list item.  For instance:

\begin{verbatim}
bug(PyObject *list) {
    PyObject *item = PyList_GetItem(list, 0);
    PyList_SetItem(list, 1, PyInt_FromLong(0L));
    PyObject_Print(item, stdout, 0); /* BUG! */
}
\end{verbatim}

This function first borrows a reference to \code{list[0]}, then
replaces \code{list[1]} with the value \code{0}, and finally prints
the borrowed reference.  Looks harmless, right?  But it's not!

Let's follow the control flow into \cfunction{PyList_SetItem()}.  The list
owns references to all its items, so when item 1 is replaced, it has
to dispose of the original item 1.  Now let's suppose the original
item 1 was an instance of a user-defined class, and let's further
suppose that the class defined a \method{__del__()} method.  If this
class instance has a reference count of 1, disposing of it will call
its \method{__del__()} method.

Since it is written in Python, the \method{__del__()} method can execute
arbitrary Python code.  Could it perhaps do something to invalidate
the reference to \code{item} in \cfunction{bug()}?  You bet!  Assuming
that the list passed into \cfunction{bug()} is accessible to the
\method{__del__()} method, it could execute a statement to the effect of
\samp{del list[0]}, and assuming this was the last reference to that
object, it would free the memory associated with it, thereby
invalidating \code{item}.

The solution, once you know the source of the problem, is easy:
temporarily increment the reference count.  The correct version of the
function reads:

\begin{verbatim}
no_bug(PyObject *list) {
    PyObject *item = PyList_GetItem(list, 0);
    Py_INCREF(item);
    PyList_SetItem(list, 1, PyInt_FromLong(0L));
    PyObject_Print(item, stdout, 0);
    Py_DECREF(item);
}
\end{verbatim}

This is a true story.  An older version of Python contained variants
of this bug and someone spent a considerable amount of time in a \C{}
debugger to figure out why his \method{__del__()} methods would fail...

The second case of problems with a borrowed reference is a variant
involving threads.  Normally, multiple threads in the Python
interpreter can't get in each other's way, because there is a global
lock protecting Python's entire object space.  However, it is possible
to temporarily release this lock using the macro
\code{Py_BEGIN_ALLOW_THREADS}, and to re-acquire it using
\code{Py_END_ALLOW_THREADS}.  This is common around blocking I/O
calls, to let other threads use the CPU while waiting for the I/O to
complete.  Obviously, the following function has the same problem as
the previous one:

\begin{verbatim}
bug(PyObject *list) {
    PyObject *item = PyList_GetItem(list, 0);
    Py_BEGIN_ALLOW_THREADS
    ...some blocking I/O call...
    Py_END_ALLOW_THREADS
    PyObject_Print(item, stdout, 0); /* BUG! */
}
\end{verbatim}

\subsection{NULL Pointers}
\label{nullPointers}

In general, functions that take object references as arguments don't
expect you to pass them \NULL{} pointers, and will dump core (or
cause later core dumps) if you do so.  Functions that return object
references generally return \NULL{} only to indicate that an
exception occurred.  The reason for not testing for \NULL{}
arguments is that functions often pass the objects they receive on to
other function --- if each function were to test for \NULL{},
there would be a lot of redundant tests and the code would run slower.

It is better to test for \NULL{} only at the ``source'', i.e.\
when a pointer that may be \NULL{} is received, e.g.\ from
\cfunction{malloc()} or from a function that may raise an exception.

The macros \cfunction{Py_INCREF()} and \cfunction{Py_DECREF()}
don't check for \NULL{} pointers --- however, their variants
\cfunction{Py_XINCREF()} and \cfunction{Py_XDECREF()} do.

The macros for checking for a particular object type
(\code{Py\var{type}_Check()}) don't check for \NULL{} pointers ---
again, there is much code that calls several of these in a row to test
an object against various different expected types, and this would
generate redundant tests.  There are no variants with \NULL{}
checking.

The \C{} function calling mechanism guarantees that the argument list
passed to \C{} functions (\code{args} in the examples) is never
\NULL{} --- in fact it guarantees that it is always a tuple.%
\footnote{These guarantees don't hold when you use the ``old'' style
calling convention --- this is still found in much existing code.}

It is a severe error to ever let a \NULL{} pointer ``escape'' to
the Python user.  


\section{Writing Extensions in \Cpp{}}
\label{cplusplus}

It is possible to write extension modules in \Cpp{}.  Some restrictions
apply.  If the main program (the Python interpreter) is compiled and
linked by the \C{} compiler, global or static objects with constructors
cannot be used.  This is not a problem if the main program is linked
by the \Cpp{} compiler.  Functions that will be called by the
Python interpreter (in particular, module initalization functions)
have to be declared using \code{extern "C"}.
It is unnecessary to enclose the Python header files in
\code{extern "C" \{...\}} --- they use this form already if the symbol
\samp{__cplusplus} is defined (all recent \Cpp{} compilers define this
symbol).

\chapter{Embedding Python in another application}
\label{embedding}

Embedding Python is similar to extending it, but not quite.  The
difference is that when you extend Python, the main program of the
application is still the Python interpreter, while if you embed
Python, the main program may have nothing to do with Python ---
instead, some parts of the application occasionally call the Python
interpreter to run some Python code.

So if you are embedding Python, you are providing your own main
program.  One of the things this main program has to do is initialize
the Python interpreter.  At the very least, you have to call the
function \cfunction{Py_Initialize()}.  There are optional calls to
pass command line arguments to Python.  Then later you can call the
interpreter from any part of the application.

There are several different ways to call the interpreter: you can pass
a string containing Python statements to
\cfunction{PyRun_SimpleString()}, or you can pass a stdio file pointer
and a file name (for identification in error messages only) to
\cfunction{PyRun_SimpleFile()}.  You can also call the lower-level
operations described in the previous chapters to construct and use
Python objects.

A simple demo of embedding Python can be found in the directory
\file{Demo/embed}.


\section{Embedding Python in \Cpp{}}
\label{embeddingInCplusplus}

It is also possible to embed Python in a \Cpp{} program; precisely how this
is done will depend on the details of the \Cpp{} system used; in general you
will need to write the main program in \Cpp{}, and use the \Cpp{} compiler
to compile and link your program.  There is no need to recompile Python
itself using \Cpp{}.


\chapter{Dynamic Loading}
\label{dynload}

On most modern systems it is possible to configure Python to support
dynamic loading of extension modules implemented in \C{}.  When shared
libraries are used dynamic loading is configured automatically;
otherwise you have to select it as a build option (see below).  Once
configured, dynamic loading is trivial to use: when a Python program
executes \code{import spam}, the search for modules tries to find a
file \file{spammodule.o} (\file{spammodule.so} when using shared
libraries) in the module search path, and if one is found, it is
loaded into the executing binary and executed.  Once loaded, the
module acts just like a built-in extension module.

The advantages of dynamic loading are twofold: the ``core'' Python
binary gets smaller, and users can extend Python with their own
modules implemented in \C{} without having to build and maintain their
own copy of the Python interpreter.  There are also disadvantages:
dynamic loading isn't available on all systems (this just means that
on some systems you have to use static loading), and dynamically
loading a module that was compiled for a different version of Python
(e.g. with a different representation of objects) may dump core.


\section{Configuring and Building the Interpreter for Dynamic Loading}
\label{dynloadConfig}

There are three styles of dynamic loading: one using shared libraries,
one using SGI IRIX 4 dynamic loading, and one using GNU dynamic
loading.

\subsection{Shared Libraries}
\label{sharedlibs}

The following systems support dynamic loading using shared libraries:
SunOS 4; Solaris 2; SGI IRIX 5 (but not SGI IRIX 4!); and probably all
systems derived from SVR4, or at least those SVR4 derivatives that
support shared libraries (are there any that don't?).

You don't need to do anything to configure dynamic loading on these
systems --- the \file{configure} detects the presence of the
\file{<dlfcn.h>} header file and automatically configures dynamic
loading.

\subsection{SGI IRIX 4 Dynamic Loading}
\label{irixDynload}

Only SGI IRIX 4 supports dynamic loading of modules using SGI dynamic
loading.  (SGI IRIX 5 might also support it but it is inferior to
using shared libraries so there is no reason to; a small test didn't
work right away so I gave up trying to support it.)

Before you build Python, you first need to fetch and build the
\code{dl} package written by Jack Jansen.  This is available by
anonymous ftp from \url{ftp://ftp.cwi.nl/pub/dynload}, file
\file{dl-1.6.tar.Z}.  (The version number may change.)  Follow the
instructions in the package's \file{README} file to build it.

Once you have built \code{dl}, you can configure Python to use it.  To
this end, you run the \file{configure} script with the option
\code{--with-dl=\var{directory}} where \var{directory} is the absolute
pathname of the \code{dl} directory.

Now build and install Python as you normally would (see the
\file{README} file in the toplevel Python directory.)

\subsection{GNU Dynamic Loading}
\label{gnuDynload}

GNU dynamic loading supports (according to its \file{README} file) the
following hardware and software combinations: VAX (Ultrix), Sun 3
(SunOS 3.4 and 4.0), Sparc (SunOS 4.0), Sequent Symmetry (Dynix), and
Atari ST.  There is no reason to use it on a Sparc; I haven't seen a
Sun 3 for years so I don't know if these have shared libraries or not.

You need to fetch and build two packages.
One is GNU DLD.  All development of this code has been done with DLD
version 3.2.3, which is available by anonymous ftp from
\url{ftp://ftp.cwi.nl/pub/dynload}, file
\file{dld-3.2.3.tar.Z}.  (A more recent version of DLD is available
via \url{http://www-swiss.ai.mit.edu/~jaffer/DLD.html} but this has
not been tested.)
The other package needed is an
emulation of Jack Jansen's \code{dl} package that I wrote on top of
GNU DLD 3.2.3.  This is available from the same host and directory,
file \file{dl-dld-1.1.tar.Z}.  (The version number may change --- but I doubt
it will.)  Follow the instructions in each package's \file{README}
file to configure and build them.

Now configure Python.  Run the \file{configure} script with the option
\code{--with-dl-dld=\var{dl-directory},\var{dld-directory}} where
\var{dl-directory} is the absolute pathname of the directory where you
have built the \file{dl-dld} package, and \var{dld-directory} is that
of the GNU DLD package.  The Python interpreter you build hereafter
will support GNU dynamic loading.


\section{Building a Dynamically Loadable Module}
\label{makedynload}

Since there are three styles of dynamic loading, there are also three
groups of instructions for building a dynamically loadable module.
Instructions common for all three styles are given first.  Assuming
your module is called \module{spam}, the source filename must be
\file{spammodule.c}, so the object name is \file{spammodule.o}.  The
module must be written as a normal Python extension module (as
described earlier).

Note that in all cases you will have to create your own Makefile that
compiles your module file(s).  This Makefile will have to pass two
\samp{-I} arguments to the \C{} compiler which will make it find the
Python header files.  If the Make variable \var{PYTHONTOP} points to
the toplevel Python directory, your \var{CFLAGS} Make variable should
contain the options \samp{-I\$(PYTHONTOP) -I\$(PYTHONTOP)/Include}.
(Most header files are in the \file{Include} subdirectory, but the
\file{config.h} header lives in the toplevel directory.)


\subsection{Shared Libraries}
\label{linking}

You must link the \file{.o} file to produce a shared library.  This is 
done using a special invocation of the \UNIX{} loader/linker,
\emph{ld}(1).  Unfortunately the invocation differs slightly per
system.

On SunOS 4, use
\begin{verbatim}
ld spammodule.o -o spammodule.so
\end{verbatim}

On Solaris 2, use
\begin{verbatim}
ld -G spammodule.o -o spammodule.so
\end{verbatim}

On SGI IRIX 5, use
\begin{verbatim}
ld -shared spammodule.o -o spammodule.so
\end{verbatim}

On other systems, consult the manual page for \manpage{ld}{1} to find
what flags, if any, must be used.

If your extension module uses system libraries that haven't already
been linked with Python (e.g. a windowing system), these must be
passed to the \program{ld} command as \samp{-l} options after the
\samp{.o} file.

The resulting file \file{spammodule.so} must be copied into a directory
along the Python module search path.


\subsection{SGI IRIX 4 Dynamic Loading}
\label{irixLinking}

\strong{IMPORTANT:} You must compile your extension module with the
additional \C{} flag \samp{-G0} (or \samp{-G 0}).  This instruct the
assembler to generate position-independent code.

You don't need to link the resulting \file{spammodule.o} file; just
copy it into a directory along the Python module search path.

The first time your extension is loaded, it takes some extra time and
a few messages may be printed.  This creates a file
\file{spammodule.ld} which is an image that can be loaded quickly into
the Python interpreter process.  When a new Python interpreter is
installed, the \code{dl} package detects this and rebuilds
\file{spammodule.ld}.  The file \file{spammodule.ld} is placed in the
directory where \file{spammodule.o} was found, unless this directory is
unwritable; in that case it is placed in a temporary
directory.\footnote{Check the manual page of the \code{dl} package for
details.}

If your extension modules uses additional system libraries, you must
create a file \file{spammodule.libs} in the same directory as the
\file{spammodule.o}.  This file should contain one or more lines with
whitespace-separated options that will be passed to the linker ---
normally only \samp{-l} options or absolute pathnames of libraries
(\samp{.a} files) should be used.


\subsection{GNU Dynamic Loading}
\label{gnuLinking}

Just copy \file{spammodule.o} into a directory along the Python module
search path.

If your extension modules uses additional system libraries, you must
create a file \file{spammodule.libs} in the same directory as the
\file{spammodule.o}.  This file should contain one or more lines with
whitespace-separated absolute pathnames of libraries (\samp{.a}
files).  No \samp{-l} options can be used.


%
\section{Extension Reference}

From the viewpoint of of C access to Python services, we have:

\begin{enumerate}
  \item "Very high level layer": two or three functions that let you exec or
    eval arbitrary Python code given as a string in a module whose name is
    given, passing C values in and getting C values out using
    mkvalue/getargs style format strings.  This does not require the user
    to declare any variables of type "PyObject *".  This should be enough
    to write a simple application that gets Python code from the user,
    execs it, and returns the output or errors.

  \item "Abstract objects layer": which is the subject of this proposal.
    It has many functions operating on objects, and lest you do many
    things from C that you can also write in Python, without going
    through the Python parser.

  \item "Concrete objects layer": This is the public type-dependent
    interface provided by the standard built-in types, such as floats,
    strings, and lists.  This interface exists and is currently
    documented by the collection of include files provides with the
    Python distributions.

  From the point of view of Python accessing services provided by C
  modules: 

  \item "Python module interface": this interface consist of the basic
    routines used to define modules and their members.  Most of the
    current extensions-writing guide deals with this interface.

  \item "Built-in object interface": this is the interface that a new
    built-in type must provide and the mechanisms and rules that a
    developer of a new built-in type must use and follow.
\end{enumerate}

  The Python C object interface provides four protocols: object,
  numeric, sequence, and mapping.  Each protocol consists of a
  collection of related operations.  If an operation that is not
  provided by a particular type is invoked, then a standard exception,
  NotImplementedError is raised with a operation name as an argument.
  In addition, for convenience this interface defines a set of
  constructors for building objects of built-in types.  This is needed
  so new objects can be returned from C functions that otherwise treat
  objects generically.

\subsubsection{Object Protocol}
     \code{int *PyObject_Print(PyObject *o, FILE *fp, int flags)}\\
         Print an object \code{o}, on file \code{fp}.  Returns -1 on error
	 The flags argument is used to enable certain printing
	 options. The only option currently supported is \code{Py_Print_RAW}. 

     \code{int PyObject_HasAttrString(PyObject *o, char *attr_name)}\\
         Returns 1 if o has the attribute attr_name, and 0 otherwise.
     This is equivalent to the Python expression:
	 \code{hasattr(o,attr_name)}.
	 This function always succeeds.

     \code{PyObject* PyObject_AttrString(PyObject *o, char *attr_name)}\\
	 Retrieve an attributed named attr_name form object o.
	 Returns the attribute value on success, or NULL on failure.
	 This is the equivalent of the Python expression: \code{o.attr_name}.


     \code{int PyObject_HasAttr(PyObject *o, PyObject *attr_name)}\\
         Returns 1 if o has the attribute attr_name, and 0 otherwise.
	 This is equivalent to the Python expression:
	 \code{hasattr(o,attr_name)}. 
	 This function always succeeds.


     \code{PyObject* PyObject_GetAttr(PyObject *o, PyObject *attr_name)}\\
	 Retrieve an attributed named attr_name form object o.
	 Returns the attribute value on success, or NULL on failure.
	 This is the equivalent of the Python expression: o.attr_name.


     \code{int PyObject_SetAttrString(PyObject *o, char *attr_name, PyObject *v)}\\
	 Set the value of the attribute named \code{attr_name}, for object \code{o},
	 to the value \code{v}. Returns -1 on failure.  This is
	 the equivalent of the Python statement: \code{o.attr_name=v}.


     \code{int PyObject_SetAttr(PyObject *o, PyObject *attr_name, PyObject *v)}\\
	 Set the value of the attribute named \code{attr_name}, for object \code{o},
	 to the value \code{v}. Returns -1 on failure.  This is
	 the equivalent of the Python statement: \code{o.attr_name=v}.


     \code{int PyObject_DelAttrString(PyObject *o, char *attr_name)}\\
	 Delete attribute named \code{attr_name}, for object \code{o}. Returns -1 on
	 failure.  This is the equivalent of the Python
	 statement: \code{del o.attr_name}.


     \code{int PyObject_DelAttr(PyObject *o, PyObject *attr_name)}\\
	 Delete attribute named \code{attr_name}, for object \code{o}. Returns -1 on
	 failure.  This is the equivalent of the Python
	 statement: \code{del o.attr_name}.


     \code{int PyObject_Cmp(PyObject *o1, PyObject *o2, int *result)}\\
	 Compare the values of \code{o1} and \code{o2} using a routine provided by
	 \code{o1}, if one exists, otherwise with a routine provided by \code{o2}.
	 The result of the comparison is returned in \code{result}.  Returns
	 -1 on failure.  This is the equivalent of the Python
	 statement: \code{result=cmp(o1,o2)}.


     \code{int PyObject_Compare(PyObject *o1, PyObject *o2)}\\
	 Compare the values of \code{o1} and \code{o2} using a routine provided by
	 \code{o1}, if one exists, otherwise with a routine provided by \code{o2}.
	 Returns the result of the comparison on success.  On error,
	 the value returned is undefined. This is equivalent to the
	 Python expression: \code{cmp(o1,o2)}.


     \code{PyObject *PyObject_Repr(PyObject *o)}\\
	 Compute the string representation of object, \code{o}.  Returns the
	 string representation on success, NULL on failure.  This is
	 the equivalent of the Python expression: \code{repr(o)}.
	 Called by the \code{repr()} built-in function and by reverse quotes.


     \code{PyObject *PyObject_Str(PyObject *o)}\\
	 Compute the string representation of object, \code{o}.  Returns the
	 string representation on success, NULL on failure.  This is
	 the equivalent of the Python expression: \code{str(o)}.
	 Called by the \code{str()} built-in function and by the \code{print}
	 statement.


     \code{int *PyCallable_Check(PyObject *o))}\\
	 Determine if the object \code{o}, is callable.  Return 1 if the
	 object is callable and 0 otherwise.
	 This function always succeeds.


     \code{PyObject *PyObject_CallObject(PyObject *callable_object, PyObject *args)}\\
	 Call a callable Python object \code{callable_object}, with
	 arguments given by the tuple \code{args}.  If no arguments are
	 needed, then args may be NULL.  Returns the result of the
	 call on success, or NULL on failure.  This is the equivalent
	 of the Python expression: \code{apply(o,args)}.

     \code{PyObject *PyObject_CallFunction(PyObject *callable_object, char *format, ...)}\\
         Call a callable Python object \code{callable_object}, with a
         variable number of C arguments. The C arguments are described
         using a mkvalue-style format string. The format may be NULL,
         indicating that no arguments are provided.  Returns the
         result of the call on success, or NULL on failure.  This is
         the equivalent of the Python expression: \code{apply(o,args)}.


     \code{PyObject *PyObject_CallMethod(PyObject *o, char *m, char *format, ...)}\\
         Call the method named \code{m} of object \code{o} with a variable number of
         C arguments.  The C arguments are described by a mkvalue
         format string.  The format may be NULL, indicating that no
         arguments are provided. Returns the result of the call on
         success, or NULL on failure.  This is the equivalent of the
         Python expression: \code{o.method(args)}.
         Note that Special method names, such as "\code{__add__}",
         "\code{__getitem__}", and so on are not supported. The specific
         abstract-object routines for these must be used.


     \code{int PyObject_Hash(PyObject *o)}\\
         Compute and return the hash value of an object \code{o}.  On
         failure, return -1.  This is the equivalent of the Python
         expression: \code{hash(o)}.


     \code{int *PyObject_IsTrue(PyObject *o)}\\
	 Returns 1 if the object \code{o} is considered to be true, and
	 0 otherwise. This is equivalent to the Python expression:
	 \code{not not o}.
	 This function always succeeds.
	 

     \code{PyObject *PyObject_Type(PyObject *o)}\\
	 On success, returns a type object corresponding to the object
	 type of object \code{o}. On failure, returns NULL.  This is
	 equivalent to the Python expression: \code{type(o)}.

     \code{int PyObject_Length(PyObject *o)}\\
         Return the length of object \code{o}.  If the object \code{o} provides
	 both sequence and mapping protocols, the sequence length is
	 returned. On error, -1 is returned.  This is the equivalent
	 to the Python expression: \code{len(o)}.


     \code{PyObject *PyObject_GetItem(PyObject *o, PyObject *key)}\\
	 Return element of \code{o} corresponding to the object \code{key} or NULL
	 on failure. This is the equivalent of the Python expression:
	 \code{o[key]}.


     \code{int PyObject_SetItem(PyObject *o, PyObject *key, PyObject *v)}\\
	 Map the object \code{key} to the value \code{v}.
	 Returns -1 on failure.  This is the equivalent
	 of the Python statement: \code{o[key]=v}.


\subsubsection{Number Protocol}

     \code{int PyNumber_Check(PyObject *o)}\\
         Returns 1 if the object \code{o} provides numeric protocols, and
	 false otherwise. 
	 This function always succeeds.


     \code{PyObject *PyNumber_Add(PyObject *o1, PyObject *o2)}\\
	 Returns the result of adding \code{o1} and \code{o2}, or null on failure.
	 This is the equivalent of the Python expression: \code{o1+o2}.


     \code{PyObject *PyNumber_Subtract(PyObject *o1, PyObject *o2)}\\
	 Returns the result of subtracting \code{o2} from \code{o1}, or null on
	 failure.  This is the equivalent of the Python expression:
	 \code{o1-o2}.


     \code{PyObject *PyNumber_Multiply(PyObject *o1, PyObject *o2)}\\
	 Returns the result of multiplying \code{o1} and \code{o2}, or null on
	 failure.  This is the equivalent of the Python expression:
	 \code{o1*o2}.


     \code{PyObject *PyNumber_Divide(PyObject *o1, PyObject *o2)}\\
	 Returns the result of dividing \code{o1} by \code{o2}, or null on failure.
	 This is the equivalent of the Python expression: \code{o1/o2}.


     \code{PyObject *PyNumber_Remainder(PyObject *o1, PyObject *o2)}\\
	 Returns the remainder of dividing \code{o1} by \code{o2}, or null on
	 failure.  This is the equivalent of the Python expression:
	 \code{o1\%o2}.


     \code{PyObject *PyNumber_Divmod(PyObject *o1, PyObject *o2)}\\
	 See the built-in function divmod.  Returns NULL on failure.
	 This is the equivalent of the Python expression:
	 \code{divmod(o1,o2)}.


     \code{PyObject *PyNumber_Power(PyObject *o1, PyObject *o2, PyObject *o3)}\\
	 See the built-in function pow.  Returns NULL on failure.
	 This is the equivalent of the Python expression:
	 \code{pow(o1,o2,o3)}, where \code{o3} is optional.


     \code{PyObject *PyNumber_Negative(PyObject *o)}\\
	 Returns the negation of \code{o} on success, or null on failure.
	 This is the equivalent of the Python expression: \code{-o}.


     \code{PyObject *PyNumber_Positive(PyObject *o)}\\
         Returns \code{o} on success, or NULL on failure.
	 This is the equivalent of the Python expression: \code{+o}.


     \code{PyObject *PyNumber_Absolute(PyObject *o)}\\
	 Returns the absolute value of \code{o}, or null on failure.  This is
	 the equivalent of the Python expression: \code{abs(o)}.


     \code{PyObject *PyNumber_Invert(PyObject *o)}\\
	 Returns the bitwise negation of \code{o} on success, or NULL on
	 failure.  This is the equivalent of the Python expression:
	 \code{~o}.


     \code{PyObject *PyNumber_Lshift(PyObject *o1, PyObject *o2)}\\
	 Returns the result of left shifting \code{o1} by \code{o2} on success, or
	 NULL on failure.  This is the equivalent of the Python
	 expression: \code{o1 << o2}.


     \code{PyObject *PyNumber_Rshift(PyObject *o1, PyObject *o2)}\\
	 Returns the result of right shifting \code{o1} by \code{o2} on success, or
	 NULL on failure.  This is the equivalent of the Python
	 expression: \code{o1 >> o2}.


     \code{PyObject *PyNumber_And(PyObject *o1, PyObject *o2)}\\
	 Returns the result of "anding" \code{o2} and \code{o2} on success and NULL
	 on failure. This is the equivalent of the Python
	 expression: \code{o1 and o2}.


     \code{PyObject *PyNumber_Xor(PyObject *o1, PyObject *o2)}\\
	 Returns the bitwise exclusive or of \code{o1} by \code{o2} on success, or
	 NULL on failure.  This is the equivalent of the Python
	 expression: \code{o1\^{ }o2}.

     \code{PyObject *PyNumber_Or(PyObject *o1, PyObject *o2)}\\
	 Returns the result or \code{o1} and \code{o2} on success, or NULL on
	 failure.  This is the equivalent of the Python expression: 
	 \code{o1 or o2}.


     \code{PyObject *PyNumber_Coerce(PyObject *o1, PyObject *o2)}\\
         On success, returns a tuple containing \code{o1} and \code{o2} converted to
	 a common numeric type, or None if no conversion is possible.
	 Returns -1 on failure. This is equivalent to the Python
	 expression: \code{coerce(o1,o2)}.


     \code{PyObject *PyNumber_Int(PyObject *o)}\\
	 Returns the \code{o} converted to an integer object on success, or
	 NULL on failure.  This is the equivalent of the Python
	 expression: \code{int(o)}.


     \code{PyObject *PyNumber_Long(PyObject *o)}\\
	 Returns the \code{o} converted to a long integer object on success,
	 or NULL on failure.  This is the equivalent of the Python
	 expression: \code{long(o)}.


     \code{PyObject *PyNumber_Float(PyObject *o)}\\
	 Returns the \code{o} converted to a float object on success, or NULL
	 on failure.  This is the equivalent of the Python expression:
	 \code{float(o)}.


\subsubsection{Sequence protocol}

     \code{int PySequence_Check(PyObject *o)}\\
         Return 1 if the object provides sequence protocol, and 0
	 otherwise.  
	 This function always succeeds.


     \code{PyObject *PySequence_Concat(PyObject *o1, PyObject *o2)}\\
	 Return the concatination of \code{o1} and \code{o2} on success, and NULL on
	 failure.   This is the equivalent of the Python
	 expression: \code{o1+o2}.


     \code{PyObject *PySequence_Repeat(PyObject *o, int count)}\\
	 Return the result of repeating sequence object \code{o} count times,
	 or NULL on failure.  This is the equivalent of the Python
	 expression: \code{o*count}.


     \code{PyObject *PySequence_GetItem(PyObject *o, int i)}\\
	 Return the ith element of \code{o}, or NULL on failure. This is the
	 equivalent of the Python expression: \code{o[i]}.


     \code{PyObject *PySequence_GetSlice(PyObject *o, int i1, int i2)}\\
	 Return the slice of sequence object \code{o} between \code{i1} and \code{i2}, or
	 NULL on failure. This is the equivalent of the Python
	 expression, \code{o[i1:i2]}.


     \code{int PySequence_SetItem(PyObject *o, int i, PyObject *v)}\\
	 Assign object \code{v} to the \code{i}th element of \code{o}.
Returns -1 on failure.  This is the equivalent of the Python
	 statement, \code{o[i]=v}.

     \code{int PySequence_SetSlice(PyObject *o, int i1, int i2, PyObject *v)}\\
         Assign the sequence object \code{v} to the slice in sequence
	 object \code{o} from \code{i1} to \code{i2}.  This is the equivalent of the Python
	 statement, \code{o[i1:i2]=v}.

     \code{PyObject *PySequence_Tuple(PyObject *o)}\\
	 Returns the \code{o} as a tuple on success, and NULL on failure.
	 This is equivalent to the Python expression: \code{tuple(o)}.

     \code{int PySequence_Count(PyObject *o, PyObject *value)}\\
         Return the number of occurrences of \code{value} on \code{o}, that is,
	 return the number of keys for which \code{o[key]==value}.  On
	 failure, return -1.  This is equivalent to the Python
	 expression: \code{o.count(value)}.

     \code{int PySequence_In(PyObject *o, PyObject *value)}\\
	 Determine if \code{o} contains \code{value}.  If an item in \code{o} is equal to
	 \code{value}, return 1, otherwise return 0.  On error, return -1.  This
	 is equivalent to the Python expression: \code{value in o}.

     \code{int PySequence_Index(PyObject *o, PyObject *value)}\\
	 Return the first index for which \code{o[i]=value}.  On error,
	 return -1.    This is equivalent to the Python
	 expression: \code{o.index(value)}.

\subsubsection{Mapping protocol}

     \code{int PyMapping_Check(PyObject *o)}\\
         Return 1 if the object provides mapping protocol, and 0
	 otherwise.  
	 This function always succeeds.


     \code{int PyMapping_Length(PyObject *o)}\\
         Returns the number of keys in object \code{o} on success, and -1 on
	 failure.  For objects that do not provide sequence protocol,
	 this is equivalent to the Python expression: \code{len(o)}.


     \code{int PyMapping_DelItemString(PyObject *o, char *key)}\\
	 Remove the mapping for object \code{key} from the object \code{o}.
	 Return -1 on failure.  This is equivalent to
	 the Python statement: \code{del o[key]}.


     \code{int PyMapping_DelItem(PyObject *o, PyObject *key)}\\
	 Remove the mapping for object \code{key} from the object \code{o}.
	 Return -1 on failure.  This is equivalent to
	 the Python statement: \code{del o[key]}.


     \code{int PyMapping_HasKeyString(PyObject *o, char *key)}\\
	 On success, return 1 if the mapping object has the key \code{key}
	 and 0 otherwise.  This is equivalent to the Python expression:
	 \code{o.has_key(key)}. 
	 This function always succeeds.


     \code{int PyMapping_HasKey(PyObject *o, PyObject *key)}\\
	 Return 1 if the mapping object has the key \code{key}
	 and 0 otherwise.  This is equivalent to the Python expression:
	 \code{o.has_key(key)}. 
	 This function always succeeds.


     \code{PyObject *PyMapping_Keys(PyObject *o)}\\
         On success, return a list of the keys in object \code{o}.  On
	 failure, return NULL. This is equivalent to the Python
	 expression: \code{o.keys()}.


     \code{PyObject *PyMapping_Values(PyObject *o)}\\
         On success, return a list of the values in object \code{o}.  On
	 failure, return NULL. This is equivalent to the Python
	 expression: \code{o.values()}.


     \code{PyObject *PyMapping_Items(PyObject *o)}\\
         On success, return a list of the items in object \code{o}, where
	 each item is a tuple containing a key-value pair.  On
	 failure, return NULL. This is equivalent to the Python
	 expression: \code{o.items()}.

     \code{int PyMapping_Clear(PyObject *o)}\\
         Make object \code{o} empty.  Returns 1 on success and 0 on failure.
	 This is equivalent to the Python statement:
	 \code{for key in o.keys(): del o[key]}


     \code{PyObject *PyMapping_GetItemString(PyObject *o, char *key)}\\
	 Return element of \code{o} corresponding to the object \code{key} or NULL
	 on failure. This is the equivalent of the Python expression:
	 \code{o[key]}.

     \code{PyObject *PyMapping_SetItemString(PyObject *o, char *key, PyObject *v)}\\
         Map the object \code{key} to the value \code{v} in object \code{o}.  Returns 
         -1 on failure.  This is the equivalent of the Python
         statement: \code{o[key]=v}.


\subsubsection{Constructors}

     \code{PyObject *PyFile_FromString(char *file_name, char *mode)}\\
	 On success, returns a new file object that is opened on the
	 file given by \code{file_name}, with a file mode given by \code{mode},
	 where \code{mode} has the same semantics as the standard C routine,
	 fopen.  On failure, return -1.
     
     \code{PyObject *PyFile_FromFile(FILE *fp, char *file_name, char *mode, int close_on_del)}\\
	 Return a new file object for an already opened standard C
	 file pointer, \code{fp}.  A file name, \code{file_name}, and open mode,
	 \code{mode}, must be provided as well as a flag, \code{close_on_del}, that
	 indicates whether the file is to be closed when the file
	 object is destroyed.  On failure, return -1.

     \code{PyObject *PyFloat_FromDouble(double v)}\\
	 Returns a new float object with the value \code{v} on success, and
	 NULL on failure.
     
     \code{PyObject *PyInt_FromLong(long v)}\\
	 Returns a new int object with the value \code{v} on success, and
	 NULL on failure.

     \code{PyObject *PyList_New(int l)}\\
	 Returns a new list of length \code{l} on success, and NULL on
	 failure.

     \code{PyObject *PyLong_FromLong(long v)}\\
	 Returns a new long object with the value \code{v} on success, and
	 NULL on failure.

     \code{PyObject *PyLong_FromDouble(double v)}\\
	 Returns a new long object with the value \code{v} on success, and
	 NULL on failure.

     \code{PyObject *PyDict_New()}\\
	 Returns a new empty dictionary on success, and NULL on
	 failure.

     \code{PyObject *PyString_FromString(char *v)}\\
	 Returns a new string object with the value \code{v} on success, and
	 NULL on failure.

     \code{PyObject *PyString_FromStringAndSize(char *v, int l)}\\
	 Returns a new string object with the value \code{v} and length \code{l}
	 on success, and NULL on failure.

     \code{PyObject *PyTuple_New(int l)}\\
	 Returns a new tuple of length \code{l} on success, and NULL on
	 failure.



\documentclass{manual}

% XXX PM explain how to add new types to Python

\title{Extending and Embedding the Python Interpreter}

\author{Guido van Rossum\\
	Fred L. Drake, Jr., editor}
\authoraddress{
	PythonLabs\\
	E-mail: \email{python-docs@python.org}
}

\date{June 15, 2001}		% XXX update before release!
\release{2.0.1c1}		% software release, not documentation
\setshortversion{2.0}		% major.minor only for software


% Tell \index to actually write the .idx file
\makeindex

\begin{document}

\maketitle

\ifhtml
\chapter*{Front Matter\label{front}}
\fi

Copyright 1991, 1992, 1993, 1994 by Stichting Mathematisch Centrum,
Amsterdam, The Netherlands.

\begin{center}
All Rights Reserved
\end{center}

Permission to use, copy, modify, and distribute this software and its
documentation for any purpose and without fee is hereby granted,
provided that the above copyright notice appear in all copies and that
both that copyright notice and this permission notice appear in
supporting documentation, and that the names of Stichting Mathematisch
Centrum or CWI not be used in advertising or publicity pertaining to
distribution of the software without specific, written prior permission.

STICHTING MATHEMATISCH CENTRUM DISCLAIMS ALL WARRANTIES WITH REGARD TO
THIS SOFTWARE, INCLUDING ALL IMPLIED WARRANTIES OF MERCHANTABILITY AND
FITNESS, IN NO EVENT SHALL STICHTING MATHEMATISCH CENTRUM BE LIABLE
FOR ANY SPECIAL, INDIRECT OR CONSEQUENTIAL DAMAGES OR ANY DAMAGES
WHATSOEVER RESULTING FROM LOSS OF USE, DATA OR PROFITS, WHETHER IN AN
ACTION OF CONTRACT, NEGLIGENCE OR OTHER TORTIOUS ACTION, ARISING OUT
OF OR IN CONNECTION WITH THE USE OR PERFORMANCE OF THIS SOFTWARE.


%begin{latexonly}
\vspace{1in}
%end{latexonly}
\strong{\large Acknowledgements}

% XXX This needs to be checked and updated manually before each
% release.

The following people have contributed sections to this document:  Jim
Fulton, Konrad Hinsen, Chris Phoenix, and Neil Schemenauer.

\begin{abstract}

\noindent
Python is an interpreted, object-oriented programming language.  This
document describes how to write modules in C or \Cpp{} to extend the
Python interpreter with new modules.  Those modules can define new
functions but also new object types and their methods.  The document
also describes how to embed the Python interpreter in another
application, for use as an extension language.  Finally, it shows how
to compile and link extension modules so that they can be loaded
dynamically (at run time) into the interpreter, if the underlying
operating system supports this feature.

This document assumes basic knowledge about Python.  For an informal
introduction to the language, see the Python Tutorial.  The \emph{Python
Reference Manual} gives a more formal definition of the language.  The
\emph{Python Library Reference} documents the existing object types,
functions and modules (both built-in and written in Python) that give
the language its wide application range.

For a detailed description of the whole Python/C API, see the separate
\emph{Python/C API Reference Manual}.  \strong{Note:} While that
manual is still in a state of flux, it is safe to say that it is much
more up to date than the manual you're reading currently (which has
been in need for an upgrade for some time now).


\end{abstract}

\tableofcontents


\chapter{Extending Python with C or \Cpp{} code}


%\section{Introduction}
\label{intro}

It is quite easy to add new built-in modules to Python, if you know
how to program in C.  Such \dfn{extension modules} can do two things
that can't be done directly in Python: they can implement new built-in
object types, and they can call C library functions and system calls.

To support extensions, the Python API (Application Programmers
Interface) defines a set of functions, macros and variables that
provide access to most aspects of the Python run-time system.  The
Python API is incorporated in a C source file by including the header
\code{"Python.h"}.

The compilation of an extension module depends on its intended use as
well as on your system setup; details are given in a later section.


\section{A Simple Example
         \label{simpleExample}}

Let's create an extension module called \samp{spam} (the favorite food
of Monty Python fans...) and let's say we want to create a Python
interface to the C library function \cfunction{system()}.\footnote{An
interface for this function already exists in the standard module
\module{os} --- it was chosen as a simple and straightfoward example.}
This function takes a null-terminated character string as argument and
returns an integer.  We want this function to be callable from Python
as follows:

\begin{verbatim}
>>> import spam
>>> status = spam.system("ls -l")
\end{verbatim}

Begin by creating a file \file{spammodule.c}.  (In general, if a
module is called \samp{spam}, the C file containing its implementation
is called \file{spammodule.c}; if the module name is very long, like
\samp{spammify}, the module name can be just \file{spammify.c}.)

The first line of our file can be:

\begin{verbatim}
#include "Python.h"
\end{verbatim}

which pulls in the Python API (you can add a comment describing the
purpose of the module and a copyright notice if you like).

All user-visible symbols defined by \code{"Python.h"} have a prefix of
\samp{Py} or \samp{PY}, except those defined in standard header files.
For convenience, and since they are used extensively by the Python
interpreter, \code{"Python.h"} includes a few standard header files:
\code{<stdio.h>}, \code{<string.h>}, \code{<errno.h>}, and
\code{<stdlib.h>}.  If the latter header file does not exist on your
system, it declares the functions \cfunction{malloc()},
\cfunction{free()} and \cfunction{realloc()} directly.

The next thing we add to our module file is the C function that will
be called when the Python expression \samp{spam.system(\var{string})}
is evaluated (we'll see shortly how it ends up being called):

\begin{verbatim}
static PyObject *
spam_system(self, args)
    PyObject *self;
    PyObject *args;
{
    char *command;
    int sts;

    if (!PyArg_ParseTuple(args, "s", &command))
        return NULL;
    sts = system(command);
    return Py_BuildValue("i", sts);
}
\end{verbatim}

There is a straightforward translation from the argument list in
Python (e.g.\ the single expression \code{"ls -l"}) to the arguments
passed to the C function.  The C function always has two arguments,
conventionally named \var{self} and \var{args}.

The \var{self} argument is only used when the C function implements a
built-in method.  This will be discussed later. In the example,
\var{self} will always be a \NULL{} pointer, since we are defining
a function, not a method.  (This is done so that the interpreter
doesn't have to understand two different types of C functions.)

The \var{args} argument will be a pointer to a Python tuple object
containing the arguments.  Each item of the tuple corresponds to an
argument in the call's argument list.  The arguments are Python
objects --- in order to do anything with them in our C function we have
to convert them to C values.  The function \cfunction{PyArg_ParseTuple()}
in the Python API checks the argument types and converts them to C
values.  It uses a template string to determine the required types of
the arguments as well as the types of the C variables into which to
store the converted values.  More about this later.

\cfunction{PyArg_ParseTuple()} returns true (nonzero) if all arguments have
the right type and its components have been stored in the variables
whose addresses are passed.  It returns false (zero) if an invalid
argument list was passed.  In the latter case it also raises an
appropriate exception by so the calling function can return
\NULL{} immediately (as we saw in the example).


\section{Intermezzo: Errors and Exceptions
         \label{errors}}

An important convention throughout the Python interpreter is the
following: when a function fails, it should set an exception condition
and return an error value (usually a \NULL{} pointer).  Exceptions
are stored in a static global variable inside the interpreter; if this
variable is \NULL{} no exception has occurred.  A second global
variable stores the ``associated value'' of the exception (the second
argument to \keyword{raise}).  A third variable contains the stack
traceback in case the error originated in Python code.  These three
variables are the C equivalents of the Python variables
\code{sys.exc_type}, \code{sys.exc_value} and \code{sys.exc_traceback} (see
the section on module \module{sys} in the \emph{Python Library
Reference}).  It is important to know about them to understand how
errors are passed around.

The Python API defines a number of functions to set various types of
exceptions.

The most common one is \cfunction{PyErr_SetString()}.  Its arguments
are an exception object and a C string.  The exception object is
usually a predefined object like \cdata{PyExc_ZeroDivisionError}.  The
C string indicates the cause of the error and is converted to a
Python string object and stored as the ``associated value'' of the
exception.

Another useful function is \cfunction{PyErr_SetFromErrno()}, which only
takes an exception argument and constructs the associated value by
inspection of the (\UNIX{}) global variable \cdata{errno}.  The most
general function is \cfunction{PyErr_SetObject()}, which takes two object
arguments, the exception and its associated value.  You don't need to
\cfunction{Py_INCREF()} the objects passed to any of these functions.

You can test non-destructively whether an exception has been set with
\cfunction{PyErr_Occurred()}.  This returns the current exception object,
or \NULL{} if no exception has occurred.  You normally don't need
to call \cfunction{PyErr_Occurred()} to see whether an error occurred in a
function call, since you should be able to tell from the return value.

When a function \var{f} that calls another function \var{g} detects
that the latter fails, \var{f} should itself return an error value
(e.g.\ \NULL{} or \code{-1}).  It should \emph{not} call one of the
\cfunction{PyErr_*()} functions --- one has already been called by \var{g}.
\var{f}'s caller is then supposed to also return an error indication
to \emph{its} caller, again \emph{without} calling \cfunction{PyErr_*()},
and so on --- the most detailed cause of the error was already
reported by the function that first detected it.  Once the error
reaches the Python interpreter's main loop, this aborts the currently
executing Python code and tries to find an exception handler specified
by the Python programmer.

(There are situations where a module can actually give a more detailed
error message by calling another \cfunction{PyErr_*()} function, and in
such cases it is fine to do so.  As a general rule, however, this is
not necessary, and can cause information about the cause of the error
to be lost: most operations can fail for a variety of reasons.)

To ignore an exception set by a function call that failed, the exception
condition must be cleared explicitly by calling \cfunction{PyErr_Clear()}. 
The only time C code should call \cfunction{PyErr_Clear()} is if it doesn't
want to pass the error on to the interpreter but wants to handle it
completely by itself (e.g.\ by trying something else or pretending
nothing happened).

Note that a failing \cfunction{malloc()} call must be turned into an
exception --- the direct caller of \cfunction{malloc()} (or
\cfunction{realloc()}) must call \cfunction{PyErr_NoMemory()} and
return a failure indicator itself.  All the object-creating functions
(\cfunction{PyInt_FromLong()} etc.) already do this, so only if you
call \cfunction{malloc()} directly this note is of importance.

Also note that, with the important exception of
\cfunction{PyArg_ParseTuple()} and friends, functions that return an
integer status usually return a positive value or zero for success and
\code{-1} for failure, like \UNIX{} system calls.

Finally, be careful to clean up garbage (by making
\cfunction{Py_XDECREF()} or \cfunction{Py_DECREF()} calls for objects
you have already created) when you return an error indicator!

The choice of which exception to raise is entirely yours.  There are
predeclared C objects corresponding to all built-in Python exceptions,
e.g.\ \cdata{PyExc_ZeroDivisionError}, which you can use directly.  Of
course, you should choose exceptions wisely --- don't use
\cdata{PyExc_TypeError} to mean that a file couldn't be opened (that
should probably be \cdata{PyExc_IOError}).  If something's wrong with
the argument list, the \cfunction{PyArg_ParseTuple()} function usually
raises \cdata{PyExc_TypeError}.  If you have an argument whose value
which must be in a particular range or must satisfy other conditions,
\cdata{PyExc_ValueError} is appropriate.

You can also define a new exception that is unique to your module.
For this, you usually declare a static object variable at the
beginning of your file, e.g.

\begin{verbatim}
static PyObject *SpamError;
\end{verbatim}

and initialize it in your module's initialization function
(\cfunction{initspam()}) with an exception object, e.g.\ (leaving out
the error checking for now):

\begin{verbatim}
void
initspam()
{
    PyObject *m, *d;

    m = Py_InitModule("spam", SpamMethods);
    d = PyModule_GetDict(m);
    SpamError = PyErr_NewException("spam.error", NULL, NULL);
    PyDict_SetItemString(d, "error", SpamError);
}
\end{verbatim}

Note that the Python name for the exception object is
\exception{spam.error}.  The \cfunction{PyErr_NewException()} function
may create either a string or class, depending on whether the
\samp{-X} flag was passed to the interpreter.  If \samp{-X} was used,
\cdata{SpamError} will be a string object, otherwise it will be a
class object with the base class being \exception{Exception},
described in the \emph{Python Library Reference} under ``Built-in
Exceptions.''


\section{Back to the Example
         \label{backToExample}}

Going back to our example function, you should now be able to
understand this statement:

\begin{verbatim}
    if (!PyArg_ParseTuple(args, "s", &command))
        return NULL;
\end{verbatim}

It returns \NULL{} (the error indicator for functions returning
object pointers) if an error is detected in the argument list, relying
on the exception set by \cfunction{PyArg_ParseTuple()}.  Otherwise the
string value of the argument has been copied to the local variable
\cdata{command}.  This is a pointer assignment and you are not supposed
to modify the string to which it points (so in Standard C, the variable
\cdata{command} should properly be declared as \samp{const char
*command}).

The next statement is a call to the \UNIX{} function
\cfunction{system()}, passing it the string we just got from
\cfunction{PyArg_ParseTuple()}:

\begin{verbatim}
    sts = system(command);
\end{verbatim}

Our \function{spam.system()} function must return the value of
\cdata{sts} as a Python object.  This is done using the function
\cfunction{Py_BuildValue()}, which is something like the inverse of
\cfunction{PyArg_ParseTuple()}: it takes a format string and an
arbitrary number of C values, and returns a new Python object.
More info on \cfunction{Py_BuildValue()} is given later.

\begin{verbatim}
    return Py_BuildValue("i", sts);
\end{verbatim}

In this case, it will return an integer object.  (Yes, even integers
are objects on the heap in Python!)

If you have a C function that returns no useful argument (a function
returning \ctype{void}), the corresponding Python function must return
\code{None}.   You need this idiom to do so:

\begin{verbatim}
    Py_INCREF(Py_None);
    return Py_None;
\end{verbatim}

\cdata{Py_None} is the C name for the special Python object
\code{None}.  It is a genuine Python object rather than a \NULL{}
pointer, which means ``error'' in most contexts, as we have seen.


\section{The Module's Method Table and Initialization Function
         \label{methodTable}}

I promised to show how \cfunction{spam_system()} is called from Python
programs.  First, we need to list its name and address in a ``method
table'':

\begin{verbatim}
static PyMethodDef SpamMethods[] = {
    ...
    {"system",  spam_system, METH_VARARGS},
    ...
    {NULL,      NULL}        /* Sentinel */
};
\end{verbatim}

Note the third entry (\samp{METH_VARARGS}).  This is a flag telling
the interpreter the calling convention to be used for the C
function.  It should normally always be \samp{METH_VARARGS} or
\samp{METH_VARARGS | METH_KEYWORDS}; a value of \code{0} means that an
obsolete variant of \cfunction{PyArg_ParseTuple()} is used.

When using only \samp{METH_VARARGS}, the function should expect
the Python-level parameters to be passed in as a tuple acceptable for
parsing via \cfunction{PyArg_ParseTuple()}; more information on this
function is provided below.

The \constant{METH_KEYWORDS} bit may be set in the third field if keyword
arguments should be passed to the function.  In this case, the C
function should accept a third \samp{PyObject *} parameter which will
be a dictionary of keywords.  Use \cfunction{PyArg_ParseTupleAndKeywords()}
to parse the arguemts to such a function.

The method table must be passed to the interpreter in the module's
initialization function (which should be the only non-\code{static}
item defined in the module file):

\begin{verbatim}
void
initspam()
{
    (void) Py_InitModule("spam", SpamMethods);
}
\end{verbatim}

When the Python program imports module \module{spam} for the first
time, \cfunction{initspam()} is called.  It calls
\cfunction{Py_InitModule()}, which creates a ``module object'' (which
is inserted in the dictionary \code{sys.modules} under the key
\code{"spam"}), and inserts built-in function objects into the newly
created module based upon the table (an array of \ctype{PyMethodDef}
structures) that was passed as its second argument.
\cfunction{Py_InitModule()} returns a pointer to the module object
that it creates (which is unused here).  It aborts with a fatal error
if the module could not be initialized satisfactorily, so the caller
doesn't need to check for errors.


\section{Compilation and Linkage
         \label{compilation}}

There are two more things to do before you can use your new extension:
compiling and linking it with the Python system.  If you use dynamic
loading, the details depend on the style of dynamic loading your
system uses; see the chapter ``Dynamic Loading'' for more information
about this.

If you can't use dynamic loading, or if you want to make your module a
permanent part of the Python interpreter, you will have to change the
configuration setup and rebuild the interpreter.  Luckily, this is
very simple: just place your file (\file{spammodule.c} for example) in
the \file{Modules} directory, add a line to the file
\file{Modules/Setup.local} describing your file:

\begin{verbatim}
spam spammodule.o
\end{verbatim}

and rebuild the interpreter by running \program{make} in the toplevel
directory.  You can also run \program{make} in the \file{Modules}
subdirectory, but then you must first rebuild \file{Makefile}
there by running `\program{make} Makefile'.  (This is necessary each
time you change the \file{Setup} file.)

If your module requires additional libraries to link with, these can
be listed on the line in the configuration file as well, for instance:

\begin{verbatim}
spam spammodule.o -lX11
\end{verbatim}

\section{Calling Python Functions from C
         \label{callingPython}}

So far we have concentrated on making C functions callable from
Python.  The reverse is also useful: calling Python functions from C.
This is especially the case for libraries that support so-called
``callback'' functions.  If a C interface makes use of callbacks, the
equivalent Python often needs to provide a callback mechanism to the
Python programmer; the implementation will require calling the Python
callback functions from a C callback.  Other uses are also imaginable.

Fortunately, the Python interpreter is easily called recursively, and
there is a standard interface to call a Python function.  (I won't
dwell on how to call the Python parser with a particular string as
input --- if you're interested, have a look at the implementation of
the \samp{-c} command line option in \file{Python/pythonmain.c} from
the Python source code.)

Calling a Python function is easy.  First, the Python program must
somehow pass you the Python function object.  You should provide a
function (or some other interface) to do this.  When this function is
called, save a pointer to the Python function object (be careful to
\cfunction{Py_INCREF()} it!) in a global variable --- or whereever you
see fit. For example, the following function might be part of a module
definition:

\begin{verbatim}
static PyObject *my_callback = NULL;

static PyObject *
my_set_callback(dummy, arg)
    PyObject *dummy, *arg;
{
    PyObject *result = NULL;
    PyObject *temp;

    if (PyArg_ParseTuple(args, "O:set_callback", &temp)) {
        if (!PyCallable_Check(temp)) {
            PyErr_SetString(PyExc_TypeError, "parameter must be callable");
            return NULL;
        }
        Py_XINCREF(temp);         /* Add a reference to new callback */
        Py_XDECREF(my_callback);  /* Dispose of previous callback */
        my_callback = temp;       /* Remember new callback */
        /* Boilerplate to return "None" */
        Py_INCREF(Py_None);
        result = Py_None;
    }
    return result;
}
\end{verbatim}

This function must be registered with the interpreter using the
\constant{METH_VARARGS} flag; this is described in Section
\ref{methodTable}, ``The Module's Method Table and Initialization
Function.''  The \cfunction{PyArg_ParseTuple()} function and its
arguments are documented in Section \ref{parseTuple}, ``Format Strings
for \cfunction{PyArg_ParseTuple()}.''

The macros \cfunction{Py_XINCREF()} and \cfunction{Py_XDECREF()}
increment/decrement the reference count of an object and are safe in
the presence of \NULL{} pointers (but note that \var{temp} will not be 
\NULL{} in this context).  More info on them in Section
\ref{refcounts}, ``Reference Counts.''

Later, when it is time to call the function, you call the C function
\cfunction{PyEval_CallObject()}.  This function has two arguments, both
pointers to arbitrary Python objects: the Python function, and the
argument list.  The argument list must always be a tuple object, whose
length is the number of arguments.  To call the Python function with
no arguments, pass an empty tuple; to call it with one argument, pass
a singleton tuple.  \cfunction{Py_BuildValue()} returns a tuple when its
format string consists of zero or more format codes between
parentheses.  For example:

\begin{verbatim}
    int arg;
    PyObject *arglist;
    PyObject *result;
    ...
    arg = 123;
    ...
    /* Time to call the callback */
    arglist = Py_BuildValue("(i)", arg);
    result = PyEval_CallObject(my_callback, arglist);
    Py_DECREF(arglist);
\end{verbatim}

\cfunction{PyEval_CallObject()} returns a Python object pointer: this is
the return value of the Python function.  \cfunction{PyEval_CallObject()} is
``reference-count-neutral'' with respect to its arguments.  In the
example a new tuple was created to serve as the argument list, which
is \cfunction{Py_DECREF()}-ed immediately after the call.

The return value of \cfunction{PyEval_CallObject()} is ``new'': either it
is a brand new object, or it is an existing object whose reference
count has been incremented.  So, unless you want to save it in a
global variable, you should somehow \cfunction{Py_DECREF()} the result,
even (especially!) if you are not interested in its value.

Before you do this, however, it is important to check that the return
value isn't \NULL{}.  If it is, the Python function terminated by
raising an exception.  If the C code that called
\cfunction{PyEval_CallObject()} is called from Python, it should now
return an error indication to its Python caller, so the interpreter
can print a stack trace, or the calling Python code can handle the
exception.  If this is not possible or desirable, the exception should
be cleared by calling \cfunction{PyErr_Clear()}.  For example:

\begin{verbatim}
    if (result == NULL)
        return NULL; /* Pass error back */
    ...use result...
    Py_DECREF(result); 
\end{verbatim}

Depending on the desired interface to the Python callback function,
you may also have to provide an argument list to
\cfunction{PyEval_CallObject()}.  In some cases the argument list is
also provided by the Python program, through the same interface that
specified the callback function.  It can then be saved and used in the
same manner as the function object.  In other cases, you may have to
construct a new tuple to pass as the argument list.  The simplest way
to do this is to call \cfunction{Py_BuildValue()}.  For example, if
you want to pass an integral event code, you might use the following
code:

\begin{verbatim}
    PyObject *arglist;
    ...
    arglist = Py_BuildValue("(l)", eventcode);
    result = PyEval_CallObject(my_callback, arglist);
    Py_DECREF(arglist);
    if (result == NULL)
        return NULL; /* Pass error back */
    /* Here maybe use the result */
    Py_DECREF(result);
\end{verbatim}

Note the placement of \samp{Py_DECREF(arglist)} immediately after the
call, before the error check!  Also note that strictly spoken this
code is not complete: \cfunction{Py_BuildValue()} may run out of
memory, and this should be checked.


\section{Format Strings for \cfunction{PyArg_ParseTuple()}
         \label{parseTuple}}

The \cfunction{PyArg_ParseTuple()} function is declared as follows:

\begin{verbatim}
int PyArg_ParseTuple(PyObject *arg, char *format, ...);
\end{verbatim}

The \var{arg} argument must be a tuple object containing an argument
list passed from Python to a C function.  The \var{format} argument
must be a format string, whose syntax is explained below.  The
remaining arguments must be addresses of variables whose type is
determined by the format string.  For the conversion to succeed, the
\var{arg} object must match the format and the format must be
exhausted.

Note that while \cfunction{PyArg_ParseTuple()} checks that the Python
arguments have the required types, it cannot check the validity of the
addresses of C variables passed to the call: if you make mistakes
there, your code will probably crash or at least overwrite random bits
in memory.  So be careful!

A format string consists of zero or more ``format units''.  A format
unit describes one Python object; it is usually a single character or
a parenthesized sequence of format units.  With a few exceptions, a
format unit that is not a parenthesized sequence normally corresponds
to a single address argument to \cfunction{PyArg_ParseTuple()}.  In the
following description, the quoted form is the format unit; the entry
in (round) parentheses is the Python object type that matches the
format unit; and the entry in [square] brackets is the type of the C
variable(s) whose address should be passed.  (Use the \samp{\&}
operator to pass a variable's address.)

\begin{description}

\item[\samp{s} (string) {[char *]}]
Convert a Python string to a C pointer to a character string.  You
must not provide storage for the string itself; a pointer to an
existing string is stored into the character pointer variable whose
address you pass.  The C string is null-terminated.  The Python string
must not contain embedded null bytes; if it does, a \exception{TypeError}
exception is raised.

\item[\samp{s\#} (string) {[char *, int]}]
This variant on \samp{s} stores into two C variables, the first one
a pointer to a character string, the second one its length.  In this
case the Python string may contain embedded null bytes.

\item[\samp{z} (string or \code{None}) {[char *]}]
Like \samp{s}, but the Python object may also be \code{None}, in which
case the C pointer is set to \NULL{}.

\item[\samp{z\#} (string or \code{None}) {[char *, int]}]
This is to \samp{s\#} as \samp{z} is to \samp{s}.

\item[\samp{b} (integer) {[char]}]
Convert a Python integer to a tiny int, stored in a C \ctype{char}.

\item[\samp{h} (integer) {[short int]}]
Convert a Python integer to a C \ctype{short int}.

\item[\samp{i} (integer) {[int]}]
Convert a Python integer to a plain C \ctype{int}.

\item[\samp{l} (integer) {[long int]}]
Convert a Python integer to a C \ctype{long int}.

\item[\samp{c} (string of length 1) {[char]}]
Convert a Python character, represented as a string of length 1, to a
C \ctype{char}.

\item[\samp{f} (float) {[float]}]
Convert a Python floating point number to a C \ctype{float}.

\item[\samp{d} (float) {[double]}]
Convert a Python floating point number to a C \ctype{double}.

\item[\samp{D} (complex) {[Py_complex]}]
Convert a Python complex number to a C \ctype{Py_complex} structure.

\item[\samp{O} (object) {[PyObject *]}]
Store a Python object (without any conversion) in a C object pointer.
The C program thus receives the actual object that was passed.  The
object's reference count is not increased.  The pointer stored is not
\NULL{}.

\item[\samp{O!} (object) {[\var{typeobject}, PyObject *]}]
Store a Python object in a C object pointer.  This is similar to
\samp{O}, but takes two C arguments: the first is the address of a
Python type object, the second is the address of the C variable (of
type \ctype{PyObject *}) into which the object pointer is stored.
If the Python object does not have the required type, a
\exception{TypeError} exception is raised.

\item[\samp{O\&} (object) {[\var{converter}, \var{anything}]}]
Convert a Python object to a C variable through a \var{converter}
function.  This takes two arguments: the first is a function, the
second is the address of a C variable (of arbitrary type), converted
to \ctype{void *}.  The \var{converter} function in turn is called as
follows:

\code{\var{status} = \var{converter}(\var{object}, \var{address});}

where \var{object} is the Python object to be converted and
\var{address} is the \ctype{void *} argument that was passed to
\cfunction{PyArg_ConvertTuple()}.  The returned \var{status} should be
\code{1} for a successful conversion and \code{0} if the conversion
has failed.  When the conversion fails, the \var{converter} function
should raise an exception.

\item[\samp{S} (string) {[PyStringObject *]}]
Like \samp{O} but requires that the Python object is a string object.
Raises a \exception{TypeError} exception if the object is not a string
object.  The C variable may also be declared as \ctype{PyObject *}.

\item[\samp{(\var{items})} (tuple) {[\var{matching-items}]}]
The object must be a Python tuple whose length is the number of format
units in \var{items}.  The C arguments must correspond to the
individual format units in \var{items}.  Format units for tuples may
be nested.

\end{description}

It is possible to pass Python long integers where integers are
requested; however no proper range checking is done --- the most
significant bits are silently truncated when the receiving field is
too small to receive the value (actually, the semantics are inherited
from downcasts in C --- your milage may vary).

A few other characters have a meaning in a format string.  These may
not occur inside nested parentheses.  They are:

\begin{description}

\item[\samp{|}]
Indicates that the remaining arguments in the Python argument list are
optional.  The C variables corresponding to optional arguments should
be initialized to their default value --- when an optional argument is
not specified, \cfunction{PyArg_ParseTuple()} does not touch the contents
of the corresponding C variable(s).

\item[\samp{:}]
The list of format units ends here; the string after the colon is used
as the function name in error messages (the ``associated value'' of
the exceptions that \cfunction{PyArg_ParseTuple()} raises).

\item[\samp{;}]
The list of format units ends here; the string after the colon is used
as the error message \emph{instead} of the default error message.
Clearly, \samp{:} and \samp{;} mutually exclude each other.

\end{description}

Some example calls:

\begin{verbatim}
    int ok;
    int i, j;
    long k, l;
    char *s;
    int size;
\end{verbatim}

\begin{verbatim}
    ok = PyArg_ParseTuple(args, ""); /* No arguments */
        /* Python call: f() */
\end{verbatim}

\begin{verbatim}
    ok = PyArg_ParseTuple(args, "s", &s); /* A string */
        /* Possible Python call: f('whoops!') */
\end{verbatim}

\begin{verbatim}
    ok = PyArg_ParseTuple(args, "lls", &k, &l, &s); /* Two longs and a string */
        /* Possible Python call: f(1, 2, 'three') */
\end{verbatim}

\begin{verbatim}
    ok = PyArg_ParseTuple(args, "(ii)s#", &i, &j, &s, &size);
        /* A pair of ints and a string, whose size is also returned */
        /* Possible Python call: f((1, 2), 'three') */
\end{verbatim}

\begin{verbatim}
    {
        char *file;
        char *mode = "r";
        int bufsize = 0;
        ok = PyArg_ParseTuple(args, "s|si", &file, &mode, &bufsize);
        /* A string, and optionally another string and an integer */
        /* Possible Python calls:
           f('spam')
           f('spam', 'w')
           f('spam', 'wb', 100000) */
    }
\end{verbatim}

\begin{verbatim}
    {
        int left, top, right, bottom, h, v;
        ok = PyArg_ParseTuple(args, "((ii)(ii))(ii)",
                 &left, &top, &right, &bottom, &h, &v);
        /* A rectangle and a point */
        /* Possible Python call:
           f(((0, 0), (400, 300)), (10, 10)) */
    }
\end{verbatim}

\begin{verbatim}
    {
        Py_complex c;
        ok = PyArg_ParseTuple(args, "D:myfunction", &c);
        /* a complex, also providing a function name for errors */
        /* Possible Python call: myfunction(1+2j) */
    }
\end{verbatim}


\section{Keyword Parsing with \cfunction{PyArg_ParseTupleAndKeywords()}
         \label{parseTupleAndKeywords}}

The \cfunction{PyArg_ParseTupleAndKeywords()} function is declared as
follows:

\begin{verbatim}
int PyArg_ParseTupleAndKeywords(PyObject *arg, PyObject *kwdict,
                                char *format, char **kwlist, ...);
\end{verbatim}

The \var{arg} and \var{format} parameters are identical to those of the
\cfunction{PyArg_ParseTuple()} function.  The \var{kwdict} parameter
is the dictionary of keywords received as the third parameter from the 
Python runtime.  The \var{kwlist} parameter is a \NULL{}-terminated
list of strings which identify the parameters; the names are matched
with the type information from \var{format} from left to right.

\strong{Note:}  Nested tuples cannot be parsed when using keyword
arguments!  Keyword parameters passed in which are not present in the
\var{kwlist} will cause \exception{TypeError} to be raised.

Here is an example module which uses keywords, based on an example by
Geoff Philbrick (\email{philbrick@hks.com}):%
\index{Philbrick, Geoff}

\begin{verbatim}
#include <stdio.h>
#include "Python.h"

static PyObject *
keywdarg_parrot(self, args, keywds)
    PyObject *self;
    PyObject *args;
    PyObject *keywds;
{  
    int voltage;
    char *state = "a stiff";
    char *action = "voom";
    char *type = "Norwegian Blue";

    static char *kwlist[] = {"voltage", "state", "action", "type", NULL};

    if (!PyArg_ParseTupleAndKeywords(args, keywds, "i|sss", kwlist, 
                                     &voltage, &state, &action, &type))
        return NULL; 
  
    printf("-- This parrot wouldn't %s if you put %i Volts through it.\n", 
           action, voltage);
    printf("-- Lovely plumage, the %s -- It's %s!\n", type, state);

    Py_INCREF(Py_None);

    return Py_None;
}

static PyMethodDef keywdarg_methods[] = {
    {"parrot", (PyCFunction)keywdarg_parrot, METH_VARARGS|METH_KEYWORDS},
    {NULL,  NULL}   /* sentinel */
};

void
initkeywdarg()
{
  /* Create the module and add the functions */
  Py_InitModule("keywdarg", keywdarg_methods);
}
\end{verbatim}


\section{The \cfunction{Py_BuildValue()} Function
         \label{buildValue}}

This function is the counterpart to \cfunction{PyArg_ParseTuple()}.  It is
declared as follows:

\begin{verbatim}
PyObject *Py_BuildValue(char *format, ...);
\end{verbatim}

It recognizes a set of format units similar to the ones recognized by
\cfunction{PyArg_ParseTuple()}, but the arguments (which are input to the
function, not output) must not be pointers, just values.  It returns a
new Python object, suitable for returning from a C function called
from Python.

One difference with \cfunction{PyArg_ParseTuple()}: while the latter
requires its first argument to be a tuple (since Python argument lists
are always represented as tuples internally),
\cfunction{Py_BuildValue()} does not always build a tuple.  It builds
a tuple only if its format string contains two or more format units.
If the format string is empty, it returns \code{None}; if it contains
exactly one format unit, it returns whatever object is described by
that format unit.  To force it to return a tuple of size 0 or one,
parenthesize the format string.

In the following description, the quoted form is the format unit; the
entry in (round) parentheses is the Python object type that the format
unit will return; and the entry in [square] brackets is the type of
the C value(s) to be passed.

The characters space, tab, colon and comma are ignored in format
strings (but not within format units such as \samp{s\#}).  This can be
used to make long format strings a tad more readable.

\begin{description}

\item[\samp{s} (string) {[char *]}]
Convert a null-terminated C string to a Python object.  If the C
string pointer is \NULL{}, \code{None} is returned.

\item[\samp{s\#} (string) {[char *, int]}]
Convert a C string and its length to a Python object.  If the C string
pointer is \NULL{}, the length is ignored and \code{None} is
returned.

\item[\samp{z} (string or \code{None}) {[char *]}]
Same as \samp{s}.

\item[\samp{z\#} (string or \code{None}) {[char *, int]}]
Same as \samp{s\#}.

\item[\samp{i} (integer) {[int]}]
Convert a plain C \ctype{int} to a Python integer object.

\item[\samp{b} (integer) {[char]}]
Same as \samp{i}.

\item[\samp{h} (integer) {[short int]}]
Same as \samp{i}.

\item[\samp{l} (integer) {[long int]}]
Convert a C \ctype{long int} to a Python integer object.

\item[\samp{c} (string of length 1) {[char]}]
Convert a C \ctype{int} representing a character to a Python string of
length 1.

\item[\samp{d} (float) {[double]}]
Convert a C \ctype{double} to a Python floating point number.

\item[\samp{f} (float) {[float]}]
Same as \samp{d}.

\item[\samp{O} (object) {[PyObject *]}]
Pass a Python object untouched (except for its reference count, which
is incremented by one).  If the object passed in is a \NULL{}
pointer, it is assumed that this was caused because the call producing
the argument found an error and set an exception.  Therefore,
\cfunction{Py_BuildValue()} will return \NULL{} but won't raise an
exception.  If no exception has been raised yet,
\cdata{PyExc_SystemError} is set.

\item[\samp{S} (object) {[PyObject *]}]
Same as \samp{O}.

\item[\samp{N} (object) {[PyObject *]}]
Same as \samp{O}, except it doesn't increment the reference count on
the object.  Useful when the object is created by a call to an object
constructor in the argument list.

\item[\samp{O\&} (object) {[\var{converter}, \var{anything}]}]
Convert \var{anything} to a Python object through a \var{converter}
function.  The function is called with \var{anything} (which should be
compatible with \ctype{void *}) as its argument and should return a
``new'' Python object, or \NULL{} if an error occurred.

\item[\samp{(\var{items})} (tuple) {[\var{matching-items}]}]
Convert a sequence of C values to a Python tuple with the same number
of items.

\item[\samp{[\var{items}]} (list) {[\var{matching-items}]}]
Convert a sequence of C values to a Python list with the same number
of items.

\item[\samp{\{\var{items}\}} (dictionary) {[\var{matching-items}]}]
Convert a sequence of C values to a Python dictionary.  Each pair of
consecutive C values adds one item to the dictionary, serving as key
and value, respectively.

\end{description}

If there is an error in the format string, the
\cdata{PyExc_SystemError} exception is raised and \NULL{} returned.

Examples (to the left the call, to the right the resulting Python value):

\begin{verbatim}
    Py_BuildValue("")                        None
    Py_BuildValue("i", 123)                  123
    Py_BuildValue("iii", 123, 456, 789)      (123, 456, 789)
    Py_BuildValue("s", "hello")              'hello'
    Py_BuildValue("ss", "hello", "world")    ('hello', 'world')
    Py_BuildValue("s#", "hello", 4)          'hell'
    Py_BuildValue("()")                      ()
    Py_BuildValue("(i)", 123)                (123,)
    Py_BuildValue("(ii)", 123, 456)          (123, 456)
    Py_BuildValue("(i,i)", 123, 456)         (123, 456)
    Py_BuildValue("[i,i]", 123, 456)         [123, 456]
    Py_BuildValue("{s:i,s:i}",
                  "abc", 123, "def", 456)    {'abc': 123, 'def': 456}
    Py_BuildValue("((ii)(ii)) (ii)",
                  1, 2, 3, 4, 5, 6)          (((1, 2), (3, 4)), (5, 6))
\end{verbatim}

\section{Reference Counts
         \label{refcounts}}

%\subsection{Introduction}

In languages like C or \Cpp{}, the programmer is responsible for
dynamic allocation and deallocation of memory on the heap.  In C,
this is done using the functions \cfunction{malloc()} and
\cfunction{free()}.  In \Cpp{}, the operators \keyword{new} and
\keyword{delete} are used with essentially the same meaning; they are
actually implemented using \cfunction{malloc()} and
\cfunction{free()}, so we'll restrict the following discussion to the
latter.

Every block of memory allocated with \cfunction{malloc()} should
eventually be returned to the pool of available memory by exactly one
call to \cfunction{free()}.  It is important to call
\cfunction{free()} at the right time.  If a block's address is
forgotten but \cfunction{free()} is not called for it, the memory it
occupies cannot be reused until the program terminates.  This is
called a \dfn{memory leak}.  On the other hand, if a program calls
\cfunction{free()} for a block and then continues to use the block, it
creates a conflict with re-use of the block through another
\cfunction{malloc()} call.  This is called \dfn{using freed memory}.
It has the same bad consequences as referencing uninitialized data ---
core dumps, wrong results, mysterious crashes.

Common causes of memory leaks are unusual paths through the code.  For
instance, a function may allocate a block of memory, do some
calculation, and then free the block again.  Now a change in the
requirements for the function may add a test to the calculation that
detects an error condition and can return prematurely from the
function.  It's easy to forget to free the allocated memory block when
taking this premature exit, especially when it is added later to the
code.  Such leaks, once introduced, often go undetected for a long
time: the error exit is taken only in a small fraction of all calls,
and most modern machines have plenty of virtual memory, so the leak
only becomes apparent in a long-running process that uses the leaking
function frequently.  Therefore, it's important to prevent leaks from
happening by having a coding convention or strategy that minimizes
this kind of errors.

Since Python makes heavy use of \cfunction{malloc()} and
\cfunction{free()}, it needs a strategy to avoid memory leaks as well
as the use of freed memory.  The chosen method is called
\dfn{reference counting}.  The principle is simple: every object
contains a counter, which is incremented when a reference to the
object is stored somewhere, and which is decremented when a reference
to it is deleted.  When the counter reaches zero, the last reference
to the object has been deleted and the object is freed.

An alternative strategy is called \dfn{automatic garbage collection}.
(Sometimes, reference counting is also referred to as a garbage
collection strategy, hence my use of ``automatic'' to distinguish the
two.)  The big advantage of automatic garbage collection is that the
user doesn't need to call \cfunction{free()} explicitly.  (Another claimed
advantage is an improvement in speed or memory usage --- this is no
hard fact however.)  The disadvantage is that for C, there is no
truly portable automatic garbage collector, while reference counting
can be implemented portably (as long as the functions \cfunction{malloc()}
and \cfunction{free()} are available --- which the C Standard guarantees).
Maybe some day a sufficiently portable automatic garbage collector
will be available for C.  Until then, we'll have to live with
reference counts.

\subsection{Reference Counting in Python
            \label{refcountsInPython}}

There are two macros, \code{Py_INCREF(x)} and \code{Py_DECREF(x)},
which handle the incrementing and decrementing of the reference count.
\cfunction{Py_DECREF()} also frees the object when the count reaches zero.
For flexibility, it doesn't call \cfunction{free()} directly --- rather, it
makes a call through a function pointer in the object's \dfn{type
object}.  For this purpose (and others), every object also contains a
pointer to its type object.

The big question now remains: when to use \code{Py_INCREF(x)} and
\code{Py_DECREF(x)}?  Let's first introduce some terms.  Nobody
``owns'' an object; however, you can \dfn{own a reference} to an
object.  An object's reference count is now defined as the number of
owned references to it.  The owner of a reference is responsible for
calling \cfunction{Py_DECREF()} when the reference is no longer
needed.  Ownership of a reference can be transferred.  There are three
ways to dispose of an owned reference: pass it on, store it, or call
\cfunction{Py_DECREF()}.  Forgetting to dispose of an owned reference
creates a memory leak.

It is also possible to \dfn{borrow}\footnote{The metaphor of
``borrowing'' a reference is not completely correct: the owner still
has a copy of the reference.} a reference to an object.  The borrower
of a reference should not call \cfunction{Py_DECREF()}.  The borrower must
not hold on to the object longer than the owner from which it was
borrowed.  Using a borrowed reference after the owner has disposed of
it risks using freed memory and should be avoided
completely.\footnote{Checking that the reference count is at least 1
\strong{does not work} --- the reference count itself could be in
freed memory and may thus be reused for another object!}

The advantage of borrowing over owning a reference is that you don't
need to take care of disposing of the reference on all possible paths
through the code --- in other words, with a borrowed reference you
don't run the risk of leaking when a premature exit is taken.  The
disadvantage of borrowing over leaking is that there are some subtle
situations where in seemingly correct code a borrowed reference can be
used after the owner from which it was borrowed has in fact disposed
of it.

A borrowed reference can be changed into an owned reference by calling
\cfunction{Py_INCREF()}.  This does not affect the status of the owner from
which the reference was borrowed --- it creates a new owned reference,
and gives full owner responsibilities (i.e., the new owner must
dispose of the reference properly, as well as the previous owner).

\subsection{Ownership Rules
            \label{ownershipRules}}

Whenever an object reference is passed into or out of a function, it
is part of the function's interface specification whether ownership is
transferred with the reference or not.

Most functions that return a reference to an object pass on ownership
with the reference.  In particular, all functions whose function it is
to create a new object, e.g.\ \cfunction{PyInt_FromLong()} and
\cfunction{Py_BuildValue()}, pass ownership to the receiver.  Even if in
fact, in some cases, you don't receive a reference to a brand new
object, you still receive ownership of the reference.  For instance,
\cfunction{PyInt_FromLong()} maintains a cache of popular values and can
return a reference to a cached item.

Many functions that extract objects from other objects also transfer
ownership with the reference, for instance
\cfunction{PyObject_GetAttrString()}.  The picture is less clear, here,
however, since a few common routines are exceptions:
\cfunction{PyTuple_GetItem()}, \cfunction{PyList_GetItem()},
\cfunction{PyDict_GetItem()}, and \cfunction{PyDict_GetItemString()}
all return references that you borrow from the tuple, list or
dictionary.

The function \cfunction{PyImport_AddModule()} also returns a borrowed
reference, even though it may actually create the object it returns:
this is possible because an owned reference to the object is stored in
\code{sys.modules}.

When you pass an object reference into another function, in general,
the function borrows the reference from you --- if it needs to store
it, it will use \cfunction{Py_INCREF()} to become an independent
owner.  There are exactly two important exceptions to this rule:
\cfunction{PyTuple_SetItem()} and \cfunction{PyList_SetItem()}.  These
functions take over ownership of the item passed to them --- even if
they fail!  (Note that \cfunction{PyDict_SetItem()} and friends don't
take over ownership --- they are ``normal.'')

When a C function is called from Python, it borrows references to its
arguments from the caller.  The caller owns a reference to the object,
so the borrowed reference's lifetime is guaranteed until the function
returns.  Only when such a borrowed reference must be stored or passed
on, it must be turned into an owned reference by calling
\cfunction{Py_INCREF()}.

The object reference returned from a C function that is called from
Python must be an owned reference --- ownership is tranferred from the
function to its caller.

\subsection{Thin Ice
            \label{thinIce}}

There are a few situations where seemingly harmless use of a borrowed
reference can lead to problems.  These all have to do with implicit
invocations of the interpreter, which can cause the owner of a
reference to dispose of it.

The first and most important case to know about is using
\cfunction{Py_DECREF()} on an unrelated object while borrowing a
reference to a list item.  For instance:

\begin{verbatim}
bug(PyObject *list) {
    PyObject *item = PyList_GetItem(list, 0);

    PyList_SetItem(list, 1, PyInt_FromLong(0L));
    PyObject_Print(item, stdout, 0); /* BUG! */
}
\end{verbatim}

This function first borrows a reference to \code{list[0]}, then
replaces \code{list[1]} with the value \code{0}, and finally prints
the borrowed reference.  Looks harmless, right?  But it's not!

Let's follow the control flow into \cfunction{PyList_SetItem()}.  The list
owns references to all its items, so when item 1 is replaced, it has
to dispose of the original item 1.  Now let's suppose the original
item 1 was an instance of a user-defined class, and let's further
suppose that the class defined a \method{__del__()} method.  If this
class instance has a reference count of 1, disposing of it will call
its \method{__del__()} method.

Since it is written in Python, the \method{__del__()} method can execute
arbitrary Python code.  Could it perhaps do something to invalidate
the reference to \code{item} in \cfunction{bug()}?  You bet!  Assuming
that the list passed into \cfunction{bug()} is accessible to the
\method{__del__()} method, it could execute a statement to the effect of
\samp{del list[0]}, and assuming this was the last reference to that
object, it would free the memory associated with it, thereby
invalidating \code{item}.

The solution, once you know the source of the problem, is easy:
temporarily increment the reference count.  The correct version of the
function reads:

\begin{verbatim}
no_bug(PyObject *list) {
    PyObject *item = PyList_GetItem(list, 0);

    Py_INCREF(item);
    PyList_SetItem(list, 1, PyInt_FromLong(0L));
    PyObject_Print(item, stdout, 0);
    Py_DECREF(item);
}
\end{verbatim}

This is a true story.  An older version of Python contained variants
of this bug and someone spent a considerable amount of time in a C
debugger to figure out why his \method{__del__()} methods would fail...

The second case of problems with a borrowed reference is a variant
involving threads.  Normally, multiple threads in the Python
interpreter can't get in each other's way, because there is a global
lock protecting Python's entire object space.  However, it is possible
to temporarily release this lock using the macro
\code{Py_BEGIN_ALLOW_THREADS}, and to re-acquire it using
\code{Py_END_ALLOW_THREADS}.  This is common around blocking I/O
calls, to let other threads use the CPU while waiting for the I/O to
complete.  Obviously, the following function has the same problem as
the previous one:

\begin{verbatim}
bug(PyObject *list) {
    PyObject *item = PyList_GetItem(list, 0);
    Py_BEGIN_ALLOW_THREADS
    ...some blocking I/O call...
    Py_END_ALLOW_THREADS
    PyObject_Print(item, stdout, 0); /* BUG! */
}
\end{verbatim}

\subsection{NULL Pointers
            \label{nullPointers}}

In general, functions that take object references as arguments do not
expect you to pass them \NULL{} pointers, and will dump core (or
cause later core dumps) if you do so.  Functions that return object
references generally return \NULL{} only to indicate that an
exception occurred.  The reason for not testing for \NULL{}
arguments is that functions often pass the objects they receive on to
other function --- if each function were to test for \NULL{},
there would be a lot of redundant tests and the code would run slower.

It is better to test for \NULL{} only at the ``source'', i.e.\ when a
pointer that may be \NULL{} is received, e.g.\ from
\cfunction{malloc()} or from a function that may raise an exception.

The macros \cfunction{Py_INCREF()} and \cfunction{Py_DECREF()}
do not check for \NULL{} pointers --- however, their variants
\cfunction{Py_XINCREF()} and \cfunction{Py_XDECREF()} do.

The macros for checking for a particular object type
(\code{Py\var{type}_Check()}) don't check for \NULL{} pointers ---
again, there is much code that calls several of these in a row to test
an object against various different expected types, and this would
generate redundant tests.  There are no variants with \NULL{}
checking.

The C function calling mechanism guarantees that the argument list
passed to C functions (\code{args} in the examples) is never
\NULL{} --- in fact it guarantees that it is always a tuple.%
\footnote{These guarantees don't hold when you use the ``old'' style
calling convention --- this is still found in much existing code.}

It is a severe error to ever let a \NULL{} pointer ``escape'' to
the Python user.  


\section{Writing Extensions in \Cpp{}
         \label{cplusplus}}

It is possible to write extension modules in \Cpp{}.  Some restrictions
apply.  If the main program (the Python interpreter) is compiled and
linked by the C compiler, global or static objects with constructors
cannot be used.  This is not a problem if the main program is linked
by the \Cpp{} compiler.  Functions that will be called by the
Python interpreter (in particular, module initalization functions)
have to be declared using \code{extern "C"}.
It is unnecessary to enclose the Python header files in
\code{extern "C" \{...\}} --- they use this form already if the symbol
\samp{__cplusplus} is defined (all recent \Cpp{} compilers define this
symbol).


\section{Providing a C API for an Extension Module
         \label{using-cobjects}}
\sectionauthor{Konrad Hinsen}{hinsen@cnrs-orleans.fr}

Many extension modules just provide new functions and types to be
used from Python, but sometimes the code in an extension module can
be useful for other extension modules. For example, an extension
module could implement a type ``collection'' which works like lists
without order. Just like the standard Python list type has a C API
which permits extension modules to create and manipulate lists, this
new collection type should have a set of C functions for direct
manipulation from other extension modules.

At first sight this seems easy: just write the functions (without
declaring them \keyword{static}, of course), provide an appropriate
header file, and document the C API. And in fact this would work if
all extension modules were always linked statically with the Python
interpreter. When modules are used as shared libraries, however, the
symbols defined in one module may not be visible to another module.
The details of visibility depend on the operating system; some systems
use one global namespace for the Python interpreter and all extension
modules (e.g.\ Windows), whereas others require an explicit list of
imported symbols at module link time (e.g.\ AIX), or offer a choice of
different strategies (most Unices). And even if symbols are globally
visible, the module whose functions one wishes to call might not have
been loaded yet!

Portability therefore requires not to make any assumptions about
symbol visibility. This means that all symbols in extension modules
should be declared \keyword{static}, except for the module's
initialization function, in order to avoid name clashes with other
extension modules (as discussed in section~\ref{methodTable}). And it
means that symbols that \emph{should} be accessible from other
extension modules must be exported in a different way.

Python provides a special mechanism to pass C-level information (i.e.
pointers) from one extension module to another one: CObjects.
A CObject is a Python data type which stores a pointer (\ctype{void
*}).  CObjects can only be created and accessed via their C API, but
they can be passed around like any other Python object. In particular, 
they can be assigned to a name in an extension module's namespace.
Other extension modules can then import this module, retrieve the
value of this name, and then retrieve the pointer from the CObject.

There are many ways in which CObjects can be used to export the C API
of an extension module. Each name could get its own CObject, or all C
API pointers could be stored in an array whose address is published in
a CObject. And the various tasks of storing and retrieving the pointers
can be distributed in different ways between the module providing the
code and the client modules.

The following example demonstrates an approach that puts most of the
burden on the writer of the exporting module, which is appropriate
for commonly used library modules. It stores all C API pointers
(just one in the example!) in an array of \ctype{void} pointers which
becomes the value of a CObject. The header file corresponding to
the module provides a macro that takes care of importing the module
and retrieving its C API pointers; client modules only have to call
this macro before accessing the C API.

The exporting module is a modification of the \module{spam} module from
section~\ref{simpleExample}. The function \function{spam.system()}
does not call the C library function \cfunction{system()} directly,
but a function \cfunction{PySpam_System()}, which would of course do
something more complicated in reality (such as adding ``spam'' to
every command). This function \cfunction{PySpam_System()} is also
exported to other extension modules.

The function \cfunction{PySpam_System()} is a plain C function,
declared \keyword{static} like everything else:

\begin{verbatim}
static int
PySpam_System(command)
    char *command;
{
    return system(command);
}
\end{verbatim}

The function \cfunction{spam_system()} is modified in a trivial way:

\begin{verbatim}
static PyObject *
spam_system(self, args)
    PyObject *self;
    PyObject *args;
{
    char *command;
    int sts;

    if (!PyArg_ParseTuple(args, "s", &command))
        return NULL;
    sts = PySpam_System(command);
    return Py_BuildValue("i", sts);
}
\end{verbatim}

In the beginning of the module, right after the line
\begin{verbatim}
#include "Python.h"
\end{verbatim}
two more lines must be added:
\begin{verbatim}
#define SPAM_MODULE
#include "spammodule.h"
\end{verbatim}

The \code{\#define} is used to tell the header file that it is being
included in the exporting module, not a client module. Finally,
the module's initialization function must take care of initializing
the C API pointer array:
\begin{verbatim}
void
initspam()
{
    PyObject *m, *d;
    static void *PySpam_API[PySpam_API_pointers];
    PyObject *c_api_object;
    m = Py_InitModule("spam", SpamMethods);

    /* Initialize the C API pointer array */
    PySpam_API[PySpam_System_NUM] = (void *)PySpam_System;

    /* Create a CObject containing the API pointer array's address */
    c_api_object = PyCObject_FromVoidPtr((void *)PySpam_API, NULL);

    /* Create a name for this object in the module's namespace */
    d = PyModule_GetDict(m);
    PyDict_SetItemString(d, "_C_API", c_api_object);
}
\end{verbatim}

Note that \code{PySpam_API} is declared \code{static}; otherwise
the pointer array would disappear when \code{initspam} terminates!

The bulk of the work is in the header file \file{spammodule.h},
which looks like this:

\begin{verbatim}
#ifndef Py_SPAMMODULE_H
#define Py_SPAMMODULE_H
#ifdef __cplusplus
extern "C" {
#endif

/* Header file for spammodule */

/* C API functions */
#define PySpam_System_NUM 0
#define PySpam_System_RETURN int
#define PySpam_System_PROTO Py_PROTO((char *command))

/* Total number of C API pointers */
#define PySpam_API_pointers 1


#ifdef SPAM_MODULE
/* This section is used when compiling spammodule.c */

static PySpam_System_RETURN PySpam_System PySpam_System_PROTO;

#else
/* This section is used in modules that use spammodule's API */

static void **PySpam_API;

#define PySpam_System \
 (*(PySpam_System_RETURN (*)PySpam_System_PROTO) PySpam_API[PySpam_System_NUM])

#define import_spam() \
{ \
  PyObject *module = PyImport_ImportModule("spam"); \
  if (module != NULL) { \
    PyObject *module_dict = PyModule_GetDict(module); \
    PyObject *c_api_object = PyDict_GetItemString(module_dict, "_C_API"); \
    if (PyCObject_Check(c_api_object)) { \
      PySpam_API = (void **)PyCObject_AsVoidPtr(c_api_object); \
    } \
  } \
}

#endif

#ifdef __cplusplus
}
#endif

#endif /* !defined(Py_SPAMMODULE_H */
\end{verbatim}

All that a client module must do in order to have access to the
function \cfunction{PySpam_System()} is to call the function (or
rather macro) \cfunction{import_spam()} in its initialization
function:

\begin{verbatim}
void
initclient()
{
    PyObject *m;

    Py_InitModule("client", ClientMethods);
    import_spam();
}
\end{verbatim}

The main disadvantage of this approach is that the file
\file{spammodule.h} is rather complicated. However, the
basic structure is the same for each function that is
exported, so it has to be learned only once.

Finally it should be mentioned that CObjects offer additional
functionality, which is especially useful for memory allocation and
deallocation of the pointer stored in a CObject. The details
are described in the \emph{Python/C API Reference Manual} in the
section ``CObjects'' and in the implementation of CObjects (files
\file{Include/cobject.h} and \file{Objects/cobject.c} in the
Python source code distribution).


\chapter{Building C and \Cpp{} Extensions on \UNIX{}
         \label{building-on-unix}}

\sectionauthor{Jim Fulton}{jim@Digicool.com}


%The make file make file, building C extensions on Unix


Starting in Python 1.4, Python provides a special make file for
building make files for building dynamically-linked extensions and
custom interpreters.  The make file make file builds a make file
that reflects various system variables determined by configure when
the Python interpreter was built, so people building module's don't
have to resupply these settings.  This vastly simplifies the process
of building extensions and custom interpreters on Unix systems.

The make file make file is distributed as the file
\file{Misc/Makefile.pre.in} in the Python source distribution.  The
first step in building extensions or custom interpreters is to copy
this make file to a development directory containing extension module
source.

The make file make file, \file{Makefile.pre.in} uses metadata
provided in a file named \file{Setup}.  The format of the \file{Setup}
file is the same as the \file{Setup} (or \file{Setup.in}) file
provided in the \file{Modules/} directory of the Python source
distribution.  The \file{Setup} file contains variable definitions:

\begin{verbatim}
EC=/projects/ExtensionClass
\end{verbatim}

and module description lines.  It can also contain blank lines and
comment lines that start with \character{\#}.

A module description line includes a module name, source files,
options, variable references, and other input files, such
as libraries or object files.  Consider a simple example::

\begin{verbatim}
ExtensionClass ExtensionClass.c
\end{verbatim}

This is the simplest form of a module definition line.  It defines a
dule, \module{ExtensionClass}, which has a single source file,
\file{ExtensionClass.c}.

Here is a slightly more complex example that uses an \strong{-I}
option to specify an include directory:

\begin{verbatim}
cPersistence cPersistence.c -I$(EC)
\end{verbatim}

This example also illustrates the format for variable references.

For systems that support dynamic linking, the \file{Setup} file should 
begin:

\begin{verbatim}
*shared*
\end{verbatim}

to indicate that the modules defined in \file{Setup} are to be built
as dynamically-linked linked modules.

Here is a complete \file{Setup} file for building a
\module{cPersistent} module:

\begin{verbatim}
# Set-up file to build the cPersistence module. 
# Note that the text should begin in the first column.
*shared*

# We need the path to the directory containing the ExtensionClass
# include file.
EC=/projects/ExtensionClass
cPersistence cPersistence.c -I$(EC)
\end{verbatim}

After the \file{Setup} file has been created, \file{Makefile.pre.in}
is run with the \samp{boot} target to create a make file:

\begin{verbatim}
make -f Makefile.pre.in boot
\end{verbatim}

This creates the file, Makefile.  To build the extensions, simply
run the created make file:

\begin{verbatim}
make
\end{verbatim}

It's not necessary to re-run \file{Makefile.pre.in} if the
\file{Setup} file is changed.  The make file automatically rebuilds
itself if the \file{Setup} file changes.

\section{Building Custom Interpreters}

The make file built by \file{Makefile.pre.in} can be run with the
\samp{static} target to build an interpreter:

\begin{verbatim}
make static
\end{verbatim}

Any modules defined in the Setup file before the \samp{*shared*} line
will be statically linked into the interpreter.  Typically, a
\samp{*shared*} line is omitted from the Setup file when a custom
interpreter is desired.

\section{Module Definition Options}

Several compiler options are supported:

\begin{tableii}{l|l}{}{Option}{Meaning}
  \lineii{-C}{Tell the C pre-processor not to discard comments}
  \lineii{-D\var{name}=\var{value}}{Define a macro}
  \lineii{-I\var{dir}}{Specify an include directory, \var{dir}}
  \lineii{-L\var{dir}}{Specify a link-time library directory, \var{dir}}
  \lineii{-R\var{dir}}{Specify a run-time library directory, \var{dir}}
  \lineii{-l\var{lib}}{Link a library, \var{lib}}
  \lineii{-U\var{name}}{Undefine a macro}
\end{tableii}

Other compiler options can be included (snuck in) by putting them
in variable variables.

Source files can include files with \file{.c}, \file{.C}, \file{.cc},
and \file{.c++} extensions. 

Other input files include files with \file{.o} or \file{.a}
extensions.


\section{Example}

Here is a more complicated example from \file{Modules/Setup.in}:

\begin{verbatim}
GMP=/ufs/guido/src/gmp
mpz mpzmodule.c -I$(GMP) $(GMP)/libgmp.a
\end{verbatim}

which could also be written as:

\begin{verbatim}
mpz mpzmodule.c -I$(GMP) -L$(GMP) -lgmp
\end{verbatim}


\section{Distributing your extension modules
         \label{distributing}}

When distributing your extension modules in source form, make sure to
include a \file{Setup} file.  The \file{Setup} file should be named
\file{Setup.in} in the distribution.  The make file make file,
\file{Makefile.pre.in}, will copy \file{Setup.in} to \file{Setup}.
Distributing a \file{Setup.in} file makes it easy for people to
customize the \file{Setup} file while keeping the original in
\file{Setup.in}.

It is a good idea to include a copy of \file{Makefile.pre.in} for
people who do not have a source distribution of Python.

Do not distribute a make file.  People building your modules
should use \file{Makefile.pre.in} to build their own make file.

Work is being done to make building and installing Python extensions
easier for all platforms; this work in likely to supplant the current
approach at some point in the future.  For more information or to
participate in the effort, refer to
\url{http://www.python.org/sigs/distutils-sig/} on the Python Web
site.


\chapter{Building C and \Cpp{} Extensions on Windows
         \label{building-on-windows}}


This chapter briefly explains how to create a Windows extension module
for Python using Microsoft Visual \Cpp{}, and follows with more
detailed background information on how it works.  The explanatory
material is useful for both the Windows programmer learning to build
Python extensions and the \UNIX{} programming interested in producing
software which can be successfully built on both \UNIX{} and Windows.

\section{A Cookbook Approach \label{win-cookbook}}

\sectionauthor{Neil Schemenauer}{neil_schemenauer@transcanada.com}

This section provides a recipe for building a Python extension on
Windows.

Grab the binary installer from \url{http://www.python.org/} and
install Python.  The binary installer has all of the required header
files except for \file{config.h}.

Get the source distribution and extract it into a convenient location.
Copy the \file{config.h} from the \file{PC/} directory into the
\file{include/} directory created by the installer.

Create a \file{Setup} file for your extension module, as described in
Chapter \ref{building-on-unix}.

Get David Ascher's \file{compile.py} script from
\url{http://starship.skyport.net/~da/compile/}.  Run the script to
create Microsoft Visual \Cpp{} project files.

Open the DSW file in V\Cpp{} and select \strong{Build}.

If your module creates a new type, you may have trouble with this line:

\begin{verbatim}
    PyObject_HEAD_INIT(&PyType_Type)
\end{verbatim}

Change it to:

\begin{verbatim}
    PyObject_HEAD_INIT(NULL)
\end{verbatim}

and add the following to the module initialization function:

\begin{verbatim}
    MyObject_Type.ob_type = &PyType_Type;
\end{verbatim}

Refer to section 3 of the Python FAQ
(\url{http://www.python.org/doc/FAQ.html}) for details on why you must
do this.


\section{Differences Between \UNIX{} and Windows
         \label{dynamic-linking}}
\sectionauthor{Chris Phoenix}{cphoenix@best.com}


\UNIX{} and Windows use completely different paradigms for run-time
loading of code.  Before you try to build a module that can be
dynamically loaded, be aware of how your system works.

In \UNIX{}, a shared object (.so) file contains code to be used by the
program, and also the names of functions and data that it expects to
find in the program.  When the file is joined to the program, all
references to those functions and data in the file's code are changed
to point to the actual locations in the program where the functions
and data are placed in memory.  This is basically a link operation.

In Windows, a dynamic-link library (\file{.dll}) file has no dangling
references.  Instead, an access to functions or data goes through a
lookup table.  So the DLL code does not have to be fixed up at runtime
to refer to the program's memory; instead, the code already uses the
DLL's lookup table, and the lookup table is modified at runtime to
point to the functions and data.

In \UNIX{}, there is only one type of library file (\file{.a}) which
contains code from several object files (\file{.o}).  During the link
step to create a shared object file (\file{.so}), the linker may find
that it doesn't know where an identifier is defined.  The linker will
look for it in the object files in the libraries; if it finds it, it
will include all the code from that object file.

In Windows, there are two types of library, a static library and an
import library (both called \file{.lib}).  A static library is like a
\UNIX{} \file{.a} file; it contains code to be included as necessary.
An import library is basically used only to reassure the linker that a
certain identifier is legal, and will be present in the program when
the DLL is loaded.  So the linker uses the information from the
import library to build the lookup table for using identifiers that
are not included in the DLL.  When an application or a DLL is linked,
an import library may be generated, which will need to be used for all
future DLLs that depend on the symbols in the application or DLL.

Suppose you are building two dynamic-load modules, B and C, which should
share another block of code A.  On \UNIX{}, you would \emph{not} pass
\file{A.a} to the linker for \file{B.so} and \file{C.so}; that would
cause it to be included twice, so that B and C would each have their
own copy.  In Windows, building \file{A.dll} will also build
\file{A.lib}.  You \emph{do} pass \file{A.lib} to the linker for B and
C.  \file{A.lib} does not contain code; it just contains information
which will be used at runtime to access A's code.  

In Windows, using an import library is sort of like using \samp{import
spam}; it gives you access to spam's names, but does not create a
separate copy.  On \UNIX{}, linking with a library is more like
\samp{from spam import *}; it does create a separate copy.


\section{Using DLLs in Practice \label{win-dlls}}
\sectionauthor{Chris Phoenix}{cphoenix@best.com}

Windows Python is built in Microsoft Visual \Cpp{}; using other
compilers may or may not work (though Borland seems to).  The rest of
this section is MSV\Cpp{} specific.

When creating DLLs in Windows, you must pass \file{python15.lib} to
the linker.  To build two DLLs, spam and ni (which uses C functions
found in spam), you could use these commands:

\begin{verbatim}
cl /LD /I/python/include spam.c ../libs/python15.lib
cl /LD /I/python/include ni.c spam.lib ../libs/python15.lib
\end{verbatim}

The first command created three files: \file{spam.obj},
\file{spam.dll} and \file{spam.lib}.  \file{Spam.dll} does not contain
any Python functions (such as \cfunction{PyArg_ParseTuple()}), but it
does know how to find the Python code thanks to \file{python15.lib}.

The second command created \file{ni.dll} (and \file{.obj} and
\file{.lib}), which knows how to find the necessary functions from
spam, and also from the Python executable.

Not every identifier is exported to the lookup table.  If you want any
other modules (including Python) to be able to see your identifiers,
you have to say \samp{_declspec(dllexport)}, as in \samp{void
_declspec(dllexport) initspam(void)} or \samp{PyObject
_declspec(dllexport) *NiGetSpamData(void)}.

Developer Studio will throw in a lot of import libraries that you do
not really need, adding about 100K to your executable.  To get rid of
them, use the Project Settings dialog, Link tab, to specify
\emph{ignore default libraries}.  Add the correct
\file{msvcrt\var{xx}.lib} to the list of libraries.


\chapter{Embedding Python in Another Application
         \label{embedding}}

Embedding Python is similar to extending it, but not quite.  The
difference is that when you extend Python, the main program of the
application is still the Python interpreter, while if you embed
Python, the main program may have nothing to do with Python ---
instead, some parts of the application occasionally call the Python
interpreter to run some Python code.

So if you are embedding Python, you are providing your own main
program.  One of the things this main program has to do is initialize
the Python interpreter.  At the very least, you have to call the
function \cfunction{Py_Initialize()}.  There are optional calls to
pass command line arguments to Python.  Then later you can call the
interpreter from any part of the application.

There are several different ways to call the interpreter: you can pass
a string containing Python statements to
\cfunction{PyRun_SimpleString()}, or you can pass a stdio file pointer
and a file name (for identification in error messages only) to
\cfunction{PyRun_SimpleFile()}.  You can also call the lower-level
operations described in the previous chapters to construct and use
Python objects.

A simple demo of embedding Python can be found in the directory
\file{Demo/embed/} of the source distribution.


\section{Embedding Python in \Cpp{}
         \label{embeddingInCplusplus}}

It is also possible to embed Python in a \Cpp{} program; precisely how this
is done will depend on the details of the \Cpp{} system used; in general you
will need to write the main program in \Cpp{}, and use the \Cpp{} compiler
to compile and link your program.  There is no need to recompile Python
itself using \Cpp{}.

\end{document}


\end{document}
