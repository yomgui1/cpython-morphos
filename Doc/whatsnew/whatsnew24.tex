\documentclass{howto}
\usepackage{distutils}
% $Id$

% Don't write extensive text for new sections; I'll do that.  
% Feel free to add commented-out reminders of things that need
% to be covered.  --amk

\title{What's New in Python 2.4}
\release{1.02}
\author{A.M.\ Kuchling}
\authoraddress{
	\strong{Python Software Foundation}\\
	Email: \email{amk@amk.ca}
}

\begin{document}
\maketitle
\tableofcontents

This article explains the new features in Python 2.4.1, released on
March~30, 2005.

Python 2.4 is a medium-sized release.  It doesn't introduce as many
changes as the radical Python 2.2, but introduces more features than
the conservative 2.3 release.  The most significant new language
features are function decorators and generator expressions; most other
changes are to the standard library.

According to the CVS change logs, there were 481 patches applied and
502 bugs fixed between Python 2.3 and 2.4.  Both figures are likely to
be underestimates.

This article doesn't attempt to provide a complete specification of
every single new feature, but instead provides a brief introduction to
each feature.  For full details, you should refer to the documentation
for Python 2.4, such as the \citetitle[../lib/lib.html]{Python Library
Reference} and the \citetitle[../ref/ref.html]{Python Reference
Manual}.  Often you will be referred to the PEP for a particular new
feature for explanations of the implementation and design rationale.


%======================================================================
\section{PEP 218: Built-In Set Objects}

Python 2.3 introduced the \module{sets} module.  C implementations of
set data types have now been added to the Python core as two new
built-in types, \function{set(\var{iterable})} and
\function{frozenset(\var{iterable})}.  They provide high speed
operations for membership testing, for eliminating duplicates from
sequences, and for mathematical operations like unions, intersections,
differences, and symmetric differences.

\begin{verbatim}
>>> a = set('abracadabra')              # form a set from a string
>>> 'z' in a                            # fast membership testing
False
>>> a                                   # unique letters in a
set(['a', 'r', 'b', 'c', 'd'])
>>> ''.join(a)                          # convert back into a string
'arbcd'

>>> b = set('alacazam')                 # form a second set
>>> a - b                               # letters in a but not in b
set(['r', 'd', 'b'])
>>> a | b                               # letters in either a or b
set(['a', 'c', 'r', 'd', 'b', 'm', 'z', 'l'])
>>> a & b                               # letters in both a and b
set(['a', 'c'])
>>> a ^ b                               # letters in a or b but not both
set(['r', 'd', 'b', 'm', 'z', 'l'])

>>> a.add('z')                          # add a new element
>>> a.update('wxy')                     # add multiple new elements
>>> a
set(['a', 'c', 'b', 'd', 'r', 'w', 'y', 'x', 'z'])       
>>> a.remove('x')                       # take one element out
>>> a
set(['a', 'c', 'b', 'd', 'r', 'w', 'y', 'z'])       
\end{verbatim}

The \function{frozenset} type is an immutable version of \function{set}.
Since it is immutable and hashable, it may be used as a dictionary key or
as a member of another set.  

The \module{sets} module remains in the standard library, and may be
useful if you wish to subclass the \class{Set} or \class{ImmutableSet}
classes.  There are currently no plans to deprecate the module.

\begin{seealso}
\seepep{218}{Adding a Built-In Set Object Type}{Originally proposed by
Greg Wilson and ultimately implemented by Raymond Hettinger.}
\end{seealso}


%======================================================================
\section{PEP 237: Unifying Long Integers and Integers}

The lengthy transition process for this PEP, begun in Python 2.2,
takes another step forward in Python 2.4.  In 2.3, certain integer
operations that would behave differently after int/long unification
triggered \exception{FutureWarning} warnings and returned values
limited to 32 or 64 bits (depending on your platform).  In 2.4, these
expressions no longer produce a warning and instead produce a
different result that's usually a long integer.

The problematic expressions are primarily left shifts and lengthy
hexadecimal and octal constants.  For example,
\code{2 \textless{}\textless{} 32} results
in a warning in 2.3, evaluating to 0 on 32-bit platforms.  In Python
2.4, this expression now returns the correct answer, 8589934592.

\begin{seealso}
\seepep{237}{Unifying Long Integers and Integers}{Original PEP
written by Moshe Zadka and GvR.  The changes for 2.4 were implemented by 
Kalle Svensson.}
\end{seealso}


%======================================================================
\section{PEP 289: Generator Expressions}

The iterator feature introduced in Python 2.2 and the
\module{itertools} module make it easier to write programs that loop
through large data sets without having the entire data set in memory
at one time.  List comprehensions don't fit into this picture very
well because they produce a Python list object containing all of the
items.  This unavoidably pulls all of the objects into memory, which
can be a problem if your data set is very large.  When trying to write
a functionally-styled program, it would be natural to write something
like:

\begin{verbatim}
links = [link for link in get_all_links() if not link.followed]
for link in links:
    ...
\end{verbatim}

instead of 

\begin{verbatim}
for link in get_all_links():
    if link.followed:
        continue
    ...
\end{verbatim}

The first form is more concise and perhaps more readable, but if
you're dealing with a large number of link objects you'd have to write
the second form to avoid having all link objects in memory at the same
time.

Generator expressions work similarly to list comprehensions but don't
materialize the entire list; instead they create a generator that will
return elements one by one.  The above example could be written as:

\begin{verbatim}
links = (link for link in get_all_links() if not link.followed)
for link in links:
    ...
\end{verbatim}

Generator expressions always have to be written inside parentheses, as
in the above example.  The parentheses signalling a function call also
count, so if you want to create an iterator that will be immediately
passed to a function you could write:

\begin{verbatim}
print sum(obj.count for obj in list_all_objects())
\end{verbatim}

Generator expressions differ from list comprehensions in various small
ways.  Most notably, the loop variable (\var{obj} in the above
example) is not accessible outside of the generator expression.  List
comprehensions leave the variable assigned to its last value; future
versions of Python will change this, making list comprehensions match
generator expressions in this respect.

\begin{seealso}
\seepep{289}{Generator Expressions}{Proposed by Raymond Hettinger and
implemented by Jiwon Seo with early efforts steered by Hye-Shik Chang.}
\end{seealso}


%======================================================================
\section{PEP 292: Simpler String Substitutions}

Some new classes in the standard library provide an alternative
mechanism for substituting variables into strings; this style of
substitution may be better for applications where untrained
users need to edit templates.

The usual way of substituting variables by name is the \code{\%}
operator:

\begin{verbatim}
>>> '%(page)i: %(title)s' % {'page':2, 'title': 'The Best of Times'}
'2: The Best of Times'
\end{verbatim}

When writing the template string, it can be easy to forget the
\samp{i} or \samp{s} after the closing parenthesis.  This isn't a big
problem if the template is in a Python module, because you run the
code, get an ``Unsupported format character'' \exception{ValueError},
and fix the problem.  However, consider an application such as Mailman
where template strings or translations are being edited by users who
aren't aware of the Python language.  The format string's syntax is
complicated to explain to such users, and if they make a mistake, it's
difficult to provide helpful feedback to them.

PEP 292 adds a \class{Template} class to the \module{string} module
that uses \samp{\$} to indicate a substitution:

\begin{verbatim}
>>> import string
>>> t = string.Template('$page: $title')
>>> t.substitute({'page':2, 'title': 'The Best of Times'})
'2: The Best of Times'
\end{verbatim}

% $ Terminate $-mode for Emacs

If a key is missing from the dictionary, the \method{substitute} method
will raise a \exception{KeyError}.  There's also a \method{safe_substitute}
method that ignores missing keys:

\begin{verbatim}
>>> t = string.Template('$page: $title')
>>> t.safe_substitute({'page':3})
'3: $title'
\end{verbatim}

% $ Terminate math-mode for Emacs


\begin{seealso}
\seepep{292}{Simpler String Substitutions}{Written and implemented 
by Barry Warsaw.}
\end{seealso}


%======================================================================
\section{PEP 318: Decorators for Functions and Methods}

Python 2.2 extended Python's object model by adding static methods and
class methods, but it didn't extend Python's syntax to provide any new
way of defining static or class methods.  Instead, you had to write a
\keyword{def} statement in the usual way, and pass the resulting
method to a \function{staticmethod()} or \function{classmethod()}
function that would wrap up the function as a method of the new type.
Your code would look like this:

\begin{verbatim}
class C:
   def meth (cls):
       ...
   
   meth = classmethod(meth)   # Rebind name to wrapped-up class method
\end{verbatim}

If the method was very long, it would be easy to miss or forget the
\function{classmethod()} invocation after the function body.  

The intention was always to add some syntax to make such definitions
more readable, but at the time of 2.2's release a good syntax was not
obvious.  Today a good syntax \emph{still} isn't obvious but users are
asking for easier access to the feature; a new syntactic feature has
been added to meet this need.

The new feature is called ``function decorators''.  The name comes
from the idea that \function{classmethod}, \function{staticmethod},
and friends are storing additional information on a function object;
they're \emph{decorating} functions with more details.

The notation borrows from Java and uses the \character{@} character as an
indicator.  Using the new syntax, the example above would be written:

\begin{verbatim}
class C:

   @classmethod
   def meth (cls):
       ...
   
\end{verbatim}

The \code{@classmethod} is shorthand for the
\code{meth=classmethod(meth)} assignment.  More generally, if you have
the following:

\begin{verbatim}
@A
@B
@C
def f ():
    ...
\end{verbatim}

It's equivalent to the following pre-decorator code:

\begin{verbatim}
def f(): ...
f = A(B(C(f)))
\end{verbatim}

Decorators must come on the line before a function definition, one decorator
per line, and can't be on the same line as the def statement, meaning that
\code{@A def f(): ...} is illegal.  You can only decorate function
definitions, either at the module level or inside a class; you can't
decorate class definitions.

A decorator is just a function that takes the function to be decorated as an
argument and returns either the same function or some new object.  The
return value of the decorator need not be callable (though it typically is),
unless further decorators will be applied to the result.  It's easy to write
your own decorators.  The following simple example just sets an attribute on
the function object:

\begin{verbatim}
>>> def deco(func):
...    func.attr = 'decorated'
...    return func
...
>>> @deco
... def f(): pass
...
>>> f
<function f at 0x402ef0d4>
>>> f.attr
'decorated'
>>>
\end{verbatim}

As a slightly more realistic example, the following decorator checks
that the supplied argument is an integer:

\begin{verbatim}
def require_int (func):
    def wrapper (arg):
        assert isinstance(arg, int)
        return func(arg)

    return wrapper

@require_int
def p1 (arg):
    print arg

@require_int
def p2(arg):
    print arg*2
\end{verbatim}

An example in \pep{318} contains a fancier version of this idea that
lets you both specify the required type and check the returned type.

Decorator functions can take arguments.  If arguments are supplied,
your decorator function is called with only those arguments and must
return a new decorator function; this function must take a single
function and return a function, as previously described.  In other
words, \code{@A @B @C(args)} becomes:

\begin{verbatim}
def f(): ...
_deco = C(args)
f = A(B(_deco(f)))
\end{verbatim}

Getting this right can be slightly brain-bending, but it's not too
difficult.

A small related change makes the \member{func_name} attribute of
functions writable.  This attribute is used to display function names
in tracebacks, so decorators should change the name of any new
function that's constructed and returned.

\begin{seealso}
\seepep{318}{Decorators for Functions, Methods and Classes}{Written 
by Kevin D. Smith, Jim Jewett, and Skip Montanaro.  Several people
wrote patches implementing function decorators, but the one that was
actually checked in was patch \#979728, written by Mark Russell.}

\seeurl{http://www.python.org/moin/PythonDecoratorLibrary}
{This Wiki page contains several examples of decorators.}

\end{seealso}


%======================================================================
\section{PEP 322: Reverse Iteration}

A new built-in function, \function{reversed(\var{seq})}, takes a sequence
and returns an iterator that loops over the elements of the sequence 
in reverse order.  

\begin{verbatim}
>>> for i in reversed(xrange(1,4)):
...    print i
... 
3
2
1
\end{verbatim}

Compared to extended slicing, such as \code{range(1,4)[::-1]},
\function{reversed()} is easier to read, runs faster, and uses
substantially less memory.

Note that \function{reversed()} only accepts sequences, not arbitrary
iterators.  If you want to reverse an iterator, first convert it to 
a list with \function{list()}.

\begin{verbatim}
>>> input = open('/etc/passwd', 'r')
>>> for line in reversed(list(input)):
...   print line
... 
root:*:0:0:System Administrator:/var/root:/bin/tcsh
  ...
\end{verbatim}

\begin{seealso}
\seepep{322}{Reverse Iteration}{Written and implemented by Raymond Hettinger.}

\end{seealso}


%======================================================================
\section{PEP 324: New subprocess Module}

The standard library provides a number of ways to execute a
subprocess, offering different features and different levels of
complexity.  \function{os.system(\var{command})} is easy to use, but
slow (it runs a shell process which executes the command) and
dangerous (you have to be careful about escaping the shell's
metacharacters).  The \module{popen2} module offers classes that can
capture standard output and standard error from the subprocess, but
the naming is confusing.  The \module{subprocess} module cleans 
this up, providing a unified interface that offers all the features
you might need.

Instead of \module{popen2}'s collection of classes,
\module{subprocess} contains a single class called \class{Popen} 
whose constructor supports a number of different keyword arguments.

\begin{verbatim}
class Popen(args, bufsize=0, executable=None,
	    stdin=None, stdout=None, stderr=None,
	    preexec_fn=None, close_fds=False, shell=False,
	    cwd=None, env=None, universal_newlines=False,
	    startupinfo=None, creationflags=0):
\end{verbatim}

\var{args} is commonly a sequence of strings that will be the
arguments to the program executed as the subprocess.  (If the
\var{shell} argument is true, \var{args} can be a string which will
then be passed on to the shell for interpretation, just as
\function{os.system()} does.)

\var{stdin}, \var{stdout}, and \var{stderr} specify what the
subprocess's input, output, and error streams will be.  You can
provide a file object or a file descriptor, or you can use the
constant \code{subprocess.PIPE} to create a pipe between the
subprocess and the parent.

The constructor has a number of handy options:

\begin{itemize}
  \item \var{close_fds} requests that all file descriptors be closed
  before running the subprocess.

  \item \var{cwd} specifies the working directory in which the
  subprocess will be executed (defaulting to whatever the parent's
  working directory is).

  \item \var{env} is a dictionary specifying environment variables.

  \item \var{preexec_fn} is a function that gets called before the
  child is started.

  \item \var{universal_newlines} opens the child's input and output
  using Python's universal newline feature.

\end{itemize}

Once you've created the \class{Popen} instance, 
you can call its \method{wait()} method to pause until the subprocess
has exited, \method{poll()} to check if it's exited without pausing, 
or \method{communicate(\var{data})} to send the string \var{data} to
the subprocess's standard input.   \method{communicate(\var{data})} 
then reads any data that the subprocess has sent to its standard output 
or standard error, returning a tuple \code{(\var{stdout_data},
\var{stderr_data})}.

\function{call()} is a shortcut that passes its arguments along to the
\class{Popen} constructor, waits for the command to complete, and
returns the status code of the subprocess.  It can serve as a safer
analog to \function{os.system()}:

\begin{verbatim}
sts = subprocess.call(['dpkg', '-i', '/tmp/new-package.deb'])
if sts == 0:
    # Success
    ...
else:
    # dpkg returned an error
    ...
\end{verbatim}

The command is invoked without use of the shell.  If you really do want to 
use the shell, you can add \code{shell=True} as a keyword argument and provide
a string instead of a sequence:

\begin{verbatim}
sts = subprocess.call('dpkg -i /tmp/new-package.deb', shell=True)
\end{verbatim}

The PEP takes various examples of shell and Python code and shows how
they'd be translated into Python code that uses \module{subprocess}. 
Reading this section of the PEP is highly recommended.

\begin{seealso}
\seepep{324}{subprocess - New process module}{Written and implemented by Peter {\AA}strand, with assistance from Fredrik Lundh and others.}
\end{seealso}


%======================================================================
\section{PEP 327: Decimal Data Type}

Python has always supported floating-point (FP) numbers, based on the
underlying C \ctype{double} type, as a data type.  However, while most
programming languages provide a floating-point type, many people (even
programmers) are unaware that floating-point numbers don't represent
certain decimal fractions accurately.  The new \class{Decimal} type
can represent these fractions accurately, up to a user-specified
precision limit.


\subsection{Why is Decimal needed?}

The limitations arise from the representation used for floating-point numbers.
FP numbers are made up of three components:

\begin{itemize}
\item The sign, which is positive or negative.
\item The mantissa, which is a single-digit binary number  
followed by a fractional part.  For example, \code{1.01} in base-2 notation
is \code{1 + 0/2 + 1/4}, or 1.25 in decimal notation.
\item The exponent, which tells where the decimal point is located in the number represented.  
\end{itemize}

For example, the number 1.25 has positive sign, a mantissa value of
1.01 (in binary), and an exponent of 0 (the decimal point doesn't need
to be shifted).  The number 5 has the same sign and mantissa, but the
exponent is 2 because the mantissa is multiplied by 4 (2 to the power
of the exponent 2); 1.25 * 4 equals 5.

Modern systems usually provide floating-point support that conforms to
a standard called IEEE 754.  C's \ctype{double} type is usually
implemented as a 64-bit IEEE 754 number, which uses 52 bits of space
for the mantissa.  This means that numbers can only be specified to 52
bits of precision.  If you're trying to represent numbers whose
expansion repeats endlessly, the expansion is cut off after 52 bits.
Unfortunately, most software needs to produce output in base 10, and
common fractions in base 10 are often repeating decimals in binary.
For example, 1.1 decimal is binary \code{1.0001100110011 ...}; .1 =
1/16 + 1/32 + 1/256 plus an infinite number of additional terms.  IEEE
754 has to chop off that infinitely repeated decimal after 52 digits,
so the representation is slightly inaccurate.

Sometimes you can see this inaccuracy when the number is printed:
\begin{verbatim}
>>> 1.1
1.1000000000000001
\end{verbatim}

The inaccuracy isn't always visible when you print the number because
the FP-to-decimal-string conversion is provided by the C library, and
most C libraries try to produce sensible output.  Even if it's not
displayed, however, the inaccuracy is still there and subsequent
operations can magnify the error.

For many applications this doesn't matter.  If I'm plotting points and
displaying them on my monitor, the difference between 1.1 and
1.1000000000000001 is too small to be visible.  Reports often limit
output to a certain number of decimal places, and if you round the
number to two or three or even eight decimal places, the error is
never apparent.  However, for applications where it does matter, 
it's a lot of work to implement your own custom arithmetic routines.

Hence, the \class{Decimal} type was created.

\subsection{The \class{Decimal} type}

A new module, \module{decimal}, was added to Python's standard
library.  It contains two classes, \class{Decimal} and
\class{Context}.  \class{Decimal} instances represent numbers, and
\class{Context} instances are used to wrap up various settings such as
the precision and default rounding mode.

\class{Decimal} instances are immutable, like regular Python integers
and FP numbers; once it's been created, you can't change the value an
instance represents.  \class{Decimal} instances can be created from
integers or strings:

\begin{verbatim}
>>> import decimal
>>> decimal.Decimal(1972)
Decimal("1972")
>>> decimal.Decimal("1.1")
Decimal("1.1")
\end{verbatim}

You can also provide tuples containing the sign, the mantissa represented 
as a tuple of decimal digits, and the exponent:

\begin{verbatim}
>>> decimal.Decimal((1, (1, 4, 7, 5), -2))
Decimal("-14.75")
\end{verbatim}

Cautionary note: the sign bit is a Boolean value, so 0 is positive and
1 is negative. 

Converting from floating-point numbers poses a bit of a problem:
should the FP number representing 1.1 turn into the decimal number for
exactly 1.1, or for 1.1 plus whatever inaccuracies are introduced?
The decision was to dodge the issue and leave such a conversion out of
the API.  Instead, you should convert the floating-point number into a
string using the desired precision and pass the string to the
\class{Decimal} constructor:

\begin{verbatim}
>>> f = 1.1
>>> decimal.Decimal(str(f))
Decimal("1.1")
>>> decimal.Decimal('%.12f' % f)
Decimal("1.100000000000")
\end{verbatim}

Once you have \class{Decimal} instances, you can perform the usual
mathematical operations on them.  One limitation: exponentiation
requires an integer exponent:

\begin{verbatim}
>>> a = decimal.Decimal('35.72')
>>> b = decimal.Decimal('1.73')
>>> a+b
Decimal("37.45")
>>> a-b
Decimal("33.99")
>>> a*b
Decimal("61.7956")
>>> a/b
Decimal("20.64739884393063583815028902")
>>> a ** 2
Decimal("1275.9184")
>>> a**b
Traceback (most recent call last):
  ...
decimal.InvalidOperation: x ** (non-integer)
\end{verbatim}

You can combine \class{Decimal} instances with integers, but not with
floating-point numbers:

\begin{verbatim}
>>> a + 4
Decimal("39.72")
>>> a + 4.5
Traceback (most recent call last):
  ...
TypeError: You can interact Decimal only with int, long or Decimal data types.
>>>
\end{verbatim}

\class{Decimal} numbers can be used with the \module{math} and
\module{cmath} modules, but note that they'll be immediately converted to 
floating-point numbers before the operation is performed, resulting in
a possible loss of precision and accuracy.  You'll also get back a
regular floating-point number and not a \class{Decimal}.  

\begin{verbatim}
>>> import math, cmath
>>> d = decimal.Decimal('123456789012.345')
>>> math.sqrt(d)
351364.18288201344
>>> cmath.sqrt(-d)
351364.18288201344j
\end{verbatim}

\class{Decimal} instances have a \method{sqrt()} method that
returns a \class{Decimal}, but if you need other things such as
trigonometric functions you'll have to implement them.

\begin{verbatim}
>>> d.sqrt()
Decimal("351364.1828820134592177245001")
\end{verbatim}


\subsection{The \class{Context} type}

Instances of the \class{Context} class encapsulate several settings for 
decimal operations:

\begin{itemize}
 \item \member{prec} is the precision, the number of decimal places.
 \item \member{rounding} specifies the rounding mode.  The \module{decimal}
       module has constants for the various possibilities:
       \constant{ROUND_DOWN}, \constant{ROUND_CEILING}, 
       \constant{ROUND_HALF_EVEN}, and various others.
 \item \member{traps} is a dictionary specifying what happens on
encountering certain error conditions: either  an exception is raised or 
a value is returned.  Some examples of error conditions are
division by zero, loss of precision, and overflow.
\end{itemize}

There's a thread-local default context available by calling
\function{getcontext()}; you can change the properties of this context
to alter the default precision, rounding, or trap handling.  The
following example shows the effect of changing the precision of the default
context:

\begin{verbatim}
>>> decimal.getcontext().prec
28
>>> decimal.Decimal(1) / decimal.Decimal(7)
Decimal("0.1428571428571428571428571429")
>>> decimal.getcontext().prec = 9 
>>> decimal.Decimal(1) / decimal.Decimal(7)
Decimal("0.142857143")
\end{verbatim}

The default action for error conditions is selectable; the module can
either return a special value such as infinity or not-a-number, or
exceptions can be raised:

\begin{verbatim}
>>> decimal.Decimal(1) / decimal.Decimal(0)
Traceback (most recent call last):
  ...
decimal.DivisionByZero: x / 0
>>> decimal.getcontext().traps[decimal.DivisionByZero] = False
>>> decimal.Decimal(1) / decimal.Decimal(0)
Decimal("Infinity")
>>> 
\end{verbatim}

The \class{Context} instance also has various methods for formatting 
numbers such as \method{to_eng_string()} and \method{to_sci_string()}.

For more information, see the documentation for the \module{decimal}
module, which includes a quick-start tutorial and a reference.

\begin{seealso}
\seepep{327}{Decimal Data Type}{Written by Facundo Batista and implemented
  by Facundo Batista, Eric Price, Raymond Hettinger, Aahz, and Tim Peters.}

\seeurl{http://research.microsoft.com/\textasciitilde hollasch/cgindex/coding/ieeefloat.html}
{A more detailed overview of the IEEE-754 representation.}

\seeurl{http://www.lahey.com/float.htm}
{The article uses Fortran code to illustrate many of the problems
that floating-point inaccuracy can cause.}

\seeurl{http://www2.hursley.ibm.com/decimal/}
{A description of a decimal-based representation.  This representation
is being proposed as a standard, and underlies the new Python decimal
type.  Much of this material was written by Mike Cowlishaw, designer of the
Rexx language.}

\end{seealso}      


%======================================================================
\section{PEP 328: Multi-line Imports}

One language change is a small syntactic tweak aimed at making it
easier to import many names from a module.  In a
\code{from \var{module} import \var{names}} statement, 
\var{names} is a sequence of names separated by commas.  If the sequence is 
very long, you can either write multiple imports from the same module,
or you can use backslashes to escape the line endings like this:

\begin{verbatim}
from SimpleXMLRPCServer import SimpleXMLRPCServer,\
            SimpleXMLRPCRequestHandler,\
            CGIXMLRPCRequestHandler,\
            resolve_dotted_attribute
\end{verbatim}

The syntactic change in Python 2.4 simply allows putting the names
within parentheses.  Python ignores newlines within a parenthesized
expression, so the backslashes are no longer needed:

\begin{verbatim}
from SimpleXMLRPCServer import (SimpleXMLRPCServer,
                                SimpleXMLRPCRequestHandler,
                                CGIXMLRPCRequestHandler,
                                resolve_dotted_attribute)
\end{verbatim}

The PEP also proposes that all \keyword{import} statements be absolute
imports, with a leading \samp{.} character to indicate a relative
import.  This part of the PEP was not implemented for Python 2.4,
but was completed for Python 2.5.

\begin{seealso}
\seepep{328}{Imports: Multi-Line and Absolute/Relative}
            {Written by Aahz.  Multi-line imports were implemented by
             Dima Dorfman.}
\end{seealso}


%======================================================================
\section{PEP 331: Locale-Independent Float/String Conversions}

The \module{locale} modules lets Python software select various
conversions and display conventions that are localized to a particular
country or language.  However, the module was careful to not change
the numeric locale because various functions in Python's
implementation required that the numeric locale remain set to the
\code{'C'} locale.  Often this was because the code was using the C library's
\cfunction{atof()} function.  

Not setting the numeric locale caused trouble for extensions that used
third-party C libraries, however, because they wouldn't have the
correct locale set.  The motivating example was GTK+, whose user
interface widgets weren't displaying numbers in the current locale.

The solution described in the PEP is to add three new functions to the
Python API that perform ASCII-only conversions, ignoring the locale
setting:

\begin{itemize}
 \item \cfunction{PyOS_ascii_strtod(\var{str}, \var{ptr})} 
and \cfunction{PyOS_ascii_atof(\var{str}, \var{ptr})} 
both convert a string to a C \ctype{double}.
 \item \cfunction{PyOS_ascii_formatd(\var{buffer}, \var{buf_len}, \var{format}, \var{d})} converts a \ctype{double} to an ASCII string.
\end{itemize}

The code for these functions came from the GLib library
(\url{http://developer.gnome.org/arch/gtk/glib.html}), whose
developers kindly relicensed the relevant functions and donated them
to the Python Software Foundation.  The \module{locale} module 
can now change the numeric locale, letting extensions such as GTK+ 
produce the correct results.

\begin{seealso}
\seepep{331}{Locale-Independent Float/String Conversions}
{Written by Christian R. Reis, and implemented by Gustavo Carneiro.}
\end{seealso}      

%======================================================================
\section{Other Language Changes}

Here are all of the changes that Python 2.4 makes to the core Python
language.

\begin{itemize}

\item Decorators for functions and methods were added (\pep{318}).

\item Built-in \function{set} and \function{frozenset} types were 
added (\pep{218}).  Other new built-ins include the \function{reversed(\var{seq})} function (\pep{322}).

\item Generator expressions were added (\pep{289}).

\item Certain numeric expressions no longer return values restricted to 32 or 64 bits (\pep{237}).

\item You can now put parentheses around the list of names in a
\code{from \var{module} import \var{names}} statement (\pep{328}).

\item The \method{dict.update()} method now accepts the same
argument forms as the \class{dict} constructor.  This includes any
mapping, any iterable of key/value pairs, and keyword arguments.
(Contributed by Raymond Hettinger.)

\item The string methods \method{ljust()}, \method{rjust()}, and
\method{center()} now take an optional argument for specifying a
fill character other than a space.
(Contributed by Raymond Hettinger.)

\item Strings also gained an \method{rsplit()} method that
works like the \method{split()} method but splits from the end of
the string.  
(Contributed by Sean Reifschneider.)

\begin{verbatim}
>>> 'www.python.org'.split('.', 1)
['www', 'python.org']
'www.python.org'.rsplit('.', 1)
['www.python', 'org']        
\end{verbatim}      

\item Three keyword parameters, \var{cmp}, \var{key}, and
\var{reverse}, were added to the \method{sort()} method of lists.
These parameters make some common usages of \method{sort()} simpler.
All of these parameters are optional.

For the \var{cmp} parameter, the value should be a comparison function
that takes two parameters and returns -1, 0, or +1 depending on how
the parameters compare.  This function will then be used to sort the
list.  Previously this was the only parameter that could be provided
to \method{sort()}.

\var{key} should be a single-parameter function that takes a list
element and returns a comparison key for the element.  The list is
then sorted using the comparison keys.  The following example sorts a
list case-insensitively:

\begin{verbatim}
>>> L = ['A', 'b', 'c', 'D']
>>> L.sort()                 # Case-sensitive sort
>>> L
['A', 'D', 'b', 'c']
>>> # Using 'key' parameter to sort list
>>> L.sort(key=lambda x: x.lower())
>>> L
['A', 'b', 'c', 'D']
>>> # Old-fashioned way
>>> L.sort(cmp=lambda x,y: cmp(x.lower(), y.lower()))
>>> L
['A', 'b', 'c', 'D']
\end{verbatim}

The last example, which uses the \var{cmp} parameter, is the old way
to perform a case-insensitive sort.  It works but is slower than using
a \var{key} parameter.  Using \var{key} calls \method{lower()} method
once for each element in the list while using \var{cmp} will call it
twice for each comparison, so using \var{key} saves on invocations of
the \method{lower()} method.

For simple key functions and comparison functions, it is often
possible to avoid a \keyword{lambda} expression by using an unbound
method instead.  For example, the above case-insensitive sort is best
written as:

\begin{verbatim}
>>> L.sort(key=str.lower)
>>> L
['A', 'b', 'c', 'D']
\end{verbatim}       

Finally, the \var{reverse} parameter takes a Boolean value.  If the
value is true, the list will be sorted into reverse order.
Instead of \code{L.sort() ; L.reverse()}, you can now write
\code{L.sort(reverse=True)}.

The results of sorting are now guaranteed to be stable.  This means
that two entries with equal keys will be returned in the same order as
they were input.  For example, you can sort a list of people by name,
and then sort the list by age, resulting in a list sorted by age where
people with the same age are in name-sorted order.

(All changes to \method{sort()} contributed by Raymond Hettinger.)

\item There is a new built-in function
\function{sorted(\var{iterable})} that works like the in-place
\method{list.sort()} method but can be used in
expressions.  The differences are:
  \begin{itemize}
  \item the input may be any iterable;
  \item a newly formed copy is sorted, leaving the original intact; and
  \item the expression returns the new sorted copy
  \end{itemize}

\begin{verbatim}
>>> L = [9,7,8,3,2,4,1,6,5]
>>> [10+i for i in sorted(L)]       # usable in a list comprehension
[11, 12, 13, 14, 15, 16, 17, 18, 19]
>>> L                               # original is left unchanged
[9,7,8,3,2,4,1,6,5]
>>> sorted('Monty Python')          # any iterable may be an input
[' ', 'M', 'P', 'h', 'n', 'n', 'o', 'o', 't', 't', 'y', 'y']

>>> # List the contents of a dict sorted by key values
>>> colormap = dict(red=1, blue=2, green=3, black=4, yellow=5)
>>> for k, v in sorted(colormap.iteritems()):
...     print k, v
...
black 4
blue 2
green 3
red 1
yellow 5
\end{verbatim}

(Contributed by Raymond Hettinger.)

\item Integer operations will no longer trigger an \exception{OverflowWarning}.
The \exception{OverflowWarning} warning will disappear in Python 2.5.

\item The interpreter gained a new switch, \programopt{-m}, that
takes a name, searches for the corresponding  module on \code{sys.path},
and runs the module as a script.  For example, 
you can now run the Python profiler with \code{python -m profile}.
(Contributed by Nick Coghlan.)

\item The \function{eval(\var{expr}, \var{globals}, \var{locals})}
and \function{execfile(\var{filename}, \var{globals}, \var{locals})}
functions and the \keyword{exec} statement now accept any mapping type
for the \var{locals} parameter.  Previously this had to be a regular
Python dictionary.  (Contributed by Raymond Hettinger.)

\item The \function{zip()} built-in function and \function{itertools.izip()}
  now return an empty list if called with no arguments.
  Previously they raised a \exception{TypeError}
  exception.  This makes them more
  suitable for use with variable length argument lists:

\begin{verbatim}
>>> def transpose(array):
...    return zip(*array)
...
>>> transpose([(1,2,3), (4,5,6)])
[(1, 4), (2, 5), (3, 6)]
>>> transpose([])
[]
\end{verbatim}
(Contributed by Raymond Hettinger.)
    
\item Encountering a failure while importing a module no longer leaves
a partially-initialized module object in \code{sys.modules}.  The
incomplete module object left behind would fool further imports of the
same module into succeeding, leading to confusing errors.  
(Fixed by Tim Peters.)

\item \constant{None} is now a constant; code that binds a new value to 
the name \samp{None} is now a syntax error.
(Contributed by Raymond Hettinger.)       

\end{itemize}


%======================================================================
\subsection{Optimizations}

\begin{itemize}

\item The inner loops for list and tuple slicing
 were optimized and now run about one-third faster.  The inner loops
 for dictionaries were also optimized, resulting in performance boosts for
 \method{keys()}, \method{values()}, \method{items()},
 \method{iterkeys()}, \method{itervalues()}, and \method{iteritems()}.
 (Contributed by Raymond Hettinger.)

\item The machinery for growing and shrinking lists was optimized for
 speed and for space efficiency.  Appending and popping from lists now
 runs faster due to more efficient code paths and less frequent use of
 the underlying system \cfunction{realloc()}.  List comprehensions
 also benefit.   \method{list.extend()} was also optimized and no
 longer converts its argument into a temporary list before extending
 the base list.  (Contributed by Raymond Hettinger.)

\item \function{list()}, \function{tuple()}, \function{map()},
  \function{filter()}, and \function{zip()} now run several times
  faster with non-sequence arguments that supply a \method{__len__()}
  method.  (Contributed by Raymond Hettinger.)

\item The methods \method{list.__getitem__()},
  \method{dict.__getitem__()}, and \method{dict.__contains__()} are
  are now implemented as \class{method_descriptor} objects rather
  than \class{wrapper_descriptor} objects.  This form of 
  access doubles their performance and makes them more suitable for
  use as arguments to functionals:
  \samp{map(mydict.__getitem__, keylist)}.
  (Contributed by Raymond Hettinger.)

\item Added a new opcode, \code{LIST_APPEND}, that simplifies
  the generated bytecode for list comprehensions and speeds them up
  by about a third.  (Contributed by Raymond Hettinger.)

\item The peephole bytecode optimizer has been improved to 
produce shorter, faster bytecode; remarkably, the resulting bytecode is 
more readable.  (Enhanced by Raymond Hettinger.)

\item String concatenations in statements of the form \code{s = s +
"abc"} and \code{s += "abc"} are now performed more efficiently in
certain circumstances.  This optimization won't be present in other
Python implementations such as Jython, so you shouldn't rely on it;
using the \method{join()} method of strings is still recommended when
you want to efficiently glue a large number of strings together.
(Contributed by Armin Rigo.)       

\end{itemize}

% pystone is almost useless for comparing different versions of Python;
% instead, it excels at predicting relative Python performance on
% different machines.
% So, this section would be more informative if it used other tools
% such as pybench and parrotbench.  For a more application oriented
% benchmark, try comparing the timings of test_decimal.py under 2.3
% and 2.4.
       
The net result of the 2.4 optimizations is that Python 2.4 runs the
pystone benchmark around 5\% faster than Python 2.3 and 35\% faster
than Python 2.2.  (pystone is not a particularly good benchmark, but
it's the most commonly used measurement of Python's performance.  Your
own applications may show greater or smaller benefits from Python~2.4.)


%======================================================================
\section{New, Improved, and Deprecated Modules}

As usual, Python's standard library received a number of enhancements and
bug fixes.  Here's a partial list of the most notable changes, sorted
alphabetically by module name. Consult the
\file{Misc/NEWS} file in the source tree for a more
complete list of changes, or look through the CVS logs for all the
details.

\begin{itemize}

\item The \module{asyncore} module's \function{loop()} function now
   has a \var{count} parameter that lets you perform a limited number
   of passes through the polling loop.  The default is still to loop
   forever.

\item The \module{base64} module now has more complete RFC 3548 support
  for Base64, Base32, and Base16 encoding and decoding, including
  optional case folding and optional alternative alphabets.
  (Contributed by Barry Warsaw.)

\item The \module{bisect} module now has an underlying C implementation
   for improved performance.
   (Contributed by Dmitry Vasiliev.)

\item The CJKCodecs collections of East Asian codecs, maintained
by Hye-Shik Chang, was integrated into 2.4.  
The new encodings are:

\begin{itemize}
 \item Chinese (PRC): gb2312, gbk, gb18030, big5hkscs, hz
 \item Chinese (ROC): big5, cp950
 \item Japanese: cp932, euc-jis-2004, euc-jp,
euc-jisx0213, iso-2022-jp, iso-2022-jp-1, iso-2022-jp-2,
 iso-2022-jp-3, iso-2022-jp-ext, iso-2022-jp-2004,
 shift-jis, shift-jisx0213, shift-jis-2004
 \item Korean: cp949, euc-kr, johab, iso-2022-kr
\end{itemize} 

\item Some other new encodings were added: HP Roman8, 
ISO_8859-11, ISO_8859-16, PCTP-154, and TIS-620.

\item The UTF-8 and UTF-16 codecs now cope better with receiving partial input.
Previously the \class{StreamReader} class would try to read more data,
making it impossible to resume decoding from the stream.  The
\method{read()} method will now return as much data as it can and future
calls will resume decoding where previous ones left off. 
(Implemented by Walter D\"orwald.)

\item There is a new \module{collections} module for 
   various specialized collection datatypes.  
   Currently it contains just one type, \class{deque}, 
   a double-ended queue that supports efficiently adding and removing
   elements from either end:

\begin{verbatim}
>>> from collections import deque
>>> d = deque('ghi')        # make a new deque with three items
>>> d.append('j')           # add a new entry to the right side
>>> d.appendleft('f')       # add a new entry to the left side
>>> d                       # show the representation of the deque
deque(['f', 'g', 'h', 'i', 'j'])
>>> d.pop()                 # return and remove the rightmost item
'j'
>>> d.popleft()             # return and remove the leftmost item
'f'
>>> list(d)                 # list the contents of the deque
['g', 'h', 'i']
>>> 'h' in d                # search the deque
True  
\end{verbatim}

Several modules, such as the \module{Queue} and \module{threading}
modules, now take advantage of \class{collections.deque} for improved
performance.  (Contributed by Raymond Hettinger.)

\item The \module{ConfigParser} classes have been enhanced slightly.
   The \method{read()} method now returns a list of the files that
   were successfully parsed, and the \method{set()} method raises
   \exception{TypeError} if passed a \var{value} argument that isn't a
   string.   (Contributed by John Belmonte and David Goodger.)

\item The \module{curses} module now supports the ncurses extension 
   \function{use_default_colors()}.  On platforms where the terminal
   supports transparency, this makes it possible to use a transparent
   background.  (Contributed by J\"org Lehmann.)

\item The \module{difflib} module now includes an \class{HtmlDiff} class
that creates an HTML table showing a side by side comparison
of two versions of a text.   (Contributed by Dan Gass.)

\item The \module{email} package was updated to version 3.0, 
which dropped various deprecated APIs and removes support for Python
versions earlier than 2.3.  The 3.0 version of the package uses a new
incremental parser for MIME messages, available in the
\module{email.FeedParser} module.  The new parser doesn't require
reading the entire message into memory, and doesn't throw exceptions
if a message is malformed; instead it records any problems in the 
\member{defect} attribute of the message.  (Developed by Anthony
Baxter, Barry Warsaw, Thomas Wouters, and others.)

\item The \module{heapq} module has been converted to C.  The resulting
   tenfold improvement in speed makes the module suitable for handling
   high volumes of data.  In addition, the module has two new functions
   \function{nlargest()} and \function{nsmallest()} that use heaps to
   find the N largest or smallest values in a dataset without the
   expense of a full sort.  (Contributed by Raymond Hettinger.)

\item The \module{httplib} module now contains constants for HTTP
status codes defined in various HTTP-related RFC documents.  Constants
have names such as \constant{OK}, \constant{CREATED},
\constant{CONTINUE}, and \constant{MOVED_PERMANENTLY}; use pydoc to
get a full list.  (Contributed by Andrew Eland.)

\item The \module{imaplib} module now supports IMAP's THREAD command
(contributed by Yves Dionne) and new \method{deleteacl()} and
\method{myrights()} methods (contributed by Arnaud Mazin).

\item The \module{itertools} module gained a
  \function{groupby(\var{iterable}\optional{, \var{func}})} function.
  \var{iterable} is something that can be iterated over to return a
  stream of elements, and the optional \var{func} parameter is a
  function that takes an element and returns a key value; if omitted,
  the key is simply the element itself.  \function{groupby()} then
  groups the elements into subsequences which have matching values of
  the key, and returns a series of 2-tuples containing the key value
  and an iterator over the subsequence.
 
Here's an example to make this clearer.  The \var{key} function simply
returns whether a number is even or odd, so the result of
\function{groupby()} is to return consecutive runs of odd or even
numbers.

\begin{verbatim}
>>> import itertools
>>> L = [2, 4, 6, 7, 8, 9, 11, 12, 14]
>>> for key_val, it in itertools.groupby(L, lambda x: x % 2):
...    print key_val, list(it)
... 
0 [2, 4, 6]
1 [7]
0 [8]
1 [9, 11]
0 [12, 14]
>>> 
\end{verbatim}

\function{groupby()} is typically used with sorted input.  The logic
for \function{groupby()} is similar to the \UNIX{} \code{uniq} filter
which makes it handy for eliminating, counting, or identifying
duplicate elements:

\begin{verbatim}
>>> word = 'abracadabra'
>>> letters = sorted(word)   # Turn string into a sorted list of letters
>>> letters 
['a', 'a', 'a', 'a', 'a', 'b', 'b', 'c', 'd', 'r', 'r']
>>> for k, g in itertools.groupby(letters):
...    print k, list(g)
... 
a ['a', 'a', 'a', 'a', 'a']
b ['b', 'b']
c ['c']
d ['d']
r ['r', 'r']
>>> # List unique letters
>>> [k for k, g in groupby(letters)]                     
['a', 'b', 'c', 'd', 'r']
>>> # Count letter occurrences
>>> [(k, len(list(g))) for k, g in groupby(letters)]     
[('a', 5), ('b', 2), ('c', 1), ('d', 1), ('r', 2)]
\end{verbatim}

(Contributed by Hye-Shik Chang.)

\item \module{itertools} also gained a function named
\function{tee(\var{iterator}, \var{N})} that returns \var{N} independent
iterators that replicate \var{iterator}.  If \var{N} is omitted, the
default is 2.

\begin{verbatim}
>>> L = [1,2,3]
>>> i1, i2 = itertools.tee(L)
>>> i1,i2
(<itertools.tee object at 0x402c2080>, <itertools.tee object at 0x402c2090>)
>>> list(i1)               # Run the first iterator to exhaustion
[1, 2, 3]
>>> list(i2)               # Run the second iterator to exhaustion
[1, 2, 3]
>\end{verbatim}

Note that \function{tee()} has to keep copies of the values returned 
by the iterator; in the worst case, it may need to keep all of them.  
This should therefore be used carefully if the leading iterator
can run far ahead of the trailing iterator in a long stream of inputs.
If the separation is large, then you might as well use 
\function{list()} instead.  When the iterators track closely with one
another, \function{tee()} is ideal.  Possible applications include
bookmarking, windowing, or lookahead iterators.
(Contributed by Raymond Hettinger.)       

\item  A number of functions were added to the \module{locale} 
module, such as \function{bind_textdomain_codeset()} to specify a
particular encoding and a family of \function{l*gettext()} functions
that return messages in the chosen encoding.
(Contributed by Gustavo Niemeyer.)

\item Some keyword arguments were added to the \module{logging}
package's \function{basicConfig} function to simplify log
configuration.  The default behavior is to log messages to standard
error, but various keyword arguments can be specified to log to a
particular file, change the logging format, or set the logging level.
For example:

\begin{verbatim}
import logging
logging.basicConfig(filename='/var/log/application.log',
    level=0,  # Log all messages
    format='%(levelname):%(process):%(thread):%(message)')	            
\end{verbatim}

Other additions to the \module{logging} package include a
\method{log(\var{level}, \var{msg})} convenience method, as well as a
\class{TimedRotatingFileHandler} class that rotates its log files at a
timed interval.  The module already had \class{RotatingFileHandler},
which rotated logs once the file exceeded a certain size.  Both
classes derive from a new \class{BaseRotatingHandler} class that can
be used to implement other rotating handlers.

(Changes implemented by Vinay Sajip.)

\item The \module{marshal} module now shares interned strings on unpacking a 
data structure.  This may shrink the size of certain pickle strings,
but the primary effect is to make \file{.pyc} files significantly smaller.
(Contributed by Martin von~L\"owis.)

\item The \module{nntplib} module's \class{NNTP} class gained
\method{description()} and \method{descriptions()} methods to retrieve 
newsgroup descriptions for a single group or for a range of groups.
(Contributed by J\"urgen A. Erhard.)

\item Two new functions were added to the \module{operator} module, 
\function{attrgetter(\var{attr})} and \function{itemgetter(\var{index})}.
Both functions return callables that take a single argument and return
the corresponding attribute or item; these callables make excellent
data extractors when used with \function{map()} or
\function{sorted()}.  For example:

\begin{verbatim}
>>> L = [('c', 2), ('d', 1), ('a', 4), ('b', 3)]
>>> map(operator.itemgetter(0), L)
['c', 'd', 'a', 'b']
>>> map(operator.itemgetter(1), L)
[2, 1, 4, 3]
>>> sorted(L, key=operator.itemgetter(1)) # Sort list by second tuple item
[('d', 1), ('c', 2), ('b', 3), ('a', 4)]
\end{verbatim}

(Contributed by Raymond Hettinger.)       

\item The \module{optparse} module was updated in various ways.  The
module now passes its messages through \function{gettext.gettext()},
making it possible to internationalize Optik's help and error
messages.  Help messages for options can now include the string
\code{'\%default'}, which will be replaced by the option's default
value.  (Contributed by Greg Ward.)

\item The long-term plan is to deprecate the \module{rfc822} module
in some future Python release in favor of the \module{email} package.
To this end, the \function{email.Utils.formatdate()} function has been
changed to make it usable as a replacement for
\function{rfc822.formatdate()}.  You may want to write new e-mail
processing code with this in mind.  (Change implemented by Anthony
Baxter.)

\item A new \function{urandom(\var{n})} function was added to the
\module{os} module, returning a string containing \var{n} bytes of
random data.  This function provides access to platform-specific
sources of randomness such as \file{/dev/urandom} on Linux or the
Windows CryptoAPI.  (Contributed by Trevor Perrin.)

\item Another new function: \function{os.path.lexists(\var{path})} 
returns true if the file specified by \var{path} exists, whether or
not it's a symbolic link.  This differs from the existing
\function{os.path.exists(\var{path})} function, which returns false if 
\var{path} is a symlink that points to a destination that doesn't exist.
(Contributed by Beni Cherniavsky.)

\item A new \function{getsid()} function was added to the
\module{posix} module that underlies the \module{os} module.
(Contributed by J. Raynor.)

\item The \module{poplib} module now supports POP over SSL.  (Contributed by
Hector Urtubia.)

\item The \module{profile} module can now profile C extension functions.
(Contributed by Nick Bastin.)

\item The \module{random} module has a new method called
   \method{getrandbits(\var{N})} that returns a long integer \var{N}
   bits in length.  The existing \method{randrange()} method now uses
   \method{getrandbits()} where appropriate, making generation of
   arbitrarily large random numbers more efficient.  (Contributed by
   Raymond Hettinger.)

\item The regular expression language accepted by the \module{re} module
   was extended with simple conditional expressions, written as
   \regexp{(?(\var{group})\var{A}|\var{B})}.  \var{group} is either a
   numeric group ID or a group name defined with \regexp{(?P<group>...)} 
   earlier in the expression.  If the specified group matched, the
   regular expression pattern \var{A} will be tested against the string; if
   the group didn't match, the pattern \var{B} will be used instead.
   (Contributed by Gustavo Niemeyer.)

\item The \module{re} module is also no longer recursive, thanks to a
massive amount of work by Gustavo Niemeyer.  In a recursive regular
expression engine, certain patterns result in a large amount of C
stack space being consumed, and it was possible to overflow the stack.
For example, if you matched a 30000-byte string of \samp{a} characters
against the expression \regexp{(a|b)+}, one stack frame was consumed
per character.  Python 2.3 tried to check for stack overflow and raise
a \exception{RuntimeError} exception, but certain patterns could
sidestep the checking and if you were unlucky Python could segfault.
Python 2.4's regular expression engine can match this pattern without
problems.

\item The \module{signal} module now performs tighter error-checking
on the parameters to the \function{signal.signal()} function.  For
example, you can't set a handler on the \constant{SIGKILL} signal;
previous versions of Python would quietly accept this, but 2.4 will
raise a \exception{RuntimeError} exception.

\item Two new functions were added to the \module{socket} module.
\function{socketpair()} returns a pair of connected sockets and
\function{getservbyport(\var{port})} looks up the service name for a
given port number. (Contributed by Dave Cole and Barry Warsaw.)

\item The \function{sys.exitfunc()} function has been deprecated.  Code
should be using the existing \module{atexit} module, which correctly
handles calling multiple exit functions.  Eventually
\function{sys.exitfunc()} will become a purely internal interface,
accessed only by \module{atexit}.

\item The \module{tarfile} module now generates GNU-format tar files
by default.  (Contributed by Lars Gustaebel.)

\item The \module{threading} module now has an elegantly simple way to support 
thread-local data.  The module contains a \class{local} class whose
attribute values are local to different threads.

\begin{verbatim}
import threading

data = threading.local()
data.number = 42
data.url = ('www.python.org', 80)
\end{verbatim}

Other threads can assign and retrieve their own values for the
\member{number} and \member{url} attributes.  You can subclass
\class{local} to initialize attributes or to add methods.
(Contributed by Jim Fulton.)

\item The \module{timeit} module now automatically disables periodic
  garbage collection during the timing loop.  This change makes
  consecutive timings more comparable.  (Contributed by Raymond Hettinger.)

\item The \module{weakref} module now supports a wider variety of objects
   including Python functions, class instances, sets, frozensets, deques,
   arrays, files, sockets, and regular expression pattern objects.
   (Contributed by Raymond Hettinger.)

\item The \module{xmlrpclib} module now supports a multi-call extension for 
transmitting multiple XML-RPC calls in a single HTTP operation.
(Contributed by Brian Quinlan.)

\item The \module{mpz}, \module{rotor}, and \module{xreadlines} modules have 
been removed.
   
\end{itemize}


%======================================================================
% whole new modules get described in subsections here

%=====================
\subsection{cookielib}

The \module{cookielib} library supports client-side handling for HTTP
cookies, mirroring the \module{Cookie} module's server-side cookie
support. Cookies are stored in cookie jars; the library transparently
stores cookies offered by the web server in the cookie jar, and
fetches the cookie from the jar when connecting to the server. As in
web browsers, policy objects control whether cookies are accepted or
not.

In order to store cookies across sessions, two implementations of
cookie jars are provided: one that stores cookies in the Netscape
format so applications can use the Mozilla or Lynx cookie files, and
one that stores cookies in the same format as the Perl libwww library.

\module{urllib2} has been changed to interact with \module{cookielib}:
\class{HTTPCookieProcessor} manages a cookie jar that is used when
accessing URLs.

This module was contributed by John J. Lee.


% ==================
\subsection{doctest}

The \module{doctest} module underwent considerable refactoring thanks
to Edward Loper and Tim Peters.  Testing can still be as simple as
running \function{doctest.testmod()}, but the refactorings allow
customizing the module's operation in various ways

The new \class{DocTestFinder} class extracts the tests from a given 
object's docstrings:

\begin{verbatim}
def f (x, y):
    """>>> f(2,2)
4
>>> f(3,2)
6
    """
    return x*y

finder = doctest.DocTestFinder()

# Get list of DocTest instances
tests = finder.find(f)
\end{verbatim}

The new \class{DocTestRunner} class then runs individual tests and can
produce a summary of the results:

\begin{verbatim}
runner = doctest.DocTestRunner()
for t in tests:
    tried, failed = runner.run(t)
    
runner.summarize(verbose=1)
\end{verbatim}

The above example produces the following output:

\begin{verbatim}
1 items passed all tests:
   2 tests in f
2 tests in 1 items.
2 passed and 0 failed.
Test passed.
\end{verbatim}

\class{DocTestRunner} uses an instance of the \class{OutputChecker}
class to compare the expected output with the actual output.  This
class takes a number of different flags that customize its behaviour;
ambitious users can also write a completely new subclass of
\class{OutputChecker}.

The default output checker provides a number of handy features.
For example, with the \constant{doctest.ELLIPSIS} option flag,
an ellipsis (\samp{...}) in the expected output matches any substring, 
making it easier to accommodate outputs that vary in minor ways:

\begin{verbatim}
def o (n):
    """>>> o(1)
<__main__.C instance at 0x...>
>>>
"""
\end{verbatim}

Another special string, \samp{<BLANKLINE>}, matches a blank line:

\begin{verbatim}
def p (n):
    """>>> p(1)
<BLANKLINE>
>>>
"""
\end{verbatim}

Another new capability is producing a diff-style display of the output
by specifying the \constant{doctest.REPORT_UDIFF} (unified diffs),
\constant{doctest.REPORT_CDIFF} (context diffs), or
\constant{doctest.REPORT_NDIFF} (delta-style) option flags.  For example:

\begin{verbatim}
def g (n):
    """>>> g(4)
here
is
a
lengthy
>>>"""
    L = 'here is a rather lengthy list of words'.split()
    for word in L[:n]:
        print word
\end{verbatim}

Running the above function's tests with
\constant{doctest.REPORT_UDIFF} specified, you get the following output:

\begin{verbatim}
**********************************************************************
File ``t.py'', line 15, in g
Failed example:
    g(4)
Differences (unified diff with -expected +actual):
    @@ -2,3 +2,3 @@
     is
     a
    -lengthy
    +rather
**********************************************************************
\end{verbatim}


% ======================================================================
\section{Build and C API Changes}

Some of the changes to Python's build process and to the C API are:

\begin{itemize}

  \item Three new convenience macros were added for common return
  values from extension functions: \csimplemacro{Py_RETURN_NONE},
  \csimplemacro{Py_RETURN_TRUE}, and \csimplemacro{Py_RETURN_FALSE}.
  (Contributed by Brett Cannon.)

  \item Another new macro, \csimplemacro{Py_CLEAR(\var{obj})}, 
  decreases the reference count of \var{obj} and sets \var{obj} to the
  null pointer.  (Contributed by Jim Fulton.)

  \item A new function, \cfunction{PyTuple_Pack(\var{N}, \var{obj1},
  \var{obj2}, ..., \var{objN})}, constructs tuples from a variable
  length argument list of Python objects.  (Contributed by Raymond Hettinger.)

  \item A new function, \cfunction{PyDict_Contains(\var{d}, \var{k})},
  implements fast dictionary lookups without masking exceptions raised
  during the look-up process.  (Contributed by Raymond Hettinger.)

  \item The \csimplemacro{Py_IS_NAN(\var{X})} macro returns 1 if 
  its float or double argument \var{X} is a NaN.  
  (Contributed by Tim Peters.)

  \item C code can avoid unnecessary locking by using the new
   \cfunction{PyEval_ThreadsInitialized()} function to tell 
   if any thread operations have been performed.  If this function 
   returns false, no lock operations are needed.
   (Contributed by Nick Coghlan.)

  \item A new function, \cfunction{PyArg_VaParseTupleAndKeywords()},
  is the same as \cfunction{PyArg_ParseTupleAndKeywords()} but takes a 
  \ctype{va_list} instead of a number of arguments.
  (Contributed by Greg Chapman.)

  \item A new method flag, \constant{METH_COEXISTS}, allows a function
  defined in slots to co-exist with a \ctype{PyCFunction} having the
  same name.  This can halve the access time for a method such as
  \method{set.__contains__()}.  (Contributed by Raymond Hettinger.)

  \item Python can now be built with additional profiling for the
  interpreter itself, intended as an aid to people developing the
  Python core.  Providing \longprogramopt{--enable-profiling} to the
  \program{configure} script will let you profile the interpreter with
  \program{gprof}, and providing the \longprogramopt{--with-tsc}
  switch enables profiling using the Pentium's Time-Stamp-Counter
  register.  Note that the \longprogramopt{--with-tsc} switch is slightly
  misnamed, because the profiling feature also works on the PowerPC
  platform, though that processor architecture doesn't call that
  register ``the TSC register''.  (Contributed by Jeremy Hylton.)
    
  \item The \ctype{tracebackobject} type has been renamed to \ctype{PyTracebackObject}.

\end{itemize}


%======================================================================
\subsection{Port-Specific Changes}

\begin{itemize}

\item The Windows port now builds under MSVC++ 7.1 as well as version 6.
  (Contributed by Martin von~L\"owis.)

\end{itemize}



%======================================================================
\section{Porting to Python 2.4}

This section lists previously described changes that may require
changes to your code:

\begin{itemize}

\item Left shifts and hexadecimal/octal constants that are too 
  large no longer trigger a \exception{FutureWarning} and return 
  a value limited to 32 or 64 bits; instead they return a long integer.

\item Integer operations will no longer trigger an \exception{OverflowWarning}.
The \exception{OverflowWarning} warning will disappear in Python 2.5.

\item The \function{zip()} built-in function and \function{itertools.izip()}
  now return  an empty list instead of raising a \exception{TypeError}
  exception if called with no arguments.

\item You can no longer compare the \class{date} and \class{datetime}
  instances provided by the \module{datetime} module.  Two 
  instances of different classes will now always be unequal, and 
  relative comparisons (\code{<}, \code{>}) will raise a \exception{TypeError}.

\item \function{dircache.listdir()} now passes exceptions to the caller
      instead of returning empty lists.

\item \function{LexicalHandler.startDTD()} used to receive the public and
  system IDs in the wrong order.  This has been corrected; applications
  relying on the wrong order need to be fixed.

\item \function{fcntl.ioctl} now warns if the \var{mutate} 
 argument is omitted and relevant.

\item The \module{tarfile} module now generates GNU-format tar files
by default.

\item Encountering a failure while importing a module no longer leaves
a partially-initialized module object in \code{sys.modules}.  

\item \constant{None} is now a constant; code that binds a new value to 
the name \samp{None} is now a syntax error.

\item The \function{signals.signal()} function now raises a
\exception{RuntimeError} exception for certain illegal values;
previously these errors would pass silently.  For example, you can no
longer set a handler on the \constant{SIGKILL} signal.

\end{itemize}


%======================================================================
\section{Acknowledgements \label{acks}}

The author would like to thank the following people for offering
suggestions, corrections and assistance with various drafts of this
article: Koray Can, Hye-Shik Chang, Michael Dyck, Raymond Hettinger,
Brian Hurt, Hamish Lawson, Fredrik Lundh, Sean Reifschneider,
Sadruddin Rejeb.

\end{document}
