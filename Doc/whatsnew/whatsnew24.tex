\documentclass{howto}
\usepackage{distutils}
% $Id$

\title{What's New in Python 2.4}
\release{0.0}
\author{A.M.\ Kuchling}
\authoraddress{\email{amk@amk.ca}}

\begin{document}
\maketitle
\tableofcontents

This article explains the new features in Python 2.4.  No release date
for Python 2.4 has been set; expect that this will happen mid-2004.

While Python 2.3 was primarily a library development release, Python
2.4 may extend the core language and interpreter in
as-yet-undetermined ways.

This article doesn't attempt to provide a complete specification of
the new features, but instead provides a convenient overview.  For
full details, you should refer to the documentation for Python 2.4.
% add hyperlink when the documentation becomes available online.
If you want to understand the complete implementation and design
rationale, refer to the PEP for a particular new feature.

%======================================================================
\section{PEP 218: Built-In Set Objects}

Two new built-in types, \function{set(iterable)} and
\function{frozenset(iterable)} provide high speed data types for
membership testing, for eliminating duplicates from sequences, and
for mathematical operations like unions, intersections, differences,
and symmetric differences.  

\begin{verbatim}
>>> a = set('abracadabra')              # form a set from a string
>>> 'z' in a                            # fast membership testing
False
>>> a                                   # unique letters in a
set(['a', 'r', 'b', 'c', 'd'])
>>> ''.join(a)                          # convert back into a string
'arbcd'

>>> b = set('alacazam')                 # form a second set
>>> a - b                               # letters in a but not in b
set(['r', 'd', 'b'])
>>> a | b                               # letters in either a or b
set(['a', 'c', 'r', 'd', 'b', 'm', 'z', 'l'])
>>> a & b                               # letters in both a and b
set(['a', 'c'])
>>> a ^ b                               # letters in a or b but not both
set(['r', 'd', 'b', 'm', 'z', 'l'])

>>> a.add('z')                          # add a new element
>>> a.update('wxy')                     # add multiple new elements
>>> a
set(['a', 'c', 'b', 'd', 'r', 'w', 'y', 'x', 'z'])       
>>> a.remove('x')                       # take one element out
>>> a
set(['a', 'c', 'b', 'd', 'r', 'w', 'y', 'z'])       
\end{verbatim}

The type \function{frozenset()} is an immutable version of \function{set()}.
Since it is immutable and hashable, it may be used as a dictionary key or
as a member of another set.  Accordingly, it does not have methods
like \method{add()} and \method{remove()} which could alter its contents.

\begin{seealso}
\seepep{218}{Adding a Built-In Set Object Type}{Originally proposed by
Greg Wilson and ultimately implemented by Raymond Hettinger.}
\end{seealso}

%======================================================================
\section{PEP 322: Reverse Iteration}

A new built-in function, \function{reversed(seq)}, takes a sequence
and returns an iterator that returns the elements of the sequence 
in reverse order.  

\begin{verbatim}
>>> for i in reversed(xrange(1,4)):
...    print i
... 
3
2
1
\end{verbatim}

Compared to extended slicing, \code{range(1,4)[::-1]}, \function{reversed()}
is easier to read, runs faster, and uses substantially less memory.

Note that \function{reversed()} only accepts sequences, not arbitrary
iterators.  If you want to reverse an iterator, first convert it to 
a list with \function{list()}.

\begin{verbatim}
>>> input = open('/etc/passwd', 'r')
>>> for line in reversed(list(input)):
...   print line
... 
root:*:0:0:System Administrator:/var/root:/bin/tcsh
  ...
\end{verbatim}

\begin{seealso}
\seepep{322}{Reverse Iteration}{Written and implemented by Raymond Hettinger.}

\end{seealso}


%======================================================================
\section{Other Language Changes}

Here are all of the changes that Python 2.4 makes to the core Python
language.

\begin{itemize}

\item The string methods, \method{ljust()}, \method{rjust()}, and
\method{center()} now take an optional argument for specifying a
fill character other than a space.

\item The \method{sort()} method of lists gained three keyword
arguments, \var{cmp}, \var{key}, and \var{reverse}.  These arguments
make some common usages of \method{sort()} simpler.  All are optional.

\var{cmp} is the same as the previous single argument to
\method{sort()}; if provided, the value should be a comparison
function that takes two arguments and returns -1, 0, or +1 depending
on how the arguments compare.  

\var{key} should be a single-argument function that takes a list
element and returns a comparison key for the element.  The list is
then sorted using the comparison keys.  The following example sorts a
list case-insensitively:

\begin{verbatim}
>>> L = ['A', 'b', 'c', 'D']
>>> L.sort()                 # Case-sensitive sort
>>> L
['A', 'D', 'b', 'c']
>>> L.sort(key=lambda x: x.lower())
>>> L
['A', 'b', 'c', 'D']
>>> L.sort(cmp=lambda x,y: cmp(x.lower(), y.lower()))
>>> L
['A', 'b', 'c', 'D']
\end{verbatim}

The last example, which uses the \var{cmp} parameter, is the old way
to perform a case-insensitive sort.  It works, but is slower than
using a \var{key} parameter.  Using \var{key} results in calling the
\method{lower()} method once for each element in the list while using
\var{cmp} will call the method twice for each comparison.

For simple key functions and comparison functions, it is often
possible to avoid a \keyword{lambda} expression by using an unbound
method instead.  For example, the above case-insensitive sort is best
coded as:

\begin{verbatim}
>>> L.sort(key=str.lower)
>>> L
['A', 'b', 'c', 'D']
\end{verbatim}       

The \var{reverse} parameter should have a Boolean value.  If the value is
\constant{True}, the list will be sorted into reverse order.  Instead
of \code{L.sort(lambda x,y: cmp(y.score, x.score))}, you can now write:
\code{L.sort(key = lambda x: x.score, reverse=True)}.

The results of sorting are now guaranteed to be stable.  This means
that two entries with equal keys will be returned in the same order as
they were input.  For example, you can sort a list of people by name,
and then sort the list by age, resulting in a list sorted by age where
people with the same age are in name-sorted order.

\item The list type gained a \method{sorted(iterable)} method that works
like the in-place \method{sort()} method but has been made suitable for
use in expressions.  The differences are:
  \begin{itemize}
  \item the input may be any iterable;
  \item a newly formed copy is sorted, leaving the original intact; and
  \item the expression returns the new sorted copy
  \end{itemize}

\begin{verbatim}
>>> L = [9,7,8,3,2,4,1,6,5]
>>> [10+i for i in list.sorted(L)]  # usable in a list comprehension
[11, 12, 13, 14, 15, 16, 17, 18, 19]
>>> L = [9,7,8,3,2,4,1,6,5]         # original is left unchanged
[9,7,8,3,2,4,1,6,5]   

>>> list.sorted('Monte Python')     # any iterable may be an input
[' ', 'M', 'P', 'e', 'h', 'n', 'n', 'o', 'o', 't', 't', 'y']

>>> # List the contents of a dict sorted by key values
>>> colormap = dict(red=1, blue=2, green=3, black=4, yellow=5)
>>> for k, v in list.sorted(colormap.iteritems()):
...     print k, v
...
black 4
blue 2
green 3
red 1
yellow 5

\end{verbatim}


\item The \function{zip()} built-in function and \function{itertools.izip()}
  now return an empty list instead of raising a \exception{TypeError}
  exception if called with no arguments.  This makes the functions more
  suitable for use with variable length argument lists:

\begin{verbatim}
>>> def transpose(array):
...    return zip(*array)
...
>>> transpose([(1,2,3), (4,5,6)])
[(1, 4), (2, 5), (3, 6)]
>>> transpose([])
[]
\end{verbatim}

\end{itemize}


%======================================================================
\subsection{Optimizations}

\begin{itemize}

\item Optimizations should be described here.

\end{itemize}

The net result of the 2.4 optimizations is that Python 2.4 runs the
pystone benchmark around XX\% faster than Python 2.3 and YY\% faster
than Python 2.2.


%======================================================================
\section{New, Improved, and Deprecated Modules}

As usual, Python's standard library received a number of enhancements and
bug fixes.  Here's a partial list of the most notable changes, sorted
alphabetically by module name. Consult the
\file{Misc/NEWS} file in the source tree for a more
complete list of changes, or look through the CVS logs for all the
details.

\begin{itemize}

\item The \module{curses} modules now supports the ncurses extension 
   \function{use_default_colors()}.   On platforms where the terminal 
   supports transparency, this makes it possible to use a transparent background.
   (Contributed by J\"org Lehmann.)

\item The \module{heapq} module has been converted to C.  The resulting
   ten-fold improvement in speed makes the module suitable for handling
   high volumes of data.

\item The \module{imaplib} module now supports IMAP's THREAD command.
(Contributed by Yves Dionne.)

\item The \module{itertools} module gained a
  \function{groupby(\var{iterable}\optional{, \var{func}})} function,
  inspired by the GROUP BY clause from SQL.
  \var{iterable} returns a succession of elements, and the optional
  \var{func} is a function that takes an element and returns a key
  value; if omitted, the key is simply the element itself.
  \function{groupby()} then groups the elements into subsequences
  which have matching values of the key, and returns a series of 2-tuples
  containing the key value and an iterator over the subsequence.
 
Here's an example.  The \var{key} function simply returns whether a
number is even or odd, so the result of \function{groupby()} is to
return consecutive runs of odd or even numbers.

\begin{verbatim}
>>> import itertools
>>> L = [2,4,6, 7,8,9,11, 12, 14]
>>> for key_val, it in itertools.groupby(L, lambda x: x % 2):
...    print key_val, list(it)
... 
0 [2, 4, 6]
1 [7]
0 [8]
1 [9, 11]
0 [12, 14]
>>> 
\end{verbatim}

Like its SQL counterpart, \function{groupby()} is typically used with
sorted input.  The logic for \function{groupby()} is similar to the
\UNIX{} \code{uniq} filter which makes it handy for eliminating,
counting, or identifying duplicate elements:

\begin{verbatim}
>>> word = 'abracadabra'
>>> [k for k, g in groupby(list.sorted(word))]
['a', 'b', 'c', 'd', 'r']
>>> [(k, len(list(g))) for k, g in groupby(list.sorted(word))]
[('a', 5), ('b', 2), ('c', 1), ('d', 1), ('r', 2)]
>>> [k for k, g in groupby(list.sorted(word)) if len(list(g)) > 1]
['a', 'b', 'r']
\end{verbatim}

\item A new \function{getsid()} function was added to the
\module{posix} module that underlies the \module{os} module.
(Contributed by J. Raynor.)

\item The \module{random} module has a new method called \method{getrandbits(N)} 
   which returns an N-bit long integer.  This method supports the existing
   \method{randrange()} method, making it possible to efficiently generate
   arbitrarily large random numbers (suitable for prime number generation in
   RSA applications).

\item The regular expression language accepted by the \module{re} module
   was extended with simple conditional expressions, written as
   \code{(?(\var{group})\var{A}|\var{B})}.  \var{group} is either a
   numeric group ID or a group name defined with \code{(?P<group>...)} 
   earlier in the expression.  If the specified group matched, the
   regular expression pattern \var{A} will be tested against the string; if
   the group didn't match, the pattern \var{B} will be used instead.
   
\end{itemize}


%======================================================================
% whole new modules get described in \subsections here


% ======================================================================
\section{Build and C API Changes}

Changes to Python's build process and to the C API include:

\begin{itemize}

  \item Three new convenience macros were added for common return
  values from extension functions: \csimplemacro{Py_RETURN_NONE},
  \csimplemacro{Py_RETURN_TRUE}, and \csimplemacro{Py_RETURN_FALSE}.

  \item A new function, \cfunction{PyTuple_Pack(N, obj1, obj2, ...,
  objN)}, constructs tuples from a variable length argument list of
  Python objects.

  \item A new function, \cfunction{PyDict_Contains(d, k)}, implements
  fast dictionary lookups without masking exceptions raised during the
  look-up process.

\end{itemize}


%======================================================================
\subsection{Port-Specific Changes}

Platform-specific changes go here.


%======================================================================
\section{Other Changes and Fixes \label{section-other}}

As usual, there were a bunch of other improvements and bugfixes
scattered throughout the source tree.  A search through the CVS change
logs finds there were XXX patches applied and YYY bugs fixed between
Python 2.3 and 2.4.  Both figures are likely to be underestimates.

Some of the more notable changes are:

\begin{itemize}

\item Details go here.

\end{itemize}


%======================================================================
\section{Porting to Python 2.4}

This section lists previously described changes that may require
changes to your code:

\begin{itemize}

\item The \function{zip()} built-in function and \function{itertools.izip()}
  now return  an empty list instead of raising a \exception{TypeError}
  exception if called with no arguments.

\item \function{dircache.listdir()} now passes exceptions to the caller
      instead of returning empty lists.

\end{itemize}


%======================================================================
\section{Acknowledgements \label{acks}}

The author would like to thank the following people for offering
suggestions, corrections and assistance with various drafts of this
article: Raymond Hettinger.

\end{document}
